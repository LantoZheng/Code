\documentclass[a4paper,12pt]{ctexart}
\usepackage[left=2.5cm, right=2.5cm, top=2.5cm, bottom=2.5cm]{geometry}
\usepackage{booktabs} % 用于美观的表格
\usepackage{enumitem} % 用于列表调整
\usepackage{titlesec} % 用于标题格式

% 标题格式设置
\titleformat{\section}{\large\bfseries}{\thesection}{1em}{}
\titleformat{\subsection}{\normalsize\bfseries}{\thesubsection}{1em}{}

\title{\textbf{赴日留学日语自学计划书 (2025-2026)}}
\author{姓名:[您的姓名] \quad 专业:物理学系}
\date{\today}

\begin{document}

\maketitle

\section{背景与目标}
\begin{itemize}[leftmargin=*]
    \item \textbf{个人情况}:北京师范大学物理学系四年级本科生。
    \item \textbf{留学计划}:计划于 \textbf{2026年10月} 前往日本攻读硕士学位(物理/经济方向)。
    \item \textbf{当前水平}:零基础入门(仅掌握五十音图)。
    \item \textbf{学习目标}:在赴日前达到能够应对日常生活及研究室基本沟通的水平(约 JLPT N3 口语/听力水平)。
    \item \textbf{投入时间}:每周约 \textbf{8--10 小时}(利用课余及周末)。
\end{itemize}

\section{使用教材与资源}
\begin{itemize}[leftmargin=*]
    \item \textbf{核心教材}:《Genki I \& II: An Integrated Course in Elementary Japanese》(第3版)。
    \item \textbf{辅助工具}:Anki(单词记忆)、OTO Navi(音频跟读)。
    \item \textbf{校内资源}:北师大图书馆分级读物、留学生语伴交流。
\end{itemize}

\section{阶段性进度规划}

\subsection*{阶段一:基础架构搭建 (2025.12 -- 2026.04)}
\textbf{目标}:完成《Genki I》(第1-12课),达到 JLPT N5 水平。
\begin{itemize}
    \item 重点掌握形容词变形、动词 Te 形 (Te-form) 及基础句式。
    \item 能够进行简单的自我介绍、购物及描述日常生活。
\end{itemize}

\subsection*{阶段二:进阶表达与动词攻坚 (2026.04 -- 2026.08)}
\textbf{目标}:完成《Genki II》(第13-23课),达到 JLPT N4 水平。
\begin{itemize}
    \item 熟练切换“普通体”与“礼貌体”。
    \item 掌握可能态、被动态、使役态(为学术日语打基础)。
\end{itemize}

\subsection*{阶段三:实战应用与学术衔接 (2026.09)}
\textbf{目标}:学术场景模拟与生存日语。
\begin{itemize}
    \item 针对租房、役所办理手续进行专项词汇突击。
    \item 学习研究室常用语(敬语基础),尝试阅读专业论文摘要。
\end{itemize}

\section{周学习时间表 (示例)}
基于“重听说、轻手写”的原则,每周安排如下:

\begin{table}[h]
\centering
\begin{tabular}{lp{10cm}}
\toprule
\textbf{时间} & \textbf{学习内容} \\ 
\midrule
\textbf{周一 (1h)} & \textbf{新课输入}:学习教材语法点,理解逻辑。 \\
\textbf{周二--周四 (1.5h)} & \textbf{碎片复习}:利用 Anki 背单词,做课后练习 (Workbook)。 \\
\textbf{周五 (1h)} & \textbf{口语训练}:影子跟读 (Shadowing) 或寻找语伴练习。 \\
\textbf{周六 (4h)} & \textbf{深度攻坚}:复习全周内容,完成单元测试,阅读简单的分级读物。 \\
\textbf{周日 (1h)} & \textbf{机动/休息}:观看日剧/动漫磨耳朵,调整下周计划。 \\
\bottomrule
\end{tabular}
\end{table}

\section{希望咨询老师的问题}
\begin{enumerate}
    \item 考虑到物理系课业较重,目前的进度安排(约两周一课)是否合理?
    \item 针对硕士留学的需求,在完成初级教材后,是否有必要专门学习《物理日语》之类的专业词汇书,还是直接看论文更好?
    \item 在自学过程中,最容易被忽视的“坏习惯”有哪些?希望老师能给一些避坑建议。
\end{enumerate}

\end{document}