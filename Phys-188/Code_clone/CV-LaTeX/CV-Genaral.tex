\documentclass[10pt]{article}
\usepackage[utf8]{inputenc}
\usepackage[T1]{fontenc}
\usepackage{geometry}
% Adjusted margins to match the original document's density, which has very narrow margins.
\geometry{a4paper, left=0.6in, right=0.6in, top=0.6in, bottom=0.6in}
\usepackage{helvet} % Use a Helvetica-like sans-serif font for a modern look.
\renewcommand{\familydefault}{\sfdefault} % Set sans-serif as the default font.
\usepackage{enumitem} % For customizing list environments (itemize, enumerate) to control spacing and bullet styles.
\usepackage{hyperref} % For creating clickable links, especially for the email address.
\usepackage{ragged2e} % For \RaggedRight to allow hyphens while generally left-aligning

% Remove page numbers
\pagestyle{empty}

% Globally reduce space before and between items in itemize lists for a compact layout.
% Adjusted leftmargin for itemize to be slightly larger to match the original's bullet indentation.
\setlist[itemize]{leftmargin=1.8em, itemsep=0pt, parsep=0pt, topsep=0pt, partopsep=0pt}
% Set enumerate list to use a circle (\(\circ\)) for its label, and also control spacing.
% Adjusted leftmargin for enumerate to match the nested indentation.
\setlist[enumerate]{label=\(\circ\), leftmargin=2.2em, itemsep=0pt, parsep=0pt, topsep=0pt, partopsep=0pt}


% --- Custom Commands and Environments ---
% Command for consistent section titles (e.g., EDUCATION, SKILLS)
\newcommand{\cvsection}[1]{%
    \vspace{8pt}% Space before section title
    \noindent\textbf{\Large #1}\\[-0.5\baselineskip]% Bold and Large, adjust vertical space
    \rule{\linewidth}{0.4pt}% Horizontal rule
    \vspace{0.5\baselineskip}% Space after the rule
}

% Environment for sub-sections like Scholarships, Language with consistent indentation
\newenvironment{cvsubsectionblock}[1]{%
    \noindent \textbf{#1:}\\%
    \begin{list}{}{\setlength{\leftmargin}{0.5cm}\setlength{\itemsep}{0pt}\setlength{\parsep}{0pt}\setlength{\topsep}{0pt}\setlength{\partopsep}{0pt}}%
    \item\relax % Start the item without an explicit bullet, just for indentation.
    \vspace{-\baselineskip}% Pull up to make the list start immediately after the label
}{\end{list}}

% --- Main Document ---
\begin{document}
% Adjust vertical space from the top of the page to match the original document's high header placement.
\vspace*{-0.5in} % Slightly adjusted compared to previous 0.7in

% Header Section: Name and Contact Information
\begin{center}
    \Huge\textbf{Xiaoyang Zheng}\vspace{5pt} % Name: Huge, Bold
    \normalsize Email: \href{mailto:xiaoyangzheng@mail.bnu.edu.cn}{xiaoyangzheng@mail.bnu.edu.cn} \quad Mob: +86 13955190184\\
    Address: Room 328, Student Hall 15, Beijing Normal University, Beijing, China
\end{center}
\vspace{10pt} % Space after the header block.

% EDUCATION Section
\cvsection{EDUCATION}

% Education Details
\noindent Beijing Normal University, Beijing, China \hfill 2021.09-Now\\
\hspace*{0.5cm}Undergraduate of Science in Theoretical Physics, Liyun Elite Program \hfill (expected in 2026.07)\\
\hspace*{0.5cm}Undergraduate of Business in Economics \hfill (expected in 2026.07)\vspace{0pt}% Added to ensure consistency after last line of this block
\noindent Ranking : 2 (out of 23)\\
\noindent GPA (overall): 3.7/4.0;\\
% Core courses with hanging indentation
\noindent\RaggedRight\textbf{Core courses:} Optics (93)/ Quantum Mechanics I\&II (89)/ Introduction to Computational Physics (95)/%
\par\hangindent=1.3cm\hangafter=1 Seminar on Optics (95)/ Mechanics (95) / Electromagnetism (97) / Electrodynamics(91) / Solid-state Physics(82)\vspace{5pt}

% Scholarships Sub-section
\begin{cvsubsectionblock}{Scholarships}
    Outstanding Freshman Scholarship \hfill 2021.09\\
    First-class Scholarship of Beijing Normal University \hfill 2024.10 \& 2022.10\\
    Beijing Normal University First-class Incentive Scholarship \hfill 2024.10 \& 2023.10 \& 2022.10
\end{cvsubsectionblock}\vspace{5pt}

% Language Sub-section
\begin{cvsubsectionblock}{Language}
    IELTS:7.5 (Reading:8.0 Listening:8.5 Speaking:6.5 Writing:6.0)\\
    TOEFL:101 (Reading:30 Listening:28 Speaking:20 Writing:22)
\end{cvsubsectionblock}\vspace{10pt}

% SKILLS Section
\cvsection{SKILLS}

% Computer Skills
\begin{cvsubsectionblock}{Computer}
    \begin{itemize} % Uses global itemize settings.
        \item Python (with PyTorch, SciPy ...)
        \item MatLab
        \item C++
    \end{itemize}
\end{cvsubsectionblock}
\vspace{-0.5\baselineskip} % Adjust spacing after the list, pulling it closer to the next line.
% Lab Skills
\noindent \textbf{Lab Skills:} Using Scanning Electron Microscopy (SEM) and Transmission Electron Microscopy (TEM) for materials characterization and analysis.\vspace{10pt}

% RESEARCH EXPERIENCES Section
\cvsection{RESEARCH EXPERIENCES}

% Project 1: Single-layer Diffractive Neural Network (D2NN)
\noindent \textbf{Single-layer Diffractive Neural Network (D2NN)} \hfill 2024.12-Now\\
\begin{itemize}
    \item \textbf{Project Description:} Participated in the design and implementation of a Single-layer Diffractive Neural Network (D2NN). D2NN aimed to transfer neural network inference to the optical domain, utilizing light diffraction for high-speed parallel computation to accelerate processing speed. Addressing the issue of significant output layer intensity loss in traditional multi-layer mask-based D2NNs, I initiated this project to explore the possibility of achieving similar performance using a single phase mask within a 4f optical system.
    \item \textbf{Key Responsibilities \& Contributions:}\vspace{-0.5\baselineskip} % Pull up for tight spacing
    \begin{enumerate} % Uses global enumerate settings for circular bullet points.
        \item Designed the optical weights by utilizing a Spatial Light Modulator (SLM) placed in the Fourier plane of a 4f system, constructing the core of the optical neural network.
        \item Formulated the optical neural network transfer function based on diffraction principles.
        \item Trained the D2NN using optical simulation methods with the PyTorch framework and the MNIST handwritten digit dataset.
    \end{enumerate}
    \vspace{-0.5\baselineskip} % Pull up for tight spacing
    \item \textbf{Achievements:} Successfully designed and trained a single-layer phase-mask-based optical neural network. Achieved high-speed inference on the MNIST dataset through optical simulation, demonstrating the potential of single-layer D2NNs for optical computing.
\end{itemize}\vspace{5pt}

% Project 2: Self-calibrating Beam Shaping Based on Reflective Spatial Light Modulator
\noindent \textbf{Self-calibrating Beam Shaping Based on Reflective Spatial Light Modulator | Course Project}\vspace{-2pt}
\begin{itemize}
    \item \textbf{Project Description:} This project focused on developing a self-calibrating optical system for real-time beam shaping and wavefront correction. The primary goal was to compensate wavefront distortions and higher-order modes commonly present in laser beams by utilizing a reflective spatial light modulator (SLM) in a feedback control loop.
    \item \textbf{Key Responsibilities \& Contributions:}\vspace{-0.5\baselineskip}
    \begin{enumerate}
        \item Designed and constructed the core optical system, integrating a reflective spatial light modulator (SLM) as the key element for wavefront phase modulation.
        \item Incorporated a CCD camera equipped with a microlens array to serve as a wavefront sensor, enabling real-time measurement of the beam's wavefront shape.
        \item Developed a Python program for the control system to process the wavefront data acquired from the CCD sensor.
        \item Implemented optimization algorithms, such as stochastic gradient descent or simulated annealing, within the Python program to minimize the square error between the measured and desired wavefronts.
    \end{enumerate}
    \vspace{-0.5\baselineskip} % Pull up for tight spacing
    \item \textbf{Achievements:} Successfully designed, simulated, and programmed a self-calibrating optical system capable of real-time wavefront correction and dynamic beam shaping. Demonstrated the ability to generate and maintain beams with specific wavefront shapes, including Gaussian and Laguerre-Gaussian profiles, by actively controlling the reflective SLM based on wavefront sensor feedback.
\end{itemize}\vspace{5pt}

% Project 3: Micro and Nano Optics | Undergraduate Research Assistant
\noindent \textbf{Micro and Nano Optics | Undergraduate Research Assistant} \hfill 2022.4-2024.6\\
\noindent Advisor: Jinwei Shi (Professor of Physics Department, Beijing Normal University)\\
\noindent Funding: Awarded funding from the Beijing Undergraduate Research Training Program.\\
\begin{itemize}
    \item \textbf{Optical Surface Structure Characterization:} Utilized Scanning Electron Microscopy (SEM) to determine the size and morphology of nanoparticles. Performed optical spectroscopy measurements to analyze their plasmon resonance properties.
    \item \textbf{FDTD Optical Properties Simulation:} Conducted simulations using the Finite-Difference Time-Domain (FDTD) method to study the electromagnetic modes of gold nanorods excited by visible light. Investigated the influence of nanorod length, width, and length uniformity on the position and width of absorption peaks induced by the second-order excitation mode in 2D materials.
\end{itemize}\vspace{5pt}

% Project 4: Atomic Co-Magnetometry | Undergraduate Research Assistant
\noindent \textbf{Atomic Co-Magnetometry | Undergraduate Research Assistant} \hfill 2023.06-2023.08\\
\noindent Advisor: Dong Sheng (Professor of Department of Precision Machinery and Instrumentation, University of Science and Technology of China)\\
\begin{itemize}
    \item Developed a working knowledge of the principles and operation of atomic magnetometers and co-magnetometers, focusing on their application in precision measurements.
    \item Conducted thermal simulations using COMSOL to optimize the design and ensure the thermal stability of critical components within the atomic co-magnetometer setup.
    \item Participated in the construction and alignment of optical paths for laser-based interrogation of atomic states and implemented electronic measurement circuits for signal acquisition and noise reduction in atomic magnetometry experiments.
\end{itemize}\vspace{10pt}

% WORK EXPERIENCES Section
\cvsection{WORK EXPERIENCES}

% Work Experience 1: BNUPA(Beijing Normal University Photographer Association)
\noindent \textbf{BNUPA(Beijing Normal University Photographer Association) | Chairman} \hfill 2023.09-2024.09(expected)\\
\begin{itemize}
    \item Hosted and organized multiple lectures and interviews with external experts. This included 3 lectures and 2 interviews, engaging approximately 200 student attendees.
    \item Delivered several lecture series, including "Optical Concepts in Photography" and "Imaging System Quality from the Perspective of Photographic Equipment".
\end{itemize}\vspace{5pt}

% Work Experience 2: Homoludens Archive
\noindent \textbf{Homoludens Archive | Undergraduate Researcher \& Archives Administrator} \hfill 2022.09-2023.07\\

\end{document}
