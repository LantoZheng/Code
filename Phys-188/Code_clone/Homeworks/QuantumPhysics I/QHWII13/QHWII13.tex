\documentclass[12pt, a4paper]{article}
\usepackage[utf8]{inputenc}
\usepackage[T1]{fontenc}
\usepackage{amsmath, amssymb}
\usepackage{geometry}
\geometry{left=2.5cm, right=2.5cm, top=2.5cm, bottom=2.5cm}
\usepackage{bm} 
\usepackage{ctex}

\begin{document}
\title{量子力学 II 作业13}
\author{郑晓旸 \\ 202111030007}
\date{\today}
\maketitle
\section*{习题 1}
考虑两个厄密算符 \(\hat{A}\) 和 \(\hat{B}\) 的共同本征态 \(|\Psi\rangle\)。如果 \(\hat{A}\hat{B} + \hat{B}\hat{A} = 0\),则 \(|\Psi\rangle\) 有什么性质?可以利用宇称算符 \(\hat{\pi}\) 和动量算符 \(\hat{\bm{p}}\) 来说明。
\section{解答}
\subsection{一般性质推导}
设 \(|\Psi\rangle\) 是厄密算符 \(\hat{A}\) 和 \(\hat{B}\) 的共同本征态,对应的本征值分别为 \(a\) 和 \(b\)。由于 \(\hat{A}\) 和 \(\hat{B}\) 是厄密算符,它们的本征值 \(a\) 和 \(b\) 都是实数。
根据定义,我们有:
\[ \hat{A}|\Psi\rangle = a|\Psi\rangle \]
\[ \hat{B}|\Psi\rangle = b|\Psi\rangle \]
其中 \(|\Psi\rangle \neq 0\)。
题目给出的条件是两个算符的反对易子为零:
\[ \hat{A}\hat{B} + \hat{B}\hat{A} = 0 \]
我们将这个算符作用在共同本征态 \(|\Psi\rangle\) 上:
\[ (\hat{A}\hat{B} + \hat{B}\hat{A})|\Psi\rangle = 0 |\Psi\rangle = 0 \]
展开左边:
\[ \hat{A}\hat{B}|\Psi\rangle + \hat{B}\hat{A}|\Psi\rangle = 0 \]
利用 \(|\Psi\rangle\) 是 \(\hat{A}\) 和 \(\hat{B}\) 的共同本征态的性质:
\[ \hat{A}(b|\Psi\rangle) + \hat{B}(a|\Psi\rangle) = 0 \]
由于 \(a\) 和 \(b\) 是标量(本征值),可以将它们提到算符作用之外:
\[ b(\hat{A}|\Psi\rangle) + a(\hat{B}|\Psi\rangle) = 0 \]
再次利用本征态的性质:
\[ b(a|\Psi\rangle) + a(b|\Psi\rangle) = 0 \]
\[ ab|\Psi\rangle + ab|\Psi\rangle = 0 \]
\[ 2ab|\Psi\rangle = 0 \]
因为 \(|\Psi\rangle\) 是本征态,所以它不是零矢量,即 \(|\Psi\rangle \neq 0\)。因此,我们必须有:
\[ 2ab = 0 \]
这意味着:
\[ ab = 0 \]
这个结论表明,对于共同本征态 \(|\Psi\rangle\),其对应的算符 \(\hat{A}\) 的本征值 \(a\) 与算符 \(\hat{B}\) 的本征值 \(b\) 的乘积必须为零。换句话说,\textbf{至少有一个本征值为零}。即,要么 \(a=0\),要么 \(b=0\),或者两者都为零。
所以,如果两个厄密算符 \(\hat{A}\) 和 \(\hat{B}\) 满足 \(\hat{A}\hat{B} + \hat{B}\hat{A} = 0\),并且它们存在共同的本征态 \(|\Psi\rangle\),那么对于这个态 \(|\Psi\rangle\),它要么是算符 \(\hat{A}\) 的零本征值本征态,要么是算符 \(\hat{B}\) 的零本征值本征态(或两者皆是)。
\subsection{利用宇称算符和动量算符说明}
我们考虑一维空间中的宇称算符 \(\hat{\pi}\) 和动量算符 \(\hat{p}\)(为简洁,这里用 \(\hat{p}\) 代表 \(\hat{p}_x\))。
\textbf{宇称算符 \(\hat{\pi}\)}:
宇称算符作用于坐标的效应是 \(\hat{\pi} \hat{x} \hat{\pi}^\dagger = -\hat{x}\)。它是一个厄密算符 (\(\hat{\pi}^\dagger = \hat{\pi}\)) 且满足 \(\hat{\pi}^2 = \hat{I}\)(其中 \(\hat{I}\) 是单位算符)。因此,宇称算符的本征值只能是 \(+1\)(偶宇称)或 \(-1\)(奇宇称)。这些本征值均不为零。
\textbf{动量算符 \(\hat{p}\)}:
动量算符 \(\hat{p} = -i\hbar \frac{d}{dx}\) 是一个厄密算符。其本征值可以是任意实数。
\textbf{宇称算符与动量算符的关系}:
宇称算符作用于动量算符的效应是:
\[ \hat{\pi} \hat{p} \hat{\pi}^\dagger = -\hat{p} \]
因为 \(\hat{\pi}^\dagger = \hat{\pi}\) 且 \(\hat{\pi}^2 = \hat{I}\),我们可以将上式右乘 \(\hat{\pi}\):
\[ \hat{\pi} \hat{p} \hat{\pi} \hat{\pi} = -\hat{p} \hat{\pi} \]
\[ \hat{\pi} \hat{p} \hat{I} = -\hat{p} \hat{\pi} \]
\[ \hat{\pi} \hat{p} = -\hat{p} \hat{\pi} \]
这可以改写为:
\[ \hat{\pi} \hat{p} + \hat{p} \hat{\pi} = 0 \]
这完全符合题目中 \(\hat{A}\hat{B} + \hat{B}\hat{A} = 0\) 的形式,其中我们可以令 \(\hat{A} = \hat{\pi}\) 和 \(\hat{B} = \hat{p}\)。
\textbf{共同本征态的性质}:
假设存在一个宇称算符 \(\hat{\pi}\) 和动量算符 \(\hat{p}\) 的共同本征态 \(|\Psi\rangle\)。设其对应的本征值分别为 \(\lambda_\pi\) 和 \(\lambda_p\):
\[ \hat{\pi}|\Psi\rangle = \lambda_\pi |\Psi\rangle \]
\[ \hat{p}|\Psi\rangle = \lambda_p |\Psi\rangle \]
根据我们上面的一般推导,对于这个共同本征态 \(|\Psi\rangle\),必须有:
\[ \lambda_\pi \lambda_p = 0 \]
我们知道宇称算符 \(\hat{\pi}\) 的本征值 \(\lambda_\pi\) 只能是 \(+1\) 或 \(-1\)。这两种情况 \(\lambda_\pi\) 都不为零。
因此,为了满足 \(\lambda_\pi \lambda_p = 0\),必须有:
\[ \lambda_p = 0 \]
这意味着,如果一个量子态同时是宇称算符和动量算符的本征态,那么这个态的动量本征值必须为零。
\textbf{物理实例}:
一个动量为零的平面波(在非相对论量子力学中,严格来说是理想化情况,常数波函数) \(\Psi(x) = C\) (其中 \(C\) 是常数) 是一个动量本征值为 \(p=0\) 的态。
同时,\(\hat{\pi}\Psi(x) = \Psi(-x) = C\)。所以 \(\hat{\pi}\Psi(x) = (+1)\Psi(x)\)。
因此,常数波函数是 \(\hat{\pi}\) 和 \(\hat{p}\) 的共同本征态,其宇称本征值为 \(+1\),动量本征值为 \(0\)。这与我们的结论 \(\lambda_p = 0\) 一致。
如果一个态是动量 \(p \neq 0\) 的本征态,例如 \(\Psi(x) = e^{ipx/\hbar}\),那么 \(\hat{\pi}\Psi(x) = \Psi(-x) = e^{-ipx/\hbar}\)。这个态通常不是宇称的本征态,除非 \(p=0\)。我们可以构造宇称本征态,如 \(\cos(px/\hbar)\) (偶宇称) 或 \(\sin(px/\hbar)\) (奇宇称),但这些态是动量 \(+p\) 和 \(-p\) 的叠加态,而不是动量算符 \(\hat{p}\) 的本征态(它们是 \(\hat{p}^2\) 的本征态)。
\subsection{结论}
如果两个厄密算符 \(\hat{A}\) 和 \(\hat{B}\) 满足反对易关系 \(\{\hat{A}, \hat{B}\} = \hat{A}\hat{B} + \hat{B}\hat{A} = 0\),并且它们有一个共同的本征态 \(|\Psi\rangle\),那么对于这个态,它所对应的 \(\hat{A}\) 的本征值与 \(\hat{B}\) 的本征值的乘积为零。这意味着至少其中一个本征值为零。
在宇称算符 \(\hat{\pi}\) 和动量算符 \(\hat{p}\) 的例子中,由于宇称算符的本征值总是非零的 (\(\pm 1\)),任何它们与动量算符的共同本征态,其动量本征值必须为零。

\section*{习题 2}
一个自旋1/2粒子在球面上运动,其哈密顿量为 $H = \frac{\mathbf{L}^2}{2\mu R^2} + a\mathbf{L}\cdot\mathbf{S}$。能量本征态为 $|l, j, m_j\rangle$。
\begin{enumerate}
    \item 给出对应波函数的奇偶性质。
    \item $H' = \mathbf{S} \cdot \mathbf{p}$,证明$H'$是赝标量 (pseudo scalar)。
    \item 如果在零时刻加上微扰,利用$H'$的属性给出跃迁定则。
\end{enumerate}
\section{解答}
\subsection{波函数的奇偶性}
能量本征态 \(|l, j, m_j\rangle\) 是由轨道角动量量子数 \(l\)、总角动量量子数 \(j\) (其中 \(\mathbf{J} = \mathbf{L} + \mathbf{S}\)) 和总角动量磁量子数 \(m_j\) 标记的。
波函数的奇偶性由宇称算符 \(\hat{\pi}\) 决定。宇称算符作用于空间坐标 \(\mathbf{r} \rightarrow -\mathbf{r}\)。
\begin{itemize}
    \item 轨道部分:与轨道角动量量子数 \(l\) 相关的空间波函数(例如球谐函数 \(Y_{lm_l}(\theta, \phi)\))在宇称变换下的行为是:
    \[ \hat{\pi} Y_{lm_l}(\theta, \phi) = Y_{lm_l}(\pi-\theta, \phi+\pi) = (-1)^l Y_{lm_l}(\theta, \phi) \]
    因此,轨道部分的宇称为 \((-1)^l\)。
    \item 自旋部分:自旋是粒子的内禀属性,宇称算符只作用于空间坐标,不直接作用于自旋自由度。因此,自旋波函数在宇称变换下不变。
    \[ \hat{\pi} |s, m_s\rangle = |s, m_s\rangle \]
\end{itemize}
态 \(|l, j, m_j\rangle\) 是通过 Clebsch-Gordan 系数将轨道角动量态 \(|l, m_l\rangle\) 和自旋态 \(|s, m_s\rangle\) (其中 \(s=1/2\)) 耦合得到的:
\[ |l, j, m_j\rangle = \sum_{m_l, m_s} C(l,s,j; m_l,m_s,m_j) |l, m_l\rangle |s, m_s\rangle \]
其中 \(C(l,s,j; m_l,m_s,m_j)\) 是 Clebsch-Gordan 系数。
对整个态作用宇称算符:
\[ \hat{\pi} |l, j, m_j\rangle = \sum_{m_l, m_s} C(l,s,j; m_l,m_s,m_j) (\hat{\pi}|l, m_l\rangle) (\hat{\pi}|s, m_s\rangle) \]
\[ = \sum_{m_l, m_s} C(l,s,j; m_l,m_s,m_j) ((-1)^l|l, m_l\rangle) (|s, m_s\rangle) \]
\[ = (-1)^l \sum_{m_l, m_s} C(l,s,j; m_l,m_s,m_j) |l, m_l\rangle |s, m_s\rangle \]
\[ = (-1)^l |l, j, m_j\rangle \]
因此,对应波函数 \(|l, j, m_j\rangle\) 的奇偶性为 \((-1)^l\)。
\subsection{证明 \(H' = \mathbf{S} \cdot \mathbf{p}\) 是赝标量}
一个算符 \(\hat{O}\) 是赝标量,如果它在宇称变换 \(\hat{\pi}\) 下满足 \(\hat{\pi} \hat{O} \hat{\pi}^\dagger = -\hat{O}\)。已知 \(\hat{\pi}^\dagger = \hat{\pi}\) 且 \(\hat{\pi}^2 = \hat{I}\)。
我们需要考察自旋算符 \(\mathbf{S}\) 和动量算符 \(\mathbf{p}\) 在宇称变换下的行为。
\begin{itemize}
    \item 动量算符 \(\mathbf{p}\):动量是极矢量 (polar vector)。在宇称变换下,\(\mathbf{r} \rightarrow -\mathbf{r}\),因此 \(\mathbf{p} = -i\hbar \nabla \rightarrow -(-i\hbar \nabla) = -\mathbf{p}\)。
    即:
    \[ \hat{\pi} \mathbf{p} \hat{\pi}^\dagger = -\mathbf{p} \]
    \item 自旋算符 \(\mathbf{S}\):自旋是内禀角动量,其变换性质与轨道角动量 \(\mathbf{L} = \mathbf{r} \times \mathbf{p}\) 类似。轨道角动量是轴矢量 (axial vector 或 pseudovector)。
    \[ \hat{\pi} \mathbf{L} \hat{\pi}^\dagger = (\hat{\pi} \mathbf{r} \hat{\pi}^\dagger) \times (\hat{\pi} \mathbf{p} \hat{\pi}^\dagger) = (-\mathbf{r}) \times (-\mathbf{p}) = \mathbf{r} \times \mathbf{p} = \mathbf{L} \]
    自旋算符 \(\mathbf{S}\) 也是轴矢量,因此:
    \[ \hat{\pi} \mathbf{S} \hat{\pi}^\dagger = \mathbf{S} \]
\end{itemize}
现在考虑 \(H' = \mathbf{S} \cdot \mathbf{p}\):
\[ \hat{\pi} H' \hat{\pi}^\dagger = \hat{\pi} (\mathbf{S} \cdot \mathbf{p}) \hat{\pi}^\dagger \]
由于点积的标量性质,我们可以将其写为:
\[ \hat{\pi} H' \hat{\pi}^\dagger = (\hat{\pi} \mathbf{S} \hat{\pi}^\dagger) \cdot (\hat{\pi} \mathbf{p} \hat{\pi}^\dagger) \]
代入 \(\mathbf{S}\) 和 \(\mathbf{p}\) 的变换性质:
\[ = (\mathbf{S}) \cdot (-\mathbf{p}) \]
\[ = -(\mathbf{S} \cdot \mathbf{p}) \]
\[ = -H' \]
由于 \(\hat{\pi} H' \hat{\pi}^\dagger = -H'\),所以 \(H' = \mathbf{S} \cdot \mathbf{p}\) 是一个赝标量。
\subsection{利用\(H'\)的属性给出跃迁定则}
微扰 \(H' = \mathbf{S} \cdot \mathbf{p}\) 引起从初态 \(|i\rangle = |l, j, m_j\rangle\) 到末态 \(|f\rangle = |l', j', m_j'\rangle\) 的跃迁。跃迁几率幅与矩阵元 \(\langle f | H' | i \rangle\) 成正比。
\[ \langle f | H' | i \rangle = \langle l', j', m_j' | H' | l, j, m_j \rangle \]
我们利用 \(H'\) 的属性来确定跃迁定则。
\textbf{宇称选择定则 (来自 \(H'\) 的赝标量属性)}:
初态 \(|i\rangle\) 的宇称为 \(\epsilon_i = (-1)^l\)。
末态 \(|f\rangle\) 的宇称为 \(\epsilon_f = (-1)^{l'}\)。
算符 \(H'\) 是赝标量,其宇称为 \(\epsilon_{H'} = -1\)。
对于矩阵元 \(\langle f | H' | i \rangle\) 不为零,必须满足:
\[ \epsilon_f^* \epsilon_{H'} \epsilon_i = 1 \]
由于宇称是实数 (\(\pm 1\)),\(\epsilon_f^* = \epsilon_f\)。所以:
\[ \epsilon_f \epsilon_{H'} \epsilon_i = 1 \]
\[ (-1)^{l'} (-1) (-1)^l = 1 \]
\[ (-1)^{l' + l + 1} = 1 \]
这意味着指数 \(l' + l + 1\) 必须是偶数。因此,\(l' + l\) 必须是奇数。
这表明 \(l\) 和 \(l'\) 的奇偶性必须不同。换句话说,轨道角动量量子数 \(l\) 必须改变宇称。
例如,如果 \(l\) 是偶数,则 \(l'\) 必须是奇数;如果 \(l\) 是奇数,则 \(l'\) 必须是偶数。
这意味着 \(\Delta l = |l' - l|\) 必须是奇数 (\(1, 3, 5, \dots\))。
\textbf{角动量选择定则 (来自 \(H'\) 的标量旋转属性)}:
\(H' = \mathbf{S} \cdot \mathbf{p}\) 是两个矢量算符的点积。尽管 \(\mathbf{S}\) 是轴矢量,\(\mathbf{p}\) 是极矢量,它们的点积在空间旋转下是一个标量。
一个在旋转下的标量算符与总角动量算符 \(\mathbf{J}\) 对易:
\[ [\hat{J}_k, H'] = 0 \quad (k=x,y,z) \]
\[ [\mathbf{J}^2, H'] = 0 \]
因此,在微扰 \(H'\) 作用下,总角动量量子数 \(j\) 和其 \(z\) 分量 \(m_j\) 必须守恒。
跃迁定则为:
\[ \Delta j = j' - j = 0 \]
\[ \Delta m_j = m_j' - m_j = 0 \]
\textbf{对 \(\Delta l\) 的进一步限制 (来自 \(\mathbf{p}\) 的性质)}:
算符 \(\mathbf{p}\) 是一个秩为1的张量算符 (矢量算符),它只作用于空间部分。根据 Wigner-Eckart 定理,对于矢量算符,轨道角动量量子数的变化为 \(\Delta l = l' - l = 0, \pm 1\)。
结合宇称选择定则 (\(l' + l\) 为奇数,即 \(\Delta l\) 为奇数),以及 \(\Delta l = 0, \pm 1\),我们得到:
\[ \Delta l = \pm 1 \]
(因为 \(\Delta l = 0\) 会导致 \(l'+l = 2l\) 为偶数,与宇称选择定则矛盾)。
综上所述,利用 \(H'\) 的属性(赝标量和旋转标量),以及构成 \(H'\) 的算符 \(\mathbf{p}\) 的性质,得到的跃迁定则为:
\begin{enumerate}
    \item \(\Delta l = \pm 1\) (轨道角动量量子数改变1,宇称改变)
    \item \(\Delta j = 0\) (总角动量量子数不变)
    \item \(\Delta m_j = 0\) (总角动量 \(z\) 分量不变)
\end{enumerate}
题目特别强调“利用\(H'\)的属性”,其作为赝标量的属性直接给出了宇称选择定则 (\(l'+l\) 为奇数),其作为旋转标量的属性给出了 \(\Delta j=0, \Delta m_j=0\)。

\end{document}
