\documentclass{article}
\usepackage{ctex}
\usepackage{braket}
\usepackage{fancyhdr}
\usepackage{extramarks}
\usepackage{amsmath}
\usepackage{amsthm}
\usepackage{amsfonts}
\usepackage{tikz}
\usepackage[plain]{algorithm}
\usepackage{algpseudocode}

\usetikzlibrary{automata,positioning}

%
% Basic Document Settings
%

\topmargin=-0.45in
\evensidemargin=0in
\oddsidemargin=0in
\textwidth=6.5in
\textheight=9.0in
\headsep=0.25in

\linespread{1.1}

\pagestyle{fancy}
\lhead{\hmwkAuthorName}
\chead{\hmwkClass\ (\hmwkClassInstructor\ \hmwkClassTime): \hmwkTitle}
\rhead{\firstxmark}
\lfoot{\lastxmark}
\cfoot{\thepage}

\renewcommand\headrulewidth{0.4pt}
\renewcommand\footrulewidth{0.4pt}

\setlength\parindent{0pt}

%
% Create Problem Sections
%

\newcommand{\enterProblemHeader}[1]{
    \nobreak\extramarks{}{Problem \arabic{#1} continued on next page\ldots}\nobreak{}
    \nobreak\extramarks{Problem \arabic{#1} (continued)}{Problem \arabic{#1} continued on next page\ldots}\nobreak{}
}

\newcommand{\exitProblemHeader}[1]{
    \nobreak\extramarks{Problem \arabic{#1} (continued)}{Problem \arabic{#1} continued on next page\ldots}\nobreak{}
    \stepcounter{#1}
    \nobreak\extramarks{Problem \arabic{#1}}{}\nobreak{}
}

\setcounter{secnumdepth}{0}
\newcounter{partCounter}
\newcounter{homeworkProblemCounter}
\setcounter{homeworkProblemCounter}{1}
\nobreak\extramarks{Problem \arabic{homeworkProblemCounter}}{}\nobreak{}

%
% Homework Problem Environment
%
% This environment takes an optional argument. When given, it will adjust the
% problem counter. This is useful for when the problems given for your
% assignment aren't sequential. See the last 3 problems of this template for an
% example.
%
\newenvironment{homeworkProblem}[1][-1]{
    \ifnum#1>0
        \setcounter{homeworkProblemCounter}{#1}
    \fi
    \section{Problem \arabic{homeworkProblemCounter}}
    \setcounter{partCounter}{1}
    \enterProblemHeader{homeworkProblemCounter}
}{
    \exitProblemHeader{homeworkProblemCounter}
}

%
% Homework Details
%   - Title
%   - Due date
%   - Class
%   - Section/Time
%   - Instructor
%   - Author
%

\newcommand{\hmwkTitle}{Homework\ 10}
\newcommand{\hmwkDueDate}{November 13,2024}
\newcommand{\hmwkClass}{Quantum Mechanics}
\newcommand{\hmwkClassTime}{}
\newcommand{\hmwkClassInstructor}{Professor Hui Shao}
\newcommand{\hmwkAuthorName}{\textbf{郑晓旸 202111030007}}

%
% Title Page
%

\title{
    \vspace{2in}
    \textmd{\textbf{\hmwkClass:\ \hmwkTitle}}\\
    \normalsize\vspace{0.1in}\small{Due\ on\ \hmwkDueDate\ at 3:10pm}\\
    \vspace{0.1in}\large{\textit{\hmwkClassInstructor\ \hmwkClassTime}}
    \vspace{3in}
}

\author{\hmwkAuthorName}
\date{}

\renewcommand{\part}[1]{\textbf{\large Part \Alph{partCounter}}\stepcounter{partCounter}\\}

%
% Various Helper Commands
%

% Useful for algorithms
\newcommand{\alg}[1]{\textsc{\bfseries \footnotesize #1}}

% For derivatives
\newcommand{\deriv}[1]{\frac{\mathrm{d}}{\mathrm{d}x} (#1)}

% For partial derivatives
\newcommand{\pderiv}[2]{\frac{\partial}{\partial #1} (#2)}

% Integral dx
\newcommand{\dx}{\mathrm{d}x}

% Alias for the Solution section header
\newcommand{\solution}{\textbf{\large Solution}}

% Probability commands: Expectation, Variance, Covariance, Bias
\newcommand{\E}{\mathrm{E}}
\newcommand{\Var}{\mathrm{Var}}
\newcommand{\Cov}{\mathrm{Cov}}
\newcommand{\Bias}{\mathrm{Bias}}

\begin{document}

\maketitle

\pagebreak

\begin{homeworkProblem}
    考虑正交完备基$\ket{1},\ket{2},\ket{3}$,且有$\ket{\alpha}=i\Ket{1}-2\Ket{2}-i\Ket{3;\Ket{\beta}=i\Ket{1}+2\Ket{3}}$,在该表象下,哈密顿算符的矩阵形式写为:\\
        $$\hat{H}=\begin{pmatrix}
            0 & 1 & 0 \\
            1 & 0 & 0 \\
            0 & 0 & 2
        \end{pmatrix}$$
    写出到能量表象的变换矩阵,并计算能量表象下的量子态  $\Ket{\alpha}$和算符$\hat{B}\equiv \ket{\beta}\bra{\beta}$.\\
    \textbf{Solution}

    我们可以通过哈密顿算符的本征态来构造能量表象的变换矩阵。哈密顿算符的本征态为:
    \[
    \begin{aligned}
    \hat{H} \ket{E_1} &= E_1 \ket{E_1} \\
    \hat{H} \ket{E_2} &= E_2 \ket{E_2} \\
    \hat{H} \ket{E_3} &= E_3 \ket{E_3}
    \end{aligned}
    \]
    其中,\( E_1 = -1, E_2 = 1, E_3 = 2 \)。因此,能量表象的变换矩阵为:
    \[
    \begin{aligned}
    \hat{U} &= \begin{pmatrix}
    \braket{1|E_1} & \braket{1|E_2} & \braket{1|E_3} \\
    \braket{2|E_1} & \braket{2|E_2} & \braket{2|E_3} \\
    \braket{3|E_1} & \braket{3|E_2} & \braket{3|E_3}
    \end{pmatrix} \\
    &= \begin{pmatrix}
    0 & 1 & 0 \\
    1 & 0 & 0 \\
    0 & 0 & 1
    \end{pmatrix}
    \end{aligned}
    \]
    我们可以通过能量表象的变换矩阵将量子态 \( \ket{\alpha} \) 和算符 \( \hat{B} \) 变换到能量表象下。量子态 \( \ket{\alpha} \) 的能量表象表示为:
    \[
    \ket{\alpha}_E = \hat{U} \ket{\alpha} = \begin{pmatrix}
    0 & 1 & 0 \\
    1 & 0 & 0 \\
    0 & 0 & 1
    \end{pmatrix} \begin{pmatrix}
    i \\
    -2 \\
    -i
    \end{pmatrix} = \begin{pmatrix}
    -2 \\
    i \\
        
    -i
    \end{pmatrix}
    \]
    算符 \( \hat{B} \) 的能量表象表示为:
    \[
    \begin{aligned}
    \hat{B}_E &= \hat{U} \hat{B} \hat{U}^\dagger \\
    &= \begin{pmatrix}
    0 & 1 & 0 \\
    1 & 0 & 0 \\
    0 & 0 & 1
    \end{pmatrix} \begin{pmatrix}
    i \\
    2 \\
    -i
    \end{pmatrix} \begin{pmatrix}
    -i & 1 & 0 \\
    1 & 0 & 0 \\
    0 & 0 & 1
    \end{pmatrix} \\
    &= \begin{pmatrix}
    0 & 1 & 0 \\
    1 & 0 & 0 \\
    0 & 0 & 1
    \end{pmatrix} \begin{pmatrix}
    -2i & 1 & 0 \\
    2 & 0 & 0 \\
    -i & 0 & 1
    \end{pmatrix} \\
    &= \begin{pmatrix}
    1 & 0 & 0 \\
    0 & 0 & 2 \\
    0 & 2 & 0
    \end{pmatrix}
    \end{aligned}
    \]
    

\end{homeworkProblem}

\pagebreak
\begin{homeworkProblem}
    对于角动量,选择一个合适的表象,写出$l= 1$ 的子空间中,算符$l^2,l_x,l_y,l_z$的矩阵表示。\\
    提示:$l\pm  = l_x ± il_y$ ,且:$$\begin{aligned} l_+ \ket{l,m}&= \sqrt{(l-m)(l+m+1)}\ket{l,m+ 1}\\
l_-\ket{l,m} &= \sqrt{(l+ m)(l- m+ 1)}\ket{l,m-1}\end{aligned}$$
\textbf{Solution}\\
    我们选择角动量表象,即$l^2l_z$的共同本征态为$\ket{l,m}$,其中$l=1$,$m=-1,0,1$。我们可以写出$l^2$的矩阵表示:
    $$
    \hat{l}^2=\begin{pmatrix}
        2 & 0 & 0 \\
        0 & 2 & 0 \\
        0 & 0 & 2
    \end{pmatrix}
    $$
    同样的,我们可以写出$l_z$的矩阵表示:
    $$
    \hat{l}_z=\begin{pmatrix}
        -1 & 0 & 0 \\
        0 & 0 & 0 \\
        0 & 0 & 1
    \end{pmatrix}
    $$
    由于$l_x$和$l_y$的矩阵表示不是对角矩阵,我们可以通过$l_+$和$l_-$的定义来求解$l_x$和$l_y$的矩阵表示:
    $$
    \begin{aligned}
        l_+\ket{l,m}&= \sqrt{(l-m)(l+m+1)}\ket{l,m+ 1}\\
        l_-\ket{l,m} &= \sqrt{(l+ m)(l- m+ 1)}\ket{l,m-1}
    \end{aligned}
    $$
    由此我们可以得到$l_x$和$l_y$的矩阵表示:
    $$
    \hat{l}_x=\frac{1}{2}\begin{pmatrix}
        0 & \sqrt{2} & 0 \\
        \sqrt{2} & 0 & \sqrt{2} \\
        0 & \sqrt{2} & 0
    \end{pmatrix}
    $$
    $$
    \hat{l}_y=\frac{1}{2}\begin{pmatrix}
        0 & -i\sqrt{2} & 0 \\
        i\sqrt{2} & 0 & -i\sqrt{2} \\
        0 & i\sqrt{2} & 0
    \end{pmatrix}
    $$

\end{homeworkProblem}
\newpage
\begin{homeworkProblem}
    写出动量表象下,动量算符和坐标算符的矩阵元 (要求有推导过程)。\\
    \textbf{Solution}\\
    在动量表象下,粒子的状态 \( |\psi\rangle \) 可以表示为动量本征态 \( |p\rangle \) 的线性组合,即:
\[
|\psi\rangle = \int \psi(p) |p\rangle \, dp
\]
动量算符 \( \hat{p} \) 作用在动量本征态 \( |p\rangle \) 上时满足:
\[
\hat{p} |p\rangle = p |p\rangle
\]
因此,动量算符在动量表象下的矩阵元为:
\[
\langle p' | \hat{p} | p \rangle = p \delta(p' - p)
\]
其中,\( \delta(p' - p) \) 是狄拉克δ函数,表示动量本征态的正交性。

接下来,坐标算符 \( \hat{x} \) 在动量表象下通过与动量算符的对易关系 \( [\hat{x}, \hat{p}] = i \hbar \) 定义为:
\[
\hat{x} = i \hbar \frac{\partial}{\partial p}
\]
因此,坐标算符的矩阵元为:
\[
\langle p' | \hat{x} | p \rangle = i \hbar \frac{\partial}{\partial p} \delta(p' - p)
\]

即动量算符和坐标算符在动量表象下的矩阵元为:
\[
\langle p' | \hat{p} | p \rangle = p \delta(p' - p), \quad \langle p' | \hat{x} | p \rangle = i \hbar \frac{\partial}{\partial p} \delta(p' - p)
\]
\end{homeworkProblem}

\begin{homeworkProblem}
    利用完备性关系,写出一维谐振子坐标表象下的波函数$\psi (x)$,坐标算符$\hat{x}$和动量算符$\hat{p}$变换到能量表象的过程和结果。\\
    \textbf{Solution}\\
    一维谐振子的波函数可以表示为能量本征态 \( |n\rangle \) 的线性组合,即:
\[
|\psi\rangle = \sum_n \psi_n |n\rangle
\]
其中, \( \psi_n \) 是波函数的展开系数。利用完备性关系,波函数 \( \psi(x) \) 可以表示为:
\[
\psi(x) = \sum_n \psi_n \langle x | n \rangle
\]
坐标算符 \( \hat{x} \) 和动量算符 \( \hat{p} \) 在能量表象下的矩阵元为:
\[
\langle n | \hat{x} | n' \rangle = \sqrt{\frac{\hbar}{2m\omega}} \left( \sqrt{n} \delta_{n', n-1} + \sqrt{n+1} \delta_{n', n+1} \right)
\]
\[ \langle n | \hat{p} | n' \rangle = -i \sqrt{\frac{\hbar m \omega}{2}} \left( \sqrt{n} \delta_{n', n-1} - \sqrt{n+1} \delta_{n', n+1} \right) \]
\end{homeworkProblem}
\newpage
\begin{homeworkProblem}
    设体系有两个彼此不对易的守恒量 $F$ 和 $G$, 即$[\hat{F},\hat{H}]=0,[\hat{G},\hat{H}]=0$, 但$[\hat{F},\hat{G}] \neq 0$,证明体系能级一般是简并的,并说明何种情况无简并。\\
    \textbf{Solution}\\
1. 守恒量与哈密顿量的对易关系:\\
   \[
   [\hat{F}, \hat{H}] = 0 \quad \text{和} \quad [\hat{G}, \hat{H}] = 0
   \]
   这意味着,\( \hat{F} \) 和 \( \hat{G} \) 都是哈密顿量的守恒量,因此它们可以在同一个本征基底下对角化。换句话说,体系的本征态可以同时是 \( \hat{F} \) 和 \( \hat{G} \) 的本征态。

2. 系统本征态的重叠:\\
   假设 \( |E\rangle \) 是哈密顿量 \( \hat{H} \) 的本征态,即:
   \[
   \hat{H} |E\rangle = E |E\rangle
   \]
   因为 \( \hat{F} \) 和 \( \hat{G} \) 与 \( \hat{H} \) 对易,\( |E\rangle \) 也可以同时是 \( \hat{F} \) 和 \( \hat{G} \) 的本征态。如果 \( \hat{F} \) 和 \( \hat{G} \) 的本征值不完全不同,系统的能级 \( E \) 会有简并性。

3. 关于 \( [\hat{F}, \hat{G}] \neq 0 \) 的影响:\\
   由于 \( \hat{F} \) 和 \( \hat{G} \) 不对易,即 \( [\hat{F}, \hat{G}] \neq 0 \),我们知道它们不能同时对角化。因此,虽然 \( \hat{F} \) 和 \( \hat{G} \) 可以在同一个能量本征态下存在共同本征值,但它们的本征值会依赖于彼此,因此产生简并。换句话说,能量本征态可能会有多个不同的本征态,这些本征态对应相同的能量 \( E \),因此系统的能级会是简并的。

4. 简并的具体情况:\\
   如果 \( [\hat{F}, \hat{G}] \neq 0 \),那么 \( \hat{F} \) 和 \( \hat{G} \) 在某些本征态上不会有共同本征值。因此,尽管这些本征态仍然是哈密顿量 \( \hat{H} \) 的本征态,但它们可能在 \( \hat{F} \) 或 \( \hat{G} \) 上的本征值不同,从而导致能级的简并。
\\
能级没有简并的情况发生在 \( \hat{F} \) 和 \( \hat{G} \) 完全交换的情况下,即:
\[
[\hat{F}, \hat{G}] = 0
\]
在这种情况下,\( \hat{F} \) 和 \( \hat{G} \) 可以同时对角化,因此它们可以在相同的基底中完全对角化。在此情况下,系统的能级不会简并,因为每个本征态对应的能量都是唯一的。
综上:\\
当 \( [\hat{F}, \hat{G}] \neq 0 \) 时,体系的能级一般是简并的。
当 \( [\hat{F}, \hat{G}] = 0 \) 时,体系的能级没有简并。
\end{homeworkProblem}


\end{document}
