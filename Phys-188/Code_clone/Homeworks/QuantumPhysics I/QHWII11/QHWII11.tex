\documentclass{article}
\usepackage{amsmath}
\usepackage{amsfonts}
\usepackage{amssymb}
\usepackage{ctex}
\usepackage{geometry}
\begin{document}

\title{量子力学 II 作业11}
\author{郑晓旸 \\ 202111030007}
\date{\today}
\maketitle

\section*{问题1}
考虑一维系统中的有限平移算符 \(T(a)\) 与位置算符 \(\hat{x}\)。
\begin{enumerate}
    \item 计算对易式 \([T(a), \hat{x}]\)。
    \item 设系统在某时刻的状态为 \(|\psi\rangle\)。当此状态经过平移算符 \(T(a)\) 作用后变为 \(|\psi'\rangle = T(a)|\psi\rangle\),计算位置算符的期望值 \(\langle \hat{x} \rangle\) 如何变化。
\end{enumerate}
\section*{解答}
\subsection*{1. 计算对易式 \([T(a), \hat{x}]\)}
有限平移算符 \(T(a)\) 的定义是它作用在任意位置本征态 \(|x\rangle\) 上时,将其平移 \(a\):
\[ T(a)|x\rangle = |x+a\rangle \]
位置算符 \(\hat{x}\) 作用在位置本征态 \(|x\rangle\) 上时,其本征值为 \(x\):
\[ \hat{x}|x\rangle = x|x\rangle \]
我们计算对易式 \([T(a), \hat{x}]\) 作用在一个任意的位置本征态 \(|x'\rangle\) 上(使用 \(x'\) 以区别算符 \(\hat{x}\) 的符号):
\[ [T(a), \hat{x}]|x'\rangle = (T(a)\hat{x} - \hat{x}T(a))|x'\rangle \]
分别计算等号右边的两项:
\[ T(a)\hat{x}|x'\rangle = T(a)(x'|x'\rangle) = x'T(a)|x'\rangle = x'|x'+a\rangle \]
\[ \hat{x}T(a)|x'\rangle = \hat{x}|x'+a\rangle = (x'+a)|x'+a\rangle \]
将这两项相减:
\[ (T(a)\hat{x} - \hat{x}T(a))|x'\rangle = x'|x'+a\rangle - (x'+a)|x'+a\rangle \]
\[ = (x' - x' - a)|x'+a\rangle \]
\[ = -a|x'+a\rangle \]
注意到 \(T(a)|x'\rangle = |x'+a\rangle\),所以上式可以写为:
\[ [T(a), \hat{x}]|x'\rangle = -a T(a)|x'\rangle \]
由于 \(|x'\rangle\) 是任意的位置本征态,且位置本征态构成一组完备基,因此上述关系对任意态都成立。故我们可以得到算符之间的关系:
\[ [T(a), \hat{x}] = -a T(a) \]
\subsection*{2. 计算 \(\langle \hat{x} \rangle\) 在平移后的变化}
假设系统原来的状态是 \(|\psi\rangle\),则位置的期望值为:
\[ \langle \hat{x} \rangle_{\psi} = \langle\psi|\hat{x}|\psi\rangle \]
经过平移算符 \(T(a)\) 作用后,系统的新状态是 \(|\psi'\rangle = T(a)|\psi\rangle\)。
在新状态下,位置的期望值为:
\[ \langle \hat{x} \rangle_{\psi'} = \langle\psi'|\hat{x}|\psi'\rangle = \langle\psi|T(a)^\dagger \hat{x} T(a)|\psi\rangle \]
我们知道平移算符 \(T(a)\) 可以由动量算符 \(\hat{p}\) 生成,其形式为 \(T(a) = e^{-i\hat{p}a/\hbar}\)。
由于动量算符 \(\hat{p}\) 是厄米算符 (\(\hat{p}^\dagger = \hat{p}\)),且 \(a\) 是实数,所以平移算符的厄米共轭为:
\[ T(a)^\dagger = (e^{-i\hat{p}a/\hbar})^\dagger = e^{i\hat{p}^\dagger a/\hbar} = e^{i\hat{p}a/\hbar} = T(-a) \]
因此,新的期望值表达式变为:
\[ \langle \hat{x} \rangle_{\psi'} = \langle\psi|T(-a) \hat{x} T(a)|\psi\rangle \]
现在我们利用第一部分得到的对易式结果来化简 \(T(-a) \hat{x} T(a)\)。
我们有 \([T(a), \hat{x}] = T(a)\hat{x} - \hat{x}T(a) = -a T(a)\)。
在该等式两边左乘 \(T(a)^\dagger = T(-a)\):
\[ T(-a)(T(a)\hat{x} - \hat{x}T(a)) = T(-a)(-a T(a)) \]
\[ T(-a)T(a)\hat{x} - T(-a)\hat{x}T(a) = -a T(-a)T(a) \]
由于 \(T(-a)T(a) = T(a)^\dagger T(a) = I\) (单位算符),上式变为:
\[ I\hat{x} - T(-a)\hat{x}T(a) = -a I \]
\[ \hat{x} - T(-a)\hat{x}T(a) = -a \]
整理得到算符关系:
\[ T(-a)\hat{x}T(a) = \hat{x} + a \]
将此结果代回到 \(\langle \hat{x} \rangle_{\psi'}\) 的表达式中:
\[ \langle \hat{x} \rangle_{\psi'} = \langle\psi|(\hat{x} + a)|\psi\rangle = \langle\psi|\hat{x}|\psi\rangle + \langle\psi|a|\psi\rangle \]
由于 \(a\) 是一个常数,可以提到积分号外:
\[ \langle \hat{x} \rangle_{\psi'} = \langle\psi|\hat{x}|\psi\rangle + a\langle\psi|\psi\rangle \]
如果态 \(|\psi\rangle\) 是归一化的,即 \(\langle\psi|\psi\rangle = 1\),则:
\[ \langle \hat{x} \rangle_{\psi'} = \langle \hat{x} \rangle_{\psi} + a \]
这表明,在平移操作 \(T(a)\) 之后,位置算符的期望值相对于原期望值增加了 \(a\)。
因此,位置期望值的变化量为:
\[ \Delta \langle \hat{x} \rangle = \langle \hat{x} \rangle_{\psi'} - \langle \hat{x} \rangle_{\psi} = a \]

\newpage

\section*{问题2}
证明 \(\vec{\sigma} \cdot \vec{n}\) 的正本征态 \(|\vec{n},+\rangle\) 可以通过对 \(|z,+\rangle\) 的两步转动得到:
\[ |\vec{n},+\rangle = e^{-i\sigma_z \phi/2} e^{-i\sigma_y \theta/2} |z,+\rangle \]
其中 \(\vec{n} = (\sin\theta\cos\phi, \sin\theta\sin\phi, \cos\theta)\),\(\vec{\sigma} = (\sigma_x, \sigma_y, \sigma_z)\) 是泡利矩阵。
\section*{解答}
我们要证明的是,通过指定的两步幺正变换作用于自旋向上的态 \(|z,+\rangle\),可以得到在 \(\vec{n}\) 方向上自旋向上的态 \(|\vec{n},+\rangle\)。
\subsection*{1. 定义与基本算符}
泡利矩阵为:
\[ \sigma_x = \begin{pmatrix} 0 & 1 \\ 1 & 0 \end{pmatrix}, \quad \sigma_y = \begin{pmatrix} 0 & -i \\ i & 0 \end{pmatrix}, \quad \sigma_z = \begin{pmatrix} 1 & 0 \\ 0 & -1 \end{pmatrix} \]
自旋沿 \(z\) 轴正方向的本征态为:
\[ |z,+\rangle = \begin{pmatrix} 1 \\ 0 \end{pmatrix} \]
对于任意泡利矩阵 \(\sigma_k\),我们有 \((\sigma_k)^2 = I\) (单位矩阵)。因此,旋转算符可以展开为:
\[ e^{-i\sigma_k \alpha/2} = I \cos(\alpha/2) - i\sigma_k \sin(\alpha/2) \]
\subsection*{2. 第一步转动:绕 \(y\) 轴旋转 \(\theta\)}
第一步转动算符为 \(U_y(\theta) = e^{-i\sigma_y \theta/2}\)。
\[ U_y(\theta) = I \cos(\theta/2) - i\sigma_y \sin(\theta/2) \]
\[ = \begin{pmatrix} 1 & 0 \\ 0 & 1 \end{pmatrix} \cos(\theta/2) - i \begin{pmatrix} 0 & -i \\ i & 0 \end{pmatrix} \sin(\theta/2) \]
\[ = \begin{pmatrix} \cos(\theta/2) & 0 \\ 0 & \cos(\theta/2) \end{pmatrix} - \begin{pmatrix} 0 & -i^2 \\ i^2 & 0 \end{pmatrix} \sin(\theta/2) \]
\[ = \begin{pmatrix} \cos(\theta/2) & 0 \\ 0 & \cos(\theta/2) \end{pmatrix} - \begin{pmatrix} 0 & 1 \\ -1 & 0 \end{pmatrix} \sin(\theta/2) \]
\[ = \begin{pmatrix} \cos(\theta/2) & -\sin(\theta/2) \\ \sin(\theta/2) & \cos(\theta/2) \end{pmatrix} \]
将此算符作用于 \(|z,+\rangle\):
\[ |\psi_1\rangle = U_y(\theta) |z,+\rangle = \begin{pmatrix} \cos(\theta/2) & -\sin(\theta/2) \\ \sin(\theta/2) & \cos(\theta/2) \end{pmatrix} \begin{pmatrix} 1 \\ 0 \end{pmatrix} = \begin{pmatrix} \cos(\theta/2) \\ \sin(\theta/2) \end{pmatrix} \]
\subsection*{3. 第二步转动:绕 \(z\) 轴旋转 \(\phi\)}
第二步转动算符为 \(U_z(\phi) = e^{-i\sigma_z \phi/2}\)。
\[ U_z(\phi) = I \cos(\phi/2) - i\sigma_z \sin(\phi/2) \]
\[ = \begin{pmatrix} 1 & 0 \\ 0 & 1 \end{pmatrix} \cos(\phi/2) - i \begin{pmatrix} 1 & 0 \\ 0 & -1 \end{pmatrix} \sin(\phi/2) \]
\[ = \begin{pmatrix} \cos(\phi/2) - i\sin(\phi/2) & 0 \\ 0 & \cos(\phi/2) + i\sin(\phi/2) \end{pmatrix} = \begin{pmatrix} e^{-i\phi/2} & 0 \\ 0 & e^{i\phi/2} \end{pmatrix} \]
将此算符作用于 \(|\psi_1\rangle\):
\[ |\psi'\rangle = U_z(\phi) |\psi_1\rangle = \begin{pmatrix} e^{-i\phi/2} & 0 \\ 0 & e^{i\phi/2} \end{pmatrix} \begin{pmatrix} \cos(\theta/2) \\ \sin(\theta/2) \end{pmatrix} = \begin{pmatrix} \cos(\theta/2)e^{-i\phi/2} \\ \sin(\theta/2)e^{i\phi/2} \end{pmatrix} \]
这就是经过两步转动后得到的态。
\subsection*{4. 验证 \(|\psi'\rangle\) 是 \(\vec{\sigma} \cdot \vec{n}\) 的正本征态}
算符 \(\vec{\sigma} \cdot \vec{n}\) 为:
\[ \vec{\sigma} \cdot \vec{n} = \sigma_x n_x + \sigma_y n_y + \sigma_z n_z \]
\[ = \sigma_x \sin\theta\cos\phi + \sigma_y \sin\theta\sin\phi + \sigma_z \cos\theta \]
\[ = \begin{pmatrix} 0 & 1 \\ 1 & 0 \end{pmatrix} \sin\theta\cos\phi + \begin{pmatrix} 0 & -i \\ i & 0 \end{pmatrix} \sin\theta\sin\phi + \begin{pmatrix} 1 & 0 \\ 0 & -1 \end{pmatrix} \cos\theta \]
\[ = \begin{pmatrix} \cos\theta & \sin\theta\cos\phi - i\sin\theta\sin\phi \\ \sin\theta\cos\phi + i\sin\theta\sin\phi & -\cos\theta \end{pmatrix} \]
\[ = \begin{pmatrix} \cos\theta & \sin\theta e^{-i\phi} \\ \sin\theta e^{i\phi} & -\cos\theta \end{pmatrix} \]
现在我们将 \(\vec{\sigma} \cdot \vec{n}\) 作用于 \(|\psi'\rangle\):
\[ (\vec{\sigma} \cdot \vec{n}) |\psi'\rangle = \begin{pmatrix} \cos\theta & \sin\theta e^{-i\phi} \\ \sin\theta e^{i\phi} & -\cos\theta \end{pmatrix} \begin{pmatrix} \cos(\theta/2)e^{-i\phi/2} \\ \sin(\theta/2)e^{i\phi/2} \end{pmatrix} \]
计算矩阵乘积的第一个分量:
\[ \cos\theta \cos(\theta/2)e^{-i\phi/2} + \sin\theta e^{-i\phi} \sin(\theta/2)e^{i\phi/2} \]
\[ = \cos\theta \cos(\theta/2)e^{-i\phi/2} + \sin\theta \sin(\theta/2)e^{-i\phi/2} \]
\[ = (\cos\theta \cos(\theta/2) + \sin\theta \sin(\theta/2))e^{-i\phi/2} \]
使用三角恒等式 \(\cos(A-B) = \cos A \cos B + \sin A \sin B\),令 \(A=\theta, B=\theta/2\):
\[ = \cos(\theta - \theta/2)e^{-i\phi/2} = \cos(\theta/2)e^{-i\phi/2} \]
计算矩阵乘积的第二个分量:
\[ \sin\theta e^{i\phi} \cos(\theta/2)e^{-i\phi/2} - \cos\theta \sin(\theta/2)e^{i\phi/2} \]
\[ = \sin\theta \cos(\theta/2)e^{i\phi/2} - \cos\theta \sin(\theta/2)e^{i\phi/2} \]
\[ = (\sin\theta \cos(\theta/2) - \cos\theta \sin(\theta/2))e^{i\phi/2} \]
使用三角恒等式 \(\sin(A-B) = \sin A \cos B - \cos A \sin B\),令 \(A=\theta, B=\theta/2\):
\[ = \sin(\theta - \theta/2)e^{i\phi/2} = \sin(\theta/2)e^{i\phi/2} \]
所以,
\[ (\vec{\sigma} \cdot \vec{n}) |\psi'\rangle = \begin{pmatrix} \cos(\theta/2)e^{-i\phi/2} \\ \sin(\theta/2)e^{i\phi/2} \end{pmatrix} = 1 \cdot |\psi'\rangle \]
这表明 \(|\psi'\rangle\) 是算符 \(\vec{\sigma} \cdot \vec{n}\) 的本征态,其本征值为 \(+1\)。因此,\(|\psi'\rangle\) 就是 \(\vec{\sigma} \cdot \vec{n}\) 的正本征态 \(|\vec{n},+\rangle\)。

通过对 \(|z,+\rangle\) 先进行绕 \(y\) 轴旋转角度 \(\theta\) 的操作 \(e^{-i\sigma_y \theta/2}\),再进行绕 \(z\) 轴旋转角度 \(\phi\) 的操作 \(e^{-i\sigma_z \phi/2}\),我们得到的态为:
\[ |\psi'\rangle = \begin{pmatrix} \cos(\theta/2)e^{-i\phi/2} \\ \sin(\theta/2)e^{i\phi/2} \end{pmatrix} \]
我们已经验证了这个态是 \(\vec{\sigma} \cdot \vec{n}\) 的本征值为 \(+1\) 的本征态。因此,
\[ |\vec{n},+\rangle = e^{-i\sigma_z \phi/2} e^{-i\sigma_y \theta/2} |z,+\rangle \]
证毕。

\end{document}
