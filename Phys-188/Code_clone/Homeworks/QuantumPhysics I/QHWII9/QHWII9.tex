\documentclass[12pt, a4paper]{article}

% --- Packages ---
\usepackage[utf8]{inputenc}
\usepackage[T1]{fontenc}
\usepackage{amsmath}
\usepackage{amssymb}
\usepackage{geometry}
\usepackage{ctex}
\usepackage{bm} % For bold math symbols like \mathbf

% --- Page Layout ---
\geometry{left=2cm, right=2cm, top=2.5cm, bottom=2.5cm}

% --- Title Information ---
\title{量子力学 II 作业 9}
\author{郑晓旸 \\ 202111030007} % Replace with actual student info
\date{\today}

% --- Custom Commands ---
\newcommand{\ket}[1]{| #1 \rangle}
\newcommand{\bra}[1]{\langle #1 |}
\newcommand{\braket}[2]{\langle #1 | #2 \rangle}
\newcommand{\matrixel}[3]{\langle #1 | #2 | #3 \rangle}
\newcommand{\nvec}{\mathbf{n}}
\newcommand{\Svec}{\mathbf{S}}
\newcommand{\svec}{\mathbf{s}}
\newcommand{\Rparam}{\mathbf{R}} % Parameter space vector

\begin{document}

\maketitle

\section*{问题}
考虑 \(S=1\) 自旋。
\begin{enumerate}
    \item 证明其相干态(\(\Svec \cdot \nvec\) 的最大本征态)\(\ket{\nvec}\) 可以表示成两个自旋 \(s=1/2\) 相干态的直积:
    \[ \ket{\nvec}_{S=1} = \ket{\nvec}_{s=1/2, A} \otimes \ket{\nvec}_{s=1/2, B} \]
    \item 已知 \(\ket{\nvec}\) 具有完备性,求出封闭路径的Berry相位 \(\gamma_B = -\Omega\),其中 \(\Omega\) 是路径在参数空间(单位球面)上所围的立体角。
    \item 推广到任意自旋 \(S\) 情况,证明 \(\gamma_B = -S\Omega\)。
\end{enumerate}
(\textit{注:采用 \(\hbar=1\) })

\section*{证明}

\subsection*{(1) \(S=1\) 相干态的直积表示}

一个自旋为 \(S\) 的相干态 \(\ket{\nvec}_S\) 定义为算符 \(\Svec \cdot \nvec\) 的具有最大本征值 \(S\) 的本征态,其中 \(\nvec = (\sin\theta\cos\phi, \sin\theta\sin\phi, \cos\theta)\) 是单位方向矢量。
\[ (\Svec \cdot \nvec) \ket{\nvec}_S = S \ket{\nvec}_S \]
对于一个自旋 \(s=1/2\) 的系统,其相干态 \(\ket{\nvec}_{1/2}\) 满足:
\[ (\svec \cdot \nvec) \ket{\nvec}_{1/2} = \frac{1}{2} \ket{\nvec}_{1/2} \]
考虑两个自旋 \(s_A=1/2\) 和 \(s_B=1/2\) 的系统,总自旋算符为 \(\Svec = \svec_A + \svec_B\)。
根据角动量耦合规则,\(1/2 \otimes 1/2 = 1 \oplus 0\)。我们关注的是 \(S=1\) 的态空间。
构造两个自旋 \(1/2\) 相干态的直积,它们都指向同一个方向 \(\nvec\):
\[ \ket{\Psi} = \ket{\nvec}_{1/2, A} \otimes \ket{\nvec}_{1/2, B} \]
考察 \(\Svec \cdot \nvec\) 作用于 \(\ket{\Psi}\):
\begin{align*}
(\Svec \cdot \nvec) \ket{\Psi} &= ((\svec_A + \svec_B) \cdot \nvec) (\ket{\nvec}_{1/2, A} \otimes \ket{\nvec}_{1/2, B}) \\
&= ((\svec_A \cdot \nvec) + (\svec_B \cdot \nvec)) (\ket{\nvec}_{1/2, A} \otimes \ket{\nvec}_{1/2, B}) \\
&= ((\svec_A \cdot \nvec) \ket{\nvec}_{1/2, A}) \otimes \ket{\nvec}_{1/2, B} + \ket{\nvec}_{1/2, A} \otimes ((\svec_B \cdot \nvec) \ket{\nvec}_{1/2, B}) \\
&= \left(\frac{1}{2} \ket{\nvec}_{1/2, A}\right) \otimes \ket{\nvec}_{1/2, B} + \ket{\nvec}_{1/2, A} \otimes \left(\frac{1}{2} \ket{\nvec}_{1/2, B}\right) \\
&= \left(\frac{1}{2} + \frac{1}{2}\right) (\ket{\nvec}_{1/2, A} \otimes \ket{\nvec}_{1/2, B}) \\
&= 1 \cdot (\ket{\nvec}_{1/2, A} \otimes \ket{\nvec}_{1/2, B}) = 1 \cdot \ket{\Psi}
\end{align*}
这表明 \(\ket{\Psi}\) 是总自旋算符 \(\Svec \cdot \nvec\) 的本征态,其本征值为 \(1\)。
对于 \(S=1\) 系统,最大的本征值就是 \(S=1\)。因此,根据自旋 \(S=1\) 相干态的定义,我们有:
\[ \ket{\nvec}_{S=1} = \ket{\nvec}_{1/2, A} \otimes \ket{\nvec}_{1/2, B} \]
其中下标 \(A\) 和 \(B\) 指代两个 \(s=1/2\) 的子系统。
\textit{证明完毕。}

\subsection*{(2) \(S=1\) 系统的Berry相位}

Berry相位定义为 \(\gamma_B = \oint_C \mathcal{A}(\Rparam) \cdot d\Rparam\),其中 \(\mathcal{A}(\Rparam) = i \bra{\nvec(\Rparam)} \nabla_{\Rparam} \ket{\nvec(\Rparam)}\) 是Berry联络,\(\Rparam\) 代表参数(如 \(\theta, \phi\))。

令 \(\ket{\nvec} \equiv \ket{\nvec}_{S=1}\),\(\ket{\nvec}_A \equiv \ket{\nvec}_{1/2, A}\),\(\ket{\nvec}_B \equiv \ket{\nvec}_{1/2, B}\)。
\[ \nabla_{\Rparam} \ket{\nvec} = (\nabla_{\Rparam} \ket{\nvec}_A) \otimes \ket{\nvec}_B + \ket{\nvec}_A \otimes (\nabla_{\Rparam} \ket{\nvec}_B) \]
所以,
\begin{align*}
\bra{\nvec} \nabla_{\Rparam} \ket{\nvec} &= (\bra{\nvec}_A \otimes \bra{\nvec}_B) \left( (\nabla_{\Rparam} \ket{\nvec}_A) \otimes \ket{\nvec}_B + \ket{\nvec}_A \otimes (\nabla_{\Rparam} \ket{\nvec}_B) \right) \\
&= \bra{\nvec}_A \nabla_{\Rparam} \ket{\nvec}_A \cdot \braket{\nvec_B}{\nvec_B} + \braket{\nvec_A}{\nvec_A} \cdot \bra{\nvec}_B \nabla_{\Rparam} \ket{\nvec}_B
\end{align*}
由于相干态是归一化的,\(\braket{\nvec_A}{\nvec_A} = 1\) 且 \(\braket{\nvec_B}{\nvec_B} = 1\)。
\[ \bra{\nvec} \nabla_{\Rparam} \ket{\nvec} = \bra{\nvec}_A \nabla_{\Rparam} \ket{\nvec}_A + \bra{\nvec}_B \nabla_{\Rparam} \ket{\nvec}_B \]
因此,Berry联络是各个子系统Berry联络之和:
\[ \mathcal{A}_{S=1}(\Rparam) = i \bra{\nvec} \nabla_{\Rparam} \ket{\nvec} = i \bra{\nvec}_A \nabla_{\Rparam} \ket{\nvec}_A + i \bra{\nvec}_B \nabla_{\Rparam} \ket{\nvec}_B = \mathcal{A}_{1/2, A}(\Rparam) + \mathcal{A}_{1/2, B}(\Rparam) \]
所以Berry相位也是相加的:
\[ \gamma_{B, S=1} = \oint_C \mathcal{A}_{S=1}(\Rparam) \cdot d\Rparam = \oint_C \mathcal{A}_{1/2, A}(\Rparam) \cdot d\Rparam + \oint_C \mathcal{A}_{1/2, B}(\Rparam) \cdot d\Rparam = \gamma_{B, 1/2, A} + \gamma_{B, 1/2, B} \]
对于一个自旋 \(s=1/2\) 的系统,其Berry相位是 \(\gamma_{B, 1/2} = -(1/2)\Omega\),其中 \(\Omega\) 是路径 \(C\) 在参数空间(单位球面)上所围的立体角。
因此,对于 \(S=1\) 系统:
\[ \gamma_{B, S=1} = -\frac{1}{2}\Omega - \frac{1}{2}\Omega = -1 \cdot \Omega = -\Omega \]
\textit{证明完毕。}

\subsection*{(3) 推广到任意自旋 \(S\)}

一个自旋为 \(S\) 的系统可以看作是由 \(2S\) 个自旋为 \(s=1/2\) 的基本粒子在完全对称态下组合而成。总自旋算符为 \(\Svec_{\text{total}} = \sum_{k=1}^{2S} \svec_k\)。
自旋 \(S\) 相干态 \(\ket{\nvec}_S\) 是 \(\Svec_{\text{total}} \cdot \nvec\) 的具有最大本征值 \(S\) 的本征态。这个态可以表示为 \(2S\) 个自旋 \(1/2\) 相干态的直积(它们都指向同一个方向 \(\nvec\)),因为这个直积态是全对称的,并且是 \(\Svec_{\text{total}} \cdot \nvec\) 的本征态,本征值为 \(\sum_{k=1}^{2S} (1/2) = S\)。
\[ \ket{\nvec}_S = \bigotimes_{k=1}^{2S} \ket{\nvec}_{1/2, k} \]
类似于第 (2) 部分的推导,总系统的Berry联络是各个 \(s=1/2\) 子系统Berry联络之和:
\[ \mathcal{A}_S(\Rparam) = \sum_{k=1}^{2S} \mathcal{A}_{1/2, k}(\Rparam) \]
因此,总系统的Berry相位是各个子系统Berry相位之和:
\[ \gamma_{B, S} = \sum_{k=1}^{2S} \gamma_{B, 1/2, k} \]
由于每个自旋 \(1/2\) 子系统贡献的Berry相位都是 \(-(1/2)\Omega\):
\[ \gamma_{B, S} = \sum_{k=1}^{2S} \left(-\frac{1}{2}\Omega\right) = 2S \cdot \left(-\frac{1}{2}\Omega\right) = -S\Omega \]
\textit{证明完毕。}

\subsubsection*{附注:自旋 \(s=1/2\) 系统的Berry联络与相位}
自旋 \(1/2\) 相干态 \(\ket{\nvec(\theta, \phi)} = \cos(\theta/2) \ket{\uparrow} + e^{i\phi} \sin(\theta/2) \ket{\downarrow}\)。
Berry联络的分量为:
\begin{align*}
\mathcal{A}_\theta &= i \bra{\nvec} \frac{\partial}{\partial\theta} \ket{\nvec} = 0 \\
\mathcal{A}_\phi &= i \bra{\nvec} \frac{\partial}{\partial\phi} \ket{\nvec} = -\sin^2(\theta/2) = -\frac{1-\cos\theta}{2}
\end{align*}
Berry相位 \(\gamma_B = \oint (\mathcal{A}_\theta d\theta + \mathcal{A}_\phi d\phi) = \oint -\frac{1-\cos\theta}{2} d\phi\)。
如果路径是沿着固定 \(\theta\) 绕行一周 (\(\phi\) 从 0 到 \(2\pi\)):
\[ \gamma_B = -\frac{1-\cos\theta}{2} \int_0^{2\pi} d\phi = -\pi(1-\cos\theta) \]
该路径所围的立体角 \(\Omega = \int_0^{2\pi} d\phi \int_0^\theta \sin\theta' d\theta' = 2\pi (1-\cos\theta)\)。
因此,\(\gamma_B = -(1/2)\Omega\)。

\end{document}
