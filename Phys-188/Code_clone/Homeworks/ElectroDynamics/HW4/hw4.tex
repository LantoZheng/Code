\documentclass{assignment}
\ProjectInfos{电动力学A}{1410100401}{2024-2025学年第一学期}{第4次作业}{截止时间:2024. 10. 08(周二)}{郑晓旸}[https://github.com/LantoZheng]{202111030007}
\begin{document}
    \begin{prob}
    设$T^{ab}=T^{[a,b]}$,证明:$T^{ab}\omega_{a}\omega_b = 0$.
    \end{prob}
    \begin{sol}
    \begin{equation}
        T^{ab}\omega_{a}\omega_b = T^{ab}\omega_{[a}\omega_{b]} = \frac{T^{ab}\omega_{a}\omega_{b} - T^{ab}\omega_{b}\omega_{a}}{2} = 0
    \end{equation}
    \end{sol}
    \begin{prob}
        已知惯性系$S$和$S'$满足洛伦兹变换,其中$\gamma = \frac{1}{\sqrt{1-v^2}}$质点在$S'$系的四速度为$v^a = (\frac{\partial}{\partial t'})^a$,求证其在$S$系的四速度为:$v^a = \gamma (\frac{\partial}{\partial t})^a + \gamma v (\frac{\partial}{\partial x})^a$
        \begin{equation}
            \left\{
            \begin{aligned}
                x' &= \gamma (x - vt) \\
                y' &= y \\
                z' &= z \\
                t' &= \gamma (t - vx)
            \end{aligned}
            \right.
        \end{equation}
    \end{prob}
    \begin{sol}
        本题中使用$c=1$,因此$\beta = -v$,由$S'$到$S$的洛伦茨变换参数形式为:
        \begin{equation}
            \Lambda^\mu_{\ \nu} = \left(
            \begin{array}{cccc}
                \gamma & \gamma v & 0 & 0 \\
                \gamma v & \gamma & 0 & 0 \\
                0 & 0 & 1 & 0 \\
                0 & 0 & 0 & 1
            \end{array}
            \right)
        \end{equation}
        在$S'$系下,四速度的分量为:
        \begin{equation}
            v'^\nu = \left( 
                \begin{array}{c}
                    1\\
                    0\\
                    0\\
                    0
                \end{array}
            \right)
        \end{equation}
        因此在$S$系下的四速度为:
        \begin{equation}
            v^\mu = \Lambda^\mu_{\ \nu}v^\nu = \left(
                \begin{array}{c}
                    \gamma\\
                    \gamma v\\
                    0\\
                    0
                \end{array}
            \right)
        \end{equation}
        因此有:
        \begin{equation}
            v^a =  \gamma (\frac{\partial}{\partial t})^a + \gamma v (\frac{\partial}{\partial x})^a
        \end{equation}
    \end{sol}
    \begin{prob}
        在地面系,静止的物体$A$在$x$方向受到恒力$\overrightarrow{F}$ ,求地面系中物体的运动轨迹;\\
        设物体$B$与物体$A$同时开始运动,$B$沿着$y$方向匀速直线运动,以$B$为参考系,求$B$参考系中$A$的速度和运动轨迹。
    \end{prob}
    \begin{sol}
        在地面系中,物体$A$受到的恒力为$\overrightarrow{F} = F\hat{i}$,因此有:
        \begin{equation}
            \frac{dp^x}{dt} = \frac{d \gamma m u^x}{dt} = F
        \end{equation}
    \end{sol}
    其中,$\gamma = \frac{1}{\sqrt{1-u^2}}$,这里取$c = 1$.
    我们得到参数方程:
    \begin{equation}
        \gamma m u^x = Ft
    \end{equation}
    展开得到:
    \begin{equation}
        \frac{dx}{dt} = u^x = \frac{Ft}{\sqrt{m^2 + F^2t^2}} = \frac{t}{\sqrt{t^2 + \frac{m^2}{F^2}}}
    \end{equation}
    积分得到:
    \begin{equation}
        x = \sqrt{t^2 + \frac{m^2}{F^2}} - \frac{m}{F}
    \end{equation}
    我们可以写出$A$的四位置:
    \begin{equation}
        x^\mu = \left(
            \begin{array}{c}
                t\\
                \sqrt{t^2 + \frac{m^2}{F^2}} - \frac{m}{F}\\
                0\\
                0
            \end{array}
        \right)
    \end{equation}
    在$B$参考系中,使用从地面系到$B$系的参数形式洛伦兹变换,其中$\gamma = \frac{1}{\sqrt{1-v^2}}$:
    \begin{equation}
        \Lambda^\nu_{\ \mu} = \left(
            \begin{array}{cccc}
                \gamma & 0 & -\gamma v & 0\\
                0 & 1 & 0 & 0\\
                -\gamma v & 0 & \gamma & 0\\
                0 & 0 & 0 & 0
            \end{array}
        \right)
    \end{equation}
    得到$A$在$B$参考系中的四位置:
    \begin{equation}
        x'^\mu = \Lambda^\mu_{\ \nu}x^\nu = \left(
            \begin{array}{c}
                \gamma t\\
                \sqrt{t^2 + \frac{m^2}{F^2}} - \frac{m}{F}\\
                -\gamma vt\\
                0
            \end{array}
        \right)
    \end{equation}
    因此$A$在$B$参考系中的速度为:
    \begin{equation}
        \begin{aligned}
        v'^x &= \frac{dx'}{dt'} = \frac{t'}{\sqrt{\gamma^2 t'^2 + \gamma^4 \frac{m^2}{F^2}}}\\
        v'^y &= -v
        \end{aligned}
    \end{equation}
\end{document}
