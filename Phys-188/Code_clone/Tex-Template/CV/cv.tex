\documentclass{article}
\usepackage{geometry}
\geometry{a4paper, margin=1in}
\usepackage{amsmath}
\usepackage{graphicx}
\usepackage{xeCJK}
\setCJKmainfont{SimSun} % 设置中文字体,这里使用宋体,你可以根据自己的喜好选择

\title{郑晓旸}
\date{}

\begin{document}

\maketitle
\begin{center}
    邮箱: xiaoyangzheng@mail.bnu.edu.cn \\
    手机: +86 13955190184 \\
    地址: 北京市北京师范大学学生公寓15号楼328
\end{center}

\section*{教育背景}
\begin{itemize}
    \item 合肥市第六中学 (重点班), 2018.09 - 2021.07
    \item 北京师范大学, 北京, 2021.09 - 至今
    \begin{itemize}
        \item 理学学士(物理学),励耘实验班(预计2026.07毕业)
        \item 经济学学士(预计2026.07毕业)
    \end{itemize}
\end{itemize}

\section*{标准化成绩}
\begin{itemize}
    \item GPA(总分): 3.7/4.0
    \item 核心课程: 光学 (93), 量子物理 (91), 计算物理导论 (95), 光学研讨课 (95), 力学 (95), 电磁学 (97), 电动力学 (91)
\end{itemize}

\section*{奖学金与奖项}
\begin{itemize}
    \item 全国中学生物理学奥林匹克竞赛 (安徽赛区) 一等奖, 2021.04
    \item 新生奖学金, 2021.09
    \item 北京师范大学一等奖学金, 2024.10 \& 2022.10
    \item 北京师范大学奖励性助学金奖学金, 2024.10 \& 2023.10 \& 2022.10
    \item 美国大学生数学建模竞赛 (MCM/ICM) Honorable Mentions, 2022.02
\end{itemize}

\section*{语言能力}
\begin{itemize}
    \item 雅思: 7.5分
    \item 四级 : 681分
\end{itemize}

\section*{研究经历}
\subsection*{微纳光学 | 本科生研究助理, 2022.4 - 2024.6}
导师: 石锦卫 教授(北京师范大学物理系)
\begin{itemize}
    \item 合成了高纯度金纳米棒并优化了合成方法:我复现了文献中描述的方法,通过使用油酸,使金纳米棒的长度均匀性提高了约 13\%(与标准方法相比)。通过将最佳反应时间从 30 分钟调整到 24 分钟(25°C),并确定了硝酸银的理想浓度,我改进了合成的金纳米棒,使其在可见光激发下表现出电四极振荡并产生二阶模式。
    \item 利用扫描电子显微镜 (SEM) 表征金纳米棒的光学表面结构,以确定粒径和形态;利用透射电子显微镜 (TEM) 分析内部结构;利用光学光谱测量其等离激元共振。
    \item 使用 COMSOL 进行了金纳米棒的光学性质模拟:模拟了金纳米棒在可见光波照射下的激发模式。研究了纳米棒的长度、宽度和长度均匀性对二维材料的二阶激发模式引起的吸收峰的位置和宽度的影响。
\end{itemize}

\subsection*{原子自旋磁力计 | 本科生研究助理, 2023.06 - 2023.08}
导师: 盛东 教授(中国科技大学精密机械与精密仪器系)
\begin{itemize}
    \item 深入理解了原子磁力计和自旋磁力计的原理和操作,重点研究了它们在精密测量中的应用。
    \item 进行了 COMSOL 热模拟,以优化原子自旋磁力计中关键部件的设计并确保其热稳定性。
    \item 参与了用于激光探测原子态的光路搭建和校准,并实现了用于信号采集和降低原子磁力测量实验中噪声的电子测量电路。
\end{itemize}

\section*{项目经历}
\subsection*{基于射线追踪和波动光学的光学系统仿真 | 课程项目}
\begin{itemize}
    \item 设计了一个基于 Matlab 的仿真程序,该程序利用蒙特卡罗射线追踪和光谱方法(3D-FFT),通过同时考虑几何和波动光学效应(衍射)来准确模拟针孔成像的成像质量。该程序模拟相干光源近场衍射图样的能力对于理解受限空间中的光传播尤为重要,这一概念适用于物理学的各个领域。
    \item 该程序利用光谱方法(三维 FFT)来模拟针孔附近的电磁场,从而能够快速计算近场衍射效应。这种方法不同于光学教科书中常见的夫琅禾费衍射计算,并且允许模拟非近轴条件下的近场相干光源的衍射图样。
    \item 对于几何光学部分,该程序使用射线追踪,以便灵活调整精度和速度。通过采用蒙特卡罗方法(该方法通过孔径发射随机光线并执行碰撞检测),它可以快速渲染针孔图像。这模拟了亮度变化以及图像清晰度和光斑特性的变化(这些变化由几何光学和针孔属性决定)。
    \item 通过将几何光学图像与衍射图样进行卷积,该程序可以生成同时考虑几何和波动光学效应的针孔成像质量的仿真。
\end{itemize}

\subsection*{基于反射式空间光调制器的自校准光束整形 | 课程项目}
\begin{itemize}
    \item 自校准光束整形基于反射式空间光调制器
    \item 设计并构建了一个基于反射式空间光调制器的图像显示系统。
    \item 开发了一个 Python 程序,该程序使用空间反演算法来计算和实现空间光调制器所需的相位和强度模式,以生成和动态控制具有特定波前形状(例如,高斯、拉盖尔-高斯)的光束。
\end{itemize}

\section*{工作经历}
\begin{itemize}
    \item 北京师范大学摄影协会 | 主席, 2023.09 - 2024.09(预计)
    \begin{itemize}
        \item 主持并组织了多次讲座和对外部专家的采访,包括 3 场讲座和 2 场采访,约有 200 名学生参加。
    \end{itemize}
    \item 讲授了多个系列讲座,包括“摄影中的光学概念”和“从摄影器材的角度看成像系统质量”
    \item Homoludens 档案馆 | 本科生研究员 \& 档案管理员, 2022.09 - 2023.07
\end{itemize}

\section*{技能与其他}
\begin{itemize}
    \item 计算机技能:
    \begin{itemize}
        \item Python (PyTorch, SciPy, RenPy)
        \item Matlab
        \item C++
    \end{itemize}
    \item 实验室技能: 精通扫描电子显微镜 (SEM) 和透射电子显微镜 (TEM),用于材料表征和分析。
\end{itemize}

\end{document}
