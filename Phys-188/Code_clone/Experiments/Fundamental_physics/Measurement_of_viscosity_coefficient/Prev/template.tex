%This is a experiment example of ZhengXiaoyang's experiment report template

\documentclass[UTF8]{ctexart}
 
\usepackage{amsmath}
\usepackage{cases}
\usepackage{cite}
\usepackage{xeCJK}
\usepackage{graphicx}
\usepackage{SIunits}
\usepackage{caption}
\usepackage{float}
\usepackage{fancyhdr}
\usepackage[margin=1in]{geometry}
\geometry{a4paper}
\pagestyle{fancy}
\fancyhf{}

\graphicspath{{picture/}}


\title{下落法测粘性系数}
\graphicspath{{picture/}}


\title{下落法测粘性系数预习报告}
\author{郑晓旸}
\date{\today}
\pagenumbering{arabic}

\begin{document}
%这里是文件的开头
\fancyhead[L]{郑晓旸}
\fancyhead[C]{粘性系数}
\fancyfoot[C]{\thepage}

\maketitle
\tableofcontents
\newpage

\section{实验仪器}
\begin{itemize}
    \item DH4606落球法液体粘滞系数测定仪
    \item 钢尺
    \item 螺旋测微器
    \item 电子天平
    \item 数字温度计
\end{itemize}

\section{实验目的}
\begin{enumerate}
    \item 学习一些基本物理量的测量方法
    \item 学习用落球法测量蓖麻油的黏性系数
\end{enumerate}

\section{实验原理}
在流体(包括气体和液体)中,惯性和黏性是影响运动的两个主要因素。气体黏性的微观机制是不同速度的分子在相邻区域之间扩散,而在液体中,分子之间的作用力是产生黏性的主要原因。

斯托克斯定律(Stokes' Law)用于描述小球在流体中匀速运动时所受的黏性阻力。对于半径为 \(r\) 的球体在黏性系数为 \(\eta\) 的无限大流体中以速度 \(v\) 运动时,黏性阻力 \(F_D\) 为:
\[
F_D = 6 \pi \eta r v
\]
当雷诺数 \(Re\) 很小时,斯托克斯定律适用。雷诺数的计算公式为:
\[
Re = \frac{\rho d v}{\eta}
\]
其中 \(\rho\) 为流体密度,\(d\) 为系统的特征尺度。

在落球法测量液体黏性系数实验中,小球在液体中下落时达到稳定速度 \(v_0\),此时重力、浮力和黏性阻力平衡。由此可得黏性系数 \(\eta\) 的计算公式为:
\[
\eta = \frac{(\rho_0 - \rho) d^2 g - \frac{27}{8} \rho d v_0^2}{18 v_0}
\]
考虑容器尺寸的修正后,公式为:
\[
\eta = \frac{(\rho_0 - \rho) d^2 g - \frac{27}{8} \rho d v_0^2}{18 v_0} \left(1 + 2.4 \frac{d}{D}\right) \left(1 + 1.6 \frac{d}{H}\right)
\]
其中 \(D\) 为容器内径,\(H\) 为液柱高度。

\section{实验过程}

\subsection{测量小球的直径与密度}
使用螺旋测微器测量不少于10个小球的直径,计算平均直径并剔除异常值。使用电子天平测量小球的总质量,计算其平均密度。

具体步骤如下:
\begin{enumerate}
    \item 使用螺旋测微器测量10个小球的直径,记录每个小球的直径值,剔除异常值后求取平均值。
    \item 使用电子天平测量这10个小球的总质量,计算平均质量,进而计算小球的平均密度。
\end{enumerate}

\subsection{调整测试架}
使用线锤调整测试架水平,使线锤对准底盘中心圆点。调节两个光电门发射端,使两激光束照在线锤线上,然后调节接收端,使激光正入射到接收器。用钢板尺测量上下激光束的距离 \(s\) 和液体深度 \(H\),用游标卡尺测量量筒内径 \(D\)。

具体步骤如下:
\begin{enumerate}
    \item 将线锤装在支撑横梁中间部位,调整测试架上的三个水平调节螺钉,使线锤对准底盘中心圆点。
    \item 调节光电门发射端,使激光束照在线锤线上,然后收起线锤,调节接收端,使激光正入射到接收器。
    \item 用钢尺测量上、下激光束之间的距离 \(s\) 以及液体的深度 \(H\)。
    \item 用游标卡尺测量量筒的内径 \(D\)。
\end{enumerate}

\subsection{测量液体温度}
使用数字温度计测量液体的初始温度和实验结束时的温度,取平均值作为实验温度。

具体步骤如下:
\begin{enumerate}
    \item 在实验开始时,用数字温度计测量液体的初始温度,并记录。
    \item 实验结束后,再次测量液体的温度,并记录。
    \item 取两次测量值的平均值作为实验过程中液体的温度。
\end{enumerate}

\subsection{测量小球的收尾速度}
多次测量(不少于10次)小球下落时间,计算平均时间 \(t\) 和收尾速度 \(v_0 = \frac{s}{t}\)。若小球下落路径偏离中线,用磁铁吸出小球并重新测量。

具体步骤如下:
\begin{enumerate}
    \item 将小球从导管顶部释放,使其自由下落。
    \item 使用计时器记录小球通过上下两个光电门的时间差。
    \item 重复步骤1和2,不少于10次,记录每次的时间差。
    \item 计算这些时间差的平均值 \(t\)。
    \item 计算小球的收尾速度 \(v_0 = \frac{s}{t}\)。
    \item 若小球下落路径偏离中线,导致光电门无法记录时间,用磁铁将小球吸出,擦拭干净后重新测量。
\end{enumerate}

\subsection{计算黏性系数}
利用公式计算雷诺数 \(Re\) 和液体的黏性系数 \(\eta\),并与参考值比较。

具体步骤如下:
\begin{enumerate}
    \item 根据实验测量数据和公式计算雷诺数 \(Re = \frac{\rho d v_0}{\eta}\)。
    \item 根据公式计算液体的黏性系数 \(\eta\):
    \[
    \eta = \frac{(\rho_0 - \rho) d^2 g - \frac{27}{8} \rho d v_0^2}{18 v_0} \left(1 + 2.4 \frac{d}{D}\right) \left(1 + 1.6 \frac{d}{H}\right)
    \]
    \item 将计算结果与实验参考值进行比较,分析误差来源。
\end{enumerate}




\bibliographystyle{plain}
\bibliography{./template}  %bib文件名

\end{document}
