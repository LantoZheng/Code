\documentclass[12pt,a4paper]{article}
\usepackage[UTF8]{ctex}
\usepackage[backend=biber]{biblatex}
\usepackage{amsmath,amsthm,amssymb,graphicx,multirow,float,caption}
\usepackage{geometry}
\geometry{left=2.54cm, right=2.54cm, top=3.18cm, bottom=3.18cm}
\usepackage{enumitem}
\usepackage{subcaption,booktabs,diagbox}
\setenumerate[1]{itemsep=0pt,partopsep=0pt,parsep=\parskip,topsep=5pt}
\setitemize[1]{itemsep=0pt,partopsep=0pt,parsep=\parskip,topsep=5pt}
\setdescription{itemsep=0pt,partopsep=0pt,parsep=\parskip,topsep=5pt}
\usepackage{adjustbox}
\usepackage[graphicx]{realboxes}
\usepackage{rotating}

\usepackage{titlesec}

\titleformat{\section}%设置section的样式
{\raggedright\large\bfseries}%右对齐,4号字,加粗
{\thesection .\quad}%标号后面有个点
{0pt}%sep label和title之间的水平距离
{}%标题前没有内容

\title{\vspace{-4cm}\Large 光纤的物理性质与应用}  %文章标题
\author{\kaishu 学号:202111030007 \hspace{2cm} 姓名:郑晓旸}   %作者的名称
\date{}

\begin{document}
\maketitle

\begin{abstract}
    (用100-200字描述本次实验的目的、方法、主要结果和主要结论) 
    关键词:(三到五个词,主要用于检索)
\end{abstract}

\section{引言}

\begin{enumerate}
\item 用简短的语言介绍实验的相关背景(发展历程和前景、用途等)、实验目的等
\end{enumerate}

\section{原理}
在理解的基础上,用简明扼要的语言叙述实验原理,切忌照抄讲义
\section{实验}
介绍实验的仪器、实验方法和主要实验过程
\section{结果分析与讨论}
\begin{enumerate}
    \item 介绍结果(定性、定量),分析规律,讨论成因
    1、以图、表的形式结合语言展示实验结果(数据、现象);
    2、分析实验结果呈现出来的规律,进一步处理;
    3、结合原理,比较实验结果与理论预期,对实验中观测到的现象和实验结果进行合理的解释;
    4、分析影响实验结果因素和造成实验误差的原因。
    \item • 图的规范
    ➢ 报告中的所有的图要统一编号。
    ➢ 所有的图必须有图题。
    ➢ 数据图的纵、横轴必须表明其物理意义和单位。
    • 表格的规范
    ➢ 报告中所有的表格也要统一编号。
    ➢ 所有的表格必须有题目,并表明每一栏数据的物理意义
    和单位。
    • 公式的规范
    ➢ 所有的公式也要统一标号。
    ➢ 公式中出现的符号在第一次出现时一定要标明其物理意
    义和单位。
\end{enumerate}
\section{总结和建议}
总结全文(主要方法,主要结果,主要结论),提出建议
\section{参考文献}
1、书的格式:作者 . 书名. 出版地. 出版社 . 出版时间
2、文献的格式: 作者, 论文题目, 期刊题目. 期刊的年、卷、期
\end{document}