\documentclass[12pt,a4paper]{article}
\usepackage[UTF8]{ctex}
\usepackage[backend=biber]{biblatex}
\DefineBibliographyStrings{english}{%
  references = {参考文献},
}
\usepackage{amsmath,amsthm,amssymb,graphicx,multirow,float,caption}
\usepackage[per-mode = symbol]{siunitx}
\usepackage{geometry}
\geometry{left=2.54cm, right=2.54cm, top=3.18cm, bottom=3.18cm}
\usepackage{enumitem}
\usepackage{subcaption,booktabs,diagbox}
\setenumerate[1]{itemsep=0pt,partopsep=0pt,parsep=\parskip,topsep=5pt}
\setitemize[1]{itemsep=0pt,partopsep=0pt,parsep=\parskip,topsep=5pt}
\setdescription{itemsep=0pt,partopsep=0pt,parsep=\parskip,topsep=5pt}
\usepackage{adjustbox}
\usepackage[graphicx]{realboxes}
\usepackage{rotating}
\usepackage{titlesec}

\addbibresource{report.bib}

\newcommand{\be}[1]{
    \begin{equation}
        #1
    \end{equation}
}

\newcommand{\bfig}[3]{
    \begin{figure}[H]
        \centering
        \includegraphics[width=#1\textwidth]{#2}
        \caption{#3}
    \end{figure}
}

\titleformat{\section}%设置section的样式
{\raggedright\large\bfseries}%右对齐,4号字,加粗
{\thesection .\quad}%标号后面有个点
{0pt}%sep label和title之间的水平距离
{}%标题前没有内容

\title{\vspace{-4cm}\Large 氦氖激光的纵横模分析和模分裂实验报告}  %文章标题
\author{\kaishu 学号:202111030007 \hspace{2cm} 姓名:郑晓旸 \hspace{2cm} 指导教师:龚文平}   %作者的名称
\date{\today}

\begin{document}
\maketitle

\begin{abstract}
    本实验利用氦氖激光器和扫描干涉仪等仪器,观察并研究了长、短管的横纵模式及模竞争现象,并用示
波器测量相应的横纵模间隔、自由光谱区、出光带宽等一系列物理量。同时,实验还定性研究了由晶体双折射
引起的纵模分裂并观察分裂光谱的偏振态。\\
    \textsl{关键词:氦氖激光器;横纵模式;模竞争;纵模分裂;偏振态}
\end{abstract}

\section{引言}
激光器一般由增益介质, 谐振腔, 和激励源组成, 谐振腔存在着分立的本征频率, 每一种本征频
率对应的光场分布称为“模”, 通常称沿轴线的光场分布称为纵模, 沿垂直于轴线平面上的光场分布
称为横模, 激光器的模谱特性是决定其应用范围的主要依据, 在生产和应用激光器的过程中, 往往也
需要了解激光器的工作的模式状态。因此, 了解激光模式的形成和掌握其测量方法是十分重要的。
“偏振”是激光器输出光束的特性之一, 当腔内有双折射元件时, 激光器必是线偏振输出。并且由
于不同偏振态的光在腔内的光程不同, 对应的谐振频率也不同, 此时等效于把激光器的一个频率“分
裂”成两个, 这又称为激光的模分裂。由于利用激光的模分裂性质, 能得到单色性, 偏振态更为一致
的激光, 所以观察与研究激光的模分裂现象是很有意义的。 \cite{textbook}

\section{原理}
\subsection{氦氖激光器的纵模、横模及其对应的频率间隔}
\subsubsection{纵模}
谐振腔轴线方向传播的光, 只有那些频率是$\frac{c}{2\mu L}$
的整数倍的光才能形成稳定的光场分布。这种
驻波场的分布称为纵模, 相邻两纵模的频率间隔为
\be{\Delta \nu_q=\frac{c}{2\mu L}}
\subsubsection{横模}
光在谐振腔中来回反射时, 由于工作物质的横截面和镜面都是有限的, 当平行光通过它们时, 因
为衍射作用, 在垂直于光的传播方向上, 也就是横向, 将出现各种不同的场强分布。每种分布形式叫
做一种横模。横向分布加上纵向分布, 可以用三个个符号 m, n, q 来标记, 记作 $TEM_{m,n,q}$ 模。
实验室采取的谐振腔为非共焦腔, 其横模频率差为
\be{\Delta \nu=\frac{c(\Delta m+\Delta n)}{2\mu L \pi}\arccos{\sqrt{(1-\frac{L}{R_1})(1-\frac{L}{R_2})}}}
其中$R_1$和$R_2$为两反射镜的曲率半径。\cite{siegman1986lasers,saleh1991fundamentals}

\subsection{氦氖激光器纵模分裂及模竞争}
\subsubsection{石英晶体双折射效应}
石英晶体是一种单轴正晶体, 既有双折射效应, 又有旋光性, 石英晶体双折射效应使 o 光和 e 光
具有光程差$\delta$, 在不考虑旋光性时, 有
\be{\delta=\left[\left(\frac{\sin ^{2} \theta}{n_{e}^{2}}+\frac{\cos ^{2} \theta}{n_{o}^{2}}\right)^{-\frac{1}{2}}-n_{o}\right] h}
其中 h 是晶片的厚度,$n_o$ 和 $n_e$ 分别是石英晶体的两个主折射率 (对于 632.8nm,$n_o$ = 1.54263 ,$n_e$ =
1.55169), $\theta$是石英晶体的晶轴和光线之间的夹角。\cite{zhao2006optics}
\subsubsection{腔内双折射效应产生激光频率分裂原理}
激光腔长 L 与纵模频率 $\nu_q$ 的关系为
\be{\nu_q=\frac{c}{2 L}q}
当一片双折射元件放入激光谐振腔中, 其引入的光程差$\delta$可看成是谐振腔长改变 $\Delta L$, 即可以取
$\Delta L=-\delta$, 对上式两边取微分可得
\be{\Delta \nu_{q}=-\frac{c}{2 L^{2}} q \Delta L=\frac{\nu}{L} \delta}
\subsection{共焦球面扫描干涉仪}
\subsubsection{结构原理}
透过干涉仪的激光频率满足$\nu=\frac{K c }{4 L}$
, 考虑到 L 是在所设计的腔长$L_0$附近作极微小的变化, 做
泰勒展开取一级近似可得
\be{\Delta \nu=-\frac{K c}{4 L_{0}^{2}} \Delta L}
说明$\nu$的变化与腔长的变化量成正比, 也就是与加在压电陶瓷环上的电压成正比, 由于示波器
的横向扫描采用与干涉仪的腔长扫描同步, 则加在压电陶瓷环上的电压与时间成正比\cite{thorlabs2020confocal}, 所以$\nu$的变
化与示波器上的时间成正比, 则存在
\be{\Delta \nu=\alpha t+\beta}
其中t为示波器上的时间,$\alpha$和$\beta$为参数。
\section{实验}
\subsection{原理:模谱频率的测量和计算}
在示波器上的一个锯齿波周期内, 调整出两个自由光谱区(free spectral range, 简写为SR)后, 可根据两个自由光谱区内对应级次
的谱线的间距, 计算出自由光谱区在示波器上的时间间隔 $t_0$, 之后测量两个纵模或者两个横模之间的
时间间隔 $\Delta t_q$ 或 $\delta t_{\Delta m+\Delta n}$, 由上节最后的公式的线性性可得到, 纵模 $\Delta \nu_q$ 或横模 $\Delta \nu_{\Delta m+\Delta n}$ 的计算公式
\be{\Delta \nu_{q}=\frac{\Delta t_{q}}{t_{0}} \Delta \nu_{S R} \quad \Delta \nu_{\Delta m+\Delta n}=\frac{\Delta t_{\Delta m+\Delta n}}{t_{0}} \Delta \nu_{S R}}
所以在实验中,可以先测量出一个自由光谱区在示波器上对应的时间间隔$ t_0$, 后测量出纵模或横模
在示波器上的时间间隔 $t_1$, 则可用如下公式计算出其对应的频率
\be{\Delta \nu=\frac{t_{1}}{t_{0}} \Delta \nu_{S R}}

\subsection{实验方法}
\subsubsection{测量量氦氖激光管的模谱分布}
测量氦氖激光管的模谱分布的实验装置示意图如下图所示:
\bfig{0.8}{实验装置示意图.png}{实验装置示意图}
在安装好激光管和扫描干涉仪, 调整好光路, 使示波器上信号达到最大后。调节偏置电压、锯齿
波幅度, 在一个锯齿波的周期内, 大约存在 2 个自由光谱区。后在一个自由光谱区内谱线尽量清晰的
情况下, 测量自由光谱区在示波器上的时间间隔; 测量编号并记录下每条谱线之间的间距; 描绘模谱
轮廓曲线; 并判断出每一条谱线对应的振动模式。
\subsubsection{观察氦氖激光管的纵模分裂和模竞争}
观察氦氖激光管的纵模分裂和模竞争的激光器内部元件示意图如下所示, 其中 PZT 为压电
陶瓷, Q 为石英晶片,R 为调整石英晶片与光轴夹角的旋钮,T 为激光器工作的增益物质。
\bfig{0.8}{模分裂装置示意图.png}{激光模分裂装置示意图}
在测量氦氖激光管的模谱分布的光路基础上, 将激光管更换为上图的激光管, 并同理
调整光路, 测量自由光谱区在示波器上的时间间隔。改变加在压电陶瓷上的电压, 测量该激光管的出
光带宽; 并描出激光管增益曲线的轮廓。调整石英晶片晶轴与光束夹角, 使纵模谱线产生足够的分裂
间距; 后使用偏振片进行激光偏振态的观测。
\section{实验结果与讨论}
\subsection{测量氦氖激光管的模谱分布}
\subsubsection{改变偏置电压、锯齿波幅度对模谱的影响}
1、改变偏置电压:改变扫描干涉仪的偏置电压, 能够使模谱在示波器上平移, 但不会改变模谱的间距。

2、锯齿波幅度:改变扫描干涉仪的锯齿波幅度, 能够改变模谱的间距, 锯齿波的幅度增大会导致模间距变小, 会
使一个锯齿波周期内的自由光谱区增多。
\subsubsection{短激光管}
实验室所选用的短氦氖激光管参数$L=0.235m$,干涉仪的自由光谱区$\Delta \nu_{SR}=1875MHz$. 由式(4)可计算得纵模间隔的理论值为
\be{\Delta \nu_q=638.3MHz}
从示波器上可得到短管的模谱如下图所示:
\bfig{0.6}{短管激光器示意图.jpeg}{短管激光器示波器图}
可以看到,一共有两组谱线,对应两个自由光谱区,每组谱线内有三条特征谱线,调节偏置电压并未出
现新的谱线,且谱线间隔无明显变化,可以判定该谱线为纵模。对这两组纵模谱线测量其示波器时间t,如下表:
\begin{table}[htbp]
    \centering
    \caption{短激光管模式分析}
    \begin{tabular}{|c|c|c|c|}
    \hline

    ms  &A  &B  &C  \\ \hline
    $t_1$    &2.507  &6.63   &10.27\\ \hline
    $t_2$    &13.85  &17.25  &20.57\\ \hline
    $t_3$    &23.58  &26.64  &29.7\\ \hline
    \end{tabular}
    \end{table}
其中A、B、C分别表示两个自由光谱区,下标表示该自由光谱区内对应级次的谱线。内容为为相应的示波器时间,单位为ms。通过对不同自由光谱区内相同级次的谱线的时间差进行平均,得
\be{\bar{t}_{SR} = 10.09\unit{ms}}\\
计算得到单位时间对应的频率间隔
\be{\Delta \nu_{sr} =\frac{\Delta \nu}{\Delta t_{sr}}= 185.91 \unit{Mhz\per ms}}\\
在一个自由光谱区内测量级次1,2和级次2,3的距离并平均得到一组纵模间隔,并将两个自由光谱区测得的纵模间隔平均,可得
\be{\Delta t_q = 3.44ms}
代入公式(9)及$\Delta \nu_{SR}=189.51 \unit{MHz\per ms}$可得
\be{\nu_q=638.38MHz}
该值与标准值的相对误差为$0.01\%$,误差非常小。这一步实验可能的导致误差的因素有
式(6)是采用一级近似得到的结果,本身有一定的偏差,即压电电压的变化量不一定严格正比于频率差.
其次自由光谱区及其内部特征谱线数量较少,能测得的数据有限,求平均值的效果一般。该因素不是误差的主
要因素。
\subsubsection{长激光管}
实验室所选用的短氦氖激光管参数$L=0.340m$,干涉仪的自由光谱区$\Delta \nu_{SR}=1875MHz$. 由式(4)可计算得纵模间隔的理论值为
\be{\Delta \nu_q=441.2MHz},共振腔$R_1=1m$, $R_2=\infty$, 有横模的理论值为$\Delta \nu_{\Delta m+\Delta n=1}=87.423MHz$.
从示波器上可得到短管的模谱如下图所示:
\bfig{0.6}{长管激光器示意图.jpeg}{长管激光器示波器图}
同样可以看到,一共有两组谱线,对应两个自由光谱区,但每组谱线内有多条特征谱线,调节偏置电压谱
线会出现明显变化,且某些谱线最终会消失,某些会得到加强,即特征谱线在不断竞争,可以判定长管谱线包
括纵、横模。每一组纵模下至多出现四个横模,如图所示,
之所以有某些横模不完整,是因为其幅度较小在输出阈值之下, 不满足激光器要求的增益大于损耗。
在两个自由光谱区分别测量7条谱线(4+3,即有一组纵模中包含的一个横模未能观察到),示波器时间如下表所示:
\\
\begin{table}[H]
    \centering
    \caption{长激光管模式分析}
    \begin{tabular}{|c|c|c|c|c|c|c|c|}
    \hline
     ms & 1     & 2     & 3     & 4     & 5    & 6     & 7     \\ \hline
    A & 13.53 &14.58    &15.48  &16.42  &18.18  &19.12  &19.95 \\ \hline
    B & 31.91 &32.7     &33.56  &34.35  &35.85  &36.67  &37.42 \\ \hline
    \end{tabular}
    \end{table}
自由光谱区与纵模的测量的方法与之前相同。
根据两个自由光谱区内对应级次的谱线的间距,可得到自由光谱区在示波器上的时间间隔$\Delta t_{sr} = 17.90 \unit{ms}$
于是,在给定自由光谱区标准值1875MHz的情况下,可得到单位时间对应的频率间隔:$\Delta \nu =104.77 \unit{Mhz \per ms}$\\
观察不同模式的变化,可以判断峰1、5,为两组横模对应的模式,\be{\Delta \nu_q=(\bar{t_5}-\bar{t_1})\cdot \Delta \nu=450.03 \unit{Mhz}}\\
与标准值的相对误差为$1.9\%$。\\
横模间隔的测量稍显复杂,需要在观察到的四组纵模中对横模间隔测量并使用逐差法进行计算,最终计算得
\be{\Delta \nu_{m,n}=\Delta t_{m,n} \cdot \Delta \nu=91.41MHz},与$\Delta m+\Delta n=1$得横模间距相近,误差为$4.5\%$。
因此基本可以判断,一组纵模中分别出现的模式(m,n)=(0,0),(1,0),(0,2)与(1,1),(2,1)四种。模式之间在不断竞争,但总体上激光的横模呈现以(0,0)为背景,隐约可以看见
两个方向上的节点的情况。如果依照示波器图,就是以(0,0)为主要模式。激光光斑图如下图所示:
\bfig{0.5}{激光光斑.jpg}{激光光斑}
边缘在两个方向上的突起暗示了(0,1)和(1,0)两种模式的存在,隐约可见的节点暗示了(0,2),整体又以均匀的(0,0)为背景,符合判断。
\subsection{观察氦氖激光管的纵模分裂和模竞争}
\subsubsection{出光带宽观测}
调整偏置电压,利用示波器的余辉功能,通过反复扫描模谱得到光器的增益曲线如下:
\bfig{0.7}{增益曲线.jpeg}{增益曲线}
根据每组增益曲线内的两条谱线,同之前一样的方法可测得自由光谱时间
\be{t_{SR}=8.85ms}测量增益曲线左右消失点间的平均距离为6.68ms, 换算得出光带宽为1417MHz.

这一步可能导致实验误差的因素有
式(9)采用一级近似得到的结果,本身有一定的偏差,即压电电压的变化量不一定严格正比于频率差
; 增益曲线及其内部特征谱线数量较少,能测得的数据有限,求平均值的效果一般。该因素不是误差的主要
因素; 模谱中各模式之间不断竞争,模谱一直在动态变化,且短时间内不会稳定,测量时是暂停示波器来测量当
前的模谱,故可能会有一定偏差。
\subsection{纵模分裂及其偏振态}
\subsubsection{相邻纵模的偏振关系}
更换配置了石英片的激光器,在石英晶片转角为0时测得模谱如下图:
\bfig{0.7}{偏振谱.jpeg}{纵模谱线}
观察到两个自由光谱区,每个区内有两组谱线,从左到右依次记为a1,a2,b1,b2四条谱线。
接下来在共振腔前放置检偏器。

实验中观察到检偏器角度在42°-60°之间, a1与b1消光;检偏器角度在132°-148°之间,a2与b2消光,同时a1与b1复原。
这表明每一个纵模都是线偏振光,并且由这两个角度区域相差90°,可以发现相邻纵模的偏振方向垂直。
\subsection{旋转石英片的纵模分裂}
在石英晶片转角不为0时测得模谱如下图:
\bfig{0.7}{偏振分裂.jpeg}{纵模谱线分裂}
与前图对比,会发现每一个纵模分裂为两个模式,这应是由双折射效应导致的。再次使用检偏器,发现
在48°-56°和140°-146°,会发现四组分裂的谱线中的两个模式分别发生消光,即分裂出来的两个模式之间偏振方向垂直。因此得到结论,两个模式分别是o光和e光。

旋转石英晶片,随着$\theta$增大,起初将观察到o模式和e模式的分裂间距不断增大。然而纵模的间距是保持不变的,模式分裂大到一定程度时会发生,k级中靠右的模式与k+1级中靠左的模式会接近,
从而看起来像同一级的模式,表现为o模式和e模式的分裂间距又减小。在$\theta$的不断增大的过程中,将观察到分裂间距的增大与减小在交替发生。
\section{总结}
本次实验分别研究长、短管激光器的输出模式,观察不同频率之间的竞争效应,测量其频率间隔并与理论
值相比较,偏差相对较小。实验还研究了由双折射效应产生的纵模分裂现象,并探究其偏振态等等。通过本次
实验,本人初步掌握了利用共焦球面扫描干涉仪研究激光器特性的方法,并复习了激光器输出模式及相关的光
学概念,加深了激光器原理及其输出特点的理解。


\printbibliography
\end{document}