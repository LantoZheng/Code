
\documentclass[lang=cn,10pt,newtx,bibend=biber,device=pad]{elegantbook}

\title{电动力学}

\setcounter{tocdepth}{3}

\logo{logo-blue.png}
\cover{cover.jpg}

% 本文档命令
\usepackage{array}
\usepackage{ulem}
\usepackage{amssymb}
\usepackage{cases}
\usepackage{booktabs}
\newcommand{\ccr}[1]{\makecell{{\color{#1}\rule{1cm}{1cm}}}}

\def\bea{\begin{eqnarray}}
\def\eea{\end{eqnarray}}
\def\nn{\nonumber}



\usepackage{tensor}
\usepackage{physics}






\newcolumntype{P}[1]{>{\Centering\hspace{0pt}}p{#1}}
\newcolumntype{Z}{>{\centering\arraybackslash}X} %Z单元格居中

\newcommand{\ds}{{S}}
\newcommand{\xs}{{s}}

\newcommand{\codename}{\textbf{Coport}}

\newcommand{\br}[1]{\left[#1\right]}
\newcommand{\ee}{\mathrm{e}}
\newcommand{\Mc}[1]{\mathcal{#1}}
\newcommand{\mJ}{\mathcal{J}}
\newcommand{\mA}{\mathcal{A}}
\newcommand{\mR}{\mathcal{R}}
\newcommand{\mS}{\mathcal{S}}
\newcommand{\mD}{\mathcal{D}}
\newcommand{\mX}{\mathcal{X}}
\newcommand{\mP}{\mathcal{P}}
\newcommand{\mQ}{\mathcal{Q}}
\newcommand{\mE}{\mathcal{E}}
\newcommand{\mY}{\mathcal{Y}}
\newcommand{\mT}{\mathcal{T}}
\newcommand{\df}{\mathrm{d}}   %微分符号
\newcommand{\dif}{\mathrm{d}}   %微分符号
\newcommand{\qpar}{\quad\par}   
\newcommand{\deri}[3]{\dfrac{\mathrm{d}^{#1}#2}{\mathrm{d}#3^{#1}}}%莱布尼茨导数记号
\newcommand{\pderi}[3]{\dfrac{\partial^{#1}#2}{\partial {#3}^{#1}}}%偏导数导数记号
\newcommand{\lrg}[1]{\langle #1\rangle }
\newcommand{\pa}[1]{\left(#1\right)}
\newcommand{\yf}[1]{\textcolor[RGB]{0,0,255}{ #1-yf}}
\newcommand{\cb}[1]{\textcolor[RGB]{255,0,0}{ #1 }}
\newcommand{\mg}[1]{\textcolor[RGB]{155,25,0}{ #1-mg}}
\newcommand{\er}[1]{\textcolor[RGB]{0,225,0}{#1-old}}




% 修改标题页的橙色带
\definecolor{customcolor}{RGB}{32,178,170}
\colorlet{coverlinecolor}{customcolor}
\usepackage{cprotect}

\addbibresource[location=local]{reference.bib} % 参考文献,不要删除
\begin{document}


\section{$\delta$函数}

\subsection{引言}

在学习电磁学的过程中,我们常常提到各种点源,比如说点电荷、质点这类有确定的总量,但是体积视作无限小的对象;如今我们在处理这类有关场的问题时更倾向使用场的方程来求解这类问题,但是这样的点源在以麦克斯韦方程组为代表的一系列方程中却很难描述:他们既有确定的总量,但同时也在空间中没有确定的密度——要么是 0,要么是无穷大。如何使用一种合适的数学结构去描述这种特殊的结构就成了急需要解决的任务,为了描述这类对象,物理学家狄拉克发明了$\delta$函数。

\subsection{$\delta$函数的定义}

$\delta$函数可以描述这样的一类物理对象:它具有确定的总量,但是其在某种空间中分布在一个确定的点上,比如说一个位于原点的点电荷$Q$,在空间中除原点以外的任何一点都找不到电荷的分布,但是空间中确实有大小为$Q $的电荷总量存在。

为了从简单的情况入手,将我们上面的描述转换为一维数学表达\cite{Robert1994}:

\begin{equation}
\begin{align}

\int_{\mathbb{R}}\rho(x)dx&=Q \\\rho(x)=0&;x\in\mathbb{R}/\{0\}

\end{align}
\end{equation}\label{eq:delta_ini}

$\delta$函数作为一种新的函数形式,它与我们之前见到的函数形式都不相同。回忆我们之前见到的黎曼积分和黎曼可积函数的定义:

\begin{definition}[可积函数]\label{def:int}
    在闭区间$[a,b]$上有分法$\{x_i|i\in {0,1,...,n};x_{i+1}>x_i;x_0 =a;x_n=b\}$,$\xi_i \in [x_{i},x_{i+1}],\Delta x_i = x_{i+1} - x_i$,$\lambda = max \{ \Delta x_i \}$。当$\lambda \rightarrow0$ 时,若$\sigma = \sum_{i=0}^{n-1}f(\xi_i)\Delta x_i$ 有限,且存在极限为$I$ 则称$f(x)$在$[a,b]$上可积.
\end{definition}
从定义~\ref{def:int} 出发,我们很容易找到$\delta$函数不可积的证据:

\begin{lemma}[不可积性]
    $\delta$函数在$\mathbb{R}$上不可积。
    考虑积分区域$\Omega' = \mathbb{R}/U(0,\delta)$,$\delta$为非负小量,显然的,$\int_{\Omega'}\rho(x)dx= 0$;若$\rho(x)$在$\mathbb{R}$上可积,则有:任意$\delta>0$,$\int_{U(0,\delta)}\rho(x)dx=Q$,这至少要求存在一点$p\in U(0,\delta)$,任意$A \in \mathbb{R},\rho(p)>A$,即$\rho(p)= \infin$。同时还要求$\int_{\{p\}} dx > 0$。
    所以这样的函数在$\Omega$上不可积。
\end{lemma}
但是我们知道,可积函数列的极限不一定是可积函数。
所以我们不妨使用一列积分值为常数,同时函数列极限满足式~\ref{eq:delta_ini}的函数列来作为这种函数的定义。
比如说,我们构建以下函数列使之满足上面的条件:$f_n(x) =  \begin{cases} 0 &\text{other}\\ nQ &\text{$|x|<\frac{1}{n}$}  \end{cases}$

这列函数关于$i\rightarrow \infin$的各种极限满足我们上面的要求:

\begin{equation}
\begin{cases}
\lim_{i\rightarrow\infin}f_i(x)=0;x\neq0\\
\lim_{i\rightarrow\infin}\int_{\mathbb{R}}f_i(x)dx = Q
\end{cases}
\end{equation}

更进一步的,为了是这样的函数一般化,我们不妨定义所有有这样趋近行为的函数为一个等价类,并且定义其积分值为 1构建一类新的函数,称为$\delta$函数:

> 定义:**$\delta$函数**​<sup>[[注]](siyuan://blocks/20241127151041-44eqiwk)</sup>
>
> ${f_i(x)}$为$\mathbb{R}$上黎曼可积函数列,满足:
>
> $$
> \begin{cases}
> \lim_{i\rightarrow\infin}f_i(x)=0;x\neq0\\
> \lim_{i\rightarrow\infin}\int_{\mathbb{R}}f_i(x)dx = 1
> \end{cases}
> $$
>
> 称所有这样的函数列${f_i(x)}$的极限为:$\delta(x)$
>
> 其值和积分定义为:
>
> * 值:
> * $$
>   \delta(x)=
>   \begin{cases}
>   0 &x\neq0 \\
>   \infin &x=0
>   \end{cases}
>   $$
> * 积分,对于任意的光滑函数$f(x)$:
> * $$
>   \int_{\Omega}f(x)\delta(x)dx=\begin{cases}0 & \Omega\not\owns0\\f(0)&\Omega\owns 0\end{cases}
>   $$
> * 值得注意的是,这里我们定义$\delta$函数时使用了其积分作为定义的一部分,并且没有使用常见的定义:
> * $$
>   \int_{0^-}^{0^+}\delta(x)dx=1
>   $$
> * 这里是要强调$\delta$函数并非是我们常见的函数,其本身的积分并没有意义,上面的式 xx 应当看作$f(x)=1$的特殊情形。

## $\delta$函数的性质

### 作为独立的函数

这样的函数列定义使得$\delta$函数满足一般的函数的求导和积分的定律:

#### 缩放和对称性<sup>[[注]](siyuan://blocks/20241127150923-9omwy9o)</sup>

对于非零标量$\alpha$,$\delta$函数满足以下缩放特性:

$$
\int_{-\infin}^{\infin}\delta(\alpha x)dx \overset{let \ u=\alpha x}{=} \int_{-\infin}^{\infin}\delta(u)\frac{du}{|\alpha|} = \frac{1}{|\alpha|}
$$

也即:

$$
\delta(\alpha x) =\frac{\delta(x)}{|\alpha|}
$$

> 证明:
>
> 给定$\mathbb{R}$上光滑紧支撑函数$f(x)$,以及$\alpha\ge 0$:
>
> $$
> \int_{-\infin}^{\infin} dx\ f(x)\delta(\alpha x) \overset{let \ x'=\alpha x}{=}\frac{1}{\alpha}\int_{-\infin}^{\infin} dx'\ f(x'/\alpha)\delta(x') = \frac{1}{\alpha}f(0)
> $$
>
> 对于$\alpha\le0$:
>
> $$
> \int_{-\infty}^{\infty}dx\ f(x)\delta(\alpha x)\overset{let\ x^{\prime}=\alpha x}{=}\frac{1}{-|\alpha|}\int_{\infty}^{-\infty}dx^{\prime}\ f(x^{\prime}/\alpha)\delta(x^{\prime})=\frac{1}{|\alpha|}\int_{-\infty}^{\infty}dx^{\prime}\ f(x^{\prime}/\alpha)\delta(x^{\prime})=\frac{1}{|\alpha|}f(0)
> $$
>
> 所以有:
>
> $$
> \delta(\alpha x) = \frac{\delta(x)}{|\alpha|}
> $$

同样的,$\delta$函数还有关于$x=0$ 的镜像对称性:

> $$
> \delta(x) = \delta(-x)
> $$

根据k 阶齐次函数定义:$f(ax) = a^k f(x)$,我们将$\delta$函数视作-1 阶齐次函数,对于$\delta(x)$,其具有$[\frac{1}{x}]$的量纲,b比如说常见的点源表达:$Q\delta(x-x_0)$这里有和(一维)电荷密度相同的量纲。

#### 在平移作用下的性质

在平移作用$x\rightarrow x-T$作用下,如何函数$f(x)$与时间延迟的狄拉克函数进行积分,则该积分挑选出函数在$x=T$时的值。该性质也被称之为:挑选性<sup>[[注]](siyuan://blocks/20241128200845-q42505a)</sup>,挑选是直接从$\delta$函数的积分定义看出来的,对于积分:

$$
\int_{-\infin}^\infin f(x)\delta(x-T)dx=\int_{-\infin}^\infin f(x'+T)\delta(x')dx' = f(x'+T)|_{x'=0}=f(T)
$$

该结果直接使用$\delta$函数积分的定义可以得到。

那么这样的$\delta$函数引入参数 T 后与任意函数$f$积分给出$f$ 在$T$ 处的值的性质让我们想到其另一种用法,也就是卷积:

$$
(f\ast\delta_T)(t) = \int_{-\infin}^\infin f(\tau)\delta(t-T-\tau)d\tau = f(t-T)
$$

可见,$\delta(t-T)$的卷积可以作为函数平移的一种工具。这点将在我们之后的讨论中派上用场。

### 高维的$\delta$函数

我们上面将$\delta$函数定义在了$\mathbb{R}$上,但是我们知道,在处理实际问题中,点源往往位于三维空间或者二维空间中,而且有时候我们还会选取如球坐标系等度规在空间各处不完全相同的坐标,在这种情况下,我们如何定义 $\delta$ 函数呢?

首先,我们要明确,我们期待的在高维空间中的$\delta$函数定义能赋予$\delta$函数的良好的性质有哪些?

在这里我们罗列出以下几点:

1. 在形式上,比如说:函数值、积分定义的意义上与$\mathbb{R}$上的 $\delta$函数类似,这样我们才可以说它是$\delta$函数在高维空间中的推广。用数学语言这么书写:

    > 设$\Omega$ 为$n$维空间,$0 \in\Omega $,$\Omega$上定义了函数$f:\Omega\to\mathbb{R}$,以及对应的积分$\int d\mathbf{x}$.
    >
    > 则若$\delta(\mathbf{x})$是$\Omega$ 上广义函数,则:
    >
    > $\delta(\mathbf{x})= \begin{cases} 0&\mathbf{x}\neq0\\ \infin&\mathbf{x}=0 \end{cases}$;
    >
    > $\int \delta(\mathbf{x})f(\mathbf{x})d\mathbf{x} = f(0)$  
    >
2. 在空间中的坐标经历变换$\mathbf{x} \mapsto \mathbf{y} = \mathbf{y}(\mathbf{x})$后,与同一函数$f$ 相乘后全空间积分结果不变。这样的性质保证了$\delta$函数是在坐标变换下不变的,能够描述一个实在的物理实体。

    > 对于空间$\Omega$上有坐标系$\mathbf{x}$,定义有函数$f(\mathbf{x}):\{x_i\}\to\mathbb{R}$,以及该坐标系下的度规$g_{ij}$;在坐标变换:$\mathbf{x} \mapsto \mathbf{y} = \mathbf{y}(\mathbf{x})$下:
    >
    > $\delta(\mathbf{y}(\mathbf{x}))=\begin{cases} 0&\mathbf{x}\neq0\\ \infin&\mathbf{x}=0 \end{cases}$  
    >
    > $\int \delta(\mathbf{x})f(\mathbf{x})d\mathbf{x} = \int \delta(\mathbf{y})f(\mathbf{y})d\mathbf{y} = f(0)$  
    >

根据上面的要求,我们可以这么定义高维空间中的$\delta$函数:

> 靠考虑 n 维空间$(\mathcal{M}, g)$,其中:
>
> * $\mathcal{M}$为一个连续的高维空间。
> * $g = g_{\mu\nu} dx^\mu \otimes dx^\nu$ 是度规张量。
>
> 定义$\delta$-函数与测试函数$f$的积分为:
>
> $ \int_{\mathbb{R}^n} f(x) \delta^{(n)}(x - x_0) \sqrt{|g|} \, d^n x = f(x_0)$
>
> 其中:
>
> * $x = (x^1, x^2, \ldots, x^n)$ 是 $n$-维坐标向量,
> * $x_0 = (x_0^1, x_0^2, \ldots, x_0^n)$是 $\delta$-函数的中心,
> * $\sqrt{|g|}$ 是度规 $g_{ij}$ 的行列式的平方根,用于引入体积元。
>
> 在这里,我们可以说,$\delta$函数有以下的形式:
>
> $ \delta^{(n)}(x - x_0) = \frac{\delta(x^1 - x_0^1) \delta(x^2 - x_0^2) \cdots \delta(x^n - x_0^n)}{\sqrt{|g(x)|}}. $

这样的高维定义可以定义出类似于三维空间中的点电荷的分布,并且保证了在类似直角坐标和球坐标的变换下,$\delta$函数积分值保持不变。

> 例:
>
> 假设二维 $\delta$-函数中心在 $(x_0, y_0)$,定义为:
>
> $$\delta$(2)(x−x_0,y−y_0)=$\delta$(x−x_0)$\delta$(y−y_0)$  
>
> 测试函数 $f(x, y)$ 的积分为:
>
> $\int_{\mathbb{R}^2} f(x, y) \delta^{(2)}(x - x_0, y - y_0) \, dx \, dy = f(x_0, y_0)$\.
>
> 2\. 在极坐标系中积分
>
> 将直角坐标 $(x, y)$ 转换为极坐标 $(r, \theta)$:
>
> $x = r \cos\theta, \quad y = r \sin\theta$\.
>
> 在极坐标中,$\delta^{(2)}(x - x_0, y - y_0)$转换为:
>
> $\delta^{(2)}(x - x_0, y - y_0) = \frac{\delta(r - r_0)}{r} \cdot \delta(\theta - \theta_0),$
>
> 其中:  
> 	•	$r_0 = \sqrt{x_0^2 + y_0^2}$,  
> 	•	$\theta_0 = \arctan(y_0 / x_0)$。
>
> 因此,测试函数 $f(r, \theta)$ 的积分为:
>
> $\int_{0}^\infty \int_{0}^{2\pi} f(r, \theta) \delta^{(2)}(x - x_0, y - y_0) r \, dr \, d\theta.$
>
> 代入 $\delta^{(2)}$ 的表达式:
>
> $\int_{0}^\infty \int_{0}^{2\pi} f(r, \theta) \frac{\delta(r - r_0)}{r} \cdot \delta(\theta - \theta_0) r \, dr \, d\theta$\.
>
> 化简后:
>
> $\int_{0}^\infty \int_{0}^{2\pi} f(r, \theta) \delta(r - r_0) \delta(\theta - \theta_0) \, dr \, d\theta$\.
>
> 执行积分:  
> 	•	对 r 积分:$\int_0^\infty f(r, \theta) \delta(r - r_0) \, dr = f(r_0, \theta)$,  
> 	•	对 $\theta$ 积分:$\int_0^{2\pi} f(r_0, \theta) \delta(\theta - \theta_0) \, d\theta = f(r_0, \theta_0)$。
>
> 最终结果为:
>
> $f(r_0, \theta_0)$\.
>
> 3\. 结果验证
>
> 注意到:
>
> $f(r_0, \theta_0) = f(x_0, y_0)$,
>
> 因此在直角坐标系和极坐标系中的积分结果相同。这验证了高维 \\delta-函数的积分在坐标变换下保持不变。

但是我们这里只关心了在 $n$ 维空间中的 $n$ 阶$$\delta$$函数,能够描述例如空间中的点电荷这样的模型,但是我们还知道,有类似于三维中的面电荷、线电荷这种模型,他们在阶数上来看可以使用 $n-k$ 阶的$\delta$函数描述,我们也在这里给出他们的定<sup>[[注]](siyuan://blocks/20241130221824-jtryhwg)</sup>义:

> 考虑 $n$维伪黎曼流形$(\mathcal{M}, g)$,其中:  
> 	•	$\mathcal{M}$ 是流形;  
> 	•	$g = g_{\mu\nu} dx\mu \otimes dx\nu$ 是度规张量。  
> 令 $f: \mathcal{M} \to \mathbb{R}^k$ 是 $k$-维标量函数,其零点集为 $\Sigma = {x \in \mathcal{M} : f(x) = 0}$,$\Sigma$ 是一个嵌入在$\mathcal{M}$ 中的 $n-k$ 维子流形。  
> 我们定义在 $\mathcal{M}$ 上的 Dirac $\delta$ 分布,描述 $f(x) = 0$ 处的筛选性质。
>
> ---
>
> 在高维空间中,$\delta(f(x))$ 被定义为在零点流形 $\Sigma$ 上的分布:  
> ​$\delta(f(x)) = \int_{\Sigma} \delta(x - x_0) |\det J(x_0)|^{-1} , d\Sigma(x_0)$,  
> 其中:  
> 	•	$J(x_0)$ 是 $f(x)$ 关于局部坐标的雅可比矩阵 $J_{ij} = \frac{\partial f_i}{\partial x^j}$;  
> 	•	$|\det J(x_0)|$ 是雅可比矩阵的行列式绝对值;  
> 	•	$d\Sigma(x_0)$ 是子流形 $\Sigma$ 上的体积形式,定义为:  
> ​$d\Sigma(x_0) = \sqrt{\det (h)} , dx1 \wedge dx2 \wedge \cdots \wedge dx^{n-k}$,  
> 其中 $h$ 是诱导度规。
>
> ---
>
> 对于任意测试函数 $g(x)$,高维 $\delta$ 函数在 $\mathcal{M}$ 上的作用定义为:  
> $\int_{\mathcal{M}} \delta(f(x)) g(x) \sqrt{|g|} , dn x = \int_\Sigma \frac{g(x)}{\sqrt{\det (J\top J)}} d\Sigma$,  
> 其中:  
> 	•	$\sqrt{|g|}$ 是流形 $\mathcal{M}$ 上的度规行列式;  
> 	•	$\sqrt{\det (J\top J)} = |\det J|$ 是雅可比矩阵的范数;
>
> 	•	$dn x$ 是流形 $\mathcal{M}$ 上的体积形式。
>
> ---
>
> 在欧几里得空间$\mathbb{R}^n$中,度规 $g_{\mu\nu} = \delta_{\mu\nu}$,$\sqrt{|g|} = 1$。积分形式简化为:
>
> $\int_{\mathbb{R}^n} \delta(f(x)) g(x) \, d^n x = \int_\Sigma \frac{g(x)}{|\det J|} d\Sigma$\.
>
> ---
>
> 考虑二维欧几里得空间上的函数 $f(x, y) = x^2 + y^2 - R^2$:  
> 	1\.	零点集合为圆周 $x^2 + y^2 = R^2$;  
> 	2\.	度规 $|g| = 1$,雅可比行列式 $|\det J| = 2R$;  
> 	3\.	$\delta(f(x, y)) = \frac{\delta(r-R)}{2R}$,其中 $r = \sqrt{x^2 + y^2}$。
>
> 积分:
>
> $\int_{\mathbb{R}^2} \delta(f(x, y)) g(x, y) dx dy = \int_0^{2\pi} g(R \cos\theta, R \sin\theta) \, d\theta$\.

‍

### $\delta$函数与其他函数复合

$\delta$函数的不可积性质使得其与其他函数的复合后的积分变得困难,因为对可积函数$f(x)$而言,不管是与$\delta$函数复合产生的$f\circ\delta(x)$,还是$\delta$函数与之复合产生的$\delta\circ f(x)$都在函数意义上不可积、不可微,甚至可能不是良定义的函数。

但是我们依然能够依据其函数列的定义给出$\delta$函数与其他函数复合产生的$\delta\circ f(x)$的值,积分的表示,至于其他的形式以及他们的微分性质我们现在没有办法直接从定义出发得到,到后面的部分,我们拥有了将$\delta$函数展开的能力时,我们才可以讨论这些内容

让我们先从$\delta$函数与其他函数的复合函数的值入手:

> **定义:**​**$\delta\circ f(x)$**​**的值**
>
> 函数列$\{g_n(x)\}$为任一极限为$\delta(x)$的函数列,f(x)为$\mathbb{R}$上光滑函数,$x_0\in \mathbb{R}$。
>
> $$
> \delta\circ f(x_0) \overset{def}{=} \lim_{n\rightarrow\infin} g_n\circ f(x_0)
> $$

根据$\delta$函数定义中,函数列的连续性以及$f(x)$的好的性质,我们很容易得到以下事实:

1. 函数$\delta\circ f(x)$在$\mathbb{R}\rightarrow\{\mathbb{R},\infin,-\infin\}$上是存在且唯一的。
2. 函数$\delta\circ f(x) =  \begin{cases} 0&f(x)\neq 0\\ \infin &f(x)=0 \end{cases}$

类似的,我们当然也可以定义其他函数和$\delta$函数复合的值:

> **定义:**​**$f\circ \delta(x)$**​**的值**
>
> 函数列$\{g_n(x)\}$为任一极限为$\delta(x)$的函数列,f(x)为$\mathbb{R}$上光滑且满足$\lim_{x\rightarrow\infin}f(x)\in\mathbb{R}$的函数,$x_0\in \mathbb{R}$。
>
> $$
> f\circ \delta(x) \overset{def}{=} \lim_{n\rightarrow\infin} f\circ g_n(x_0)
> $$

但是这将导致以下问题:

1. 这样的函数看上去没有什么物理意义,没有人会指望将一种密度无限大的源做除了积分、微分以外的操作,这可以想象将导致某种发散,也就是违反上面定义中的$\lim_{x\rightarrow\infin}f(x)\in\mathbb{R}$。
2. 这样操作后的函数做积分也容易导致在$x=0$处不可积,并且不方便使用我们上面提到的函数列方法定义其在 0 附近的积分或者微分值;而其在$x\neq 0$处的值也不是很有意义,它挑选出$f(0)$的值并将其填满全空间,这一般($f(0)\neq 0$时)也会导致该函数全空间积分发散。

所以我们往往不认为$f\circ \delta(x)$有良好的定义<sup>[[注]](siyuan://blocks/20241128224944-cj9n93j)</sup>

其实从这一点,我们就能想到类似的这种作用与函数的算子的定义域应当有的性质了,比如说:光滑、在无穷远处光滑的收敛到零等等。这类函数叫做速降函数,是类似$\delta$函数这类奇异广义函数的积分泛函的定义域。<sup>[[注]](siyuan://blocks/20241129204526-l0u7c5u)</sup>

现在,我们再来关心一下形如$\delta\circ f(x)$的函数的积分性质:

要计算$\int\delta\circ f(x) dx$,我们依旧从定义出发,构建一个函数列$\{g_n(x)\}$满足定义 xx,然后取积分$\int g_n(f(x)) dx$的极限。

我们先从性质较好的 $f(x)$入手:

> 设$f(x)$为$\mathbb{R}$上光滑函数,且对所有的$x_i$满足$f(x_i) = 0$,$f'(x_i) \neq 0,|\{x_i\}| \in\mathbb{R}$。
>
> 我们先断言:$\int_{-\infin}^{\infin} \delta\circ f(x) dx = \sum_i \frac{1}{|f'(x_i)|}$,然后证明这一结论。
>
> 首先计算:
>
> $$
> \int_{-\infin}^{\infin} g_n\circ f(x) dx = \int_{-\infin}^{x^-_0} g_n\circ f(x) dx+\sum_i \int_{x^+_i}^{x^-_{i+1}} g_n\circ f(x) dx + \int^{+\infin}_{x_{max}} g_n\circ f(x) dx + \sum_i \int_{x_i^-}^{x_i^+} g_n\circ f(x) dx
> $$
>
> 对两段取$n\rightarrow \infin$,若极限存在(这里不妨假设存在,证明可以从$g_n$除了奇异点的部分的收敛和$f$的可积性得到):
>
> $$
> \begin{align*}
> \lim_{n\rightarrow\infin}\int_{-\infin}^{x^-_0} g_n\circ f(x) dx &= \int_{-\infin}^{x^-_0}\lim_{n\rightarrow\infin}g_n\circ f(x) dx = 0\\
> \lim_{n\rightarrow\infin}\int_{x^+_i}^{x^-_{i+1}} g_n\circ f(x) dx &= \int_{x^+_i}^{x^-_{i+1}} \lim_{n\rightarrow\infin} g_n\circ f(x) dx = 0\\
>
> \end{align*}
> $$
>
> 而对于任何$x_i$,以下积分:
>
> $$
> \int_{x_i^-}^{x_i^+} g_n\circ f(x) dx = \int_{x_i^-}^{x_i^+} \frac{g_n\circ f(x)}{f'(x)}d(f(x))=\frac{1}{|f'(x_i)|}\int_{0^-}^{0^+} g_n(f) df = \frac{1}{|f'(x_i)|}
> $$
>
> 因此,我们可以说上面的结果基本上正确。

进一步的,我们考虑$|\{x_i\}|=\infin$的情况,可以预见,这就要取决于$\sum_i\frac{1}{|f'(x_i)|}$是否收敛了。如果收敛,那么上面积分的结果形式依然不变。

在这种意义上,我们可以将这种复合函数也写作一系列$\delta$函数平移后的线性组合,这样的结果在数值和积分上与该复合函数在行为上没有差别:

> $$
> \delta\circ f(x) = \sum_i \frac{\delta(x-x_i)}{|f'(x_i)|}
> $$

再进一步,我们还能讨论当 $f'(x_i)=0$ 时的情况,在这种情况下,我们一般在传统意义上认为积分发散或者说积分不是良定义的。但如果必须要求给出一个有那么些意义的值,我们需要使用一些数学小技巧计算$x_i$ 附近的积分。

考虑到一定需要得到一个在$x_i$附近的值,我们有以下这几种方法:

1. 光滑函数正则化方法

    既然在$x_i$附近直接使用 $\delta$函数的性质无法得到良好的积分结果,我们不妨直接给定一个 $\delta$函数的趋近方法,比如说:给定函数$\rho_\epsilon(x) = \exp(-\epsilon|x|)$,然后直接对给定的积分区域$\Omega \owns x_i$,计算:$\lim_{\epsilon\rightarrow \infin} \int_{\Omega} \rho_\epsilon(f(x)) dx$。这样我们往往有可能得到一个合理的值。当然,在物理学的计算中计算结果往往对$\rho_\epsilon$ 的选取不敏感。<sup>[[注]](siyuan://blocks/20241129224544-3q620ym)</sup>
2. 维数正则化

    1. 将问题推广到$d$维空间。$x_0$ 是$f(x) = 0$的一个重根,因此在$x_0$ 附近泰勒展开的领头阶$f(x) \approx  A(x-x_0)^n$ ,零点附近的积分可表示为:$\int \delta(f(x)) dx \sim \int \delta(u) |J| du$其中  $u = f(x)$ ,而  $J$  是雅可比行列式, $J = |df/du|$ 。在  $f{\prime}(x_0) = 0$  时,积分在  $d=1$  维下发散,但在  $d \to d-\epsilon$  维中可以重新定义。
    2. 在非整数维数下,使用高斯积分和 Gamma 函数近似:$\int \delta(f(x)) dx \to \int \delta(u) |u|^{-\epsilon} du$\.通过 Gamma 函数的解析延拓,将发散部分隔离。例如:$\Gamma(-\epsilon) = \frac{1}{\epsilon} + \text{有限项}$\.
    3. 将发散部分作为“正则化常数”吸收到物理量的重整化过程中,得到一个有限值。最终结果依赖于:$\int \delta(f(x)) dx = \frac{1}{|f^{(n)}(x_0)|} \cdot \text{校正因子}$,

        校正因子与  $n$ 、维度  $d$  和  $\epsilon$  有关。
    4. 例子:

        > 计算积分:$\int \delta(f(x))dx$的积分,其中$f(x) = (x-x_0)^n$。
        >
        > 考虑积分:$ I = \int \delta((x - x_0)^n) dx $,当  $n > 1$  时, $f{\prime}(x_0) = 0$,此积分需要正则化以确保其意义。
        >
        > $\delta$函数性质:
        >
        > $\delta((x - x_0)^n) = \frac{\delta(x - x_0)}{|f{\prime}(x_0)|}.$
        >
        > 但  $f{\prime}(x_0) = 0$  时,上述形式无意义。
        >
        > 将  $\delta(f(x))$  扩展到  $d = 1 - \epsilon$  维,并引入正则化参数  $\epsilon$  控制发散。重新定义:
        >
        > $\delta((x - x_0)^n) = \frac{1}{|f^{(n)}(x_0)|} |x - x_0|^{-\epsilon} \delta(x - x_0).$
        >
        > 其中  $f^{(n)}(x_0) = n!$,是  $f(x)$  的  $n$  阶导数。
        >
        > 校正因子用于描述维数正则化对积分的贡献。考虑正则化后:
        >
        > $\int \delta((x - x_0)^n) dx = \frac{1}{|f^{(n)}(x_0)|} \int_{-\infty}^\infty |x - x_0|^{-\epsilon} \delta(x - x_0) dx$\.
        >
        > 利用 $\delta$函数的筛选性质,仅保留  $x = x_0$  处的值:
        >
        > $\int \delta((x - x_0)^n) dx = \frac{1}{|f^{(n)}(x_0)|} \cdot \lim_{\epsilon \to 0} |x_0 - x_0|^{-\epsilon}$\.
        >
        > 因  $|x_0 - x_0| = 0$ ,产生奇异性。
        >
        > 通过引入维数正则化的解析延拓技巧,定义积分在  $\epsilon \to 0$  下的有限部分。此时:
        >
        > $\int \delta((x - x_0)^n) dx = \frac{\Gamma(-\epsilon)}{|f^{(n)}(x_0)|}$\.
        >
        > 其中:
        >
        > $\Gamma(-\epsilon) \sim \frac{1}{\epsilon} - \gamma + \mathcal{O}(\epsilon)$,
        >
        > $\gamma$是欧拉常数。
        >
        > 最终积分结果表达为:
        >
        > $\int \delta((x - x_0)^n) dx = \frac{\text{有限值} + \text{发散项}}{n!}=\frac{\frac{1}{\epsilon} - \gamma}{n!}$\.
        >
        > 发散项通过重整化被吸收到系统的参数中,仅保留有限值作为物理意义的结果。
        >
        > 即,上面的积分结果我们只保留有限项部分:
        >
        > $\int \delta((x-x_0)^n)x = -\frac{\gamma}{n!}$
        >
        > 这么做的物理背景是:我们在处理实际问题时,可以将发散项$\frac{1}{\epsilon}$视为多余的独立项,并且使用全空间归一化条件将其消除。
        >

### $\delta$函数的微分性质

由于 $\delta$ 函数不是一个在$\mathbb{R}$上有良好定义的函数,我们很难直接使用常见的$\delta - \epsilon$语言分析其微分性质。我们可以尝试将其类比于一般的函数然后通过其积分性质得到它的微分结果:

> $\int\delta'(x)\varphi(x)dx=\delta(x)\varphi(x)-\int \delta(x)\varphi'(x)dx = -\varphi'(0)$

考虑到$\delta$函数的值并没有具体意义,我们也可以形式化的将其写成:

$\delta'(x) = \lim_{h\to0}\frac{\delta(x+h)-\delta(x-h)}{2h}$

不难看出$\delta'$有以下性质:

$\delta'(-x) = -\delta'(x)\\x\delta'(x) = -\delta(x)$  

## $\delta$函数在偏微分方程中

在偏微分方程中出现 $\delta$函数显然很难处理特别是在处理点源这样的问题时,泊松方程中出现$\delta$函数作为场源,其奇异性使得传统的解法无法分析这样的函数。在这里我们也许可以使用泊松方程的特征函数对$\delta$函数进行展开。

考虑拉普拉斯方程:$\nabla^2\varphi = \rho(\mathbf{x})$

其特征方程为:$\nabla^2\varphi = k^2 \varphi$。

对于自由边界条件下,其特征解为:$ \psi(x, y, z) = (C_x e^{i k_x x} + D_x e^{-i k_x x})(C_y e^{i k_y y} + D_y e^{-i k_y y})(C_z e^{i k_z z} + D_z e^{-i k_z z}) $

因此,我们可以按$\psi_\mathbf{k}(\mathbf{x}) = e^{\pm i\mathbf{k}\cdot\mathbf{x}}$展开$\rho(\mathbf{x})$从而得到拉普拉斯方程的特解。

对于$\rho=\delta(\mathbf{x})$的拉普拉斯方程,我们也才用这样的思路求解点源产生的场分布。

所以,我们需要求解$\delta$函数对$e^{\pm i kx}$的展开,这正是我们熟悉的傅里叶展开。

### $\delta$函数的傅里叶展开

#### 高维空间中的傅里叶展开

要解决$\delta$函数的傅里叶展开,我们首先要给出高维空间中,傅里叶变换的形式。回顾一维的傅里叶变换及其逆变换:

> 一维傅里叶变换:
>
> $f(x)$为$\mathbb{R}$上可积函数,$f$ 的傅里叶变换为:
>
> $$
> \mathcal{F}[f(x)]=F(\omega)=\frac{1}{\sqrt{2\pi}}\int_{\mathbb{R}}f(x)e^{-i\omega x}dx
> $$
>
> 逆变换为:
>
> $$
> \mathcal{F}^{-1}[F(\omega)]=f(x)=\frac{1}{\sqrt{2\pi}}\int_{\mathbb{R}}F(\omega)e^{i\omega x}d\omega
> $$

其中的函数基底为:$\{\frac{1}{\sqrt{2\pi}}e^{i\omega x}\}$;对于高维来说,函数基底为:$\{\frac{1}{\sqrt{2\pi}^{d}}e^{i\mathbf{k}\cdot\mathbf{x}}\}$;

因此 d 维傅里叶变换和逆变换可以写为:

> 高维傅里叶变换:
>
> 给定$\mathbb{R}^d$​空间中的可积函数$f(\mathbf{r})$,其傅里叶变换定义为:
>
> $$
> \mathcal{F}[f(\mathbf{r})] = F(\mathbf{k}) =\frac{1}{\sqrt{2\pi}^{d}} \int_{\mathbb{R}^d} f(\mathbf{r}) e^{-i \mathbf{k} \cdot \mathbf{r}} d\mathbf{r}
> $$
>
> 逆变换定义为:
>
> $$
> \mathcal{F}^{-1}[F(\mathbf{k})] = f(\mathbf{r}) =\frac{1}{\sqrt{2\pi}^{d}} \int_{\mathbb{R}^d} F(\mathbf{k}) e^{i \mathbf{k} \cdot \mathbf{r}} d\mathbf{k}
> $$

#### 计算$\delta$函数展开

利用以上定义,计算定义在$\mathbb{R}^d$上的 $\delta^{(n)}(\mathbf{r})$的傅里叶展开:

$$
\mathcal{F}[\delta^{(n)}(\mathbf{r})]  =\frac{1}{\sqrt{2\pi}^{d}} \int_{\mathbb{R}^d} \delta^{(n)}(\mathbf{r}) e^{-i \mathbf{k} \cdot \mathbf{r}} d\mathbf{r} = \frac{1}{\sqrt{2\pi}^{d}}
$$

可见$\delta$函数的傅里叶展开结果是一个常数,这非常符合我们对$\delta$函数的一般想象,因为这个常数函数的积分性质几乎和 $\delta$ 函数一样差——它在全空间也是不可积的。

#### 利用$\delta$函数的傅里叶展开求解拉普拉斯方程

现在我们开始求解场源为$\delta$函数的拉普拉斯方程的解:

$$
\nabla^2 \varphi(\mathbf{r}) = \delta^{(3)}(\mathbf{r}-\mathbf{r'})
$$

对两侧取傅里叶变换,并简化:

$$
\mathcal{F}[\nabla^2 \varphi(\mathbf{r})] = \mathcal{F}[\delta^{(3)}(\mathbf{r}-\mathbf{r'})]\\
-\mathbf{k}^2\mathcal{F}[\varphi(\mathbf{r})]=\frac{1}{(2\pi)^{3/2}}e^{-i\mathbf{k}\cdot\mathbf{r'}}\\
\mathcal{F}[\varphi(\mathbf{r})]=-\frac{1}{(2\pi)^{3/2}}\frac{e^{-i\mathbf{k}\cdot\mathbf{r'}}}{\mathbf{k}^2}
$$

对右侧取傅里叶逆变换:

$$
\mathcal{F}^{-1}[-\frac{1}{(2\pi)^{3/2}}\frac{e^{-i\mathbf{k}\cdot\mathbf{r'}}}{\mathbf{k}^2}] = -\frac{1}{4\pi|\mathbf{r}-\mathbf{r'}|}
$$

最终我们得到:

$$
\nabla^2 \frac{1}{|\mathbf{r}-\mathbf{r'}|} = -4\pi \delta^{(3)}(\mathbf{r}-\mathbf{r'})
$$

## 脚注

> 缩放和对称性 [[↩️]](siyuan://blocks/20241127150734-xsjfq5s)​[Strichartz, R.](https://en.wikipedia.org/wiki/Robert_Strichartz "Robert Strichartz") (1994), *[A Guide to Distribution Theory and Fourier Transforms](https://books.google.com/books?id=T7vEOGGDCh4C)*, CRC Press, [ISBN](https://en.wikipedia.org/wiki/ISBN_(identifier) "ISBN (identifier)") [978-0-8493-8273-4](https://en.wikipedia.org/wiki/Special:BookSources/978-0-8493-8273-4 "Special:BookSources/978-0-8493-8273-4").Problem 2.6.2

​

> $\delta$函数 [[↩️]](siyuan://blocks/20241124112317-up7j2at)​[Strichartz, R.](https://en.wikipedia.org/wiki/Robert_Strichartz "Robert Strichartz") (1994), *[A Guide to Distribution Theory and Fourier Transforms](https://books.google.com/books?id=T7vEOGGDCh4C)*, CRC Press, [ISBN](https://en.wikipedia.org/wiki/ISBN_(identifier) "ISBN (identifier)") [978-0-8493-8273-4](https://en.wikipedia.org/wiki/Special:BookSources/978-0-8493-8273-4 "Special:BookSources/978-0-8493-8273-4"). Chapter 2.2

​

> 数学表达 [[↩️]](siyuan://blocks/20241123222319-oxgtooj)​[Gelfand, I. M.](https://en.wikipedia.org/wiki/Israel_Gelfand "Israel Gelfand"); Shilov, G. E. (1966–1968), *[Generalized functions](https://books.google.com/books?id=nAnjBQAAQBAJ)*, vol. 1–5, Academic Press, [ISBN](https://en.wikipedia.org/wiki/ISBN_(identifier) "ISBN (identifier)") [9781483262246](https://en.wikipedia.org/wiki/Special:BookSources/9781483262246 "Special:BookSources/9781483262246"). Chapter1.1

​

> 挑选性 [[↩️]](siyuan://blocks/20241128180126-lg9epdn)TY  - BOOK  
> T1  - The Fourier transform and its applications  
> AU  - Bracewell, Ronald N. (Ronald Newbold), 1921-2007  
> PY  - [1978]  
> PB  - Second edition. New York : McGraw-Hill, [1978] ©1978  
> AB  - xvi, 444 pages : illustrations ; 24 cm  
> CN  - QA403.5 B7 1978  
> CN  - QA403.5 .B7 1978  
> N1  - Includes index.  
> ID  - 999495438302121  
> UR  - https://search.library.wisc.edu/catalog/999495438302121  
> CH - 3.3



> 不认为$f\circ \delta(x)$</span></span></span>有良好的定义 [[↩️]](siyuan://blocks/20241128224546-sce3z1l)NIcholas Wheeler,1987,Simplified production of Dirac Delta Function identities,Reed Colledge Physics Department
‍

> 这类函数叫做速降函数,是类似$\delta$函数这类奇异广义函数定义的积分泛函的定义域。 [[↩️]](siyuan://blocks/20241129204112-rzmvfrp)这类积分泛函的更多详细讨论可以参见:程建春,2004,科学出版社,数学物理方程及其近似方法,Ch 3.1

​

> 当然,在物理学的计算中计算结果往往对$\rho_\epsilon$<span class="vlist" style="height:0.15em;"><span></span></span></span></span></span></span></span></span></span> 的选取不敏感。 [[↩️]](siyuan://blocks/20241129222811-7437vor)@article{10.1119/1.19485,  
>     author = {Nyeo, Su-Long},  
>     title = {Regularization methods for delta-function potential in two-dimensional quantum mechanics},  
>     journal = {American Journal of Physics},  
>     volume = {68},  
>     number = {6},  
>     pages = {571-575},  
>     year = {2000},  
>     month = {06},  
>     abstract = {The quantum mechanics of a bound particle in the delta-function potential in two dimensions is studied with a discussion of its regularization and renormalization. A simple regularization approach is considered with the introduction of several regularizing functions for defining the quantum system. More systematic regularization is introduced from the mathematical viewpoint of the theory of distributions. The renormalization scheme independence of the physical observable is demonstrated.},  
>     issn = {0002-9505},  
>     doi = {10.1119/1.19485},  
>     url = {https://doi.org/10.1119/1.19485},  
>     eprint = {https://pubs.aip.org/aapt/ajp/article-pdf/68/6/571/7528703/571\_1\_online.pdf},  
> }

​

> [[↩️]Schutz, B., Geometrical Methods of Mathematical Physics.](siyuan://blocks/20241130221158-wzjh0d0)
>
> Courant, R., & Hilbert, D., Methods of Mathematical Physics.

​

‍
\end{document}