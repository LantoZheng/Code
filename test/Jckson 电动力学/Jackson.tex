\documentclass[lang=cn,10pt,newtx,bibend=biber,device=pad]{elegantbook}

\title{经典电动力学(第三版)}

\setcounter{tocdepth}{3}

\logo{logo-blue.png}
\cover{cover.jpg}

% 本文档命令
\usepackage{array}
\usepackage{ulem}
\usepackage{amssymb}
\usepackage{cases}
\usepackage{booktabs}
\newcommand{\ccr}[1]{\makecell{{\color{#1}\rule{1cm}{1cm}}}}

\def\bea{\begin{eqnarray}}
\def\eea{\end{eqnarray}}
\def\nn{\nonumber}



\usepackage{tensor}
\usepackage{physics}






\newcolumntype{P}[1]{>{\Centering\hspace{0pt}}p{#1}}
\newcolumntype{Z}{>{\centering\arraybackslash}X} %Z单元格居中

\newcommand{\ds}{{S}}
\newcommand{\xs}{{s}}

\newcommand{\codename}{\textbf{Coport}}

\newcommand{\br}[1]{\left[#1\right]}
\newcommand{\ee}{\mathrm{e}}
\newcommand{\Mc}[1]{\mathcal{#1}}
\newcommand{\mJ}{\mathcal{J}}
\newcommand{\mA}{\mathcal{A}}
\newcommand{\mR}{\mathcal{R}}
\newcommand{\mS}{\mathcal{S}}
\newcommand{\mD}{\mathcal{D}}
\newcommand{\mX}{\mathcal{X}}
\newcommand{\mP}{\mathcal{P}}
\newcommand{\mQ}{\mathcal{Q}}
\newcommand{\mE}{\mathcal{E}}
\newcommand{\mY}{\mathcal{Y}}
\newcommand{\mT}{\mathcal{T}}
\newcommand{\df}{\mathrm{d}}   %微分符号
\newcommand{\dif}{\mathrm{d}}   %微分符号
\newcommand{\qpar}{\quad\par}   
\newcommand{\deri}[3]{\dfrac{\mathrm{d}^{#1}#2}{\mathrm{d}#3^{#1}}}%莱布尼茨导数记号
\newcommand{\pderi}[3]{\dfrac{\partial^{#1}#2}{\partial {#3}^{#1}}}%偏导数导数记号
\newcommand{\lrg}[1]{\langle #1\rangle }
\newcommand{\pa}[1]{\left(#1\right)}
\newcommand{\yf}[1]{\textcolor[RGB]{0,0,255}{ #1-yf}}
\newcommand{\cb}[1]{\textcolor[RGB]{255,0,0}{ #1 }}
\newcommand{\mg}[1]{\textcolor[RGB]{155,25,0}{ #1-mg}}
\newcommand{\er}[1]{\textcolor[RGB]{0,225,0}{#1-old}}




% 修改标题页的橙色带
\definecolor{customcolor}{RGB}{32,178,170}
\colorlet{coverlinecolor}{customcolor}
\usepackage{cprotect}

\addbibresource[location=local]{reference.bib} % 参考文献,不要删除

\begin{document}
	
\maketitle
\frontmatter

\tableofcontents

\mainmatter

\chapter{静电学边值问题II}
\section{球协函数的补充定理}
\begin{figure}[h]
    \centering
    \includegraphics[width=0.4\textwidth]{figure/SphericalAdd.png}
    \caption{}
    \label{fig:fig1}
\end{figure}
为了计算这个系数,我们注意到,根据公式\ref{eq:3.60},它可以看作是函数 $\sqrt{4\pi/(2l + 1)} Y_l^m(\theta, \phi)$ 在以式\ref{eq:3.64} 中的参考轴(即带撇号的坐标轴)上的球谐函数 $Y_l^m(\gamma, \beta)$ 展开中 $m{\prime} = 0$ 的系数。由式\ref{eq:3.59} 可以发现,由于只有一个 $l$ 值存在,系数\ref{eq:3.66}表达为:
\begin{equation}\label{eq:3.67}
    A_m(\theta',\phi')=\frac{4\pi}{2l+1}\{Y_{lm}^{*}[\theta(\gamma,\beta),\phi(\gamma,\beta)]\}_{\gamma=0}
\end{equation}
当 $\gamma \to 0$ 时,角 $(\theta, \phi)$ 作为 $(\gamma, \beta)$ 的函数,将趋近于 $(\theta{\prime}, \phi{\prime})$。由此,附加定理 \ref{eq:3.62} 得证。有时该定理会使用 $\mathrm{P}l^m(\cos \theta)$ 的形式而不是 $Y_l^m$ 的形式来表示。此时,其形式为:
\begin{equation}\label{eq:3.68}
P_l(\cos \gamma) = P_l(\cos \theta) P_l(\cos \theta{\prime}) + 2 \sum{m=1}^l \frac{(l - m)!}{(l + m)!} \mathrm{P}_l^m(\cos \theta) \mathrm{P}_l^m(\cos \theta{\prime}) \cos[m(\phi - \phi{\prime})] \tag{3.68}
\end{equation}

若角度 $\gamma$ 趋于零,也可以推导出球协函数平方的“求和规则”:
\begin{equation}\label{eq:3.69}
    \sum_{m=-l}^l |Y_l^m(\theta, \phi)|^2 = \frac{2l + 1}{4\pi} \tag{3.69}    
\end{equation}

附加定理还可以用于将位于 $\mathbf{x}{\prime}$ 的单位电荷在 $\mathbf{x}$ 处产生的势的展开式\ref{eq:3.38}以最简洁的形式写出。

将式\ref{eq:3.62} 代入 $P_l(\cos \gamma)$,得到:
\begin{equation}\label{eq:3.70}
\frac{1}{|\mathbf{x} - \mathbf{x}{\prime}|} = 4\pi \sum_{l=0}^\infty \sum_{m=-l}^l \frac{1}{2l + 1} \left(\frac{r_<^l}{r_>^{l+1}}\right) Y_l^{m*}(\theta{\prime}, \phi{\prime}) Y_l^m(\theta, \phi) \tag{3.70}
\end{equation}

式 \ref{eq:3.70} 给出了坐标 $\mathbf{x}$ 和 $\mathbf{x}{\prime}$ 中因式分解的势形式。这对于涉及电荷密度积分的情况非常有用,其中一个变量是积分变量,另一个是观测点的坐标。然而,这样做的代价是用双重求和代替了单次求和。
\section{柱坐标系下的拉普拉斯方程;贝塞尔方程}
如图\ref{fig:cylin_coor}所示,在柱坐标系$(\rho,\phi,z)$下,拉普拉斯方程为如下的形式:
\begin{equation}\label{eq:3.71}
    \pderi{2}{\Phi}{\rho} + \frac{1}{\rho} \pderi{}{\Phi}{\rho} + \frac{1}{\rho^2} \pderi{2}{\Phi}{\phi} + \pderi{2}{\Phi}{z} = 0 
\end{equation}
\begin{figure}[h]
    \centering
    \includegraphics[width=0.4\textwidth]{figure/clyin_coor.png}
    \caption{}
    \label{fig:cylin_coor}
\end{figure}

变量分离由如下的变换给出:
\begin{equation}\label{eq:3.72}
    \Phi(\rho,\phi,z) = R(\rho) Q(\phi) Z(z)
\end{equation}
一般情况下,这将把上面的偏微分方程化为三个常微分方程:
\begin{equation}\label{eq:3.73}
    \begin{aligned}
        \deri{2}{Z}{z}-k^2Z &= 0 \\
        \deri{2}{Q}{\phi}+\nu^2Q &= 0 \\
        \deri{2}{R}{\rho}+\frac{1}{\rho}\deri{R}{\rho}+\left(k^2-\frac{\nu^2}{\rho^2}\right)R &= 0
    \end{aligned}
\end{equation}
前两个常微分方程的解是基本的:
\begin{equation}\label{eq:3.74}
    \begin{aligned}
        Z(z)&=\ee^{\pm ikz} \\
        Q(\phi)&=\ee^{\pm i\nu\phi}
    \end{aligned}
\end{equation}
为了使势函数在整个方位角范围内是单值的,$\nu$ 必须是一个整数。但如果在 $z$ 方向没有任何边界条件的限制,参数 $k$ 是任意的。目前,我们假设 k 是实数且为正值。

径向的方程可以在变换$x=k\rho$下变为标准形式:
\begin{equation}\label{eq:3.75}
    \deri{2}{R}{x}+\frac{1}{x}\deri{R}{x}+\left(1-\frac{\nu^2}{x^2}\right)R=0
\end{equation}
这是贝塞尔方程的标准形式,该方程的解被称为$\nu$阶的贝塞尔函数。
假设解的形式为幂级数展开的形式:
\begin{equation}\label{eq:3.76}
    R(x)=x^\alpha\sum_{j=0}^{\infty}a_jx^J
\end{equation}
可以发现:
\begin{equation}\label{eq:3.77}
    \alpha = \pm \nu
\end{equation}
且对所有$j=1,2,3,\cdots$有:
\begin{equation}\label{eq:3.78}
    a_{2j}=-\frac{1}{4j(j+\alpha)}a_{2j-2}
\end{equation}
所有奇次项的系数都为零。这样,我们就可以通过递推法给出所有的系数:
\begin{equation}
    a_{2j}=\frac{(-1)^j\Gamma(\alpha+1)}{2^{2j}j!\Gamma(j+\alpha+1)}a_0
\end{equation}
\end{document}

