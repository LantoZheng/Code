\documentclass{article}

\usepackage{ctex}
\usepackage{graphicx}
\usepackage{amsmath}
\graphicspath{{image/}}

\title{实验报告}
\author{
    郑晓旸\thanks{Funded by China Scholarship Council}
}
\date{\today}

\begin{document}
\maketitle
这里是一片文档\newline
\textbf{这是粗体}
\newline
\textit{这是斜体}
\newline
\underline{这是下划线}
\newline
\underline{\textbf{它们可以组合使用}}
\newline
\emph{这是强调,可以取决于上下文,给定强调方式}
\newline
\textit{在斜体中\emph{它是正体}作为强调}
\begin{figure}[h]
    \centering
    \includegraphics[width=0.5\textwidth]{example}
    \caption{example}
    \label{fig:example}
\end{figure}

这就是图 \ref{fig:example}
\begin{itemize}
    \item 这个是列表
    \item 第二行list
    \item 第三行list
\end{itemize}
\begin{enumerate}
    \item 这是序列1
    \item 这是序列2
    \item 这是序列3
\end{enumerate}
\begin{description}
    \item[First] 这是描述
    \item[Second] 这是描述
    \item[Third] 这是描述
\end{description}
接下来我们来看看数学公式,这是一个行内公式 $a^2 + b^2 = c^2$,这是一个行间公式
\[
    a^2 + b^2 = c^2
\]
\[
    \int_0^1 x^2 \, dx
\]


Subscripts in math mode are written as $a_b$ and superscripts are written as $a^b$. These can be combined and nested to write expressions such as

\[ T^{i_1 i_2 \dots i_p}_{j_1 j_2 \dots j_q} = T(x^{i_1},\dots,x^{i_p},e_{j_1},\dots,e_{j_q}) \]
 
We write integrals using $\int$ and fractions using $\frac{a}{b}$. Limits are placed on integrals using superscripts and subscripts:

\[ \int_0^1 \frac{dx}{e^x} =  \frac{e-1}{e} \]

Lower case Greek letters are written as $\omega$ $\delta$ etc. while upper case Greek letters are written as $\Omega$ $\Delta$.

Mathematical operators are prefixed with a backslash as $\sin(\beta)$, $\cos(\alpha)$, $\log(x)$ etc.


The well-known Pythagorean theorem \(x^2 + y^2 = z^2\) was proved to be invalid for other exponents, meaning the next equation has no integer solutions for \(n>2\):

\[ x^n + y^n = z^n \]

\section{Second example}

This is a simple math expression \(\sqrt{x^2+1}\) inside text. 
And this is also the same: 
\begin{math}
\sqrt{x^2+1}
\end{math}
but by using another command.

This is a simple math expression without numbering
\[\sqrt{x^2+1}\] 
separated from text.

This is also the same:
\begin{displaymath}
\sqrt{x^2+1}
\end{displaymath}

\ldots and this:
\begin{equation*}
\sqrt{x^2+1}
\end{equation*}

\end{document}
