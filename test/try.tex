\documentclass[12pt]{article}
\usepackage{amsmath, amssymb, amsthm, geometry, hyperref, amsfonts,ctex}
\usepackage{bm}
\geometry{a4paper, margin=1in}

\title{δ函数讲义}
\author{}
\date{}

\begin{document}
\maketitle

\section*{引言}

在学习电磁学的过程中,我们常常提到各种点源,比如说点电荷、质点这类有确定的总量,但是体积视作无限小的对象;如今我们在处理这类有关场的问题时更倾向使用场的方程来求解这类问题,但是这样的点源在以麦克斯韦方程组为代表的一系列方程中却很难描述:他们既有确定的总量,但同时也在空间中没有确定的密度——要么是 $0$,要么是无穷大。如何使用一种合适的数学结构去描述这种特殊的结构就成了急需要解决的任务。为了描述这类对象,物理学家狄拉克发明了 $\delta$ 函数。

\section*{δ函数的定义}

$\delta$ 函数可以描述这样的一类物理对象:它具有确定的总量,但是其在某种空间中分布在一个确定的点上,比如说一个位于原点的点电荷 $Q$,在空间中除原点以外的任何一点都找不到电荷的分布,但是空间中确实有大小为 $Q$ 的电荷总量存在。

为了从简单的情况入手,将我们上面的描述转换为一维数学表达:
\begin{align*}
\int_{\mathbb{R}} \rho(x) \, dx &= Q, \\
\rho(x) &= 0, \quad x \in \mathbb{R} \setminus \{0\}.
\end{align*}

δ 函数作为一种新的函数形式,它与我们之前见到的函数形式都不相同。回忆我们之前见到的黎曼积分和黎曼可积函数的定义:

\begin{quote}
\textbf{可积函数}

定义:在闭区间 $[a,b]$ 上有分法 $\{x_i \mid i \in \{0,1,...,n\}; x_{i+1}>x_i; x_0 =a; x_n=b\}$, $\xi_i \in [x_{i},x_{i+1}]$, $\Delta x_i = x_{i+1} - x_i$, $\lambda = \max \{\Delta x_i\}$。当 $\lambda \to 0$ 时,若 $\sigma = \sum_{i=0}^{n-1} f(\xi_i)\Delta x_i$ 有限,且存在极限为 $I$ 则称 $f(x)$ 在 $[a,b]$ 上可积。
\end{quote}

从该定义出发,我们很容易找到 δ 函数不可积的证据:

\begin{quote}
考虑积分区域 $\Omega' = \mathbb{R} \setminus U(0,\delta)$,$\delta$ 为非负小量,显然 $\int_{\Omega'} \rho(x) dx = 0$; 若 $\rho(x)$ 在 $\mathbb{R}$ 上可积,则有:任意 $\delta>0$, $\int_{U(0,\delta)} \rho(x) dx = Q$, 这至少要求存在一点 $p \in U(0,\delta)$, 任意 $A \in \mathbb{R}, \rho(p) > A$,即 $\rho(p)= \infty$。同时还要求 $\int_{\{p\}} dx > 0$。
\end{quote}

所以这样的函数在 $\Omega$ 上不可积。

然而,我们知道,可积函数列的极限不一定是可积函数。

所以我们不妨使用一列积分值为常数,同时函数列极限满足上述描述的函数列来作为这种函数的定义。

例如,我们构建以下函数列使之满足这些条件:
\[
f_n(x) =
\begin{cases}
0 & \text{其他} \\
nQ & |x| < \frac{1}{n}.
\end{cases}
\]

这列函数关于 $n \to \infty$ 的各种极限满足要求:
\[
\begin{cases}
\lim_{n \to \infty} f_n(x) = 0, & x \neq 0, \\
\lim_{n \to \infty} \int_{\mathbb{R}} f_n(x) \, dx = Q.
\end{cases}
\]

进一步,为了使这样的函数一般化,我们定义所有具有上述趋近行为的函数为一个等价类,并且定义其积分值为 $1$ 来构建一类新的函数,称为 $\delta$ 函数:

\begin{quote}
\textbf{定义:δ函数}

${f_n(x)}$ 为 $\mathbb{R}$ 上黎曼可积函数列,满足:
\[
\begin{cases}
\lim_{n \to \infty} f_n(x) = 0, & x \neq 0, \\
\lim_{n \to \infty} \int_{\mathbb{R}} f_n(x) \, dx = 1.
\end{cases}
\]
称所有这样的函数列 ${f_n(x)}$ 的极限为 $\delta(x)$。

其值和积分定义为:
\[
\delta(x) =
\begin{cases}
0, & x \neq 0, \\
\infty, & x = 0.
\end{cases}
\]
积分,对于任意光滑函数 $f(x)$:
\[
\int_\Omega f(x) \delta(x) \, dx =
\begin{cases}
0, & 0 \notin \Omega, \\
f(0), & 0 \in \Omega.
\end{cases}
\]
值得注意的是,这里我们定义 $\delta$ 函数时使用了其积分作为定义的一部分,并且没有使用常见的定义 $\int_{0^-}^{0^+} \delta(x) \, dx = 1$。
\end{quote}

\section*{δ函数的性质}

\subsection*{作为独立的函数}

这样的函数列定义使得 δ 函数满足一般函数的求导和积分的定律:

\subsubsection*{缩放和对称性}

对于非零标量 $\alpha$,δ 函数满足以下缩放特性:
\[
\int_{-\infty}^\infty \delta(\alpha x) \, dx =
\int_{-\infty}^\infty \delta(u) \frac{du}{|\alpha|} = \frac{1}{|\alpha|}.
\]

即:
\[
\delta(\alpha x) = \frac{\delta(x)}{|\alpha|}.
\]

证明过程略,详细参考文件中的公式推导。

$\delta(x)$ 关于 $x = 0$ 的镜像对称性:
\[
\delta(x) = \delta(-x).
\]

\subsection*{在平移作用下的性质}

在平移作用 $x \to x - T$ 下,若函数 $f(x)$ 与时间延迟的狄拉克函数进行积分,则该积分挑选出函数在 $x = T$ 时的值。该性质被称为挑选性,可以直接从 $\delta$ 函数的积分定义中看出。

\[
\int_{-\infty}^\infty f(x) \delta(x - T) \, dx =
\int_{-\infty}^\infty f(x' + T) \delta(x') \, dx' = f(x' + T) \big|_{x'=0} = f(T).
\]

因此,$\delta(x - T)$ 的卷积可以作为函数平移的一种工具:
\[
(f * \delta_T)(t) = \int_{-\infty}^\infty f(\tau) \delta(t - T - \tau) \, d\tau = f(t - T).
\]

\subsection*{高维的 δ 函数}

在高维空间中定义 $\delta$ 函数需要考虑以下几点性质:

1. 在形式上,与一维 $\delta$ 函数的值、积分定义一致。
2. 在坐标变换下积分不变性。

基于上述要求,高维空间中的 $\delta$ 函数定义如下:

\[
\int_{\mathbb{R}^n} f(x) \delta^{(n)}(x - x_0) \sqrt{|g|} \, d^n x = f(x_0),
\]
其中:
- $x = (x^1, x^2, \ldots, x^n)$ 是 $n$ 维坐标向量,
- $x_0 = (x_0^1, x_0^2, \ldots, x_0^n)$ 是 $\delta$ 函数的中心,
- $\sqrt{|g|}$ 是度规 $g_{ij}$ 的行列式的平方根,用于引入体积元。

这可进一步写为:
\[
\delta^{(n)}(x - x_0) = \frac{\delta(x^1 - x_0^1) \delta(x^2 - x_0^2) \cdots \delta(x^n - x_0^n)}{\sqrt{|g(x)|}}.
\]

 例子:二维空间中的 $\delta$ 函数

在二维欧几里得空间中,$\delta^{(2)}(x - x_0, y - y_0)$ 定义为:
\[
\delta^{(2)}(x - x_0, y - y_0) = \delta(x - x_0) \delta(y - y_0).
\]

测试函数 $f(x, y)$ 的积分为:
\[
\int_{\mathbb{R}^2} f(x, y) \delta^{(2)}(x - x_0, y - y_0) \, dx \, dy = f(x_0, y_0).
\]

 极坐标系下的积分

将直角坐标 $(x, y)$ 转换为极坐标 $(r, \theta)$:
\[
x = r \cos\theta, \quad y = r \sin\theta.
\]

$\delta^{(2)}(x - x_0, y - y_0)$ 转换为:
\[
\delta^{(2)}(x - x_0, y - y_0) = \frac{\delta(r - r_0)}{r} \delta(\theta - \theta_0),
\]
其中:
- $r_0 = \sqrt{x_0^2 + y_0^2}$,
- $\theta_0 = \arctan(y_0 / x_0)$.

积分结果为:
\[
\int_0^\infty \int_0^{2\pi} f(r, \theta) \delta^{(2)}(x - x_0, y - y_0) r \, dr \, d\theta = f(r_0, \theta_0).
\]

这验证了高维 $\delta$ 函数在坐标变换下积分结果不变。

\subsection*{δ 函数与其他函数复合}

δ 函数的不可积性质使得其与其他函数复合后的积分变得复杂。例如,形如 $\delta(f(x))$ 的复合函数可以通过以下形式表示:
\[
\int \delta(f(x)) \, dx = \sum_i \frac{1}{|f'(x_i)|},
\]
其中 $x_i$ 是 $f(x) = 0$ 的所有解,且 $f'(x_i) \neq 0$。

若 $f'(x_i) = 0$,则需要通过正则化方法处理,如:
1. 光滑函数正则化方法:构建 $\rho_\epsilon(x)$ 来近似 $\delta$ 函数。
2. 维数正则化:将问题推广到高维空间,通过 Gamma 函数处理发散项。

\subsection*{δ 函数的微分性质}

δ 函数的微分定义为:
\[
\int \delta'(x) \varphi(x) \, dx = -\varphi'(0).
\]

其导数具有以下性质:
\[
\delta'(-x) = -\delta'(x), \quad x \delta'(x) = -\delta(x).
\]




\end{document}