% 示例 LaTeX 文档用于测试 AI 检测器

\documentclass{article}
\usepackage[utf8]{inputenc}
\usepackage{amsmath}

\title{AI Content Detection Example}
\author{Test Author}
\date{\today}

\begin{document}

\maketitle

\section{Introduction}

This is a sample document to demonstrate the AI content detection system. 
The tool analyzes each word and calculates its perplexity based on the preceding context.

\section{Methodology}

The detection algorithm works by computing the likelihood of each token given the previous tokens.
Lower perplexity indicates higher predictability, which often correlates with AI-generated content.

\subsection{Technical Details}

We use the OpenAI API to estimate token probabilities. The formula for perplexity is:

\[
\text{Perplexity} = \frac{1}{P(\text{token}|\text{context})}
\]

where $P(\text{token}|\text{context})$ represents the conditional probability.

\section{Results}

The system successfully identifies patterns that are characteristic of language model outputs.
Color-coded highlighting makes it easy to visualize which portions may have been AI-generated.

\section{Conclusion}

This tool provides a useful heuristic for detecting AI-generated text in academic documents.
However, it should be used as one of several indicators rather than definitive proof.

\end{document}
