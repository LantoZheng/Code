\documentclass[11pt, a4paper]{article}

% --- UNIVERSAL PREAMBLE BLOCK ---
\usepackage[a4paper, top=2.5cm, bottom=2.5cm, left=2cm, right=2cm]{geometry}
\usepackage{fontspec}

\usepackage[english, bidi=basic, provide=*]{babel}

\babelprovide[import, onchar=ids fonts]{english}

% Set default/Latin font to Sans Serif in the main (rm) slot
\babelfont{rm}{Noto Sans}

% --- PACKAGES ---
\usepackage{amsmath} % For advanced math environments
\usepackage{hyperref} % For hyperlinks, should be loaded last

% --- DOCUMENT METADATA ---
\title{Solution Draft: Zee's QFT Problem on Loop Integrals}
\author{A. Zee, Chapter I.7}
\date{\today}

\begin{document}
\maketitle
\thispagestyle{empty}

\textbf{Goal:} Show that for both internal particles in the loop diagram (eq. 23) to be real, the total energy must satisfy $E \ge 2m$.

\section*{Mathematical Derivation}

\begin{enumerate}
    \item \textbf{Internal Particles:} The loop integral has two propagators for internal particles with 4-momenta:
    \begin{itemize}
        \item Particle 1: $k$
        \item Particle 2: $k' = k_1 + k_2 - k$
    \end{itemize}

    \item \textbf{On-Shell Condition:} A particle is real (on-shell) when its propagator pole is hit: $p^2 = m^2$. For both particles to be real:
    \begin{align}
        k^2 &= m^2 \label{eq:cond1} \\
        (k_1 + k_2 - k)^2 &= m^2 \label{eq:cond2}
    \end{align}

    \item \textbf{Center-of-Mass Frame:} In the center-of-mass (CM) frame, the total 4-momentum $P$ is:
    $$P = k_1 + k_2 = (E, \vec{0})$$
    The on-shell conditions become:
    \begin{align}
        k^2 &= m^2 \label{eq:cond3} \\
        (P - k)^2 &= m^2 \label{eq:cond4}
    \end{align}

    \item \textbf{Solve for Energy $E$:} Let $k = (k^0, \vec{k})$. The second particle's momentum is $k' = (E - k^0, -\vec{k})$. Applying the on-shell condition $p^2 = (p^0)^2 - |\vec{p}|^2 = m^2$ to both:
    \begin{itemize}
        \item Particle 1: $(k^0)^2 - |\vec{k}|^2 = m^2 \implies k^0 = \sqrt{|\vec{k}|^2 + m^2}$
        \item Particle 2: $(E - k^0)^2 - |-\vec{k}|^2 = m^2 \implies E - k^0 = \sqrt{|\vec{k}|^2 + m^2}$
    \end{itemize}
    From these two equations, $E = 2k^0$. Substituting for $k^0$:
    $$E = 2\sqrt{|\vec{k}|^2 + m^2}$$

    \item \textbf{Find Minimum Energy Threshold:} To find the minimum energy, we note that spatial momentum squared must be non-negative:
    $$|\vec{k}|^2 \ge 0$$
    The minimum energy therefore occurs when $|\vec{k}|^2 = 0$:
    $$E_{\text{min}} = 2\sqrt{0 + m^2} = 2m$$
    This sets the minimum energy threshold for the on-shell condition to be met.
    \[ \mathbf{E \ge 2m} \]
\end{enumerate}

\section*{Physical Interpretation (Draft)}

The condition $\mathbf{E \ge 2m}$ is the \textbf{threshold energy for particle production}.

The total energy $E$ in the CM frame is available for a quantum fluctuation to create the two intermediate particles.
\begin{itemize}
    \item If $\mathbf{E < 2m}$: There is insufficient energy to create two real particles of mass $m$. The particles must remain \textbf{virtual}.
    \item If $\mathbf{E \ge 2m}$: Sufficient energy exists to create two \textbf{real, physical particles}. The excess energy ($E - 2m$) becomes their kinetic energy.
\end{itemize}

\textbf{Summary:} The loop integral develops a singularity (an imaginary part) when the energy $E$ is high enough to promote the virtual particles in the loop to real, on-shell particles. The poles of propagators correspond to the physical thresholds for particle production.

\end{document}

