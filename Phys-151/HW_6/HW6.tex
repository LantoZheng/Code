\documentclass[12pt]{article}
\usepackage{amsmath}
\usepackage{amssymb}
\usepackage{amsfonts}
\usepackage[margin=1in]{geometry} % Sets 1-inch margins

\title{Quantum Field Theory Homework \\ \large A. Zee, \textit{Quantum Field Theory in a Nutshell}}
\author{Xiaoyang Zheng} % You can change this
\date{\today}

\begin{document}

\maketitle

\section*{Problems}
\begin{enumerate}

    \item[\textbf{I.8.3}] For the complex scalar field discussed in the text calculate $\langle 0 | T[\varphi(x)\varphi^\dagger(0)] | 0 \rangle$.

    \item[\textbf{I.8.4}] Show that $[Q, \varphi(x)] = -\varphi(x)$.

    \item[\textbf{II.2.1}] Use Noether's theorem to derive the conserved current $J^\mu = \bar{\psi}\gamma^\mu\psi$. Calculate $[Q, \psi]$, thus showing that $b$ and $d^\dagger$ must carry the same charge.

    \item[\textbf{III.1.2}] Regard (1) as an analytic function of $K^2$. Show that it has a cut extending from $4m^2$ to infinity. [Hint: If you can't extract this result directly from (1) look at (14). An extensive discussion of this exercise will be given in chapter III.8.]

    \item[\textbf{III.1.3}] Change $\Lambda$ to $e^\epsilon \Lambda$. Show that for $\mathcal{M}$ not to change, to the order indicated $\lambda$ must change by $\delta\lambda = 6\epsilon C\lambda^2 + O(\lambda^3)$, that is,
    $$
    \Lambda \frac{d\lambda}{d\Lambda} = 6C\lambda^2 + O(\lambda^3)
    $$

\end{enumerate}

\end{document}