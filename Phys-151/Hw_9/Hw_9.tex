\documentclass[12pt]{article}
\usepackage{amsmath}
\usepackage{amssymb}
\usepackage{amsthm}
\usepackage{geometry}
\usepackage{slashed}
\geometry{margin=1in}

\title{Zee QFT Chapter Problems for HW 9}
\author{Xiaoyang Zheng}
\date{\today}

\begin{document}

\maketitle

\section*{Problem 1: Noether Current and Charge Commutator}

\subsection*{(a) Derivation of the Conserved Current using Noether's Theorem}

The Dirac Lagrangian density is given by:
\[
\mathcal{L} = \bar{\psi}(i\gamma^\mu\partial_\mu - m)\psi
\]

Consider a global $U(1)$ transformation:
\[
\psi \to e^{i\alpha}\psi, \quad \bar{\psi} \to \bar{\psi}e^{-i\alpha}
\]

For infinitesimal transformations:
\[
\delta\psi = i\alpha\psi, \quad \delta\bar{\psi} = -i\alpha\bar{\psi}
\]

According to Noether's theorem, the conserved current is:
\[
J^\mu = \frac{\partial\mathcal{L}}{\partial(\partial_\mu\psi)}\delta\psi + \frac{\partial\mathcal{L}}{\partial(\partial_\mu\bar{\psi})}\delta\bar{\psi}
\]

Computing the derivatives:
\[
\frac{\partial\mathcal{L}}{\partial(\partial_\mu\psi)} = i\bar{\psi}\gamma^\mu, \quad \frac{\partial\mathcal{L}}{\partial(\partial_\mu\bar{\psi})} = 0
\]

Therefore:
\[
J^\mu = (i\bar{\psi}\gamma^\mu)(i\alpha\psi) + 0 = \alpha\bar{\psi}\gamma^\mu\psi
\]

Setting $\alpha=1$, we obtain the standard form:
\[
J^\mu = \bar{\psi}\gamma^\mu\psi
\]

The conserved charge is:
\[
Q = \int d^3x\, J^0 = \int d^3x\, \psi^\dagger\psi
\]

\subsection*{(b) Calculation of $[Q, \psi]$ and Charge Properties}

Using the equal-time anticommutation relations for the Dirac field:
\[
\{\psi_a(\mathbf{x}), \psi_b^\dagger(\mathbf{y})\} = \delta_{ab}\delta^3(\mathbf{x}-\mathbf{y})
\]
\[
\{\psi_a(\mathbf{x}), \psi_b(\mathbf{y})\} = \{\psi_a^\dagger(\mathbf{x}), \psi_b^\dagger(\mathbf{y})\} = 0
\]

We compute the commutator:
\[
[Q, \psi(\mathbf{y})] = \int d^3x\, [\psi^\dagger(\mathbf{x})\psi(\mathbf{x}), \psi(\mathbf{y})]
\]

Using the identity $[AB, C] = A\{B,C\} - \{A,C\}B$:
\[
[\psi^\dagger(\mathbf{x})\psi(\mathbf{x}), \psi(\mathbf{y})] = \psi^\dagger(\mathbf{x})\{\psi(\mathbf{x}), \psi(\mathbf{y})\} - \{\psi^\dagger(\mathbf{x}), \psi(\mathbf{y})\}\psi(\mathbf{x})
\]
\[
= 0 - \delta^3(\mathbf{x}-\mathbf{y})\psi(\mathbf{x}) = -\delta^3(\mathbf{x}-\mathbf{y})\psi(\mathbf{y})
\]

Thus:
\[
[Q, \psi(\mathbf{y})] = -\psi(\mathbf{y})
\]

This shows that $\psi$ carries charge $-1$. Similarly:
\[
[Q, \psi^\dagger(\mathbf{y})] = \psi^\dagger(\mathbf{y})
\]

This shows that $\psi^\dagger$ carries charge $+1$. Although the specific charge values differ, both fields carry non-zero $U(1)$ charge, demonstrating they must share the same charge property.

\section*{Problem 2: Trouble with Commutation Relations}

\subsection*{Calculation of $[J^0(\mathbf{x},0), J^0(\mathbf{0})]$ with Commutation Relations}

If we incorrectly quantize the Dirac field using commutation relations instead of anticommutation relations:
\[
[\psi_a(\mathbf{x}), \psi_b^\dagger(\mathbf{y})] = \delta_{ab}\delta^3(\mathbf{x}-\mathbf{y})
\]
\[
[\psi_a(\mathbf{x}), \psi_b(\mathbf{y})] = [\psi_a^\dagger(\mathbf{x}), \psi_b^\dagger(\mathbf{y})] = 0
\]

The charge density is:
\[
J^0(\mathbf{x}) = \psi^\dagger(\mathbf{x})\psi(\mathbf{x})
\]

We compute the commutator:
\[
[J^0(\mathbf{x}), J^0(\mathbf{0})] = [\psi^\dagger(\mathbf{x})\psi(\mathbf{x}), \psi^\dagger(\mathbf{0})\psi(\mathbf{0})]
\]

Using the identity $[AB, CD] = A[B,C]D + AC[B,D] + [A,C]DB + C[A,D]B$:
\begin{align*}
[\psi^\dagger(\mathbf{x})\psi(\mathbf{x}), \psi^\dagger(\mathbf{0})\psi(\mathbf{0})] &= \psi^\dagger(\mathbf{x})[\psi(\mathbf{x}), \psi^\dagger(\mathbf{0})]\psi(\mathbf{0}) \\
&\quad + \psi^\dagger(\mathbf{x})\psi^\dagger(\mathbf{0})[\psi(\mathbf{x}), \psi(\mathbf{0})] \\
&\quad + [\psi^\dagger(\mathbf{x}), \psi^\dagger(\mathbf{0})]\psi(\mathbf{0})\psi(\mathbf{x}) \\
&\quad + \psi^\dagger(\mathbf{0})[\psi^\dagger(\mathbf{x}), \psi(\mathbf{0})]\psi(\mathbf{x})
\end{align*}

Using the commutation relations:
\[
[\psi(\mathbf{x}), \psi^\dagger(\mathbf{0})] = \delta^3(\mathbf{x}), \quad
[\psi(\mathbf{x}), \psi(\mathbf{0})] = 0
\]
\[
[\psi^\dagger(\mathbf{x}), \psi^\dagger(\mathbf{0})] = 0, \quad
[\psi^\dagger(\mathbf{x}), \psi(\mathbf{0})] = -\delta^3(\mathbf{x})
\]

Substituting these results:
\begin{align*}
[J^0(\mathbf{x}), J^0(\mathbf{0})] &= \psi^\dagger(\mathbf{x})\delta^3(\mathbf{x})\psi(\mathbf{0}) + 0 + 0 + \psi^\dagger(\mathbf{0})(-\delta^3(\mathbf{x}))\psi(\mathbf{x}) \\
&= \delta^3(\mathbf{x})[\psi^\dagger(\mathbf{x})\psi(\mathbf{0}) - \psi^\dagger(\mathbf{0})\psi(\mathbf{x})]
\end{align*}

This result does not vanish for $\mathbf{x} \neq 0$, violating causality. In contrast, with the correct anticommutation relations, this commutator vanishes for spacelike separations, ensuring causal behavior of the theory.

This demonstrates why Dirac fields must be quantized using anticommutation relations (fermionic statistics) rather than commutation relations, to maintain causality and consistency of the quantum field theory.

\end{document}