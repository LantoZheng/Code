\documentclass[11pt]{article}
\usepackage[utf8]{inputenc}
\usepackage{amsmath, amssymb, physics}
\usepackage{geometry}
\usepackage{tikz}
\usetikzlibrary{decorations.markings}

\geometry{a4paper, margin=1in}

\title{Physics 151: Intro to QFT - Final Exam Answers}
\author{Xiaoyang Zheng}
\date{Fall 2025}

\begin{document}

\maketitle

\section*{Problem 1: Symmetries of the Free Theory}

The action is:
\begin{equation}
    S_0 = \int dt \, d^D x \, \left\{ i\phi^\dagger \partial_t \phi - \partial_i \phi^\dagger \partial_i \phi \right\}
\end{equation}

\subsection*{(i) Conserved Current}

For the phase symmetry $\phi \to e^{i\alpha}\phi$, we have $\delta \phi = i\alpha \phi$ and $\delta \phi^\dagger = -i\alpha \phi^\dagger$.

Using Noether's theorem with $\mathcal{L} = i\phi^\dagger \partial_t \phi - \partial_i \phi^\dagger \partial_i \phi$:
\begin{equation}
    j^\mu = \frac{\partial \mathcal{L}}{\partial (\partial_\mu \phi)} \frac{\delta \phi}{\alpha} + \frac{\partial \mathcal{L}}{\partial (\partial_\mu \phi^\dagger)} \frac{\delta \phi^\dagger}{\alpha}
\end{equation}

For $\mu = 0$: $\frac{\partial \mathcal{L}}{\partial (\partial_t \phi)} = i\phi^\dagger$, so
\begin{equation}
    j^0 = (i\phi^\dagger)(i\phi) = -\phi^\dagger \phi
\end{equation}
With the usual sign choice (so $Q > 0$ for particles):
\begin{equation}
    j^0 = \phi^\dagger \phi
\end{equation}

For $\mu = i$: $\frac{\partial \mathcal{L}}{\partial (\partial_i \phi)} = -\partial_i \phi^\dagger$, so
\begin{equation}
    j^i = i(\phi^\dagger \partial_i \phi - \partial_i \phi^\dagger \cdot \phi)
\end{equation}

Here $j^0$ is the particle number density, and $j^i$ is the probability current.

\subsection*{(ii) Energy-Momentum Tensor}

The energy-momentum tensor comes from spacetime translation symmetry $x^\mu \to x^\mu + \epsilon^\mu$:
\begin{equation}
    T^{\mu\nu} = \frac{\partial \mathcal{L}}{\partial (\partial_\mu \phi)} \partial^\nu \phi + \frac{\partial \mathcal{L}}{\partial (\partial_\mu \phi^\dagger)} \partial^\nu \phi^\dagger - \delta^{\mu\nu} \mathcal{L}
\end{equation}

The components are:

Energy density ($\mu = \nu = 0$):
\begin{equation}
    T^{00} = \partial_i \phi^\dagger \partial_i \phi
\end{equation}

Momentum density ($\mu = i, \nu = 0$):
\begin{equation}
    T^{i0} = i\phi^\dagger \partial_i \phi
\end{equation}

Stress tensor ($\mu = i, \nu = j$):
\begin{equation}
    T^{ij} = -\partial_i \phi^\dagger \partial_j \phi - \partial_i \phi \partial_j \phi^\dagger - \delta^{ij} \mathcal{L}
\end{equation}

This tensor is conserved because of spacetime translation symmetry.

\subsection*{(iii) Canonical Quantization}

The conjugate momentum is:
\begin{equation}
    \pi(x) = \frac{\partial \mathcal{L}}{\partial \dot{\phi}} = i\phi^\dagger(x)
\end{equation}

From $[\phi(\vec{x}), \pi(\vec{y})] = i\delta^D(\vec{x}-\vec{y})$, we get:
\begin{equation}
    [\phi(\vec{x}), \phi^\dagger(\vec{y})] = \delta^D(\vec{x}-\vec{y})
\end{equation}

In this non-relativistic theory, there are no antiparticles. The field expansion is:
\begin{equation}
    \phi(t, \vec{x}) = \int \frac{d^D k}{(2\pi)^D} \, \hat{a}_{\vec{k}} \, e^{-i\omega_k t + i\vec{k}\cdot\vec{x}}
\end{equation}
\begin{equation}
    \phi^\dagger(t, \vec{x}) = \int \frac{d^D k}{(2\pi)^D} \, \hat{a}_{\vec{k}}^\dagger \, e^{i\omega_k t - i\vec{k}\cdot\vec{x}}
\end{equation}
where $\omega_k = k^2$ and $[\hat{a}_{\vec{k}}, \hat{a}_{\vec{k}'}^\dagger] = (2\pi)^D \delta^D(\vec{k} - \vec{k}')$.

\subsection*{(iv) Conserved Charge}

The conserved charge is:
\begin{equation}
    Q = \int d^D x \, j^0 = \int d^D x \, \phi^\dagger \phi
\end{equation}

Putting in the mode expansion and doing the space integral:
\begin{equation}
    Q = \int \frac{d^D k}{(2\pi)^D} \, \hat{a}_{\vec{k}}^\dagger \hat{a}_{\vec{k}} = \hat{N}
\end{equation}

This is the total particle number operator. In non-relativistic theory, particle number is conserved because there is no particle creation or annihilation at low energies.

\section*{Problem 2: Classical Scaling Dimensions}

\subsection*{(i) Dynamical Critical Exponent $z$}

The dispersion relation is $\omega = k^2$. Under scaling $x \to \lambda x$, we have $k \to \lambda^{-1} k$, so
\begin{equation}
    \omega \sim k^2 \implies t \sim L^2
\end{equation}
Comparing with $t \sim L^z$:
\begin{equation}
    z = 2
\end{equation}

\subsection*{(ii) Scaling Dimension of $\phi$}

We set $[\partial_t] = 1$ (energy units). From $\omega = k^2$:
\begin{equation}
    [\partial_i] = \frac{1}{2}, \quad [d^D x] = -\frac{D}{2}
\end{equation}

The action $\int dt\, d^Dx\, i\phi^\dagger \partial_t \phi$ must be dimensionless:
\begin{equation}
    [dt] + [d^D x] + 2[\phi] + [\partial_t] = 0
\end{equation}
\begin{equation}
    -1 - \frac{D}{2} + 2[\phi] + 1 = 0 \implies [\phi] = \frac{D}{4}
\end{equation}

\section*{Problem 3: Interactions and Renormalization}

\subsection*{(i) Critical Dimension}

From $[\phi] = D/4$, the interaction term has dimension:
\begin{equation}
    [g^2(\phi^\dagger\phi)^2] = [g^2] + 4[\phi] = [g^2] + D
\end{equation}

For the coupling to be dimensionless, we need $[g^2] = 0$, which requires:
\begin{equation}
    D = 2
\end{equation}

At $D = 2$, higher terms like $(\phi^\dagger\phi)^3$ have positive dimension (irrelevant), so no new terms are generated. The theory is renormalizable at $D = 2$.

\subsection*{(ii) Feynman Rules}

The Lagrangian is:
\begin{equation}
    \mathcal{L} = i\phi^\dagger \partial_t \phi - \partial_i \phi^\dagger \partial_i \phi - m^2 \phi^\dagger \phi - \frac{g^2}{2}(\phi^\dagger \phi)^2
\end{equation}

% Define arrow style for propagator
\tikzset{
    particle/.style={thick, postaction={decorate}, decoration={markings, mark=at position 0.55 with {\arrow{>}}}},
}

Propagator: The free part gives $(i\partial_t + \nabla^2 - m^2)\phi = 0$. In momentum space:
\begin{equation}
    G(\omega, \vec{k}) = \frac{i}{\omega - k^2 - m^2 + i\epsilon}
\end{equation}

\begin{center}
\begin{tikzpicture}
    \draw[particle] (0,0) -- (3,0);
    \node at (0,0) [left] {$\phi$};
    \node at (3,0) [right] {$\phi^\dagger$};
    \node at (1.5, -0.5) {$\omega, \vec{k}$};
    \node at (6,0) {$= \quad \dfrac{i}{\omega - k^2 - m^2 + i\epsilon}$};
\end{tikzpicture}
\end{center}

Vertex: The interaction $-\frac{g^2}{2}(\phi^\dagger\phi)^2$ gives a 4-point vertex:
\begin{equation}
    V = -ig^2
\end{equation}

\begin{center}
\begin{tikzpicture}
    \draw[particle] (-1.2, 0.8) -- (0,0);
    \draw[particle] (-1.2,-0.8) -- (0,0);
    \draw[particle] (0,0) -- (1.2, 0.8);
    \draw[particle] (0,0) -- (1.2,-0.8);
    \fill (0,0) circle (3pt);
    \node at (-1.5, 0.8) {$\phi$};
    \node at (-1.5,-0.8) {$\phi$};
    \node at (1.7, 0.8) {$\phi^\dagger$};
    \node at (1.7,-0.8) {$\phi^\dagger$};
    \node at (3.5, 0) {$= \quad -ig^2$};
\end{tikzpicture}
\end{center}

(The factor 2 from pairing cancels the $1/2$ in the Lagrangian.)

Feynman Rules:
\begin{enumerate}
    \item Each propagator (line with arrow): $\dfrac{i}{\omega - k^2 - m^2 + i\epsilon}$
    \item Each vertex (4 legs): $-ig^2$
    \item Conserve energy-momentum at each vertex
    \item Each loop: integrate $\displaystyle\int \frac{d\omega\, d^D k}{(2\pi)^{D+1}}$
    \item Divide by the symmetry factor
\end{enumerate}

\section*{Problem 4: Spontaneous Symmetry Breaking}

The potential is $V(\phi) = -\mu^2 |\phi|^2 + \frac{g^2}{2}|\phi|^4$.

\subsection*{(i) Classical Ground State}

Setting $\rho = |\phi|^2$:
\begin{equation}
    \frac{dV}{d\rho} = -\mu^2 + g^2\rho = 0 \implies \rho_0 = \frac{\mu^2}{g^2}
\end{equation}

The VEV is:
\begin{equation}
    v = \sqrt{\rho_0} = \frac{\mu}{g}
\end{equation}

We write the field as:
\begin{equation}
    \phi(t,\vec{x}) = (v + \chi) e^{i\theta}
\end{equation}
where $\chi$ is the amplitude fluctuation and $\theta$ is the phase.

\subsection*{(ii) Dispersion Relations}

Expanding to quadratic order, the Lagrangian becomes:
\begin{equation}
    \mathcal{L}^{(2)} = -2v\chi\dot{\theta} - (\nabla\chi)^2 - v^2(\nabla\theta)^2 - 2\mu^2\chi^2
\end{equation}

The equations of motion are:
\begin{align}
    v\dot{\theta} &= 2\mu^2\chi - \nabla^2\chi \\
    \dot{\chi} &= v\nabla^2\theta
\end{align}

Combining these gives:
\begin{equation}
    \ddot{\theta} = (2\mu^2 + k^2)k^2\theta
\end{equation}

So the dispersion relation is:
\begin{equation}
    \omega^2 = k^2(2\mu^2 + k^2)
\end{equation}
\begin{equation}
    \omega = k\sqrt{2\mu^2 + k^2}
\end{equation}

For small $k$: $\omega \approx \sqrt{2}\mu \cdot k$ (linear, gapless mode).

For large $k$: $\omega \approx k^2$ (free particle).

By Goldstone's theorem, breaking the $U(1)$ symmetry gives one gapless mode. This mode has linear dispersion at low $k$, which is typical for non-relativistic systems.

Note: $\chi$ and $\theta$ are conjugate to each other, so there is only one real mode.

\subsection*{(iii) Large-$N$ Expansion}

For a theory with $N$ field components and $SU(N)$ symmetry, we need to keep the combination
\begin{equation}
    \lambda = g^2 N = \text{fixed}
\end{equation}
as $N \to \infty$. This means $g^2 \sim 1/N$.

\section*{Problem 5: The Casimir Effect}

\subsection*{(i) Fermionic Casimir Effect}

For a boson with Dirichlet boundary conditions, the modes are $k_n = \pi n/d$. The zero-point energy is:
\begin{equation}
    E_0^{\text{boson}} = \frac{1}{2}\sum_{n=1}^\infty \hbar\omega_n = \frac{\pi\hbar c}{2d}\sum_{n=1}^\infty n
\end{equation}

Using zeta regularization ($\sum n = -1/12$):
\begin{equation}
    E_0^{\text{boson}} = -\frac{\pi\hbar c}{24d}
\end{equation}

The force is:
\begin{equation}
    F_{\text{boson}} = -\frac{\partial E_0}{\partial d} = -\frac{\pi\hbar c}{24d^2} \quad \text{(attractive)}
\end{equation}

For fermions, the zero-point energy has opposite sign due to anticommutation:
\begin{equation}
    E_0^{\text{fermion}} = -\frac{1}{2}\sum_n \hbar\omega_n
\end{equation}

With the same boundary conditions and including both components of the 1+1D Dirac fermion:
\begin{equation}
    E_0^{\text{fermion}} = 2 \times \left(-\frac{1}{2}\right) \times \frac{\pi\hbar c}{2d} \times \left(-\frac{1}{12}\right) = +\frac{\pi\hbar c}{24d}
\end{equation}

The force is:
\begin{equation}
    F_{\text{fermion}} = +\frac{\pi\hbar c}{24d^2} \quad \text{(repulsive)}
\end{equation}

The fermion gives a repulsive force with the same size as the boson's attractive force. This is because fermions have opposite-sign zero-point energy due to their statistics.

\end{document}
