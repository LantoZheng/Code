\documentclass[12pt]{article}
\usepackage{amsmath}
\usepackage{amssymb}
\usepackage{amsthm}
\usepackage{geometry}
\geometry{margin=1in}

\title{Zee QFT Chapter VI.8 Problems 1 and 2}
\author{Xiaoyang Zheng}
\date{\today}

\begin{document}

\maketitle

\section*{Problem VI.8.1}

\textbf{Problem:} Show that the solution of $dg/dt = -bg^3 + \cdots$ is given by
$$\frac{1}{\alpha(t)} = \frac{1}{\alpha(0)} + 8\pi b t + \cdots$$
where we defined $\alpha(t) = g(t)^2/4\pi$.

\vspace{1em}

\textbf{Solution:}

Given $\alpha(t) = g^2/4\pi$, differentiate with respect to $t$:
\begin{equation}
\frac{d\alpha}{dt} = \frac{g}{2\pi}\frac{dg}{dt} = \frac{g}{2\pi}(-bg^3) = -\frac{bg^4}{2\pi}
\end{equation}

Since $g^4 = (4\pi\alpha)^2 = 16\pi^2\alpha^2$:
\begin{equation}
\frac{d\alpha}{dt} = -8\pi b\alpha^2
\end{equation}

Separating variables and integrating:
\begin{equation}
\int_{\alpha(0)}^{\alpha(t)} \frac{d\alpha'}{\alpha'^2} = -8\pi b \int_0^t dt' \quad \Rightarrow \quad -\frac{1}{\alpha(t)} + \frac{1}{\alpha(0)} = -8\pi b t
\end{equation}

Therefore:
\begin{equation}
\boxed{\frac{1}{\alpha(t)} = \frac{1}{\alpha(0)} + 8\pi b t + \cdots}
\end{equation}

\textbf{Interpretation:} For $b > 0$, $\alpha$ decreases with $t$ (\textbf{asymptotic freedom}); for $b < 0$, $\alpha$ grows with $t$.

\newpage

\section*{Problem VI.8.2}

\textbf{Problem:} Study the threshold effect when the renormalization scale $\mu$ crosses a particle mass $m$. In the crude approximation, particles with $m < \mu$ contribute to RG flow (set $m=0$) while those with $m > \mu$ decouple (set $m=\infty$). Investigate how particles actually decouple smoothly as $\mu$ decreases below $m$.

\vspace{1em}

\textbf{Solution (QED):}

\subsection*{Beta Function with Threshold Effects}

For QED with $N_f$ fermions of mass $m_i$, the beta function is:
\begin{equation}
\beta(e) = \frac{e^3}{12\pi^2}\sum_{i=1}^{N_f} f(m_i/\mu)
\end{equation}

The threshold function $f(r)$ with $r = m/\mu$ satisfies:
\begin{equation}
f(r) \to \begin{cases}
1 & r \to 0 \quad (\mu \gg m) \\
0 & r \to \infty \quad (\mu \ll m)
\end{cases}
\end{equation}

\subsection*{Explicit Form}

From vacuum polarization calculations:
\begin{equation}
f(r) \approx \begin{cases}
1 - \frac{10}{3}r^2 + \mathcal{O}(r^4) & r \ll 1 \\
\text{smooth drop to 0} & r \sim \frac{1}{2} \\
0 & r > \frac{1}{2}
\end{cases}
\end{equation}

\subsection*{Physical Regimes}

\begin{itemize}
\item \textbf{High energy} ($\mu \gg m$): Fermion effectively massless, full contribution to vacuum polarization, $f \approx 1$.

\item \textbf{Threshold} ($\mu \sim 2m$): Pair production becomes kinematically suppressed, smooth transition.

\item \textbf{Low energy} ($\mu \ll m$): Fermion decouples, virtual pairs cannot be produced, $f \approx 0$ (\textbf{decoupling theorem}).
\end{itemize}

\subsection*{Crude vs. Smooth Approximation}

\textbf{Crude:} $f(r) = \Theta(1/r - 1)$ (step function at $\mu = m$)

\textbf{Reality:} Smooth interpolation with characteristic width $\Delta\mu \sim m$ around threshold.

\subsection*{Running Coupling}

For QED with leptons ($m_e, m_\mu, m_\tau$):
\begin{equation}
\frac{d\alpha}{d\ln\mu} = \frac{\alpha^2}{3\pi}\sum_{\ell=e,\mu,\tau} f(m_\ell/\mu)
\end{equation}

In crude approximation between thresholds:
\begin{align}
\mu < m_e: &\quad \beta = 0 \\
m_e < \mu < m_\mu: &\quad \beta = \frac{\alpha^2}{3\pi} \\
m_\mu < \mu < m_\tau: &\quad \beta = \frac{2\alpha^2}{3\pi} \\
\mu > m_\tau: &\quad \beta = \frac{\alpha^2}{\pi}
\end{align}

\subsection*{QCD Application}

Critical for strong coupling $\alpha_s(\mu)$. The number of active quark flavors changes at mass thresholds:
\begin{equation}
\beta_{\text{QCD}} = -\frac{\alpha_s^2}{2\pi}\left(\frac{33-2N_f}{12}\right)
\end{equation}

$N_f$ changes: 6 flavors at $\mu > m_t$, down to 3 flavors ($u,d,s$) at $\mu \sim 1$ GeV. Each threshold modifies the running.

\subsection*{Matching Condition}

At threshold $\mu = m_i$, continuity requires:
\begin{equation}
\alpha^{(N_f)}(m_i^-) = \alpha^{(N_f-1)}(m_i^+) + \mathcal{O}(\alpha^2)
\end{equation}

Higher-order corrections come from threshold loop diagrams.

\vspace{1em}


\end{document}
