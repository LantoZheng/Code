% MIT License

\documentclass[UTF8]{ctexart}
 
\usepackage{amsmath}
\usepackage{cases}
\usepackage{cite}
\usepackage{xeCJK}
\usepackage{graphicx}
\usepackage[margin=1in]{geometry}
\geometry{a4paper}
\usepackage{fancyhdr}
\pagestyle{fancy}
\fancyhf{}

\title{光纤光谱仪实验报告}
\author{郑晓旸}
\date{\today}
\pagenumbering{arabic}

\begin{document}
%这里是文件的开头
\fancyhead[L]{郑晓旸}
\fancyhead[C]{光纤光谱仪}
\fancyfoot[C]{\thepage}

\maketitle
\tableofcontents
\newpage

\section{实验目的}
    \begin{enumerate}
            \item 了fguanyu解光谱分析的基本原理
            \item 掌握光纤光谱仪的正确使用方法
            \item 了解光谱分析在物理学中的应用.
            \item 正确的
    \end{enumerate} 


\section{实验仪器}
\begin{enumerate}
    \item Ocean2000型光纤光谱仪
    \item 组合气体放电灯
    \item 组合LED灯源
    \item 光纤光源
    \item 待测吸光液体
\end{enumerate}

\section{实验原理}

\subsection{光谱仪工作原理}


\subsection{xxx情况下的边界条件和xx现象}
xxxx时发生xxxx现象。由xxx方程可知,xxx波形为$y^+=f(vt+x)$,xxx波形为$y^-=f(vt-x)$。

\subsection{xx在xxx条件下的xxx现象}
Complete by yourself!


\section{实验过程与数据分析}
\subsection{A.在xx条件下测量xxx}
\subsubsection{$a1. $计算出xx的电阻和电感}
在xx上将xx的两端串联xx和xx相连,将xx的两端串联进xx,分别将xx接在$L_1$,$L_2$,xx的两端测量xx并记录。
\subsubsection{$a2. $Complete by yourself!}
Complete by yourself!
\subsubsection{$a3. $Complete by yourself!}
实验得到的数据如下:

\begin{center}
\begin{tabular}{|c|c|c|c|c|c|}
 \hline
线圈名称 & R'(Ω) & Va(V) & V(V) & Vr'(V) & Vo(V)\\
 \hline
线圈1(空气芯) & 123 & 456 & 789 & 012 & 345\\
 \hline
线圈2(空气芯) & 123 & 456 & 789 & 012 & 345\\
 \hline
线圈3(铝芯) & 123 & 456 & 789 & 012 & 345\\
 \hline
线圈4(铝芯) & 123 & 456 & 789 & 012 & 345\\
 \hline
\end{tabular}
\end{center}

\subsection{展示一下行间公式}
\subsubsection{行间公式}
% 行间公式用 $$ $$ 或者 \[ \] 来框住都可以,但在 LaTeX 中前者会改变行文的默认行间距,因此不推荐采用。
\paragraph{}这是一个不确定度计算。
\[
U_k=tinv(x,y)xs_k=xxx
\]
\subsubsection{相对于行内公式}
这是一个不确定度计算:$U_k=tinv(x,y)xs_k=xxx$


\section{分析与讨论}

\subsection{误差分析}

\subsubsection{实验中的系统误差}
来自xxx的精度影响。

受空间内xx与xx的干扰。

\subsubsection{实验中的偶然误差}
接线时可能有xxx情况,导致xxx。xx上的xx在某情况下有xx的问题存在,经反复调整后得以正常测量。

\subsection{实验后的思考}
可说明自己做本实验的总结、收获和体会,对实验中发现的问题提出自己的建议。

\newpage
%图一般很大,建议换页。
\section{原始数据}
\begin{center}
    Change the picture by yourself
    
    
    \includegraphics{picture/example.png}
\end{center}



\bibliographystyle{plain}
\bibliography{./template}  %bib文件名

\end{document}
