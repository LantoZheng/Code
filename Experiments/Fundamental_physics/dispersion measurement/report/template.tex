%This is a experiment example of ZhengXiaoyang's experiment report template

\documentclass[UTF8]{ctexart}
 
\usepackage{pythonhighlight}
\usepackage{amsmath}
\usepackage{cases}
\usepackage{cite}
\usepackage{xeCJK}
\usepackage{graphicx}
\usepackage[margin=1in]{geometry}
\geometry{a4paper}
\usepackage{fancyhdr}
\pagestyle{fancy}
\fancyhf{}

\graphicspath{{picture/}}


\title{利用分光计测量玻璃的色散曲线实验报告}
\graphicspath{{picture/}}


\title{利用分光计测量玻璃的色散曲线实验报告}
\author{郑晓旸}
\date{\today}
\pagenumbering{arabic}

\begin{document}
%这里是文件的开头
\fancyhead[L]{郑晓旸}
\fancyhead[C]{色散曲线}
\fancyfoot[C]{\thepage}

\maketitle
\tableofcontents
\newpage

\section{实验目的}
    \begin{enumerate}
            \item 掌握分光计的调整方法;
            \item 掌握利用最小偏向法测量三棱镜折射率的方法。
    \end{enumerate} 


\section{实验仪器}
\begin{enumerate}
    \item 分光计
    \item 低压汞灯
    \item 三棱镜
    \item 双面平面镜
    \item 其他辅助器件
\end{enumerate}

\section{实验原理}
\subsection{光的色散}
\subsubsection{介质中的光速}
与真空中光速不同,介质中光速对于不同频率(波长)的光是不同的,这一关系称之为色散关系,一般使用光的与圆频率和波数之间的关系表达,即$k(\omega)$,相对的,可以计算光的群速度和相速度:\\
\begin{align}
    v_p&=\frac{\omega}{k}=\frac{c}{n}\\v_g&=\frac{d\omega}{dk}=v_p+\frac{c\lambda dn}{n^2d\lambda}
\end{align}
当然,对于复色光来说,发生折射时由于相速度的不同引起的折射率不通会使光束中不同频率的分量发生大小不同的折射,使得它们向不同方向传播。
\\
通常的色散关系为$\frac{dn}{d\lambda}<0$,称为正常色散;反之则为反常色散。
\subsubsection{Cauchy公式}
对于光的色散关系,在可见光波段,有经验公式(cauchy方程)描述:\\
\begin{align}
    n(\lambda)=A+\frac{B}{\lambda^2}+\frac{C}{\lambda^4}
\end{align}

\subsection{三棱镜的最小偏向角}
由光学知识,假设一束单色光以$i_1$角入射到AB面上,经棱镜两次折射后,从AC面折射出来,出射角为$i_2$。入射光和出射光之间的夹角$\delta$称为偏向角。当棱镜顶角A一定时,偏向角$\delta$的大小会随入射角的变化而变化。由光学知识,当$i_1'=i_2'$时,$\delta$为最小。这时的偏向角为最小偏向角,记作$\delta_{min}$。\\
由图中看出借助折射关系和上述的最小偏向角条件:
\begin{align}
    \delta=\arcsin(\frac{n}{n_0}\sin(i_1'))+\arcsin(\frac{n}{n_0}\sin(i_1'))-A
\end{align}
\begin{figure}[h]
    \centering
    \includegraphics[width=0.3\textwidth]{tri.png}
    \caption{三棱镜折射示意图}
    \label{fig:tri}
\end{figure}
\\
根据几何关系,$A=i_1'+i_2'$;可以得到最小偏向角为:
\begin{align}
    \delta_{min}=2\arcsin(\frac{n_{\lambda}}{n_0}\sin(A/2))-A
\end{align}
\subsection{三棱镜的顶角}
三棱镜的顶角A在制造过程中可能存在误差,与制造标定的数值存在差异,因此我们采用反射法测定顶角:\\
首先将平行光管对准三棱镜顶角,光路如图所示:
\begin{figure}[h]
    \centering
    \includegraphics[width=0.3\textwidth]{A.png}
    \caption{顶角测量示意图}
    \label{fig:A}
\end{figure}
\\
然后转动望远镜,分别观察到从两表面反射产生的狭缝像,对两游标作适当标记,两次分别记录两个游标游标1和游标2的读数。进而求出载物台转过的角度:\\
\begin{align}
    \Phi=\frac{1}{2}[|\theta_1-\theta_1'|+|\theta_2-\theta_2'|]
\end{align}
又由几何关系, $\Phi$是三棱镜顶角的两倍,故:
\begin{align}
    A=\Phi/2
\end{align}

\section{实验过程和数据分析}
\subsection{调整分光计}
调整分光计,需要达到下列要求:
\begin{enumerate}
    \item 平行光管发出平行光:
    \item 望远调整分光计,最后要达到下列要求:镜对平行光聚焦(即接收平行光)。
    \item 望远镜、平行光管的光轴垂直仪器公共轴
\end{enumerate}
分光计调整的关键是\textbf{调好望远镜},其他的调整可以以望远镜为基准。
\subsection{测量三棱镜顶角}
在这里我们使用原理中提到的反射法进行测量,多次测量取平均值,记录数据如下:
\\
\begin{table}[h]
    \begin{center}
        \begin{tabular}{|c|c|c|c|c|c|}
            \hline
           测量序号 & \(\theta_1\) & \(\theta_1'\) & \(\theta_2\) & \(\theta_2'\) & A \\
            \hline
           1 & \(108^\circ 7'\) & \(288^\circ 7'\) & \(228^\circ 10'\) & \(48^\circ 10'\) & \(60^\circ 1'30''\) \\
            \hline
           2 & \(94^\circ 20'\) & \(274^\circ 19'\) & \(214^\circ 21'\) & \(34^\circ 19'\) & \(60^\circ 0'30''\) \\
            \hline
          3 & \(115^\circ 23'\) & \(295^\circ 21'\) & \(235^\circ 25'\) & \(55^\circ 25'\) & \(60^\circ 3'\) \\
            \hline
        \end{tabular}
        \caption{三棱镜顶角测量数据}
        \label{table:1}
    \end{center}  
\end{table}

角度测量的不确定度为:
\begin{align}
    \delta_{Aa} &=\sqrt{\frac{\Sigma_i (A_i-A)^2}{n(n-1)}}=43'' \\ \delta_{Ab}&=\frac{1'}{\sqrt3}=34'' \\ \delta_A&=\sqrt{\delta_{Aa}^2+\delta_{Ab}^2}=55''\approx 1'
\end{align}
故测得顶角为:
\[A=60^\circ 1' \pm 1' \]
\subsection{测量三棱镜的色散曲线}
使用汞灯作为光源,用上文所述的最小偏向角法测量汞灯各谱线在玻璃中的折射率;\\
汞灯的发光谱线如下:\\
\begin{table}[h]
    \begin{center}
        \begin{tabular}{|c|c|c|c|c|c|c|c|c|}  
            \hline    
            波长(nm)& 404.66 & 435.84 & 496.16 & 546.07 & 576.96 & 578.97 & 623.44 & 690.72 \\
            \hline
            颜色 & 紫 & 蓝紫  & 青 & 绿 & 黄 & 黄 & 红 & 深红 \\
            \hline
            强度 &强 &很强 & 强 &很强 & 强 & 强 & 中 & 弱 \\
            \hline
        \end{tabular}
        \caption{汞灯发光谱线}
    \end{center}
\end{table}
\\
由于汞灯亮度和实验器材所限,肉眼仅能观察到404nm,435nm,496nm,546nm,576nm五条谱线。
\\
以下是它们在三棱镜中的最小偏向角数据:\\
\begin{table}[h]
    \begin{center}
        \begin{tabular}{|c|c|c|c|c|c|}
        \hline
        \(\lambda \ (nm)\)&404.66 & 435.84 & 496.16 & 546.07 & 576.96 \\
        \hline
        \(\theta\)&320°12'&320°30'&320°53'&321°13'&321°20'\\
        \hline
        \(\theta'\)&140°12'&140°30'&140°52'&141°11'&141°18'\\
        \hline
        \(\delta\)&39°48'&39°30'&39°07'&38°45'&38°41'\\
        \hline
        \end{tabular}
        \caption{汞灯谱线最小偏向角数据}
    \end{center}
\end{table}\\
其中,546nm的谱线进行了三次测量,数据记录如下:\\
\begin{table}[h]
    \begin{center}
        \begin{tabular}{|c|c|c|c|}
            \hline
            测量序号 & \(\theta\) & \(\theta'\) & \(\delta_i\)\\
            \hline
            1 & 320°53' & 140°52' & 39°7'30'' \\
            \hline
            2 & 321°14' & 141°13' & 38°46'30'' \\
            \hline
            3 & 321°14' & 141°13' & 38°46'30'' \\
            \hline
        \end{tabular}
        \caption{546nm谱线最小偏向角多次测量数据}
    \end{center}
\end{table}
\subsubsection{计算折射率}
根据公式(4)和(5)计算得到的所有谱线折射率数据如下:\\
\begin{table}[h]
    \begin{center}
        \begin{tabular}{|c|c|c|c|c|c|}
            \hline
            波长(nm)& 404.66 & 435.84 & 496.16 & 546.07 & 576.96 \\
            \hline
            折射率 & 1.5300 & 1.5266 & 1.5223 & 1.5181 & 1.5174 \\
            \hline
        \end{tabular}
        \caption{汞灯谱线折射率数据}
    \end{center}
\end{table}
\\
计算得到的546nm谱线折射率数据如下:\\
\begin{align}
    n_\lambda=\frac{n_0}{\sin(A/2)}\sin(\frac{\delta_{min}+A}{2})=1.5223
\end{align}
计算不确定度:
\begin{align}
    u_\delta&=\sqrt{u_{\delta_a}^2+u_{\delta_b}^2}=\sqrt{\frac{\Sigma_1^3(\delta_i-\bar{\delta})^2}{6}+\frac{1'^2}{3}} \\
    u_{n}&=\sqrt{(\frac{\partial n_{\lambda}}{\partial A} u_A)^2+(\frac{\partial n_{\lambda}}{\partial \delta_{\lambda}^{min}}u_{\delta})^2}
\end{align}
得到最终实验结果:
\begin{align}
    n_{546nm}=1.52\pm 0.018
\end{align}
\\
\subsubsection{绘制色散曲线、拟合Cauchy方程}
根据表绘制色散曲线,并使用拟合方法拟合Cauchy方程,得到拟合结果如下:
\\
\begin{figure}[h]
    \centering
    \includegraphics[width=0.8\textwidth]{fit.png}
    \caption{\(n_\lambda - \lambda\)关系(拟合)}
    \label{fig:fit}
\end{figure}
\\
其中拟合结果如下所示:
\begin{center}
    A=1.504870 (nm), B=4146.749084 (\(nm^2\)), C=1.000000 (\(nm^4\))
\end{center}
拟合所用程序见附录。
\section{分析与讨论}

\subsection{误差分析}

\subsubsection{实验中的系统误差}
\begin{enumerate}
    \item 三棱镜顶角测量中,由于测量仪器的精度限制,可能存在一定的系统误差。
    \item 由于测量时的环境因素,如温度、湿度等,可能会对空气折射率产生一定的影响。
    \item 由于汞灯亮度和实验器材所限,肉眼仅能观察到404nm,435nm,496nm,546nm,576nm五条谱线,无法为cauchy方程拟合提供足够多数据点。
    \item 受汞灯温度和稳定性影响,波长可能存在一定的波动。
\end{enumerate}



\subsubsection{实验中的偶然误差}
\begin{enumerate}
    \item 分光计调整中,由于仪器的精度限制,可能存在一定的偶然误差。
    \item 使用最小偏向角法测量时,肉眼难以准确观察到最小偏向角,可能存在一定的偶然误差。
\end{enumerate}

\subsection{复习思考题}
\begin{enumerate}
    \item 总结分光计的调整步骤(每一步的目的、方法和达到要求的判据)。
    \begin{enumerate}
        \item 调节望远镜
        \begin{itemize}
            \item 目的:使望远镜聚焦在无穷远处,光轴与分光计的中心轴垂直,分划板上竖线与中心轴平行。
            \item 方法:
            \begin{enumerate}
                \item 目镜调焦:转动目镜调焦手轮,使分划板刻线成像清晰。
                \item 望远镜调焦(自准直法):放置双面反射镜,调节望远镜调焦手轮,使反射回来的亮十字像与叉丝无视差。
                \item 望远镜俯仰调节:采用1/2渐进调节法,使平面镜两面反射的绿十字像均与分划板上部十字叉丝重合。
                \item 望远镜分划线方向调节:轻微转动载物台,调节目镜筒角度,使反射十字像保持在横向分划线上运动。
            \end{enumerate}
            \item 判据:平面镜反射回来的绿十字像与分划板上的双十字叉丝的上十字叉丝重合,且载物台旋转180°时,反射绿十字像的高度不变,仍和上十字叉丝重合。
        \end{itemize}
        \item 调节载物台
        \begin{itemize}
            \item 目的:使载物台与分光计中心轴垂直。
            \item 方法:
            \begin{enumerate}
                \item 粗调:目测并调节载物台的三个水平调节螺钉,使载物台大致与主轴垂直。
                \item 精调:在完成望远镜调节后,轻微转动载物台,观察三棱镜光学面反射十字像。若反射十字像偏离横向分划线,则调节载物台调平螺丝,使反射十字像保持在横向分划线上运动。
            \end{enumerate}
            \item 判据:转动载物台时,反射十字像应始终沿着横向分划线运动,不偏离横向分划线。
        \end{itemize}
        \item 调节平行光管
        \begin{itemize}
            \item 目的:平行光管能发出平行光,光轴与分光计的中心轴垂直,狭缝方向与中心轴平行。
            \item 方法:
            \begin{enumerate}
                \item 调节平行光管产生平行光:前后移动狭缝套管,使望远镜中看到清晰的狭缝像,且像与望远镜分划板竖叉丝无视差。
                \item 调整平行光管光轴与狭缝方向:
            \end{enumerate}
            \item 判据:
            \begin{itemize}
                \item 望远镜中能看到清晰的狭缝像,狭缝像与望远镜分划板竖叉丝无视差。
                \item 狭缝像与分划板中心的横向分划线重合,且与竖向分划线平行。
            \end{itemize}
        \end{itemize}
    \end{enumerate}
    \item 在通过最小偏向角法测量折射率时,由于\(n_0\)非常接近1,在计算中是否可以取\(n_0 = 1\)?(从不确定度的角度考虑)\\ 不可以。因为\(\partial n / \partial n_0\)较大,所以\(n_0\)的不确定度会对\(n_{\lambda}\)的不确定度产生较大影响。\\
\end{enumerate}
\newpage

\section{附录}
\begin{python}
    
    import pandas as pd
    import re
    import math
    import matplotlib.pyplot as plt
    from scipy.optimize import curve_fit

    def dms_to_radians(dms_str):
        parts = re.split(r'[°\'\"]', dms_str)
        degrees = int(parts[0])
        minutes = int(parts[1]) if len(parts) > 1 and parts[1] else 0
        seconds = int(parts[2]) if len(parts) > 2 and parts[2] else 0
    
        decimal_degrees = degrees + minutes / 60 + seconds / 3600
        radians = math.radians(decimal_degrees)
        return radians

    df = pd.read_csv('data.csv')
    df['n_lambda'] = n_0 / math.sin(A_rad / 2) * df['delta(rad)'].apply(lambda x: math.sin((x + A_rad) / 2))

    def dispersion_equation(wavelength, A, B, C):
        return A + B / wavelength**2 + C / wavelength**4

    popt, _ = curve_fit(dispersion_equation, df['lambda(nm)'], df['n_lambda'])

    A, B, C = popt

    wavelength_range = pd.Series(range(int(df['lambda(nm)'].min()), int(df['lambda(nm)'].max()) + 1))
    fitted_n = dispersion_equation(wavelength_range, A, B, C)

    plt.figure(figsize=(8, 6))
    plt.plot(wavelength_range, fitted_n, label='Fitted Curve')
    plt.scatter(df['lambda(nm)'], df['n_lambda'], color='red', marker='o', facecolors='none', label='Observed Points')

    for i, row in df.iterrows():
        plt.annotate(f"({row['lambda (nm)']}, {row['n_lambda']:.6f})", (row['lambda (nm)'], row['n_lambda']), textcoords="offset points", xytext=(0, 10), ha='center')

    plt.xlabel('Wavelength lambda (nm)') 
    plt.ylabel('Refractive Index n_lambda')
    plt.title('Refractive Index vs. Wavelength')
    plt.legend()
    plt.grid(True)
    plt.show()

    print(f"Fitted Parameters: A={A:.6f}, B={B:.6f}, C={C:.6f}")
    
\end{python}
\bibliographystyle{plain}
\bibliography{./template}  %bib文件名

\end{document}
