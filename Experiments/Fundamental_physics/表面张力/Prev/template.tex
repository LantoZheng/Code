%This is a experiment example of ZhengXiaoyang's experiment report template

\documentclass[UTF8]{ctexart}
 
\usepackage{amsmath}
\usepackage{cases}
\usepackage{cite}
\usepackage{xeCJK}
\usepackage{graphicx}
\usepackage{SIunits}
\usepackage{caption}
\usepackage{float}
\usepackage{fancyhdr}
\usepackage[margin=1in]{geometry}
\geometry{a4paper}
\pagestyle{fancy}
\fancyhf{}

\graphicspath{{picture/}}


\title{拉脱法测量液体的表面张力系数}
\graphicspath{{picture/}}


\title{拉脱法测量液体的表面张力系数预习报告}
\author{郑晓旸}
\date{\today}
\pagenumbering{arabic}

\begin{document}
%这里是文件的开头
\fancyhead[L]{郑晓旸202111030007}
\fancyhead[C]{粘性系数}
\fancyfoot[C]{\thepage}

\maketitle
\tableofcontents
\newpage


\section{实验目的}

\begin{enumerate}
    \item 掌握拉脱法测量液体表面张力系数的原理和方法。
    \item 学习微力传感器的标定方法。
\end{enumerate}

\section{实验仪器}

\begin{itemize}
    \item 液体表面张力系数测量实验仪
    \item 数字示波器
    \item 试样品(去离子水)
    \item 微力传感器及标定设备
\end{itemize}

\section{实验原理}

液体分子存在短程的相互吸引力。在液体内部,分子所受吸引力来自不同方向,平均值为零。但在液体表面,分子所受吸引力只来自液体内部,导致表面有向内收缩的趋势,宏观上造成表面张力现象。定义表面张力系数为:

\begin{equation}
\sigma = \frac{f}{L}
\end{equation}

其中,\(\sigma\) 的量纲为 \(\text{N/m}\),物理意义为液体增加单位表面积所需的能量。

实验采用拉脱法测量液体的表面张力系数。将金属吊环浸没于液体中并缓慢拉起,记录环上的拉力。在液膜破裂瞬间,拉力突然减小,差值 \(\Delta f\) 为液膜的拉力,即:

\begin{equation}
\Delta f = \sigma \pi (D_1 + D_2)
\end{equation}

式中 \(D_1\)、\(D_2\) 分别为吊环的外径和内径。液体表面张力系数为:

\begin{equation}
\sigma = \frac{\Delta f}{\pi (D_1 + D_2)}
\end{equation}

实际操作中,使用微力传感器测量拉力,输出电压与拉力的关系为线性关系:

\begin{equation}
U_k = a + b f_k
\end{equation}

标定后,液体表面张力系数可通过电压变化 \(\Delta U\) 计算得出:

\begin{equation}
\sigma = \frac{\Delta U}{\pi (D_1 + D_2) b}
\end{equation}

\section{实验过程}

\subsection{准备工作}

\begin{enumerate}
    \item 连接各部件,测量吊环内外直径,清洗玻璃盘和吊环。
    \item 给水箱装置加水,验证水量足够。
\end{enumerate}

\subsection{标定力传感器}

\begin{enumerate}
    \item 将吊环挂在力传感器钩上,转至容器外部,减少晃动后传感器输出电压逐渐平稳。
    \item 用镊子安放砝码对传感器进行定标,记录数据并作直线拟合,得到传感器的灵敏度 \(b\)。
\end{enumerate}

\subsection{测量表面张力系数}

\begin{enumerate}
    \item 将待测液体倒入玻璃盘中,小心放入塑料容盘中,并一起放入水箱上室。
    \item 将力传感器转至容器内,挂上吊环,轻触吊环减小晃动。
    \item 关闭阀门,反复挤压气囊使上室内水面上升,当吊环下沿均与待测液体接触时,松开阀门,使水面缓慢下降。
    \item 观察吊环从液体中拉起的物理过程,示波器观察传感器输出的变化趋势。
    \item 在液膜破裂,传感器输出发生突变后,按下示波器“STOP”按钮,测量突变前后的电压值 \(U_1, U_2\),计算电压差 \(\Delta U\),根据标定系数换算拉力。
\end{enumerate}

\subsection{重复测量}

\begin{enumerate}
    \item 为提高测量结果准确度,至少测量3次,估算结果的不确定度。
    \item 验证力传感器的稳定性,实验结束前再测量一次传感器的灵敏度。
\end{enumerate}

\section{注意事项}

\begin{enumerate}
    \item 实验前吊环需严格处理干净。
    \item 仪器开机预热5分钟。
    \item 手指不要接触被测液体。
    \item 力敏传感器使用时用力不宜大于0.1N。
    \item 液体上升有一定惯性,打气速度不可过快,以免产生测量误差。
\end{enumerate}

\section{预习思考题}

\subsection{举出生活中一些由表面张力引起的物理现象}

\begin{itemize}
    \item 水珠在荷叶上形成滚动的小球。
    \item 昆虫(如水黾)在水面上行走。
    \item 肥皂泡的形成和保持形状。
    \item 毛细现象,如毛细管中液体的上升或下降。
\end{itemize}

\subsection{查阅资料,说明一种有别于拉脱法的测量表面张力系数的方法的原理}

另一种测量表面张力的方法是毛细管上升法。该方法基于毛细现象,将细管插入液体中,液体在管中上升,直到液体的表面张力与重力达到平衡。通过测量液柱的高度 \(h\) 和毛细管的半径 \(r\),表面张力 \(\sigma\) 可通过以下公式计算:

\begin{equation}
\sigma = \frac{h \rho g r}{2}
\end{equation}

其中,\(\rho\) 是液体密度,\(g\) 是重力加速度。

\section{拓展问题}

在液膜拉断之前,如果测出吊环在不同上升高度的变化曲线,可以用来计算液体的表面张力系数。具体来说,吊环在不同高度时的拉力变化可以反映液膜的形变和受力情况,通过对这些数据进行分析,可以间接计算表面张力。

\end{document}
