\documentclass[12pt,a4paper]{article}
\usepackage[UTF8]{ctex}
\usepackage[backend=bibtex]{biblatex}
\usepackage{amsmath,amsthm,amssymb,graphicx,multirow,float,caption}
\usepackage{geometry}
\geometry{left=2.54cm, right=2.54cm, top=3.18cm, bottom=3.18cm}
\usepackage{enumitem}
\usepackage{subcaption,booktabs,diagbox}
\setenumerate[1]{itemsep=0pt,partopsep=0pt,parsep=\parskip,topsep=5pt}
\setitemize[1]{itemsep=0pt,partopsep=0pt,parsep=\parskip,topsep=5pt}
\setdescription{itemsep=0pt,partopsep=0pt,parsep=\parskip,topsep=5pt}
\usepackage{adjustbox}
\usepackage[graphicx]{realboxes}
\usepackage{rotating}

\usepackage{titlesec}

\newcommand{\be}[1]{
    \begin{equation}
        #1
    \end{equation}
}

\newcommand{\bfig}[3]{
    \begin{figure}[H]
        \centering
        \includegraphics[width=#1\textwidth]{#2}
        \caption{#3}
    \end{figure}
}

\titleformat{\section}%设置section的样式
{\raggedright\large\bfseries}%右对齐,4号字,加粗
{\thesection .\quad}%标号后面有个点
{0pt}%sep label和title之间的水平距离
{}%标题前没有内容

\title{\vspace{-4cm}\Large 热光关联成像}  %文章标题
\author{\kaishu 学号:202111999064 \hspace{2cm} 姓名:郑力恒}   %作者的名称
\date{2024年3月14日}

\begin{document}
\maketitle

\begin{abstract}
    本次实验的主要了解热光关联成像的原理, 并掌握测量计算关联函数的技
术, 并且据此测量了光斑的关联时间, 和利用关联成像原理进行了双孔一维成
像, 并验证了热光关联成像公式。

实验采用 HBT 光路测量了二阶关联函数的水平位置分布, 并且用高斯函
数拟合了二阶关联函数水平变化趋势; 并同理在同一空间点, 利用高斯函数对
空间拟合, 计算得到了所用热光的关联长度 $c = 0.3105 mm$。
之后以双小孔作为物体, 在放大倍数 $\beta = 1.5$ 的成像条件下, 测量了像平面
一维的二阶关联函数$ g^{(2)}(x_2) $的水平分布, 并且基于数据点, 对测量得到的关
联函数进行了双峰高斯拟合, 得到双孔间距测量值$ d^{\prime} = 2.305mm$, 与真实值
$d = 2.2645mm $相对偏差相对误差$ \alpha = 1.8\%$, 
在实验误差允许的范围内, 验证了
热光联成像公式。
\end{abstract}

\section{引言}
1995 年, Y.Shih 等人利用自发参量下转换产生的纠缠光子对实现了鬼成像。将纠缠的两个光子分别送
到两个分别被称为取样臂和参考臂的光学线性传输系统中。光在取样臂中照亮一个待成像的物体, 产生的
物光由一个桶探测器进行探测, 并在参考臂的探测平面上探测空间各点的光强, 通过合理安排这两个光学
系统, 并在它们的输出平面进行符合测量, 就可以得到物体的像, 这种成像技术被称为鬼成像, 也被称为纠
缠光的符合成像或关联成像。

2002 年, Bennink 等人利用转动的激光模拟热光, 实现了关联成像。随着研究的深入, 人们逐渐认识到
利用经典热光源可以模拟量子纠缠光的部分性质, 实现关联成像。北京师范大学物理系汪凯戈小组从理论
上证明, 经典热光源发出的光束在远场和近场中都存在空间关联, 满足关联成像的要求, 他们还预言用经
典热光源和量子纠缠源来做关联成像会得到相似的结果, 而且从理论上推导出了经典热光关联成像所满
足的成像公式, 为热光关联成像实验的实现做了充分的理论铺垫。

本实验主要目的是了解热光关联成像的原理, 并且采用激光器和毛玻璃模拟单色多模热光, 研究测量
热光的关联时间并验证热光关联成像公式。


\section{原理}
\subsection{实验基本公式}
任意多模热光场可以表示为如下形式:
\be{E(\vec{r}, t)=\sum_{k} E_{k} \exp \left[i\left(\vec{k} \cdot \vec{r}-\omega_{k} t\right)\right]}
其中 $\vec k$ 为模的波矢,$\omega_k$ 为对应的频率。

定义两个时空点光场分布 $E(\vec{r_1}, t_1)$ 和 $E(\vec{r_2}, t_2)$ 的一阶相关函数 $G^{(1)}(\vec{r_1},\vec{r_2}, \tau)$ 和相应的归一化函数
$g^{(1)}(\vec{r_1},\vec{r_2}, \tau)$ 为 (记 $\tau = t_2 − t_1$):
\be{G^{(1)}\left(\overrightarrow{r_{1}}, \overrightarrow{r_{2}}, \tau\right)=\left\langle E^{*}\left(\overrightarrow{r_{1}}, t\right) E\left(\overrightarrow{r_{2}}, t+\tau\right)\right\rangle}
\be{g^{(1)}\left(\overrightarrow{r_{1}}, \overrightarrow{r_{2}}, \tau\right)=\frac{\left\langle E^{*}\left(\overrightarrow{r_{1}}, t\right) E\left(\overrightarrow{r_{2}}, t+\tau\right)\right\rangle}{\sqrt{\left\langle\left|E\left(\overrightarrow{r_{1}}\right)\right|^{2}\right\rangle\left\langle\left|E\left(\overrightarrow{r_{2}}\right)\right|^{2}\right\rangle}}}
定义光场的二阶相关函数$ G^{(2)} $和相应的归一化函数$ g(2) $为:
\be{G^{(2)}\left(\overrightarrow{r_{1}}, \overrightarrow{r_{2}}, \tau\right)=\left\langle E^{*}\left(\overrightarrow{r_{1}}, t_{1}\right) E\left(\overrightarrow{r_{1}}, t_{1}\right) E^{*}\left(\overrightarrow{r_{2}}, t_{2}\right) E\left(\overrightarrow{r_{2}}, t_{2}\right)\right\rangle}
\be{g^{(2)}\left(\overrightarrow{r_{1}}, \overrightarrow{r_{2}}, \tau\right)=\frac{G^{(2)}\left(\overrightarrow{r_{1}}, \overrightarrow{r_{2}}, \tau\right)}{\left\langle\left|E\left(\overrightarrow{r_{1}}, t_{1}\right)\right|^{2}\right\rangle\left\langle\left|E\left(\overrightarrow{r_{2}}, t_{2}\right)\right|^{2}\right\rangle}}
定义 $g^{(2)}$ 取值下降到最大值的 10\% 对应的延迟时间为光源的关联时间 $\tau_0$, 即:
\be{g^{(2)}\left(\tau_{0}\right)=1+0.1\left(g_{\max }^{(2)}-1\right)}
下面公式涉及的物理量定义如下:
\begin{itemize}
\item  $x_{0}$  : 光源输出平面水平坐标
\item  $x_{1}$  : 参考臂探测器平面水平坐标
\item  $x_{2}$  : 取样臂探测器平面水平坐标
\item  $f$  : 透镜焦距
\item  $z_{1}$  : 光源到参考臂探测器的距离
\item  $z_{2}$  : 光源到透镜的距离
\item  $z_{3}$  : 透镜到物体的距离
\end{itemize}

参考臂和取样臂满足成像公式 (热光关联成像的高斯公式):
\be{\frac{1}{z_{2}-z_{1}}+\frac{1}{z_{3}}=\frac{1}{f}}

此时探测器 $D_1$, $D_2$ 的水平光场分布 $E_1(x_1)$, $E_2(x_2)$ 对应的二阶相关函数位置项为:
\be{f\left(x_{1}, x_{2}\right) \propto \sqrt{\frac{\left|z_{2}-z_{1}\right|}{z_{3}}} \exp (i \phi) T\left(x_{2}\right) \delta\left(x_{1}-x_{2} \frac{z_{2}-z_{1}}{z_{3}}\right)}
其中 $\phi$ 为相位项,$T(x_2) $为透明物体的分布函数。
\be{\phi=\arg \left(\sqrt{\frac{z_{3}}{z_{1}-z_{2}}}\right)+k\left(z_{2}-z_{1}+z_{3}-\frac{x_{1}^{2}}{2 z_{1}}+\frac{x_{2}^{2}}{2 z_{1}} \frac{f-z_{2}}{f-z_{3}}\right)}
像相对于物的放大倍数:
\be{\beta=\frac{z_{2}-z_{1}}{z_{3}}}
\subsection{实验装置}
实验装置和光路图如下所示:
\bfig{0.8}{热光成像实验装置图.png}{热光成像实验装置图}
激光由透镜会聚后, 照射在旋转的毛玻璃上, 毛玻璃的散射使其形成赝热光源, 赝热光经过分束镜分
束后, 分别经过取样臂和参考臂照射到探测器 $D_1$ 和$ D_2$ 上, $D_1$ 和 $D_2$ 的水平位置和垂直位置分别可调, 故
可在不同的时空点测量计算 $D_1$ 和 $D_2$ 的关联函数。
\section{实验数据处理与分析}
\subsection{测量热光的关联时间}
\subsubsection{关联函数水平分布测量}
在示波器 $X − Y$ 图收敛至细锐直线处, 表明两个探测器所探测的位置进入关联区域, 如下图所示: 
\bfig{0.6}{x-y图.jpg}{发生关联的标志: $X-Y$图收敛成直线}
固定探测器 $D_2$ 的位置, 水平移动探测器 $D_1$ 水平位置 $x_1$, 每隔
0.1mm 测量一次关联函数, 根据测量结果作出 $x_1 − g^{(2)}$图像如下所示:
\bfig{0.8}{空间关联.png}{关联函数$g^{(2)}$空间分布}
其中散点为测量值, 虚线为采取高斯函数$f(x)=a \exp \left(\frac{-\left(x-b\right)^{2}}{2 c^{2}}\right)+d$拟合出的趋势线, 并且根据拟合
结果得到趋势线的表达式为:
\be{g^{(2)}(x_1)= 0.459 \exp{(-2.5937(x-9.8480).^2)}+0.9734}
拟合优度为$R=0.984643$, 拟合结果得到最大关联函数值和相应的位置为
\be{x_{c}=9.8480\mathrm{~mm} \quad g^{(2)}\left(x_{c}\right)=1.4324}
而实验直接得到的 $g^{(2)}$ 最大值的位置为 $x_0 = 9.7(mm)$ , 对应的二阶相关函数 $g^{(2)}(x_0) = 1.44307$ 与
拟合结果大致相符, 说明拿高斯函数进行拟合是有一定道理的。

拟合得到的方差$c=0.3105 \mathrm{~mm}$, 其含义为空间有效关联的尺寸。
\subsection{验证热光关联成像公式}
实验中采用的物体为两个可通光的小孔, 两个小孔水平放置, 如图所示:
\bfig{0.8}{通光孔.jpg}{通光孔}
两个小孔的宽度不同, 其中心间距理论值 $d = 2.2645mm$ 仔细调整好光路, 物体到成像透镜的距离为
$z_3 = 10cm$, 分光镜到成像透镜的距离为 $z_2 = 25cm$, 分光镜到扫描探测器$ D_1$ 的距离为 $z_1 = 10cm$, 得到
的物体的像放大倍数 $\beta =\frac{z_2 − z_1}{z_3}= 1.5$。

\subsubsection{背景与物体信号对比度预测量}


调节好相应光路后, 改变探测器 $D_1$ 的位置, 可以大致测量得到小孔像和背景的二阶关联函数取值如
下:
\begin{table}[H]
    \centering
    \begin{tabular}{|c|c|c|}
    \hline
    位置     & 水平坐标$x(mm)$ & 关联函数$g^{(2)}$  \\ \hline
    小孔$x_1$   & 10.2  & 1.01345 \\ \hline
    中心$x_c$   & 12.9  & 1.00195 \\ \hline
    小孔$x_2$   & 15.2  & 1.01228 \\ \hline
    边缘位置$x_{\infty}$ & 16.5  & 1.00305 \\ \hline
    \end{tabular}
    \caption{对比度预测量}
    \end{table}


因此对比度为:
\be{\eta=\frac{g^{(2)}\left(x_{2}\right)-1}{g^{(2)}\left(x_{c}\right)-1}=6.297}

\subsubsection{关联函数水平位置测量}
调节探测器 $D_1$ 的位置, 从激光光斑边缘位置, 每隔 $0.1mm$ 进行一次测量, 直至另一侧激光光斑边缘位
置停止。根据实验测量得到的数据点, 作出图像如下图所示:
\bfig{0.8}{空间关联2.png}{ 热光关联成像 $g^{(2)}$ 图}
考虑到散点分布近似于双峰高斯分布, 则选择将关联系数 $g^{(2)}$ 与位置坐标 $x$ 作如下形式高斯拟合,
\be{g^{(2)}(x)=a_{1} \exp \left(-\frac{\left(x-b_{1}\right)^{2}}{2 c_{1}^{2}}\right)+a_{2} \exp \left(-\frac{\left(x-b_{2}\right)^{2}}{2 c_{2}^{2}}\right)+C}
得到具体参数如下所示:
\begin{table}[H]
    \centering
    \begin{tabular}{|c|c|c|c|c|c|c|c|c|}
    \hline
    参数 & $a_1$          & $b_1$          & $c_1$          & $a_2$          & $b_2$         & $c_2 $         &$ C$           & 拟合优度$R$  \\ \hline
       & 0.0134 & 10.5364 & 0.3051 & 0.00997 & 15.021 & 0.4966 & 1.00252 & 0.9418 \\ \hline
    \end{tabular}
    \end{table}

对于拟合系数的结果分析如下:
\begin{itemize}
\item 高斯分布的尺度参数 $c_1 < c_2$ , 表明波包 1 的宽度小于波包 2 的宽度, 大孔的波包较宽, 与实验用两
个大小不一致的小孔相符合。
\item $C = 1.00252 \approx 1$, 当两点之间距离无穷远时, 应该是完全不相干的状态, 即 $$\lim _{x_{1} \rightarrow \infty} g^{(2)}\left(x_{1}, x_{2}\right)=1$$, 与理论
预期大致符合。
\end{itemize}
取两个峰中心间距当作孔中心的间距, 即得到双孔的像间距:
\be{\tilde{d}=b_{2}-b_{1}=4.485(\mathrm{~mm})}
为了换算得到实际的孔间距, 实验结束后再次对$z_1$, $z_2$, $z_3$进行测量, 有$z_1=8.00cm$, $z_2=26.00cm$, $z_3=9.25cm$, 
此时的放大倍率为
\be{\beta^{\prime}=\frac{z_2-z_1}{z_3}=1.946}
这一变化的原因是在设置距离时以光具座中心为准, 而探测器的中心并不在光具座中心的位置。基于该实际放大倍率, 实验测得的空间距应为: 
\be{d^{\prime}=\frac{\tilde{d}}{\beta^{\prime}}=2.305(mm)}
与实际值 $d = 2.2645(mm)$ 对比, 相对误差:
\be{\alpha=\frac{|d-d^{\prime}|}{d} \times 100 \% \approx 1.8 \%}
这一误差出现的主要原因, 是$z_1$, $z_2$, $z_3$没有严格地满足式(7), 但误差也不算特别大。
\section{实验总结}
本次实验的主要了解热光关联成像的原理, 并掌握测量计算关联函数的技
术, 并且据此测量了光斑的关联时间, 和利用关联成像原理进行了双孔一维成
像, 并验证了热光关联成像公式。

实验采用 HBT 光路测量了二阶关联函数的水平位置分布, 并且用高斯函
数拟合了二阶关联函数水平变化趋势; 并同理在同一空间点, 利用高斯函数对
空间拟合, 计算得到了所用热光的关联长度 $c = 0.3105 mm$。
之后以双小孔作为物体, 在放大倍数 $\beta = 1.5$ 的成像条件下, 测量了像平面
一维的二阶关联函数$ g^{(2)}(x_2) $的水平分布, 并且基于数据点, 对测量得到的关
联函数进行了双峰高斯拟合, 得到双孔间距测量值$ d^{\prime} = 2.305mm$, 与真实值
$d = 2.2645mm $相对偏差相对误差$ \alpha = 1.8\%$, 
在实验误差允许的范围内, 验证了
热光联成像公式。
\end{document}