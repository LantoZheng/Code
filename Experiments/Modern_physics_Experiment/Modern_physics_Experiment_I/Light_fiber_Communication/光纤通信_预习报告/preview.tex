\documentclass[12pt,a4paper]{article}
\usepackage[UTF8]{ctex}
\usepackage[backend=biber]{biblatex}
\usepackage{amsmath,amsthm,amssymb,graphicx,multirow,float,caption}
\usepackage{geometry}
\geometry{left=2.54cm, right=2.54cm, top=3.18cm, bottom=3.18cm}
\usepackage{enumitem}
\usepackage{subcaption,booktabs,diagbox}
\setenumerate[1]{itemsep=0pt,partopsep=0pt,parsep=\parskip,topsep=5pt}
\setitemize[1]{itemsep=0pt,partopsep=0pt,parsep=\parskip,topsep=5pt}
\setdescription{itemsep=0pt,partopsep=0pt,parsep=\parskip,topsep=5pt}
\usepackage{adjustbox}
\usepackage[graphicx]{realboxes}
\usepackage{rotating}

\usepackage{titlesec}

\titleformat{\section}%设置section的样式
{\raggedright\large\bfseries}%右对齐,4号字,加粗
{\thesection .\quad}%标号后面有个点
{0pt}%sep label和title之间的水平距离
{}%标题前没有内容

\title{\vspace{-4cm}\Large 光纤通信 预习报告}  %文章标题
\author{\kaishu 郑晓旸 202111030007}   %作者的名称
\date{}

\begin{document}
\maketitle
\section{实验目的}
\begin{enumerate}
\item 了解光纤光学的基础知识
\item 学习测量光纤数值孔径和损耗特性的方法
\item 了解光纤温度传感器的工作原理
\item 了解光纤音频通信的基本原理和系统构成
\end{enumerate}
\section{实验原理}
\subsection{光纤的损耗}
光纤是一种介质波导, 利用光学全反射原理, 将光的能量约束在
光吸收和光散射都非常小的波导界面内. 因此, 光纤通信的一大特征就是损耗小, 
光纤的技术的发展历史也是围绕着降低光纤材料的损耗率进行的. 

光纤的损耗主要来自材料的两种固有损耗: 散射与吸收, 它们有着不同的损耗机制(即, 损耗与波长的函数关系不同).
在本实验中, 方便起见, 不对损耗的机制进行区分, 测量一个总的损耗系数. 这样, 光功率的指数衰减可以简单描述为
$$P(L)=P(0)\exp{(-\alpha(\lambda)L)}$$
损耗作为一种材料属性, 我们关心其单位长度上的损耗; 同时这样一个指数结构, 可以用声学上的分贝类似地描述:
\begin{equation}
\alpha(\lambda)=\frac{A(\lambda)}{L}=\frac{10}{L}\lg{\frac{P(0)}{P(L)}}
\end{equation}

\subsection{光纤的数值孔径}
损耗特性是光纤得以广泛应用的基础, 下面进一步考虑这种光学波导的细节. 

光纤的数值孔径(NA)是表征光纤集光能力的一个重要物理量. NA与端面处光线的最大入射角有关, 当然由于光路的可逆性, 
该角也是出射角. 因此, NA既反映了光纤的入射性质, 又反映了光纤的出射性质. 

之所以集光能力与最大入射角有关, 源自于光纤的内表面全反射原理. 在光纤内部, 光线从光密介质进入光纤介质而发生全反射, 这只有在介质面
入射角足够大时才能发生. 而这个入射角越大, 就意味着端面处的入射角越小, 意味着端面处的入射角有一个上确界$\theta$. 我们将这个上确界角定义
为光纤的数值孔径, 即
\begin{equation}
    NA=\sin{\theta}
\end{equation}

内表面全反射的性质还决定了光纤材料构成上是分层的. 通常, 光纤的内全反射由折射率较高的纤芯和其外侧折射率较低
的包层完成; 包层外的涂敷层与套塑则用于加强光纤的机械强度.
\subsection{光源与光纤的耦合效率}
光纤作为一种光波的传输材料, 与光源是分开的. 因此, 光线从光源进入光纤时, 有一个耦合的过程. 
为了获得最大的耦合效率, 考虑光纤的特征长度, 即端面的半径a. 对于激光, 总是可以选择合适的焦距的透镜, 对其高斯光束的束腰$W_0$进行调节. 
当$$2W_0=2a$$时, 耦合效率最佳. 在此基础上, 耦合的影响就在于入射到端面时的功率$P_0$到光纤中的功率$P$变化. 此时, 耦合效率定义为
$$\gamma=\frac{P}{P_0}$$

\subsection{光纤温度传感器}
通过光纤测量物理量, 关键在于物理量变化对光信号的影响. 温度传感器就是这样一个例子. 通过光纤分束器, 将激光的相干光源分别输入两根长度一致的光纤中, 就能发生双缝干涉. 
此处, 相位差可以用光程写出:
\begin{equation}
    \phi(T)=\frac{2\pi}{\lambda}n(T)L(T)-\phi_0
\end{equation}
其中, $\phi_0$为参考光纤的相位, 通过改变探测光纤的温度, 其折射率和长度都会发生变化, 最终导致相位差的变化. 

\subsection{音频信号通信}
光纤技术的核心是对光信号的无损耗传输. 想要应用光纤技术, 就要将光信号与其他信号相互转换. 
本实验中, 使用LED进行电流到光信号的转换, 使用硅光电二极管将光信号转换为光电流. 在转换的过程中, 需要输入端和输出端都保持光电流与光功率成正比, 
防止光信号失真. 

由于LED的光功率输出曲线在低功率时不是线性的, 需要选择一个合适的偏置电流, 使光信号无畸变.

\section{实验内容}
\subsection{光纤损耗}
根据公式(1), 需要光纤传输过程中距离为L的两个光功率值P(0)和P(L). 由于光源和光纤耦合的存在, 不能直接选取光源处的光功率, 而使用'截断法'.

首先,在稳定的光强输入条件下,测量长度为
L 的整根光纤的输出功率 P2; 然后, 保持耦合条件不变, 在离光纤输入端约 l 处截断光纤, 
测量此短光纤的输出功率 P1. 当$L>>l$时,短光纤损耗可以忽略,故可近似认为 P1 和 P2 是
被截断光纤(长度为 L-l)的输入功率和输出功率. 

为了能实现光纤的截断, 我们使用便于处理的塑料光纤. 对其处理有几个要点:(1)端面需要尽量的垂直与平滑.(2)用铜管固定光纤, 需要注意光路的调节, 光功率计的设置等, 
反复微调使输出功率最大, 以测量较为准确的损耗系数. 

\subsection{数值孔径和耦合系数}
数值孔径类似于一个光锥, 以端面中心为顶点. 由于公式(2), 我们要测量的角度是很小的, 于是我们使用'远场光斑法'. 

由光
纤出射的光照射到观察屏上,测出光纤端面与观察屏之间的距离 h, 以及观察屏上光斑直径
2r 之后,就可以由下式求出光纤的数值孔径为
$$NA=\sin{\theta}=\frac{r}{\sqrt{r^2+h^2}}$$

由于不需要截断, 这里使用的是商用石英光纤. 需要注意的有(1)光纤的长度.(2)用光纤剥线钳处理光纤端面.(3)耦合系数和数值孔径依赖于端面的处理, 需要使用显微镜检查.
(4)光纤输出功率需要调至最大, 以保证远场光斑测量的准确性. 
\subsection{光纤温度传感器}
探测光纤的温度发生微小改变时, 通过折射率与长度的形式发生相位差变化, 
$$\Delta \phi=\frac{2\pi}{\lambda}(L\Delta n + n \Delta L)$$
相位差的变化的直接影响是干涉条纹的移动. 这里需要选择分束-干涉系统, 在显示器上调节出亮度合适且清晰的干涉条纹.

探测光纤需要放入加热器狭缝中. 实验中在30°C-40°C的范围内调节温度, 测量温度系数$\Delta N/\Delta T$.

\section{思考题}
\begin{enumerate}
    \item 请列出光纤能够成为信息传播介质的物理原因.
    
    答: 由于出色的低损耗性能, 光纤传输距离远, 信号稳定性好. 由于材料特性, 成本低, 寿命长, 耐腐蚀. 
    由于以光波作为信息载体, 通信容量大. 
    \item 光纤的数值孔径与光纤的内全反射有什么关系?
    
    答: 内全反射对内入射角有一个下确界限制, 这反映在光纤端面入射角的上确界上. 低于该上确界, 就能实现内全反射. 数值孔径即为此上确界的正弦值. 
    这样, 数值孔径越大, 允许的入射角范围越大, 光纤的集光能力就越强.
    \item 引起光纤损耗的因素有哪些?列出实验中光传输系统中可能引起光信号的衰减的环节。
    
    答: 主要因素是材料的散射与吸收, 即材料中杂质颗粒的散射与材料和杂质带来的吸收, 分为紫外吸收和红外吸收. 此外, 还有辐射损耗, 接续损耗, 弯曲损耗等.
    在实验中, 与光源耦合, 端面未处理平滑与垂直会影响光纤的损耗. 此外光路未调节至水平等, 也会引起测量上的损耗.
    \item 结合 LED 特性曲线,说明如何实现用音频信号线性不失真地调节半导体激光二极管的发
    射光强。
    
    答: LED特性曲线在大小适中的电流处呈现线性. 为了使光信号-电流的转换相一致, 输入的电流要调节到线性区域, 方法是加一个偏置电流, 一般取线性部分中点对应的电流值或LED 最大允许工作电流的一半.
    \item 切割光纤对操作和纤芯截面有什么的技术要求?哪个参数直接反映纤芯断面是否理想?
    
    答: 用专用光纤切割器切割端面,并在显微镜下观察光纤端面,端面需光滑、垂直、干净。
    如果达不到要求则需再次切割; 耦合系数与数值孔径反映了端面是否理想.
\end{enumerate}
\end{document}