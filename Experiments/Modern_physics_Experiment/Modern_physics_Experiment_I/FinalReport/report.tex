\documentclass[12pt, a4paper]{article}
\usepackage[utf8]{inputenc}
\usepackage{amsmath}
\usepackage{ctex}
\usepackage{amssymb}
\usepackage{geometry}
\geometry{a4paper, margin=1in}
\usepackage{graphicx}
\usepackage{float}

\title{近代物理实验总结}
\author{郑晓旸 \texttt{202111030007}}
\date{}

\begin{document}
\maketitle

\section{实验原理}

\section{液晶物性实验}

\subsection{液晶相的划分}
近晶相、向列相和胆甾相。
\begin{itemize}
    \item 近晶相分子排列成层,层内分子平行排列;
    \item 向列相液晶分子平行排列,但分子重心混乱无序;
    \item 胆甾相是向列相的特殊形式,分子排列成层,层内分子取向有序,不同层分子取向稍有变化,沿层的法线方向排列成螺旋结构。
\end{itemize}
本实验采用的液晶是向列相液晶。

\subsection{液晶的基本物理性质}

\subsubsection{2.1 液晶的介电各向异性}
由于各向异性,当外电场平行于分子长轴或垂直于分子长轴时,分子的极化率不同,分别用$\alpha_{//}$和$\alpha_{\perp}$表示。
\begin{itemize}
    \item $\alpha_{//} > \alpha_{\perp}$时,电场使液晶分子的长轴趋于沿电场方向排列
    \item $\alpha_{//} < \alpha_{\perp}$时,电场使液晶分子的长轴趋于垂直电场方向排列
\end{itemize}

\subsubsection{液晶的光学各向异性}
由于各向异性,光在液晶中传播会发生双折射现象,产生寻常光(o光)和非寻常光(e光),据此可分为正光性材料和负光性材料。
\begin{itemize}
    \item 正光性材料:折射率 $n_e>n_o$,传播速度 $v_e<v_o$
    \item 负光性材料:的折射率$n_e<n_o$,传播速度 $v_e>v_o$
\end{itemize}
由于双折射效应,液晶引入的光程差为
\begin{equation}
    \delta = \frac{(n_e-n_o)wd}{c}
\end{equation}

\subsection{液晶盒的结构及旋光性}
\textbf{液晶盒}:材料被封装在两个镀有透明导电薄膜的玻璃基片之间,玻璃的表面经过特殊处理,液晶分子的排列将受表面的影响。若上下两个基片的取向成一定角度,两个基片间液晶分子取向将均匀扭曲。

TN 模式使得光在通过该液晶层时,其偏振面发生的旋转就与波长无关。液晶盒旋光本领在可见光范围内变化较大,可以看到明显的旋光色散。其旋光本领为:
\begin{equation}
    \alpha = \frac{2\pi}{Po} \frac{\Delta \epsilon^2}{8(1+\epsilon^2)}
\end{equation}
其中$Po$是液晶的螺距。

\subsection{液晶的电光效应}

\subsubsection{电光响应曲线}
液晶在外电场作用下,分子取向发生改变,光通过液晶盒时偏振状态也发生变化,若透过液晶盒后检偏器透光位置不变,系统透光强度将发生变化,透过率与外加电压的关系曲线称为电光响应曲线。

常称无电场时的白画面为“常白模式”,相反即为常黑模式。当加电场使电压大于阈值时,液晶分子原有的均匀扭曲结构被破坏,会沿电场方向排列,旋光性被破坏,显示为暗态。

透过率最大值与最小值之比称为对比度(反差):
\begin{equation}
    C = \frac{T_{max}}{T_{min}}
\end{equation}
对比度越高,显示的画面就更加生动亮丽,反之则会显得平淡单调。故对比度的大小直接影响到液晶显示器的显示质量。

定义以下参量:
\begin{itemize}
    \item (1)阈值电压$V_{th}$:透过率为90\%时所对应的电压;
    \item (2)饱和电压$V_s$:透过率为 10\%时所对应的电压;
    \item (3)阈值锐度$\beta$:饱和电压与阈值电压之比,即$\beta = V_s / V_{th}$, $\beta>1$。
\end{itemize}
\subsubsection{电光响应时间}
当施加在液晶上的电压改变时,液晶改变原排列方式所需要的时间。实际上就是液晶由全亮变为全暗、再由全暗变成全亮的反应时间。分别用上升沿时间(上升沿时间$T_{on}$:透过率由最小值升到最大值的 90\%时所需的时间)和下降沿时间(下降沿时间$T_{off}$:透过率由最大值降到最大值的 10\%时所需的时间)来衡量液晶对外界驱动信号的响应速度。

\subsection{4.3 液晶衍射}
施加在液晶盒上的低频电压高于某一阈值时,带电杂质的运动将引起液晶分子的环流,导致整个液晶盒中液晶取向的有规则形变,形成折射率的周期变化,使得通过样品的光聚焦在明暗交替的带上,构成一个衍射光栅,在远场观察液晶的出射光强时会看到衍射图样。衍射环的数目与液晶材料的双折率的关系:
\begin{equation}
    N \approx \frac{\Delta n d}{\lambda}
\end{equation}
液晶位相光栅满足一般的光栅方程$asin\theta = k\lambda$。其中,$a$是光栅常数;$\theta$是衍射角;$k$为衍射级次。

\section{铷原子的光泵磁共振实验}
\subsection{Rb 原子基态及最低激发态能级}

Rb 电子轨道量子数 $L=0$,自旋量子数 $S=1/2$,轨道角动量与自旋角动量耦合后总角动量为$J=L+S,......,|L-S|$,故 Rb 的基态 $J=1/2$,记作 $5^2S_{1/2}$。Rb 原子中,离5s 能级最近的激发态是 $5p$, $L=1$, $S=1/2$,总角动量J有两个:J=1/2 的 $5^2P_{1/2}$态和J=3/2 的$5^2P_{3/2}$态,故为双重态。电子由5p 跃迁到 5s 所产生的光是 Rb 原子主线系的第一条线(双线),$5^2P_{1/2}$ 到$5^2S_{1/2}$ 跃迁产生的谱线称为 D1 线,波长是 794.8nm, $5^2P_{3/2}$ 到 $5^2S_{1/2}$ 跃迁产生的谱线称为 D2 线,波长为 780.0nm。

核自旋 I=0时,原子的价电子经 L-S 耦合后总角动量 $P_J$与原子总磁矩$\mu_J$的大小关系为:
\begin{equation}
\mu_J = -g_J \frac{e}{2m_e} P_J,   g_J = 1 + \frac{J(J+1)+S(S+1)-L(L+1)}{2J(J+1)}
\end{equation}
I$\neq 0$时,原子总角动量还应考虑核的贡献。设核自旋角动量为 $P_I$,核磁矩为 $\mu_I$, $P_I$ 与 $P_J$耦合成 $P_F$,于是有 $P_F=P_I+P_J$,耦合后总量子数F=$I+J,...,|I-J|$。因此, $^{87}$Rb 的基态为$3/2\pm1/2=2$或1, $^{85}$Rb 的基态为$5/2\pm1/2=3$或2。原子总角动量 $P_F$与总磁矩$\mu_F$之间的大小关系为:
\begin{equation}
    \mu_F = -g_F \frac{e}{2m_e} P_F,  g_F = g_J \frac{F(F+1)+J(J+1)-I(I+1)}{2F(F+1)}
\end{equation}
弱磁场条件下,其能量本征值为:
\begin{equation}
    E=E_0+a_h[F(F+1)-J(J+1)-I(I+1)]+g_F\mu_B H_{ext}
\end{equation}
$\mu_B$为玻尔磁子,$a_h$为磁偶极相互作用常数。故基态$5^2S_{1/2}$的两个超精细能级之间的能量差为
\begin{equation}
    \Delta E=a_h[F'(F'+1)-F(F+1)],
\end{equation}
相邻 Zeeman 子能级之间的能量则为
\begin{equation}
    \Delta E_{m_F} = g_F\mu_B B_{ext}
\end{equation}

\subsection{圆偏振光对Rb原子的激发与光抽运效应}
能量守恒要求光子的能量hv与跃迁能级间的能量变化相等;

角动量是矢量,在考虑角动量守恒时还要考虑光的偏振状态。左旋圆偏振光($\sigma^+$)的自旋角动量为$\hbar$,方向指向光的传播方向;右旋圆偏振光($\sigma^-$)的自旋角动量为$-\hbar$,方向与光的传播方向相反。电子吸收左旋圆偏振光后跃迁选择定则为:
\begin{equation}
\Delta F = 0, \pm 1, \Delta m_F = +1.
\end{equation}

若用 Rb 光谱的 D1 线的$\sigma^+$光激发 Rb 原子,由于只允许$\Delta m_F=+1$的跃迁发生,所以处于 $5^2S_{1/2}$的 $m_F=+2$ 子能级上的粒子不能被激发至$5^2P_{1/2}$态; 原子从$5^2P_{1/2}$经历自发辐射和无辐射跃迁回到$5^2S_{1/2}$时,粒子返回基态各个子能级的几率大致相等,若干循环后,基态$m_F=+2$子能级上的粒子数就会大大增加,即光抽运效应。
同样的,右旋圆偏振光将大量的粒子抽运到基态子能级 $m_F =-2$上。

\subsection{弛豫过程}
光抽运个别子能级上的粒子数大大的增加,使系统处于非热平衡状态。光抽运造成 Zeeman 子能级间的粒数差比玻尔兹曼分布造成的粒子数差要大几个数量级。
在Rb 原子系统中主要有:
\begin{itemize}
    \item 1)Rb 原子与容器壁的碰撞: 导致子能级之间的跃迁,失去光抽运所造成的偏极化;
    \item 2)Rb 原子间的碰撞:使粒子的磁矩发生改变而失去偏极化;
    \item 3)外场为零时,Zeeman 子能级简并,使原子回到热平衡分布。
    \item 4)Rb 原子与缓冲气体之间的碰撞。选分子磁性很小的气体(如N$_2$)作为缓冲气体,缓冲气体与 Rb 原子的碰撞对 Rb 的磁能态扰动极小,基本对原子偏极无影响。
\end{itemize}

Rb 原子与器壁碰撞是失去偏极化的主要原因。缓冲气体分子不能完全抑制子能级之间的跃迁,主要作用是使基态由非热平衡分布恢复到热平衡分布的弛豫时间增加(约为 $10^2s$ 数量级)。且处于$5^2P_{1/2}$态的原子需与缓冲气体分子碰撞多次才有可能发生能量转移,由于主要以无辐射跃迁的形式交换能量,所以返回到基态八个 Zeeman 子能级的几率均等,因此缓冲气体分子还有将粒子更快地抽运到m=+2子能级的作用。

若想获得较强的共振信号,Rb 原子蒸汽的最佳温度范围在 40℃-60℃之间,过高和过低均不适宜。

\subsection{Zeeman 子能级之间的磁共振}
在垂直于恒定磁场 $B_0$的方向上加一圆频率为$\omega_1$ 的线偏振射频场 $B_1$,此射频场可分解为左旋圆偏振磁场与右旋圆偏振磁场,当满足共振条件
\begin{equation}
   \hbar\omega_1=\Delta E_{m_F} =g_F\mu_BB_{ext},
\end{equation}
被抽运到基态$m_F=+2$ 子能级上的大量粒子在射频场 $B_1$ 的作用,由 $m_F=+2$ 跃迁到 $m_F=+1$。同时由于光抽运,处于基态非 $m_F=+2$ 子能级上的粒子又被抽运到$m_F=+2$ 子能级上。感应跃迁与光抽运将达到一个新的动态平衡。

\section{He-Ne 激光的纵横模分析和模分裂实验}
\subsection{1.气体激光器}
激光器由增益介质、光学谐振腔和激励能源组成。激光谐振腔有本征频率,每个频率对应一种光场分布。纵模描述轴向光场分布状态,横模描述横向光场分布状态。
气体激光器是空腔,均匀的增益介质充入谐振腔不改变由空腔得到的模式结构。气体增益介质充入空腔后使之成为有源谐振腔,只有那些在谐振腔内往返一次增益大于损耗的光才能建立稳定的振荡,故只包含少量的模式。

\subsection{He-Ne 激光器的纵模、横模及其对应的频率间隔}

\subsubsection{纵模}
激光器包括增益介质和光学谐振腔。He-Ne 激光器谐振腔由二片直径为 2a、间隔为L的介质膜反射镜相对放置组成。反射镜之间混有激光工作物质 He、Ne 混合气体,通常状态下工作物质的粒子数分布为低能级的粒子数多于高能级的粒子数。用放电激励的方法使某个上能级E2的粒子数多于下能级E1的粒子数,称为粒子数反转。该状态时由于自发辐射,产生初始的光在反射镜间不断反射不断通过增益介质放大,形成稳定的激光分布。两列沿轴向相对传播的同频率的光波相干迭加形成驻波,当$2\mu L=q\lambda$时,在腔內形成的驻波场是稳定的。式中$\mu$是增益介质的折射率,L是谐振腔长,$\lambda$是波长,q是整数。由此可得谐振腔允许的激光频率
\begin{equation}
    \nu= q \frac{c}{2\mu L}
\end{equation}

只有频率是$\frac{c}{2\mu L}$的整数倍的光才能形成稳定的光场分布,这种分布称为纵模。

相邻两纵模的频率间隔为:
\begin{equation}
    \Delta \nu =  \frac{c}{2\mu L}
\end{equation}
相邻纵模的频率间隔是相等的。

\subsubsection{横模}
光在谐振腔中来回反射时,若为平行光,因为衍射作用使出射光波的波阵面发生畸变,在横向出现各种不同的场强分布,每种分布形式叫做一种横模。横向分布是二维的,再考虑纵模序数,故用三个符号 $m, n, q$ 标记,即 $TEM_{mnq}$ 模。$TEM_{mnq}$ (m≠0,n≠0)为离轴模,$m$ 表示沿 $x$ 轴场强为零的节点数,$n$是沿 y 轴场强为零的节点数,$q$表示驻波在激光器轴线上的节点数。则纵模的频率间隔为

$\Delta \nu = \nu_{m,n,q+\Delta q} - \nu_{m,n,q} = \frac{c}{2L}\Delta q$,横模的频率间隔为$\Delta \nu = \nu_{m+\Delta m,n,q} - \nu_{m,n,q}$.
旋转对称腔中的模式是旋转对称模,用 $TEM_{plq}$ 标记,$p$ 表示暗环的数目,$l$表示暗直径的数目。若增益介质不均匀或调整不仔细,就不能得到这种光斑。横模的频率间隔与谐振腔的二块反射镜的曲率半径及腔长有关。
共焦腔的横模频率间隔为:
$\Delta \nu = \frac{c}{4l}(\Delta m +\Delta n)$,相邻横模间隔为相邻纵模间隔的一半。

\subsection{氦氖激光器纵模分裂及模竞争}
\subsubsection{石英晶体双折射效应}
石英晶体双折射效应使o光和e光具有光程差$\delta$。不考虑旋光性时,$\delta=(n''-n')h$,
$n'' = (\frac{sin^2\theta}{n_o^2} + \frac{cos^2\theta}{n_e^2} )^{-\frac{1}{2}},n'=n_o$,h是晶片的厚度,$n'$和$n''$是o光和e光的折射率,$n_o$和 $n_e$ 分别是石英晶体的两个主折射率,$\theta$是石英晶体的晶轴和光线之间的夹角。

\subsubsection{腔内双折射效应产生激光频率分裂原理}
波长$\lambda$和激光腔总光程L应满足:
$L= \frac{\lambda}{2}q$,故$\Delta \nu = \frac{c}{2L} \Delta q$,$\Delta \lambda = -\frac{\lambda^2}{L}\Delta L$。当一片双折射元件放入激光谐振腔中,其引入的光程差$\delta$可看成是谐振腔长之差$\Delta L$,故
\begin{equation}
\Delta \nu= -\frac{c}{L^2}\delta
\end{equation}
\subsection{共焦球面扫描干涉仪}

\subsubsection{结构}
由二个曲率半径相等的球面反射镜组成,距离为曲率半径。一面镜子固定不动,另一面镜子固定在压电陶瓷环上。压电陶瓷环加锯齿波电压,腔长L做周期性的变化。波长为$\lambda$的光接近光轴方向入射到干涉仪内,反射光线走一闭合路径,四次反射后与入射光线重合,光程差$\Delta=4L$。一束入射光有二组透射光,分别反射 4m 次和 4m+2次,若相邻两束光光程差满足4L=K$\lambda$,则透射光束产生干涉极大。

\section{光学多道与氢氘同位素光谱}
由于原子能级分立,由高能级向低能级跃迁时会发射一些特定频率的光,在光谱仪上表现为一条条分立的光谱线。对于 H 原子有
\begin{equation}
 \frac{1}{\lambda} = R_H(\frac{1}{2^2} - \frac{1}{n^2}),
\end{equation}
R为H原子的里德堡常数。H的同位素D的巴耳末系的公式类似,H和D的巴耳末系对应谱线的波长差为
\begin{equation}
    \Delta \lambda = \lambda_H - \lambda_D = (\frac{1}{R_H} - \frac{1}{R_D})(\frac{1}{2^2} - \frac{1}{n^2}),n=3,4,5,...
\end{equation}
差别在于里德堡常数不同。
\begin{equation}
R_H = \frac{m_p}{m_p + m_e}R_0 \qquad
R_D = \frac{m_D}{m_D + m_e}R_0
\end{equation}
里德堡常数可分别写为
\begin{equation}
    \Delta \lambda = \lambda_H - \lambda_D =
\end{equation}
$R_0=109737.31cm^{-1}$ 表示原子核质量为无穷大时的里德堡常数。求二同位素原子的里德堡常数之比并代入 H和 D的巴耳末系对应谱线的波长差,得
\begin{equation}
    \Delta\lambda = \frac{m_e}{2m_p+m_e}\lambda
\end{equation}
忽略电子质量得
\begin{equation}
    \Delta\lambda=\frac{m_e}{2m_p}\lambda
\end{equation}
故若能从实验中测出对应谱线的波长$\lambda$和波长差$\Delta \lambda$,即可得出电子和质子的质量比。

\section{光纤的物理性质与应用}
\section{光源与光纤的耦合效率}
为降低耦合损耗,使更多的光功率注入光纤,应考虑具体的耦合方法。
He-Ne 激光器输出的高斯光束经过透镜后仍为高斯光束。调节透镜焦距$f$使经透镜耦合后光束的束腰(光束中最窄的位置)与纤芯直径相等($2W_0=2a$)。应将光纤端面置于光束的焦点处以便获得最佳的耦合效率。耦合效率定义为$\frac{P}{P_0}$,$P_0$是入射到光纤端面的光功率,$P$是经耦合后输入光纤中的光功率。

\section{光纤的数值孔径}
数值孔径表征了光纤集光能力,同时反映了光纤的入射性质和出射性质。数值孔径越大,光纤端面接收或会聚光的能力越强。由几何光学,设$\theta$是入射光线与光纤轴之间的夹角,则这个角度的正弦值就定义为光纤的数值孔径 NA,即 $NA = sin\theta$。
采用“远场光斑法”近似测量光纤的数值孔径,由光纤出射的光照射到观察屏上,测出光纤端面与观察屏之间的距离h 及观察屏上光斑直径2r后得光纤的数值孔径
\begin{equation}
    NA=sin\theta = \frac{r}{\sqrt{r^2+h^2}}
\end{equation}
若光线在纤芯和包层界面的入射角为临界角 $\theta_c=90^\circ-\theta$,则所有与光纤光轴夹角小于$\theta$的光线都能约束在光纤内。

\subsection{光纤的损耗}
光波在光纤中传播主要来自于材料的两种固有损耗:散射与吸收。
\begin{itemize}
    \item 散射:主要由微小颗粒(尺度小于波长)或折射率不均匀导致的散射,损耗与光波长的四次方成反比,之外气泡、较大的杂质颗粒等也会引起米氏散射;
    \item 吸收:可分为紫外吸收与红外吸收。
\end{itemize}
光纤对光波产生的衰减作用称为光纤损耗,影响光纤通信的中继距离。光纤实际传输过程中,随传播距离的增加,光功率以指数形式逐渐衰减,即$P(L)=P(0)e^{-2\alpha(\lambda)L}$,$P(0)$为光纤的输入功率,$P(L)$为光波传输$L$距离后光纤的输出功率,$\alpha(\lambda)$为损耗系数。

损耗以分贝(dB)为单位,采取光纤损耗定义式:
\begin{equation}
    A(\lambda) = 10lg\frac{P(0)}{P(L)} (dB)
\end{equation}
,光纤的损耗系数定义为单位距离上的损耗,即
\begin{equation}
  \alpha(\lambda) = \frac{2}{L}A(\lambda) (dB/Km)
\end{equation}

\subsection{光纤温度传感器}
是一种相位调制型光纤传感器,原理基于光纤双光束干涉相位的变化。激光器发出的相干光经光纤分束器送入两根长度基本相同的光纤中,从两根光纤输出的激光束叠加形成干涉条纹。干涉场的光强$I$ 正比于 $1+cos\phi$,表达式为:
\begin{equation}
   P(T)=n(T)L(T)\frac{2\pi}{\lambda}\Delta L
\end{equation}
$\lambda$为波长,$n$是光纤折射率,$L$是光纤的长度,$T$是温度。外界的溫度作用在探测臂上时,光纤长度和折射率都将发生变化,则相位$\phi$也会发生变化并导致干涉条纹移动。
\section{高温超导实验}
\subsection{零电阻现象}
用液氮冷却水银线并通以几毫安电流,测量其端电压时发现当温度稍低于液氦的正常沸点时,水银线的电阻突然跌落到零,即零电阻现象。
\begin{itemize}
    \item 临界温度:限制其它条件时超导体呈现超导态的最高温度。电阻法测临界温度时,降温过程中电阻温度曲线开始从直线偏离处的温度称为起始转变温度,待测样品电阻从起始转变处下降到一半时对应的温度为临界转变温度,电阻变化10\%到90\%所对应的温度间隔为转变宽度,电阻刚刚完全降到零时的温度为完全转变温度。
    \item 转变宽度的大小反映了材料品质的好坏,均匀单相的样品转变宽度较窄,反之较宽。
\end{itemize}
\subsection{迈斯纳效应}
无论加磁场的次序如何,超导体内磁场感应强度总是等于零。超导体即使在处于外磁场中冷却到超导态,也永远没有内部磁场,它与加磁场的历史无关。

\subsection{临界磁场 $H_c$}
磁场加到超导体上后,部分磁场能量用来建立屏蔽电流的磁场以抵消超导体的内部磁场。磁场达到某定值后,能量上有利于使样品返回正常态,破坏了超导电性。若超导体存在杂质,在不同处有不同的 $H_c$,转变将在一个很宽的范围内完成,把$\rho=\rho_0/2$ 相应的磁场叫临界磁场。
一般的超导体临界磁场随T下降而增加,$H_c$与$T$遵循抛物线近似关系。对于第II类超导体,超导态和正常态之间存在过渡中间态,故存在两个临界磁场 $H_{c1}$ 和 $H_{c2}$.

$H<H_{c1}$ 时有和第1类超导体相同的磁矩; $H>H_{c1}$ 磁场进入超导体,但体系仍有无阻的能力,伴随H增加,超导体中磁场进入增多,超导态比例减少,磁化曲线随H的增加磁矩减小至零,超导体完全恢复到正常态。高温超导体为第II类超导体。

\subsection{临界电流密度 $I_c$/$J_c$}
超导体通电流时,无阻的超流态受电流大小的限制,电流达临界值后超导体将恢复至正常态。大多数金属超导体正常态的恢复是突变的,但某些超导体不是突变,随 $I$ 增加渐变到正常电阻。临界电流与临界磁场强度呈负相关。临界磁场强度随温度升高而减小并在转变温度 $T_c$时降为零。临界电流密度也在较高温度下减小。临界温度,临界电流密度和临界磁场都与物质的内部微观结构有关。必须将其同时置于这3个临界值以下,否则超导态会被破坏。

\subsection{实用超导体——非理想的第II类超导体}
对于第 II 类超导体,外磁场从零开始增加,H<Hc1 时,超导体处在迈斯纳态,故$-M=H$; H>Hc1 时,不存在完全的迈斯纳效应,磁通线要进入到大块超导体中。磁场去掉后,大块物质中还残留了俘获磁通。磁场以磁通量子进入超导体,缺陷阻碍了磁通线的进入,磁通线进入超导体受到阻力,直到磁场继续增加克服阻力后才能进入超导体,故−M−H 曲线上 H>Hc1 还要继续上升;H 从 Hc1 开始下降时,磁通线受到阻力不容易排出,就在非理想第II类超导体中形成了部分磁通。

\subsection{纯金属材料的电阻温度特性}
材料中存在的杂质破坏周期性势场引起电子的散射。电阻率为$\rho=\rho_L(T) + \rho_r$.$\rho_L(T)$表示晶格热振动对电子散射引起的电阻率与温度有关。在高温区,T>$\theta_D$/2 时,$\rho_L(T)$与T成正比;在低温区,T<$\theta_D$/10时,$\rho_L(T)$与T成正比。其中$\theta_D$为德拜温度。$\rho_r$表示杂质和缺陷对电子的散射所引起的电阻率,金属中杂质和缺陷散射不依赖于温度,与杂质和缺陷的密度成正比,称为剩余电阻率。故杂质和缺陷可以改变金属电阻率,不改变 d$\rho$/dT。在液氮正常沸点到室温温度范围内,铂电阻与温度具有良好的线性关系。铂电阻温度计是符合 13.8~ 630.74K 温度范围的国际实用基准温度计。

\subsection{半导体材料的电阻温度特性}
本征半导体的电阻率$p_i$为
\begin{equation}
    p_i = \frac{1}{ne\mu_n+pe\mu_p}
\end{equation}
,$n,p$随温度增高指数上升,迁移率$\mu_n$,$\mu_p$随温度增高而下降较慢,故本征半导体的电阻率$p_i$ 随温度上升而单调下降,具有负的温度系数(即 $d\rho_i/dT<0$)。对于杂质半导体,载流子由杂质电离及本征激发产生,比较复杂。半导体一定温度范围内有负的电阻温度系数,用半导体材料做成的温度计可弥补金属电阻温度计在低温区电阻值和灵敏度降低的缺陷,在相当宽的温度范围内有较好的线性关系和较高的灵敏度。

\section{实验内容}

\subsection{液晶物性实验}
\begin{enumerate}
    \item 观测液晶中的旋光现象和双折射现象
    \begin{itemize}
        \item 调节光路中的起偏器,使入射到液晶表面的光强最大。调节检偏器,测量无液晶时,光的线偏度$L_0$。
        \item 保持起偏器不动旋转检偏器,使系统处于消光状态即光功率计示数最小值。在起偏器与检偏器之间加入液晶盒,再次旋转检偏器至系统处于消光状态,将两次检偏器的角度做差即为实验液晶的扭曲角。旋转检偏器使系统处于消光状态,从0开始旋转液晶盒至180°,选择合适的间隔,每次旋转液晶盒后,都要选择检偏器找到光功率的最大、最小值记录并进一步推算出此时的线偏度。利用线偏度与液晶盒的旋转角度判断液晶的双折射现象。
    \end{itemize}

    \item 测量液晶盒的电光响应曲线
    \begin{itemize}
        \item 在考虑液晶衍射现象的条件下,可选择“常黑模式”或“常白模式”,分别测量升压和降压过程的电光响应曲线,求出阈值电压、饱和电压、阈值锐度,并对结果分析。
    \end{itemize}
    
    \item 测量液晶盒的电光响应时间
    \begin{itemize}
        \item 利用液晶驱动电源为液晶盒提供驱动,选择驱动电源的工作模式为间隙模式,调节驱动频率先观察电光响应的弛豫现象。再调节一个较为合适的频率使间隙频率和驱动频率能达到稳定且曲线平滑,用光标分别测出上升沿、下降沿中最小、最大值,以及通过率为90\%和 10\%时所需的时间,从而得到上升沿与下降沿的时间。
    \end{itemize}
    
    \item 液晶衍射现象观测
     \begin{itemize}
         \item 取下液晶盒,缓慢调节液晶盒上的调制电压(连续状态),观察液晶表面的形态变化。将液晶放入光路中,取下光探测器,换上白屏,用白屏观察衍射情况。缓慢增加调制电压至 12V 左右,观察液晶的衍射现象,记录下衍射条纹出现和消失时对应的调制电压值。同样,缓慢降低调制电压,观察液晶的衍射现象,记录下衍射条纹出现和消失时对应的调制电压值。分析两过程中衍射现象的差异。取下检偏器,估算液晶“光栅”的周期。
     \end{itemize}
\end{enumerate}

\subsection{铷原子的光泵磁共振}
\begin{enumerate}
    \item 加热样品泡和 Rb 灯。
    \item 调节光源、透镜、样品泡、光电池等元件的位置,使打到样品泡上的光为平行光,调节 L2 使光电池受光量最大。
    \item 消除地磁场垂直分量对信号的影响。
    \begin{itemize}
    \item 扫场线圈的输出方式设为方波,调节其振幅使磁场为 0.5-1Gs。加上外磁场的瞬间,基态各 Zeeman 子能级上的粒子数大致相等(接近热平衡状态),因此,这一瞬间有总粒子数的7/8 可吸收D1 的 $\sigma^+$光,对光的吸收最强。随着粒子逐渐被抽运到 mF=+2 子能级上,能够吸收光的粒子数逐渐减少,因而透过样品的光强逐渐增加。当 mF=+2 子能级上的粒子数达到饱和时,透过样品的光强达到最大。方波扫过零并反向时,Zeeman 子能级随之发生简并及再分裂,重新分裂后,各Zeeman 子能级的粒子数又近似相等,对D1 光的吸收又达最大值。这就是光抽运信号。地磁场对光抽运信号有很大影响,特别是地磁场的垂直分量。
    \end{itemize}
    
    \item 观察磁共振信号。
    \begin{itemize}
    \item 采用扫场法测量磁共振信号,即保持射频场的频率不变,通过改变稳恒磁场的大小得到共振信号。首先给样品泡加上射频场 $B_1$,扫场信号选择锯齿波输出,改变水平磁场的大小,测量 $^{87}$Rb 及 $^{85}$Rb 发生共振时磁场的大小,计算其$g_F$值。
    \end{itemize}

    \item 测地磁场大小。
\end{enumerate}

\subsection{He-Ne 激光的纵横模分析和模分裂}
\begin{enumerate}
    \item 分别测量两根氦氖激光管的模谱分布
    \begin{itemize}
    \item 在导轨的两个光具座上安装好激光管和扫描干涉仪,用 JDW-3 型激光电源给激光管供电。
    \item 光路调节。接好线后打开激光电源和扫描干涉仪驱动电源。调整两个光具座使得从扫描干涉仪入口反射回到激光器输出端的光斑大体与激光束同心。仔细调整光路使得在示波器上看到的模谱信号为最大。
    \item 改变偏置电压、锯齿波幅度,观察这些因素对模谱的影响。
    \item 测量激光管的相邻纵模频率间隔和相邻橫模频率间隔,确定扫描干涉仪自由光谱区范围,并测量模谱间隔。
    \item 根据讲义中橫模频率间隔公式结合观测横向光场分布,判断包含哪些横模。
    \item 观察并记录一个自由光谱区的模谱图,并描绘模谱轮廓曲线。
    \end{itemize}
    
    \item 观测氦氖激光器的纵模分裂和模竞争
    \begin{itemize}
    \item 搭建光路,连接仪器。将“选择”置于Ⅱ,“粗调”由0拨至1,调整细调钮,使电流达到5mA。光路调整
    \item 出光带宽观测。改变加在压电陶瓷上的电压,模谱将在示波器上移动并改变幅值。记下谱线左边和右边消失点,二消失点的频率间隔即是出光带宽,并在这左右两个消失点中选测 3-4个点,描出激光管增益曲线的大致轮廓。
    \item 激光偏振态的观测。调整石英晶片晶轴与光束夹角,使纵模谱线产生足够的分裂间距。在激光纵模分裂后,将偏振片置于激光器输出镜和扫描干涉仪之间,旋转偏振片,在示波器上观察两个分裂谱线的幅值变化情况,确定两分裂谱线间的偏振关系,并解释原因。
    \item 放回仪器。
    \end{itemize}
\end{enumerate}


\subsection{光学多道与氢氘同位素光谱}
\begin{enumerate}
    \item 仪器结构
    \begin{itemize}
    \item 光栅光谱仪由光栅色散系统、CCD 接收单元、光电倍增管、电子子信号处理单元、A/D 采集单元和计算机组成。
    \end{itemize}
    
    \item 光栅光谱仪
    \begin{itemize}
    \item 如图是光栅光谱仪的光路图。通过入射狭缝$S_1$的光经平面镜 $M_1$ 反射后,被凹面镜 $M_2$ 反射为平行光投射到光栅G上。由于光栅衍射,不同波长的光被反射到不同的方向上,再经凹面物镜$M_3$ 反射成像在光电倍增管$P$上,还可由可旋转的平面镜$M_4$ 反射到CCD上。
    \end{itemize}
    
    \item CCD 光电探测器
    \begin{itemize}
    \item CCD 将光学图像转换为电学“图像”,故可以“瞬时”记录光信号的空间分布。其灵敏度受光电转换二极管电荷改变量测量极限的限制,受材料无规则热运动所致暗电流形成的表面漏电的影响,还受放大器噪声的影响。OMA 的分辨率主要取决于多色仪,也受 CCD 像元大小的限制。其中,电荷耦合器件简称为CCD; OMA 是一种采用多通道方法检测和显示微弱光谱信号的光电子仪器。
    \end{itemize}

    \item 光电倍增管
    \begin{itemize}
    \item 光电倍增管:将微弱光信号转换成电信号的真空电子器件。可以工作在紫外、可见和近红外区的光谱区。如图所示为端窗型光电倍增管的剖面结构图和一般光电倍增管的电原理图。当光照射到光阴极时,光阴极向真空中激发出光电子。光电子按聚焦极电场进入倍增系统,并通过进一步的二次发射倍增放大。然后把放大后的电子用阳极收集作为信号输出。实际使用中,光电倍增管的工作电压显著影响光信号的探测效果。通常工作电压越大,信号增益越大。但过大的工作电压容易使信号失真,甚至导致光电极烧毁。
    \end{itemize}
    
     \item 实验仪器操作要点
    \begin{itemize}
        \item \textbf{光源使用}
            \begin{itemize}
             \item 光源皆为高压气体放电灯,电压大于 2000V。氢氘灯使用寿命有限,只有测量时才能打开电源,测完及时关闭电源。
             \item 转换光源时,先关闭电源开关,再拔出电极棒并插入所需光源位置后,再打开电源开关。
            \end{itemize}

        \item \textbf{光谱仪开机顺序}
            \begin{itemize}
                \item 检查光电倍增管的高压电源是否为零,若不为零,先把电源置于零,然后打开电源总开关。关机时需先将PMT 电源高压降为零,再关闭总电源。
                \item 打开电脑,将光谱仪左下侧的 CCD 和 PMT 转换开关置于所需位置后点击相应的软件。CCD 的控制软件是:WGD-8A-CCD;PMT 的控制软件是:WDG-8A 光电倍增管。等待软件进行检零和初始位置设置并进入数据采集界面。
            \end{itemize}
            
        \item \textbf{CCD 操作}
            \begin{itemize}
                \item  CCD 共 2048 道,波长采集范围为22nm。中心波长(1024 道对应的波长)的设置范围为300~660nm;
                \item 依次单击“运行”、“实时采集”,使计算机进入光谱采集状态。调节中心波长使之位于待测谱线附近;
                \item 中心波长的调节:按“手动前进”和“手动后退”调整中心波长。在调节中心波长时,每次只能点击一次手动前进或手动后退,一次操作完成后才能进行下一次调节。每次调整波长的步长为1nm~50nm;
                \item 调节氢灯光源的位置,确认能否观察到氢谱线。再换上标准灯(氦灯或氖灯),反复调节 CCD 中心波长的位置,使在同一个摄谱范围内既可观察到待测的氢谱线,也可观察到至少两条标准灯谱线;
                \item 由于机械误差,显示的中心波长有 0~10nm 的误差,在采集到相应谱线后,需自行判断,如果中心波长与实际波长差别较大,应先对光谱仪进行波长修正;
                \item CCD 道址的波长定标:通常常用线性定标,定标时除了待测谱线外还应选择至少3条已知谱线(氦灯和氖灯),其中两条谱线的波长用于确定线性方程中的未知参数,一条用于检验定标的正确性。只要改变了中心波长就要重新定标;
                \item  定标的过程:(a)反复调节中心波长使待测谱线和3条标准谱线均处于 CCD 显示波长范围内,确定最佳中心波长并保持不变。根据光强选曝光时间、累积次数,光强的最大值不能超过4000;
                \item  (b)获得理想的标准光谱谱图后暂停采集。“数据处理”、“手动定标”,用“←”、“→”键将“X”移到该谱线峰尖,回车,见“输入显示波长”框,在白框中输入选定谱线的波长值,然后点击添加下一点,用同样的方法输入第二条谱线的数值。如此,图谱的横坐标变为以波长表示。在“寻峰”中寻出所有的峰并对照检查无误;
                \item  (c)选择对标准谱线寄存器进行“自动寻峰”,确定定标误差(小于0.5Å),若超出范围需重新定标;
                \item   (d)确认定标无误后,测量得到待测谱线的波长值;
                \item  (e)将中心波长定在另一条氢谱线附近,重复上述步骤 2-6 测量其波长。
            \end{itemize}
        \item \textbf{PMT 操作}
            \begin{itemize}
                \item 先在 CCD 模式中调节光源的位置与距离,使光路最佳(也就是可以明显观察到434 双峰)。退出 CCD 进入 PMT 采集状态。PMT 的横轴为波长值,但其波长值存在误差,需消除零误差;
                \item 在屏幕左侧相应的位置输入光谱测量的起始波长(小于 300nm)和终止波长(不大于660nm),选择扫描步长,PMT 高压电源电压调制600V,按“单程”扫描采集光谱;
                \item 根据光强大小和分辨率要求调节高压(通常为 600~1000 V)、采集次数、前后光缝等参数,直到获得理想的光谱。如果谱线不光滑可以光滑处理,可选择扩展功能缩小谱线显示范围;
                \item 采用自动寻峰功能获得谱线波长;
                \item 记录氢、氘的 n=3,4,5,6 四条谱线的波长及其波长差。
            \end{itemize}
        \end{itemize}
    
\end{enumerate}

\subsection{光纤的物理性质与应用实验}
\begin{enumerate}
    \item 实验装置
    \begin{itemize}
    \item 实验使用的仪器及光学元件有:He-Ne 激光器、光功率计、FD-OFT-A 型音频信号光纤传输实验仪、显微物镜、五维调整架、光纤、光纤切割刀、光纤钳。
    \item He-Ne 激光器:激光提供源,通过激光来测量光纤的相关物理性质;
    \item 光功率计:检测光强的仪器;
    \item 显微物镜:观察商用石英光纤的切口是否与轴垂直;
    \item 光纤:实验主体,测量它的相关性质;
    \item 光纤钳:剥离商用石英光纤的三层外表皮。
    \end{itemize}
    
    \item 实验内容
    \begin{itemize}
        \item \textbf{光路调节}
        \begin{itemize}
            \item 利用光学平台和可调小孔光阑确认激光输出是否水平。方法:小孔光阑置于激光的出光孔附近,调节高度使其与光斑等高,孔的大小与激光光斑大小一致,平移小孔光阑远离激光器,观察激光光斑是否刚好能通过光阑。若能,说明光束水平;若不能,则需调节激光器的俯仰,并重复步骤直至光束水平;
             \item 五维调节架放入光学平台,使激光入射到聚焦目镜上,观察光束的反射斑判断激光束是否与其光轴共轴(判断依据是:四个反射光斑重合至一点)。若不共轴,调节目镜的位置、方向和高低直至共轴。打开磁性表座开关固定五维调节架;
             \item 观察激光是否和固定光纤的通光孔共轴,若不共轴,反复调节通光孔的上下、左右位置以及俯仰角度,直至激光与通光孔共轴。
        \end{itemize}

       \item \textbf{塑料光纤的耦合效率和损耗系数测量}
            \begin{itemize}
                 \item 取一段塑料光纤(1.5~3m),用剪刀处理端面,使其尽量垂直与平滑;
                 \item 用铜套固定光纤的一端,光纤露出铜管的长度控制在 2mm 以内,将铜管固定在五维调节架上;
                  \item 观察激光光斑与光纤端面的相对位置,调节五维调节架上的铜管的左右与上下位置至激光光斑打在光纤端面上,光纤的另一端也用铜管固定在光纤架上,并将光纤输出端尽量贴近光功率计感光面反复微调铜管位置、光纤与聚焦目镜的距离,使输出功率最大。计算塑料光纤的耦合效率,损耗系数和数值孔径。
           \end{itemize}
        \item \textbf{商用石英光纤的耦合效率和数值孔径测量}
        \begin{itemize}
            \item 取商用石英光纤,长度应能从五维调节架连接到桌面光纤耦合插头;
            \item 处理光纤端面:用光纤剥线钳的套塑孔剥除 6-10cm 的黄色塑套;用涂覆层孔剥除白色涂覆层,每次剥除长度最好不超过1cm,露出裸光纤 2cm 左右;用包层孔剥除包层,保留包层的长度不超过2mm;
             \item 将处理好的光纤装入螺纹铜套并固定在五维调节架上,调节光纤位置至输出功率最大。
         \end{itemize}
         \item \textbf{光纤温度传感器的温度系数测量}
           \begin{itemize}
             \item 保持石英光纤处于最佳耦合状态,将光纤插头接入分束-干涉系统,选择“一分四”光纤分束器的两个输出端(选择依据为光强足够大),将一支光纤放入桌面上的加热器狭缝中并固定好。加热臂与参考臂两光纤输出端固定在光纤座上,调整两光纤的相对位置及光纤与 CCD 探头的相对位置直至在显示器上观察到清晰的干涉条纹,调节光纤耦合使干涉条纹亮度合适。
             \item 加热光纤,在30℃~40℃ 温度范围测量并计算光纤传感器的温度系数,分析干涉条纹产生及移动的原因。
           \end{itemize}
    \end{itemize}

\end{enumerate}

\subsection{高温超导实验}
\begin{enumerate}
    \item 实验装置
    \begin{itemize}
        \item 装置由以下部分组成:
    \begin{itemize}
        \item 低温温度的获得和控制:低温恒温器和不锈钢杜瓦容器;
        \item 电测量部分:W2 型高温超导材料特性测试装置和型号直流数字电压表;
        \item 高温超导体的磁悬浮演示装置。
      \end{itemize}
     \item \textbf{低温恒温器和不锈钢杜瓦容器}
       \begin{itemize}
       \item 低温恒温器的目的是得到从液氮的正常沸点 77.3K 到室温范围内的任意温度。其核心部件是安装有超导样品和铂电阻温度计、硅二极管温度计、康-铜温差电偶及 25Ω 锰铜加热器线圈的紫铜恒温块。液氮盛在具有真空夹层的不锈钢杜瓦容器中。实验的主要工作是测量超导转变曲线,在液氮正常沸点附近的温度范围内(例如140K 到 77K)标定温度计。控温程序是从高温到低温,液氮的温度为77.3K 装在杜瓦瓶内,简便易行的方法是利用液面以上空间存在的温度梯度来获得所需温度。样品温度及降温速率的控制是靠在测量过程中改变低温恒温器在杜瓦容器內的位置来实现,只要降温速率足够慢,就可认为在每一时刻都达到了温度的动态平衡。
        \end{itemize}
    \end{itemize}
    \item \textbf{电测量原理}
       \begin{itemize}
          \item 测量电流由恒流源提供,可由标准电阻 $R_n$ 上的电压 $U_n$ 的测量得出,即 $I=U_n/R_n$。
            \item 如果测量得到了待测样品上的电压 $U_x$,则待测样品的电阻 $R_x$ 为 $R_x = \frac{U_x}{I} = \frac{U_x}{U_n}R_n$。
          \item 测量引线通常又细又长,其阻值有可能远远超过待测样品(如超导样品)的阻值。因此采用“四引线测量法”(如图)。恒流源通过两根电流引线将测量电流$I$提供给待测样品,而数字电压表则是通过两根电压引线来测量电流$I$在样品上所形成 的电势差 $U$。 由于两根电压引线与样品的接点处在两根电流引线的接点之间,因此排除了电流引线与样品间的接触电阻对测量的影响;由于数字电压表的输入阻抗很高,电压引线的引线电阻以及它们与样品之间的接触电阻对测量的影响可以忽略不计。四引线测量法减小了引线和接触电阻对测量的影响。
        \end{itemize}
     \item 铂电阻和硅二极管测量电路
        \begin{itemize}
            \item 使用两个内置的灵敏度分别为 10μV 和 100μV 的 1/2 位数字电压表,通过转换开关分别测量铂电阻、硅二极管以及相应的标准电阻上的电压,可确定紫铜恒温块的温度。
        \end{itemize}
        \item 实验内容
         \begin{itemize}
            \item \textbf{1.室温测量}
                 \begin{itemize}
                     \item 连接电路
                     \item 室温检测
                  \end{itemize}
            \item  \textbf{2.液氮灌注}
            \item  \textbf{3.低温恒温器降温速率的控制}
            \item  \textbf{4.低温温度计的比对}
               \begin{itemize}
                  \item 当紫铜恒温块的温度开始降低时,观察和测量各种温度计及超导样品电阻随温度的变化, 每隔一定时间测量一次各温度计的测温参量(如:铂电阻温度计的电阻、硅二极管温度计的正向电压、温差电偶的电动势),即进行温度计的比对。
               \end{itemize}
            \item \textbf{5. 超导转变曲线的测量}
               \begin{itemize}
                  \item 当紫铜恒温块的溫度降低到130K附近时,开始测量超导体的电阻以及这时铂电阻温度计所给出的温度,测量点的选取可视电阻变化的快慢而定,在这些测量点,应同时测量各温度计的测温参量,进行低温温度计的比对。
               \end{itemize}
            \item \textbf{6.高温超导体的磁悬浮演示}
            \item \textbf{7.高温超导体的磁悬浮力测量}
               \begin{itemize}
                   \item 通过改变高温超导盘片与磁块之间的距离,定量测量高温超导体磁悬浮力的变化
               \end{itemize}
        \end{itemize}
\end{enumerate}

\section*{学习心得}

\subsection*{实验原理部分}
\begin{enumerate}
    \item 认知了部分在理论物理课内无法了解到的前沿科技(如光镊),及应用科技(如混沌通信);
    \item 深化理解了部分理论物理课内学习的知识,如:
    \begin{itemize}
        \item 光抽运现象和本学期的量子力学中学习的自旋对应;
        \item 液晶的偏振和电动力学中学习的电磁波的传播和性质对应;
        \item 光学多道和氢氘同位素光谱和原子物理中学习的同里德伯公式相对应;
    \end{itemize}
    \item 对部分内容进行了深入探究,如对偏振光自旋角动量的性质进行了深入研究;
\end{enumerate}

\subsection*{实验操作部分}
\begin{enumerate}
    \item 熟悉了很多陌生的实验器材的操作使用及要点,
    \begin{itemize}
        \item 如 CCD 光学多道分析仪的定标方法:采用线性定标,定标时除了待测谱线外还应选择至少3条已知谱线(氦灯和氖灯),其中两条谱线的波长用于确定线性方程中的未知参数,一条用于检验定标的正确性......篇幅限制不再赘述。
    \end{itemize}
    \item 复习回顾了很多已掌握仪器的使用方法及部分未完全掌握的仪器的使用方法
    \begin{itemize}
        \item 如示波器的使用一直是我面临的一大难题,不同种类示波器的按钮不相同,以及将眼花缭乱的功能按钮区分一直是我的弱点,本学期通过多次示波器的使用(液晶物性实验,非线性电路等),使我对示波器的各种功能有了进一步的认知,并能在不询问老师的情况下自行使用需要的功能。
    \end{itemize}
\end{enumerate}

\subsection*{后续发展部分}
\begin{itemize}
    \item 通过这学期的理论课学习及实验课操作,自己找准了自己的优势所在(电学),准备后续往电磁场和电路的研究上再深造发展。
\end{itemize}

\section*{改进与不足}
\begin{enumerate}
    \item 实验预习有时不够到位,老师上课提出的问题无法全部回答,理论知识掌握并不牢固;
    \item 实验操作时不够专注,实验速度较慢,往往落后于其他组员;
    \item 实验数据测量并不够精确,往往造成误差较大,不能合理减小误差;
    \item 前期实验报告攥写不清楚格式和要点,导致实验报告攥写很糟糕,后期有逐渐改正;
    \item 误差分析不到位,往往无法分析出误差来源的要点。
\end{enumerate}
\end{document}