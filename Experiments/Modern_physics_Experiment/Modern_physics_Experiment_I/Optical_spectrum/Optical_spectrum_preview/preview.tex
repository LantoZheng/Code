\documentclass[12pt,a4paper]{article}
\usepackage[UTF8]{ctex}
\usepackage[backend=biber]{biblatex}
\usepackage{amsmath,amsthm,amssymb,graphicx,multirow,float,caption}
\usepackage{geometry}
\geometry{left=2.54cm, right=2.54cm, top=3.18cm, bottom=3.18cm}
\usepackage{enumitem}
\usepackage{subcaption,booktabs,diagbox}
\setenumerate[1]{itemsep=0pt,partopsep=0pt,parsep=\parskip,topsep=5pt}
\setitemize[1]{itemsep=0pt,partopsep=0pt,parsep=\parskip,topsep=5pt}
\setdescription{itemsep=0pt,partopsep=0pt,parsep=\parskip,topsep=5pt}
\usepackage{adjustbox}
\usepackage[graphicx]{realboxes}
\usepackage{rotating}

\usepackage{titlesec}

\titleformat{\section}%设置section的样式
{\raggedright\large\bfseries}%右对齐,4号字,加粗
{\thesection .\quad}%标号后面有个点
{0pt}%sep label和title之间的水平距离
{}%标题前没有内容

\title{\vspace{-4cm}\Large 光纤多道 预习报告}  %文章标题
\author{\kaishu 郑晓旸 202111030007}   %作者的名称
\date{}

\begin{document}
\maketitle
\section{实验目的}
利用光学多道分析仪研究 H 的同位素光谱,了解 H、D 原子谱线特点,学习光学多道分析仪的使用方法及基本的光谱学技术。

\section{实验原理(物理、实验、仪器)}
\subsection{物理理论}
研究类氢原子, 通常会得到下面波长的公式:
$$\frac{1}{\lambda_{X}}=R_{X}\left(\frac{1}{n_{1}^{2}}-\frac{1}{n_{2}^{2}}\right)$$
在可见光区, 通常取$n1=2$, 即巴尔末线系. 

与原子本身特性有关的是$R_{X}$. 它与类氢原子体系的折合质量有关, 具体地, 对于H和D有: 
$$R_{\mathrm{H}}=R_{\infty} \frac{m_{\mathrm{p}}}{m_{\mathrm{p}}+m_{\mathrm{e}}}, \quad R_{\mathrm{D}}=R_{\infty} \frac{2 m_{\mathrm{p}}}{2 m_{\mathrm{p}}+m_{\mathrm{e}}}$$
其中$R_{\infty}=1109737.31cm^{-1}$

由此容易推知, 相同n下, H线和D线的差距是: 
$$\Delta \lambda=\frac{m_{e}}{2 m_{p}+m_{e}}\left[R_{D}\left(\frac{1}{2^{2}}-\frac{1}{n^{2}}\right)\right]^{-1}=\frac{m_{e}}{2 m_{p}+m_{e}} \lambda_{D}\approx \frac{m_e}{2m_p} \lambda_{D}$$

通过对谱线的精密测量, 可以得到质子和电子的质量比. 
\section{实验内容}
实验中测量光谱线主要使用光纤光谱仪. 光栅光谱仪主要由光源、光栅色散系统、光电接收系统三部分组成。本实验中的光电接收系统分为CCD和PMT. 

\subsection{CCD}
CCD 探测器件就是由MOS
制作的光电转换二极管作为感光像元,排成面阵列或线阵列形成的固体成像器件。可以将光学图像转换为
电学“图像”,即电荷量与各成像点照
度大致成正比的电荷包空间分布,因
此,它可以“瞬时”记录光信号的空间
分布。

CCD的实验操作如下: 
(1)本实验采用的 CCD 共有 2048 道,波长采集范围为 22nm。中心波长(1024 道对应的波
长)的设置范围为 300到660nm.

(2)调节中心波长,使之位
于待测谱线附近。

(3)调节 CCD 中心波长的位置,使在同一个摄谱范围内既可观察到待测的氢谱线,也可
观察到至少两条标准灯谱线。

(4)由于机械误差的存在,显示的中心波长并不一定准确. 因此如果中心波长与实际波长差别较大,应先对光谱仪进
行波长定标. 定标的流程简化一下应该是: 先用三条标准谱线进行拟合, 再用拟合的结果重新测量, 直到定标误差小于0.5埃, 最后再测量待测谱线. 

(5)将中心波长定在另一条待测谱线, 重复上述操作.  需要测量H原子和D原子的巴尔末系n=3,4,5的谱线. 

\subsection{PMT}
光电倍增管(PMT)是将微弱光信号转换成电信号的真空电子器件. 当光照射到光阴极时,光阴极向真空中激
发出光电子。这些光电子按聚焦极电场进入倍增系统,并通过进一步的二次发射得到的倍增
放大。然后把放大后的电子用阳极收集作为信号输出。在光电倍增管的实际使用中,光电倍
增管的的工作电压将显著影响光信号的探测效果。通常是工作电压越大,信号增益越大。但
过大的工作电压容易使信号失真. 

PMT的实验操作如下: 
(1)先在 CCD 模式中调节光源的位置与距离,使光路最佳. 想办法消除波长上的误差. 

(2)设置波长光谱的扫描范围, 选择扫描步长和高压电源调制600V. 

(3)不断调节高压, 采集次数, 前后光缝等参数, 直到获得理想的光谱. 

(3)用自动寻峰, 获得谱线波长. 

(5)记录氢、氘的 n=3,4,5,6 四条谱线的波长及其波长差,要求打印一幅 400nm-660nm 范
围谱图和一幅扩展后的、434nm 氢氘光谱线谱图.

\subsection{实验操作注意}
(1)氢氘灯的使用寿命有限,只
有测量时才能打开电源,测量完毕及时关闭电源,避免氢氘灯长时间工作。多组灯在转
换光源时,应先关闭电源开关,再拔出电极棒并插入所需光源位置后,再打开电源开关。

(2)检查光电倍增管的高压电源是否为零,若不为零,必需先把电源置于零,然后打开电
源总开关。关机时需先将 PMT 电源高压降为零,然后才能关闭总电源。
然后打开电脑,将光谱仪左下侧的 CCD 和 PMT 转换开关置于所需位置,然后点击相
应的软件,CCD 的控制软件是: WGD-8A-CCD, PMT 的控制软件是:WDG-8A 光
电倍增管。等待软件进行检零和初始位置设置并进入数据采集界面。
\section{思考题}
\begin{enumerate}
    \item 什么是光谱?如何根据 H、D 光谱计算电子、质子质量比?
    
    答: 光谱(spectrum),是复色光经过色散系统(如棱镜、光栅)分光后,被色散开的单色光按波长(或频率)大小而依次排列的图案,全称为光学频谱. 
    H, D光谱中特征谱线是很接近的. 接近的程度就和电子, 质子的质量比有关了: 
    $\frac{\Delta \lambda}{\lambda_{D}}=\frac{m_e}{2m_p}$
    \item 根据光谱实验总结出来的巴尔末公式如何说明氢原子的能级是分立的?
    
    答: 如果氢原子的能级是连续的, 那么光谱实验将测出连续光谱. 光谱的分离就是能级的分立导致的结果. 

    \item 说明光栅光谱仪的分光原理和主要光路,描述单色仪和多色仪的区别和工作方式。
    
    答: 分光原理: 光谱仪分为准直系统、色散系统和成像系统. 色散系统由光栅承担. 这样, 不同波长的光, 经光栅向不同的方向衍射, 从而分光. 
    如果像平面处装上出射狭缝,经过色散系统得到的单色光可从
    狭缝相继出射,这种仪器通常叫单色仪;如果像平面处有系列狭缝或矩形开口,可同时出射
    多个单色光,这种仪器通常叫多色仪。

    \item 什么是光学多道分析方法?在实验中如何利用光学多道分析方法测量 H、D 谱线?
    
    答:光学多道分析仪简称 OMA,主要由光栅色散系统、电荷耦合器件(简称 CCD)或光导摄像
    管,以及带有专用微处理器组成的数据处理系统三部分组成。具有高灵敏度,高效率,瞬时宽光谱探测范围和时
    间分辨等优点。
    实验中的多道体现在CCD的多道. 实验中使用的CCD有2048道, 每一道覆盖 22nm. CCD定标时要求待测谱线和标准谱线在同一道内, 即22nm的范围内. 
    \item 实验结果的分辨率受哪些具体因素影响?
    
    答: 光栅的衍射极限, 光谱线的线形, 强度, 定标的准确性, 扫描步长, 高压的数值等都会影响分辨率. 这些影响最终的结果都是使得光谱线线形不佳, 不利于判断峰值. 
\end{enumerate}
\end{document}