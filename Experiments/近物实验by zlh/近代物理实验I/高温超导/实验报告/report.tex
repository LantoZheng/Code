\documentclass[12pt,a4paper]{article}
\usepackage[UTF8]{ctex}
\usepackage[backend=bibtex]{biblatex}
\usepackage{amsmath,amsthm,amssymb,graphicx,multirow,float,caption}
\usepackage{geometry}
\geometry{left=2.54cm, right=2.54cm, top=3.18cm, bottom=3.18cm}
\usepackage{enumitem}
\usepackage{subcaption,booktabs,diagbox}
\setenumerate[1]{itemsep=0pt,partopsep=0pt,parsep=\parskip,topsep=5pt}
\setitemize[1]{itemsep=0pt,partopsep=0pt,parsep=\parskip,topsep=5pt}
\setdescription{itemsep=0pt,partopsep=0pt,parsep=\parskip,topsep=5pt}
\usepackage{adjustbox}
\usepackage[graphicx]{realboxes}
\usepackage{rotating}

\usepackage{titlesec}

\newcommand{\be}[1]{
    \begin{equation}
        #1
    \end{equation}
}

\newcommand{\bfig}[3]{
    \begin{figure}[H]
        \centering
        \includegraphics[width=#1\textwidth]{#2}
        \caption{#3}
    \end{figure}
}

\titleformat{\section}%设置section的样式
{\raggedright\large\bfseries}%右对齐,4号字,加粗
{\thesection .\quad}%标号后面有个点
{0pt}%sep label和title之间的水平距离
{}%标题前没有内容

\title{\vspace{-4cm}\Large 高温超导材料的特性与表征}  %文章标题
\author{\kaishu 学号:202111999064 \hspace{2cm} 姓名:郑力恒}   %作者的名称
\date{}

\begin{document}
\maketitle

\begin{abstract}
    本实验利用液氮冷却实现了 77.3K-290K 的可控的温度范围变化. 并且利用液氮降温超导样品, 通
    过四引线法测量超导样品的电阻随温度的变化, 并通过插值得到了转变曲线和相应的温度参数. 超导的起
    始转变温度 Tc,onset = 107.4K, 临界温度 Tc = 92.65K , 零电阻温度 Tc0 = 90.8K, 转变宽度 ∆T = 8.32K. 该结果不是很理想, 是仪器对转变区电阻的分辨能力不够导致的. 
    同时完成了硅二极管正向电阻和温差电偶电阻的定标. 使用线性函数对两者进行了拟合, 均有良好的拟合优度. 
    最后进行磁悬浮的演示实验, 并对磁悬浮能稳定产生进行了定性的理论解释. 后对超导体在场冷和零
    场冷的两种情况下测量了其磁悬浮力——位移曲线, 并对曲线的趋势做出了定性分析.
    
\end{abstract}

\section{引言}
超导态具备电学和磁学上的超导特性, 即完全导电性和完全抗磁性. 完全导电性是说, 在某个温度下材料的电阻突然消失到0. 完全抗磁性是说, 超导态内部不含磁场. 
本实验将使用液氮低温技术探究超导体的这两个技术. 

超导体起初是在液氦温区发现的. 出于产业应用的需求, 人们逐渐发现了超导转变温度在液氮温区的金属氧化物材料, 例如Y-Ba-Cu-O(钇钡铜氧, YBCO)材料. 超导电性正在应用于超导计算机, 超导微波器件甚至于物理量计量标准, 具有十分广阔的前景. 

\section{原理}
\subsection{低温材料的电阻特性}
\subsubsection{纯金属}
纯金属的电阻产生于晶体的电子被晶格本身和晶格中的缺陷的热振动散射, 因此可以分为两部分:
\be{\rho=\rho_{L}(T)+\rho_{r}}
其中$\rho_{L}(T)$来源于晶格热振动带来的电阻散射, 与温度强相关. $\rho_r$表示晶格中的杂质和缺陷对电子的散射所引起的电阻率, 一般不对$\frac{d\rho}{dT}$贡献. 

实验中使用的铂金属电阻在液氮温区至室温的范围内具有良好的线性关系, 在冷热循环中电阻的稳定性很好. 在实验中, 起到指示温度的作用. 
\subsubsection{半导体}
半导体区分于金属的一个重要特征就是负的温度系数. 一般本征半导体的电阻率$\rho$满足
\be{\rho=\frac{1}{n e (\mu_{e}+\mu_{p})}}
即$\rho$由载流子浓度$n$及迁移率$\mu$决定. 载流子浓度$n$随温度增高而指数上升, 比迁移率$\mu$下降的速度要快, 这就导致了负的温度系数. 

半导体温度计常用于弥补金属温度计在低温区电阻值和灵敏度降低的缺陷. 实验中使用硅二极管电阻, 是想在低温条件下观察这种负温度系数的现象. 

\subsection{超导体的电学及磁学特征}
\subsubsection{电阻温度转变曲线}
作为一种相变现象, 超导材料从普通相转变为超导相存在一个转变温度$T_c$. 转变温度以上, 电阻对温度呈线性下降. 达到临界温度以后, 电阻将迅速下降. 一般将电阻变化10\%到90\%所对应的温度间隔定义为转变宽度$\Delta T_c$, 
反映了材料品质的好坏. 样品越均匀, 相越单一, $\Delta T_c$越窄. 在这个过程中将转变温度$T_{cm}$定义为待测样品电
阻从起始转变处下降到一半时对应的温度. 
\subsubsection{电阻测量方法--四引线法}
由于在低温下超导体的电阻非常小, 需要规避导线电阻和接触电阻的影响. 其基本电路是恒流源串联电路, 通过标准电阻上的电压测定电流, 通过待测样品上的电压确定其电阻. 所谓"四引线"的含义, 是指每个电阻元件都采用四根引线, 两根电流引线, 两根电压引线, 其要点是电压引线与样品的接点应处于
两根电流引线的接点之间. 
\subsubsection{超导判断方法--电流反向}
电路中由于材料的不均匀性和温差的存在, 会产生乱真电动势. 在测得样品电压小于最低可分辨值时, 电流反向后的电压测定若结果相同, 基本可以认为已经达到超导态. 
\subsection{超导体的磁效应}
\subsubsection{超导体的分类}
迈斯纳效应表明, 对于理想的超导体来说, 无论外界如何加磁场, 加磁场的历史如何, 超导体内的磁感应强度总是为0. 对于实际中的超导体, 外界施加
的磁场达到一定值以后, 将促使样品返回正常态. 超过临界磁场即从超导态返回正常态的, 称为第I类超导体. 如果中间还存在超导态和正常态的混合态的, 称为第II类超导体. 

高温超导态一般都是非理想的第II类超导体. 其载流区域一般呈层状结构, 整体上不均匀, 从而存在钉扎效应. 钉扎效应导致超导体倾向于俘获稳定磁通, 这种阻碍磁通线运动的力就叫钉扎力, 产生钉扎的缺陷称为钉扎中心. 
\section{实验及结果讨论}
\subsection{材料的电阻-温度曲线}
实验中使用BW2型高温超导材料特性测试装置进行电性测量, 其中包含铂电阻, 硅二极管, 超导样品三个电阻测量电路, 每一电路均包含恒流源、标准电阻、待测电阻、数字电压表和转换开关等五个
主要部件. 温度计和电阻样品被一同置于紫铜恒温块中, 恒温块通过靠近液氮以逐渐降温. 其中, 铂电阻具有线性的电阻-温度曲线, 用于指示恒温块此时的温度. 

初始, 室温条件下, 调节电流使铂电阻电路, 硅二极管电路上的标准电阻分别100mV和1V的电压, 同时设置超导样品电路的恒流源电流10mA. 测得电压及计算的电阻如下表: 
\begin{table}[H]
    \centering
    \begin{tabular}{|c|c|c|c|}
    \hline
             & Pt电阻   & Si二极管 & 样品     \\ \hline
    电压$U(mV)$  & 107.73 & 533.4 & 0.046  \\ \hline
    电阻$R(\Omega)$ & 107.73 & 5334  & 0.0046 \\ \hline
    \end{tabular}
    \caption{室温测量结果}
    \end{table}
我们预期, 在降温的过程中, 铂电阻和样品的测量电压都会下降, 硅二极管的测量电压则会上升. 
\subsubsection{硅二极管}
据测量的数据, 可作图得到半导体硅电阻与随温度变化曲线如下图:
\bfig{0.8}{rsi.jpg}{硅二极管电阻-温度曲线}
可以看到, 硅二极管的电阻随温度的下降而上升, 在液氮沸点至250K左右基本呈线性, 拟合优度r=0.9998. 
其电阻表达式为: 
\be{R_{Si}=-0.024*T+12.27(k\Omega)}
\subsubsection{热偶电阻}
热偶电阻测量紫铜恒温块与液氮的温度差. 实验中随着液氮的挥发, 需频繁下移恒温块的位置以保持较好的降温速率, 这导致热偶电阻随温度呈分段连续. 
为测量其温度系数, 取样品达到超导以后浸没在液氮中降温时的数据, 测量结果如下图所示: 
\bfig{0.8}{rtc.jpg}{热电偶(thermocouple, 图中简写为tc)电阻-温度曲线}
其中计算其电阻时, 使用的电流是超导样品电路的电流, 这是由于温差电阻与超导样品通过转置开关共用同一电路的缘故. 
测得温差电偶电阻的表达式为: 
\be{R_{tc}=0.00125*T-0.097(\Omega)}
拟合优度r=0.9992, 呈良好的线性. 

还可以注意到该表达式在T轴上的截距为$0.097/0.00125*K=77.6K$, 基本在液氮的沸点附近, 即在液氮沸点处其电阻降为0, 符合实验的预期. 这时温差电偶的电阻表达式也可以写为: 
\be{R_{tc}=0.00125*(T-T_N)(\Omega)}
其中$T_N$表示液氮沸点. 

\subsubsection{超导样品}
根据测量所得的数据, 可作图得到超导样品电阻与随温度变化的散点图如下图:
\bfig{0.8}{rsc1.jpg}{超导样品(superconductor, 图中简写为sc)电阻-温度关系散点图}
由于温度变化比电阻变化快, 记录数据采取相同电阻记录首尾两个温度的缘故, 电阻在线性区域呈现部分平台化, 但这并不影响这段温度区间内电阻为线性下降的结论. 
图中可以观察到在T=110K左右与线性区域发生偏离. 从实验数据中能判断的$T_{c,onset}=108K$. 在该温度以下至超导转变完成, 使用matlab中的interp1进行插值, 如图所示
\bfig{0.8}{rsc2.jpg}{超导样品相变区电阻-温度插值曲线}
$T_{c,onst}$处$R_n=0.0029\Omega$由插值函数, 可以确定$T_{c0}=91.68K$ , $T_{c}=92.65K$, $\Delta T=8.32K$. 这个结果是显然偏大的, 这是测量用仪器的分辨率不够高, 无法分辨这段温度电阻变化程度导致的. 从图中可以看出, 如果选取的样本数据更平滑一些, 转变宽度会小3K左右. 

此外, 通过电流反向的手段, 可以确定$T_{c0}=90.8K$. 总的来说, 超导转变曲线的几个关键参数测量如下: 
\begin{table}[H]
    \centering
    \begin{tabular}{|c|c|c|c|}
    \hline
    $T_{c,onset}$ & $T_c$     & $T_{c0}$   & $\Delta T$ \\ \hline
    107.4K  & 92.65K & 90.8K & 8.32K  \\ \hline
    \end{tabular}
    \caption{超导温度转变曲线的基本参数}
    \end{table}

\subsection{磁悬浮演示}
\subsubsection{零场冷}
将液氮缓缓倾倒在样品之上, 待超导样品温度降至其临界温度以下, 可将铷磁铁置于超导样品上, 将发现磁铁感受到斥力, 但无法达成稳定悬浮. 
这是由完全抗磁性导致的. 
\subsubsection{场冷}
将超导样品恢复至室温后, 将铷磁铁悬于超导样品上方. 此时由于样品呈正常态, 移动铷磁铁没有阻力. 在此情形下, 将液氮缓缓倒在超导样品之上, 达到临界温度以下, 将发现磁铁可以稳定悬浮于样品上. 
若拿起磁铁, 则超导样品会跟着被拿起, 向下按压磁铁, 会感受到较强的阻力.  磁铁的整体移动将受到阻力, 但是磁铁绕其质心转动时不受阻碍. 这是由于旋转不影响场冷时钉扎的那部分磁通. 

如果详细分析, 铷磁铁放置于超导样品之上, 在室温下, 有磁场穿过超导样品, 在超导样品从正常态冷却至转变温度
以下后进入混合态时, 该磁通会钉扎于超导体内. 之后若超导体远离铷磁铁, 则此时穿过超导体的磁场有
减弱的趋势, 故磁通量子在向外逸出的过程中受到钉扎磁通的吸引, 对铷磁铁呈现吸引力. 超导体靠近铷
磁铁, 则此时穿过超导体的磁场有增强的趋势, 但新磁通的进入受到内部磁通量子的排斥,对铷磁铁呈现
排斥力. 故超导体所悬浮位置为稳定平衡点, 所以超导体被固定在原处. 
\subsection{压力位移曲线测量}
调整实验装置, 在无作用力时调零作用力后降下铷磁铁至接触超导体至恰好无作用力, 归零位移示数。
后倒入液氮, 分别在场冷和零场冷的情况下测量磁力-位移曲线。
\subsubsection{零场冷}
在外磁场为零的情况下进入超导态, 得到压力位移曲线如下图所示: 
\bfig{0.8}{zfc.png}{零场冷}
纵轴正值表示磁铁受到的排斥力, 负值表示磁铁受到吸引力. 排斥力随位移基本呈磁滞回线. 而产生吸引力的原因一般是没有完全做到零场冷, 即在冷却的过程中, 超导体还是俘获了一部分磁通.  
在实验中发现, 在不恢复室温的情况下重复零场冷情况, 将发现吸引力的情况越来越明显, 即俘获的磁通越来越多了. 消除这一点的办法只有将样品恢复室温重新冷却. 
\subsubsection{场冷}
在磁铁近距离靠近超导体时进行冷却, 得到压力位移曲线如下图所示: 
\bfig{0.8}{fc.png}{场冷}
可以观察到滞回过程中明显的吸引力. 
\section{总结和建议}
本实验利用液氮冷却实现了 77.3K-290K 的可控的温度范围变化. 并且利用液氮降温超导样品, 通
过四引线法测量超导样品的电阻随温度的变化, 并通过插值得到了转变曲线和相应的温度参数. 超导的起
始转变温度 Tc,onset = 107.4K, 临界温度 Tc = 92.65K , 零电阻温度 Tc0 = 90.8K, 转变宽度 ∆T = 8.32K. 该结果不是很理想, 是仪器对转变区电阻的分辨能力不够导致的. 
同时完成了硅二极管正向电阻和温差电偶电阻的定标. 使用线性函数对两者进行了拟合, 均有良好的拟合优度. 
最后进行磁悬浮的演示实验, 并对磁悬浮能稳定产生进行了定性的理论解释. 后对超导体在场冷和零
场冷的两种情况下测量了其磁悬浮力——位移曲线, 并对曲线的趋势做出了定性分析. 

\section{附录: 部分实验数据}
\bfig{0.3}{数据1.jpg}{}
\end{document}