\documentclass[12pt, a4paper]{article}
\usepackage{amsmath, amssymb, amsthm}
\usepackage[utf8]{inputenc}
\usepackage[T1]{fontenc}
\usepackage{geometry}
\usepackage{ctex}
\geometry{left=2.5cm, right=2.5cm, top=2.5cm, bottom=2.5cm}
\usepackage{physics} % For \bra{}, \ket{}, \braket{}, \expval{}
\usepackage{bm}

\newtheorem*{theorem}{Theorem}
\newtheorem*{conclusion}{Conclusion}
\renewcommand{\qedsymbol}{$\blacksquare$}

% 定义常用物理常量符号
\newcommand{\kb}{k_{\mathrm{B}}} % 玻尔兹曼常数
\newcommand{\mub}{\mu_{\mathrm{B}}} % 玻尔磁子

\begin{document}

\title{量子力学 II 作业 5}
\author{姓名: 郑晓旸 \\ 学号: 202111030007}
\date{ \today}
\maketitle

\section*{问题1}

考虑一个由自旋 1/2 粒子构成的纯态系综。如果我们已知该纯态下泡利算符 \(\sigma_x\) 和 \(\sigma_z\) 的期望值 \(\expval{\sigma_x}\) 和 \(\expval{\sigma_z}\),以及 \(\sigma_y\) 的期望值 \(\expval{\sigma_y}\) 的符号(即 \(\text{sgn}(\expval{\sigma_y})\)),我们能否唯一地确定这个纯态(除去一个整体相位因子)?

\subsection*{解答}

\subsubsection*{自旋1/2纯态的参数化}
一个任意的自旋 1/2 纯态 \(\ket{\psi}\) 可以表示为自旋向上 \(\ket{+}_z = \begin{pmatrix} 1 \\ 0 \end{pmatrix}\) 和自旋向下 \(\ket{-}_z = \begin{pmatrix} 0 \\ 1 \end{pmatrix}\) (\(\sigma_z\) 的本征态)的线性叠加:
\[
\ket{\psi} = a \ket{+}_z + b \ket{-}_z = \begin{pmatrix} a \\ b \end{pmatrix}
\]
其中 \(a, b \in \mathbb{C}\) 且满足归一化条件 \(|a|^2 + |b|^2 = 1\)。
由于整体相位不影响物理状态,我们可以选择 \(a\) 为实数且非负。因此,可以将状态参数化为:
\[
\ket{\psi(\theta, \phi)} = \cos(\theta/2) \ket{+}_z + e^{i\phi} \sin(\theta/2) \ket{-}_z = \begin{pmatrix} \cos(\theta/2) \\ e^{i\phi} \sin(\theta/2) \end{pmatrix}
\]
其中 \(0 \le \theta \le \pi\) 是极角,\(0 \le \phi < 2\pi\) 是方位角。这对应于布洛赫球面上的一个点。我们的目标是利用给定信息确定 \(\theta\) 和 \(\phi\)。

\subsubsection*{泡利算符的期望值}
泡利算符为:
\[
\sigma_x = \begin{pmatrix} 0 & 1 \\ 1 & 0 \end{pmatrix}, \quad
\sigma_y = \begin{pmatrix} 0 & -i \\ i & 0 \end{pmatrix}, \quad
\sigma_z = \begin{pmatrix} 1 & 0 \\ 0 & -1 \end{pmatrix}
\]
在态 \(\ket{\psi(\theta, \phi)}\) 下的期望值 \(\expval{\sigma_i} = \bra{\psi} \sigma_i \ket{\psi}\) 计算如下:
\begin{align*}
\expval{\sigma_x} &= \bra{\psi} \sigma_x \ket{\psi} = \begin{pmatrix} \cos(\theta/2) & e^{-i\phi} \sin(\theta/2) \end{pmatrix} \begin{pmatrix} 0 & 1 \\ 1 & 0 \end{pmatrix} \begin{pmatrix} \cos(\theta/2) \\ e^{i\phi} \sin(\theta/2) \end{pmatrix} \\
&= \cos(\theta/2) (e^{i\phi} \sin(\theta/2)) + (e^{-i\phi} \sin(\theta/2)) \cos(\theta/2) \\
&= \sin(\theta/2)\cos(\theta/2) (e^{i\phi} + e^{-i\phi}) = \frac{1}{2} \sin(\theta) (2\cos(\phi)) \\
&= \sin(\theta) \cos(\phi)
\end{align*}
\begin{align*}
\expval{\sigma_y} &= \bra{\psi} \sigma_y \ket{\psi} = \begin{pmatrix} \cos(\theta/2) & e^{-i\phi} \sin(\theta/2) \end{pmatrix} \begin{pmatrix} 0 & -i \\ i & 0 \end{pmatrix} \begin{pmatrix} \cos(\theta/2) \\ e^{i\phi} \sin(\theta/2) \end{pmatrix} \\
&= \cos(\theta/2) (-i e^{i\phi} \sin(\theta/2)) + (e^{-i\phi} \sin(\theta/2)) (i \cos(\theta/2)) \\
&= \sin(\theta/2)\cos(\theta/2) (-i e^{i\phi} + i e^{-i\phi}) = \frac{1}{2} \sin(\theta) i(e^{-i\phi} - e^{i\phi}) \\
&= \frac{1}{2} \sin(\theta) i(-2i \sin(\phi)) \\
&= \sin(\theta) \sin(\phi)
\end{align*}
\begin{align*}
\expval{\sigma_z} &= \bra{\psi} \sigma_z \ket{\psi} = \begin{pmatrix} \cos(\theta/2) & e^{-i\phi} \sin(\theta/2) \end{pmatrix} \begin{pmatrix} 1 & 0 \\ 0 & -1 \end{pmatrix} \begin{pmatrix} \cos(\theta/2) \\ e^{i\phi} \sin(\theta/2) \end{pmatrix} \\
&= \cos^2(\theta/2) - \sin^2(\theta/2) \\
&= \cos(\theta)
\end{align*}
总结得到布洛赫向量 \(\mathbf{S} = (\expval{\sigma_x}, \expval{\sigma_y}, \expval{\sigma_z}) = (\sin\theta \cos\phi, \sin\theta \sin\phi, \cos\theta)\)。对于纯态,其模长 \(|\mathbf{S}|=1\)。

\subsubsection*{利用给定信息确定 \(\theta\) 和 \(\phi\)}
我们已知 \(\expval{\sigma_x}\), \(\expval{\sigma_z}\) 的值和 \(\text{sgn}(\expval{\sigma_y})\)。

\paragraph{确定 \(\theta\):}
由 \(\expval{\sigma_z} = \cos(\theta)\) 和 \(0 \le \theta \le \pi\),\(\cos(\theta)\) 的值唯一确定了 \(\theta\):
\[
\theta = \arccos(\expval{\sigma_z})
\]
确定 \(\theta\) 后,\(\sin(\theta)\) 的值也随之确定,且 \(\sin(\theta) = \sqrt{1 - \cos^2(\theta)} = \sqrt{1 - \expval{\sigma_z}^2} \ge 0\)。

\paragraph{确定 \(\phi\):}
我们需要分情况讨论:

\subparagraph{情况 1: \(\sin(\theta) \neq 0\) (即 \(\theta \neq 0\) 且 \(\theta \neq \pi\),对应 \(\expval{\sigma_z} \neq \pm 1\))}
在这种情况下,我们可以从 \(\expval{\sigma_x} = \sin(\theta) \cos(\phi)\) 计算 \(\cos(\phi)\):
\[
\cos(\phi) = \frac{\expval{\sigma_x}}{\sin(\theta)} = \frac{\expval{\sigma_x}}{\sqrt{1 - \expval{\sigma_z}^2}}
\]
知道 \(\cos(\phi)\) 的值通常会给出两个可能的 \(\phi\) 值,记为 \(\phi_0\) 和 \(2\pi - \phi_0\)(或 \(\phi_0\) 和 \(-\phi_0\) 模 \(2\pi\))。
此时,我们利用 \(\expval{\sigma_y}\) 的符号信息。我们有 \(\expval{\sigma_y} = \sin(\theta) \sin(\phi)\)。因为我们已确定 \(\sin(\theta) > 0\),所以 \(\expval{\sigma_y}\) 的符号与 \(\sin(\phi)\) 的符号完全相同:
\[
\text{sgn}(\expval{\sigma_y}) = \text{sgn}(\sin(\phi))
\]
结合 \(\cos(\phi)\) 的值和 \(\sin(\phi)\) 的符号,可以唯一地确定角 \(\phi\) 在 \([0, 2\pi)\) 区间内的值。例如:
\begin{itemize}
    \item 如果 \(\cos(\phi) > 0\) 且 \(\text{sgn}(\expval{\sigma_y}) > 0\) (\(\sin(\phi) > 0\)),则 \(\phi\) 在第一象限。
    \item 如果 \(\cos(\phi) < 0\) 且 \(\text{sgn}(\expval{\sigma_y}) > 0\) (\(\sin(\phi) > 0\)),则 \(\phi\) 在第二象限。
    \item 如果 \(\cos(\phi) < 0\) 且 \(\text{sgn}(\expval{\sigma_y}) < 0\) (\(\sin(\phi) < 0\)),则 \(\phi\) 在第三象限。
    \item 如果 \(\cos(\phi) > 0\) 且 \(\text{sgn}(\expval{\sigma_y}) < 0\) (\(\sin(\phi) < 0\)),则 \(\phi\) 在第四象限。
    \item 如果 \(\text{sgn}(\expval{\sigma_y}) = 0\) (\(\sin(\phi) = 0\)),则 \(\phi = 0\) 或 \(\pi\)。由 \(\cos(\phi) = \expval{\sigma_x}/\sin(\theta)\) 的值 (\(\pm 1\)) 可以确定是哪一个。
    \item 如果 \(\expval{\sigma_x} = 0\) (\(\cos(\phi) = 0\)),则 \(\phi = \pi/2\) 或 \(3\pi/2\)。由 \(\text{sgn}(\expval{\sigma_y})\) 的符号 (\(\pm 1\)) 可以确定是哪一个。
\end{itemize}

\subparagraph{情况 2: \(\sin(\theta) = 0\) (即 \(\theta = 0\) 或 \(\theta = \pi\),对应 \(\expval{\sigma_z} = \pm 1\))}
\begin{itemize}
    \item 如果 \(\theta = 0\),则 \(\expval{\sigma_z} = 1\)。此时必然有 \(\expval{\sigma_x} = 0 \times \cos(\phi) = 0\) 和 \(\expval{\sigma_y} = 0 \times \sin(\phi) = 0\)。给定的信息 \(\expval{\sigma_x} = 0\),\(\expval{\sigma_z} = 1\),\(\text{sgn}(\expval{\sigma_y}) = 0\) 唯一确定了状态为 \(\ket{+}_z\)。此时 \(\phi\) 无定义,但这不影响状态的唯一性。
    \item 如果 \(\theta = \pi\),则 \(\expval{\sigma_z} = -1\)。此时必然有 \(\expval{\sigma_x} = 0\) 和 \(\expval{\sigma_y} = 0\)。给定的信息 \(\expval{\sigma_x} = 0\),\(\expval{\sigma_z} = -1\),\(\text{sgn}(\expval{\sigma_y}) = 0\) 唯一确定了状态为 \(\ket{-}_z\)。此时 \(\phi\) 也无定义。
\end{itemize}

\subsubsection*{结论}
在所有情况下,给定的信息 \(\expval{\sigma_x}\),\(\expval{\sigma_z}\) 和 \(\text{sgn}(\expval{\sigma_y})\) 都足以唯一地确定参数 \(\theta\) 和 \(\phi\)(或者在 \(\theta=0, \pi\) 的情况下直接确定状态)。由于纯态 \(\ket{\psi}\) 由 \(\theta\) 和 \(\phi\) 唯一确定(除去整体相位),因此该纯态可以被唯一确定。
\section*{问题 2.1}

设一个量子系统的密度算符在时刻 \(t_0\) 为 \(\rho(t_0)\)。在幺正演化下,系统在时刻 \(t\) 的密度算符 \(\rho(t)\) 由下式给出:
\[
\rho(t) = U(t, t_0) \rho(t_0) U^\dagger(t, t_0)
\]
其中 \(U(t, t_0)\) 是从时刻 \(t_0\) 到 \(t\) 的时间演化算符,满足 \(U^\dagger(t, t_0) U(t, t_0) = I\) 且 \(U(t_0, t_0) = I\)。

\begin{proof}
考虑一个在时刻 \(t_0\) 由系综 \(\{ p_i, |\psi_i(t_0)\rangle \}\) 描述的量子系统,其中 \(p_i\) 是处于归一化纯态 \(|\psi_i(t_0)\rangle\) 的概率,且 \(\sum_i p_i = 1\)。
根据定义,时刻 \(t_0\) 的密度算符为:
\[
\rho(t_0) = \sum_i p_i |\psi_i(t_0)\rangle \langle \psi_i(t_0)|
\]
根据量子力学的时间演化原理,状态向量从 \(t_0\) 到 \(t\) 的演化为:
\[
|\psi_i(t)\rangle = U(t, t_0) |\psi_i(t_0)\rangle
\]
其对应的右矢 (bra) 演化为:
\[
\langle \psi_i(t)| = \langle \psi_i(t_0)| U^\dagger(t, t_0)
\]
在幺正演化过程中,概率 \(p_i\) 保持不变。因此,在时刻 \(t\),系统由系综 \(\{ p_i, |\psi_i(t)\rangle \}\) 描述。
时刻 \(t\) 的密度算符 \(\rho(t)\) 为:
\[
\rho(t) = \sum_i p_i |\psi_i(t)\rangle \langle \psi_i(t)|
\]
将 \(|\psi_i(t)\rangle\) 和 \(\langle \psi_i(t)|\) 的表达式代入:
\begin{align*}
\rho(t) &= \sum_i p_i \left( U(t, t_0) |\psi_i(t_0)\rangle \right) \left( \langle \psi_i(t_0)| U^\dagger(t, t_0) \right) \\
&= U(t, t_0) \left( \sum_i p_i |\psi_i(t_0)\rangle \langle \psi_i(t_0)| \right) U^\dagger(t, t_0) \quad \text{(因为 \(U\) 和 \(U^\dagger\) 与 \(i\) 无关)} \\
&= U(t, t_0) \rho(t_0) U^\dagger(t, t_0)
\end{align*}
证毕。
\end{proof}

\section*{问题 2.2}

如果一个量子系统在时刻 \(t_0\) 处于一个纯态,那么在幺正时间演化下,它在任意时刻 \(t\) 仍然处于一个纯态。换言之,纯态系综不可能通过幺正演化变成混合态系综。

\begin{proof}
我们利用密度算符的纯度 (Purity) \(\mathcal{P} = \Tr(\rho^2)\) 来判别状态的类型。
\begin{itemize}
    \item 系统处于纯态 \(\iff \rho = |\psi\rangle\langle\psi| \iff \rho^2 = \rho \iff \Tr(\rho^2) = 1\)。
    \item 系统处于混合态 \(\iff \rho = \sum_i p_i |\psi_i\rangle\langle\psi_i|\) (至少两个 \(p_i > 0\)) \(\iff \rho^2 \neq \rho \iff \Tr(\rho^2) < 1\)。
\end{itemize}
(注意:对于所有密度算符,\(\Tr(\rho) = 1\) 且 \(0 < \Tr(\rho^2) \le 1\))。
假设系统在时刻 \(t_0\) 处于纯态。这意味着 \(\rho(t_0)\) 描述一个纯态,因此:
\[
\Tr(\rho(t_0)^2) = 1
\]
系统在时刻 \(t\) 的密度算符为 \(\rho(t) = U(t, t_0) \rho(t_0) U^\dagger(t, t_0)\)。我们计算 \(\rho(t)\) 的纯度:
\[
\rho(t)^2 = \left( U(t, t_0) \rho(t_0) U^\dagger(t, t_0) \right) \left( U(t, t_0) \rho(t_0) U^\dagger(t, t_0) \right)
\]
利用时间演化算符的幺正性 \(U^\dagger(t, t_0) U(t, t_0) = I\),我们得到:
\[
\rho(t)^2 = U(t, t_0) \rho(t_0) (U^\dagger(t, t_0) U(t, t_0)) \rho(t_0) U^\dagger(t, t_0)
\]
\[
\rho(t)^2 = U(t, t_0) \rho(t_0) I \rho(t_0) U^\dagger(t, t_0)
\]
\[
\rho(t)^2 = U(t, t_0) \rho(t_0)^2 U^\dagger(t, t_0)
\]
现在计算其迹:
\[
\Tr(\rho(t)^2) = \Tr\left( U(t, t_0) \rho(t_0)^2 U^\dagger(t, t_0) \right)
\]
利用迹的循环不变性 \(\Tr(ABC) = \Tr(CAB)\):
\[
\Tr(\rho(t)^2) = \Tr\left( U^\dagger(t, t_0) U(t, t_0) \rho(t_0)^2 \right)
\]
再次利用幺正性 \(U^\dagger(t, t_0) U(t, t_0) = I\):
\[
\Tr(\rho(t)^2) = \Tr\left( I \rho(t_0)^2 \right) = \Tr(\rho(t_0)^2)
\]
因此,我们证明了 \(\Tr(\rho(t)^2) = \Tr(\rho(t_0)^2)\)。
由于初始状态是纯态,\(\Tr(\rho(t_0)^2) = 1\),所以对于任意时刻 \(t\),必然有:
\[
\Tr(\rho(t)^2) = 1
\]
这表明 \(\rho(t)\) 在任意时刻 \(t\) 都描述一个纯态。因此,一个纯态系综在幺正时间演化下永远保持为纯态,不会演化成混合态。
证毕。
\end{proof}
\subsection*{结论}
密度算符的时间演化遵循 \(\rho(t) = U(t, t_0) \rho(t_0) U^\dagger(t, t_0)\) 的规律。幺正演化保持系统的纯度 \(\Tr(\rho^2)\) 不变,这意味着纯态在封闭系统的演化中始终保持为纯态。混合态的出现需要系统与环境发生相互作用(开放系统)或经历测量过程。

\section*{问题 3}
考虑一个由 \(N\) 个独立的、可分辨的自旋 1/2 粒子组成的系综。系统处于沿 \(z\) 轴方向的均匀外磁场 \(\vec{B} = B \hat{z}\) 中,并与温度为 \(T\) 的热库达到热平衡。计算该系统的亥姆霍兹自由能 \(F\)。
\subsection*{推导过程}
\subsubsection*{系统描述与哈密顿量}
每个自旋 1/2 粒子的磁矩算符与其自旋算符 \(\vec{S}\) 相关。对于电子,其关系为 \(\vec{\mu} = -g \frac{\mub}{\hbar} \vec{S}\),其中 \(\mub = \frac{e\hbar}{2m_e}\) 是玻尔磁子,\(g\) 是朗德 g 因子(对电子 \(g \approx 2\)),\(\hbar\) 是约化普朗克常数。
单个粒子在磁场 \(\vec{B} = B \hat{z}\) 中的相互作用哈密顿量为:
% 使用 \hat 表示算符
\[
\hat{H}_{\text{single}} = -\vec{\mu} \cdot \vec{B} = - \left(-g \frac{\mub}{\hbar} \hat{S}_z\right) B = g \frac{\mub B}{\hbar} \hat{S}_z
\]
其中 \(\hat{S}_z = \frac{\hbar}{2} \hat{\sigma}_z\) 是自旋沿 \(z\) 方向的分量算符,\(\hat{\sigma}_z = \begin{pmatrix} 1 & 0 \\ 0 & -1 \end{pmatrix}\) 是泡利矩阵。将 \(\hat{S}_z\) 的表达式代入哈密顿量:
\[
\hat{H}_{\text{single}} = g \frac{\mub B}{\hbar} \left(\frac{\hbar}{2} \hat{\sigma}_z\right) = \frac{1}{2} g \mub B \hat{\sigma}_z
\]
为了简化书写,我们定义一个能量 \(\epsilon_0 = \frac{1}{2} g \mub B\)。则单粒子哈密顿量可写为:
\[
\hat{H}_{\text{single}} = \epsilon_0 \hat{\sigma}_z
\]
\subsubsection*{能级}
哈密顿量 \(\hat{H}_{\text{single}}\) 的本征态是自旋向上态 \(|\uparrow\rangle = \begin{pmatrix} 1 \\ 0 \end{pmatrix}\) 和自旋向下态 \(|\downarrow\rangle = \begin{pmatrix} 0 \\ 1 \end{pmatrix}\)。它们分别是 \(\hat{\sigma}_z\) 算符本征值为 +1 和 -1 的本征态。对应的能量本征值为:
\begin{itemize}
    \item 对于自旋向上 \(|\uparrow\rangle\) (\(\sigma_z = +1\)): 
    \[ E_\uparrow = \epsilon_0 \cdot (+1) = +\epsilon_0 = +\frac{1}{2} g \mub B \]
    \item 对于自旋向下 \(|\downarrow\rangle\) (\(\sigma_z = -1\)): 
    \[ E_\downarrow = \epsilon_0 \cdot (-1) = -\epsilon_0 = -\frac{1}{2} g \mub B \]
\end{itemize}
因此,每个粒子只有这两个可能的能量状态。
\subsubsection*{单粒子配分函数}
系统与温度为 \(T\) 的热库达到热平衡,处于正则系综。单个粒子的配分函数 \(Z_1\) 由其所有可能状态的玻尔兹曼因子 \(e^{-E_i / (\kb T)}\) 求和得到:
\[
Z_1 = \sum_{\text{states } i} e^{-E_i / (\kb T)}
\]
其中 \(\kb\) 是玻尔兹曼常数。对于这个二能级系统:
\[
Z_1 = e^{-E_\downarrow / (\kb T)} + e^{-E_\uparrow / (\kb T)} = e^{-(-\epsilon_0) / (\kb T)} + e^{-(+\epsilon_0) / (\kb T)}
\]
\[
Z_1 = e^{\epsilon_0 / (\kb T)} + e^{-\epsilon_0 / (\kb T)}
\]
这个表达式可以使用双曲余弦函数 \(\cosh(x) = \frac{e^x + e^{-x}}{2}\) 来简化。令 \(x = \frac{\epsilon_0}{\kb T} = \frac{g \mub B}{2 \kb T}\),则:
\[
Z_1 = 2 \cosh\left(\frac{\epsilon_0}{\kb T}\right) = 2 \cosh\left(\frac{g \mub B}{2 \kb T}\right)
\]
\subsection*{N 粒子总配分函数}
由于题目假设 \(N\) 个粒子是独立的且可分辨的(例如,它们被固定在晶格的不同位置上),整个系统的总配分函数 \(Z_N\) 等于单粒子配分函数 \(Z_1\) 的 \(N\) 次方:
\[
Z_N = (Z_1)^N = \left[ 2 \cosh\left(\frac{g \mub B}{2 \kb T}\right) \right]^N
\]
(注意:如果粒子是不可分辨的全同费米子或玻色子,计算方式会不同。)
\subsubsection*{亥姆霍兹自由能}
亥姆霍兹自由能 \(F\) 与正则系综配分函数 \(Z_N\) 的关系由下式给出:
\[
F = -\kb T \ln Z_N
\]
将我们得到的 \(Z_N\) 代入:
\[
F = -\kb T \ln \left\{ \left[ 2 \cosh\left(\frac{g \mub B}{2 \kb T}\right) \right]^N \right\}
\]
利用对数的基本性质 \(\ln(a^N) = N \ln(a)\),我们可以简化上式:
\[
F = -N \kb T \ln \left[ 2 \cosh\left(\frac{g \mub B}{2 \kb T}\right) \right]
\]
\subsubsection*{结论}
对于一个由 \(N\) 个独立的、可分辨的自旋 1/2 粒子组成的系综,在均匀外磁场 \(B\) 中,当系统与温度为 \(T\) 的热库达到热平衡时,其亥姆霍兹自由能 \(F(T, B, N)\) 为:
\[
F(T, B, N) = -N \kb T \ln \left[ 2 \cosh\left(\frac{g \mub B}{2 \kb T}\right) \right]
\]


\end{document}
