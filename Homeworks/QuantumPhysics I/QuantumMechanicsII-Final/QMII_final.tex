\documentclass{article}
\usepackage[a4paper, margin=1in]{geometry}
\usepackage{amsmath}
\usepackage{amssymb}
\usepackage{physics} % For \ket{}, \bra{}, \expval{}, etc.
\usepackage{xeCJK}
\setCJKmainfont{SimSun} % Or any other CJK font available on your system
\usepackage{hyperref} % Optional

\begin{document}

\title{量子力学II (2025) 期末 - 答卷}
\author{郑晓旸 \\ 202111030007}
\date{\today}
\maketitle

\section*{问题 1 (30分)}
\noindent (a) 利用 \(J_z\) 时间反演为奇,证明 \(\Theta|j, m\rangle = c|j, -m\rangle\),其中 \(c\) 为常数。\\
\noindent (b) 再利用 \(J_\pm |jm\rangle = e^{i\delta} c_\pm(j, m) |jm \pm 1\rangle\),进一步证明:
\[ \Theta|j, m\rangle = e^{i\delta} (-1)^m |j, -m\rangle. \]
\rule[1ex]{\textwidth}{0.4pt}
\subsection*{(a) 证明:}
时间反演算符 \(\Theta\) 是一个反幺正算符,其重要性质之一是对角动量算符的作用:
\[ \Theta \mathbf{J} \Theta^{-1} = -\mathbf{J} \]
这意味着 \(\Theta J_k \Theta^{-1} = -J_k\) 对于 \(k=x, y, z\) 都成立。

首先考虑总角动量算符 \(\mathbf{J}^2 = J_x^2 + J_y^2 + J_z^2\) 在时间反演下的变换:
\[ \Theta \mathbf{J}^2 \Theta^{-1} = \Theta (J_x^2 + J_y^2 + J_z^2) \Theta^{-1} \]
\[ = (\Theta J_x \Theta^{-1})^2 + (\Theta J_y \Theta^{-1})^2 + (\Theta J_z \Theta^{-1})^2 \]
\[ = (-J_x)^2 + (-J_y)^2 + (-J_z)^2 \]
\[ = J_x^2 + J_y^2 + J_z^2 = \mathbf{J}^2 \]
因此,\(\mathbf{J}^2\) 在时间反演下是不变的,即 \([\mathbf{J}^2, \Theta] = 0\)。

已知角动量本征态 \(\ket{j, m}\) 是 \(\mathbf{J}^2\) 的本征态:
\[ \mathbf{J}^2 \ket{j, m} = j(j+1)\hbar^2 \ket{j, m} \]
将 \(\mathbf{J}^2\) 作用到态 \(\Theta\ket{j, m}\) 上:
\[ \mathbf{J}^2 (\Theta\ket{j, m}) = (\mathbf{J}^2 \Theta) \ket{j, m} = \Theta (\mathbf{J}^2 \ket{j, m}) \]
\[ = \Theta (j(j+1)\hbar^2 \ket{j, m}) \]
由于 \(j(j+1)\hbar^2\) 是一个实数,且 \(\Theta\) 是反幺正算符 (\(\Theta c = c^* \Theta\)),对实数作用时可直接提出:
\[ \mathbf{J}^2 (\Theta\ket{j, m}) = j(j+1)\hbar^2 (\Theta\ket{j, m}) \]
这表明 \(\Theta\ket{j, m}\) 仍然是 \(\mathbf{J}^2\) 的本征态,其本征值与 \(\ket{j, m}\) 相同,为 \(j(j+1)\hbar^2\)。

接下来考虑 \(J_z\) 算符。题目给定 \(J_z\) 时间反演为奇 (\(\Theta J_z \Theta^{-1} = -J_z\))。
这意味着 \(J_z \Theta = -\Theta J_z\)。

已知 \(\ket{j, m}\) 是 \(J_z\) 的本征态:
\[ J_z \ket{j, m} = m\hbar \ket{j, m} \]
将 \(J_z\) 作用到态 \(\Theta\ket{j, m}\) 上:
\[ J_z (\Theta\ket{j, m}) = (-\Theta J_z) \ket{j, m} = -\Theta (J_z \ket{j, m}) \]
\[ = -\Theta (m\hbar \ket{j, m}) \]
由于 \(m\hbar\) 是一个实数:
\[ J_z (\Theta\ket{j, m}) = -m\hbar (\Theta\ket{j, m}) \]
这表明 \(\Theta\ket{j, m}\) 是 \(J_z\) 的本征态,其本征值为 \(-m\hbar\)。

综上所述,态 \(\Theta\ket{j, m}\) 是 \(\mathbf{J}^2\) 和 \(J_z\) 的共同本征态,其本征值分别为 \(j(j+1)\hbar^2\) 和 \(-m\hbar\)。在角动量量子数 \(j\) 给定的子空间内,由 \(\mathbf{J}^2\) 和 \(J_z\) 的本征值唯一确定了态 \(\ket{j, -m}\) (不考虑相位)。因此,\(\Theta\ket{j, m}\) 必须与 \(\ket{j, -m}\) 成正比:
\[ \Theta\ket{j, m} = c_{j,m}\ket{j, -m} \]
其中 \(c_{j,m}\) 是一个复常数,其可能依赖于 \(j\) 和 \(m\)。

\subsection*{(b) 证明:}
首先,确定升降算符 \(J_\pm = J_x \pm iJ_y\) 在时间反演下的变换:
\[ \Theta J_+ \Theta^{-1} = \Theta (J_x + iJ_y) \Theta^{-1} \]
\[ = \Theta J_x \Theta^{-1} + \Theta i \Theta^{-1} \Theta J_y \Theta^{-1} \]
\[ = (-J_x) + (-i) (-J_y) \]
\[ = -J_x + iJ_y = -(J_x - iJ_y) = -J_- \]
所以,\(\Theta J_+ = -J_- \Theta\)。类似地,可得 \(\Theta J_- = -J_+ \Theta\)。

考虑将时间反演算符 \(\Theta\) 作用到 \(J_+\ket{j, m}\) 的等式两边:
\[ \Theta (J_+\ket{j, m}) = \Theta (e^{i\delta} c_+(j, m) \ket{j, m+1}) \]
左边利用 \(\Theta J_+ = -J_- \Theta\) 和 (a) 的结果 \(\Theta\ket{j, k} = c_k\ket{j, -k}\) (这里用 \(c_k\) 表示依赖于 \(m\) 的常数 \(c_{j,k}\)):
\[ \text{LHS} = (\Theta J_+) \ket{j, m} = (-J_- \Theta) \ket{j, m} = -J_- (\Theta\ket{j, m}) = -J_- (c_m\ket{j, -m}) \]
由于 \(c_m\) 是一个系数,可以提出:
\[ \text{LHS} = -c_m J_-\ket{j, -m} \]
根据题目给定的 \(J_-\) 作用形式 (将 \(m\) 替换为 \(-m\)):
\[ J_-\ket{j, -m} = e^{i\delta} c_-(j, -m) \ket{j, -m-1} \]
所以,
\[ \text{LHS} = -c_m e^{i\delta} c_-(j, -m) \ket{j, -m-1} \]

右边利用 \(\Theta\) 的反幺正性 (\(\Theta c = c^* \Theta\)) 和 (a) 的结果:
\[ \text{RHS} = \Theta (e^{i\delta} c_+(j, m) \ket{j, m+1}) = (e^{i\delta})^* c_+(j, m) \Theta\ket{j, m+1} \]
\[ = e^{-i\delta} c_+(j, m) c_{m+1} \ket{j, -(m+1)} = e^{-i\delta} c_+(j, m) c_{m+1} \ket{j, -m-1} \]
令 LHS = RHS,并比较 \(\ket{j, -m-1}\) 前的系数 (假设 \(\ket{j, -m-1} \neq 0\)):
\[ -c_m e^{i\delta} c_-(j, -m) = e^{-i\delta} c_+(j, m) c_{m+1} \]
\[ c_{m+1} = -c_m \frac{e^{i\delta}}{e^{-i\delta}} \frac{c_-(j, -m)}{c_+(j, m)} = -c_m e^{2i\delta} \frac{c_-(j, -m)}{c_+(j, m)} \]
这是系数 \(c_m\) 的递推关系。

要证明的形式是 \(\Theta|j, m\rangle = e^{i\delta} (-1)^m |j, -m\rangle\),这意味着 \(c_m = e^{i\delta} (-1)^m\)。
如果 \(c_m = e^{i\delta} (-1)^m\),那么 \(c_{m+1} = e^{i\delta} (-1)^{m+1} = -e^{i\delta} (-1)^m = -c_m\)。
将此代入递推关系中:
\[ -c_m = -c_m e^{2i\delta} \frac{c_-(j, -m)}{c_+(j, m)} \]
若 \(c_m \neq 0\),则要求:
\[ 1 = e^{2i\delta} \frac{c_-(j, -m)}{c_+(j, m)} \]
即 \(c_+(j, m) = e^{2i\delta} c_-(j, -m)\)。
这个条件是关于题目中给定的 \(J_\pm\) 算符系数 \(c_\pm(j,m)\) 和相位 \(e^{i\delta}\) 之间的一个一致性要求。
如果这个一致性条件成立,那么递推关系 \(c_{m+1} = -c_m e^{2i\delta} \frac{c_-(j, -m)}{c_+(j, m)}\) 就简化为 \(c_{m+1} = -c_m\)。
此递推关系的通解为 \(c_m = C_0 (-1)^m\),其中 \(C_0\) 是一个不依赖于 \(m\) 的常数 (但可能依赖于 \(j\))。
为了与目标形式 \(\Theta|j, m\rangle = e^{i\delta} (-1)^m |j, -m\rangle\) 相符,选择 \(C_0 = e^{i\delta}\)。
因此,在题目给定的 \(J_\pm\) 作用形式与 \(\Theta\) 作用结果形式内在一致(即 \(c_+(j, m) = e^{2i\delta} c_-(j, -m)\))的前提下,证明了:
\[ \Theta|j, m\rangle = e^{i\delta} (-1)^m |j, -m\rangle. \]
\newpage
\section*{问题 2 (30分)}
\noindent 一个无自旋粒子束缚在固定点,其束缚势能 \(V(\mathbf{x})\) 不是中心势,故能级都是非简并的。利用时间反演对称性证明 \(\langle \mathbf{L} \rangle = 0\)。

\rule[1ex]{\textwidth}{0.4pt}

\subsection*{证明:}
\begin{enumerate}
\item \textbf{哈密顿量与时间反演对称性}

系统的哈密顿量为
\[ H = \frac{\mathbf{p}^2}{2m} + V(\mathbf{x}) \]
其中 \(\mathbf{p}\) 是动量算符,\(V(\mathbf{x})\) 是束缚势能。时间反演算符为 \(\Theta\)。对于无自旋粒子,时间反演算符对坐标算符 \(\mathbf{x}\) 和动量算符 \(\mathbf{p}\) 的作用如下:
\[ \Theta \mathbf{x} \Theta^{-1} = \mathbf{x} \]
\[ \Theta \mathbf{p} \Theta^{-1} = -\mathbf{p} \]
因此,哈密顿量在时间反演下的变换为:
\[ \Theta H \Theta^{-1} = \Theta \left( \frac{\mathbf{p}^2}{2m} + V(\mathbf{x}) \right) \Theta^{-1} \]
\[ = \frac{1}{2m} (\Theta \mathbf{p} \Theta^{-1}) \cdot (\Theta \mathbf{p} \Theta^{-1}) + V(\Theta \mathbf{x} \Theta^{-1}) \]
\[ = \frac{1}{2m} (-\mathbf{p}) \cdot (-\mathbf{p}) + V(\mathbf{x}) \]
\[ = \frac{\mathbf{p}^2}{2m} + V(\mathbf{x}) = H \]
这表明哈密顿量 \(H\) 在时间反演下是不变的,即系统具有时间反演对称性:\([\Theta, H] = 0\)。

\item \textbf{非简并能级与时间反演态}

设 \(|n\rangle\) 是哈密顿量 \(H\) 的一个能量本征态,对应的能量本征值为 \(E_n\),即:
\[ H |n\rangle = E_n |n\rangle \]
由于哈密顿量具有时间反演对称性,\(\Theta |n\rangle\) 也是 \(H\) 的一个本征态,且具有相同的能量本征值 \(E_n\):
\[ H (\Theta |n\rangle) = (\Theta H \Theta^{-1}) (\Theta |n\rangle) = \Theta (H |n\rangle) = \Theta (E_n |n\rangle) = E_n (\Theta |n\rangle) \]
(因为能量 \(E_n\) 是实数)。
题目中已说明能级都是非简并的。这意味着对于每一个能量本征值 \(E_n\),只存在一个(除去整体相位因子外)线性无关的本征态。因此,\(\Theta |n\rangle\) 必须与 \(|n\rangle\) 成正比:
\[ \Theta |n\rangle = c_n |n\rangle \]
其中 \(c_n\) 是一个复数。
对于无自旋粒子,时间反演算符的平方 \(\Theta^2 = 1\)。将 \(\Theta\) 再次作用于上式:
\[ \Theta^2 |n\rangle = \Theta (c_n |n\rangle) \]
由于 \(\Theta\) 是反幺正算符,\(\Theta (c_n |n\rangle) = c_n^* (\Theta |n\rangle) = c_n^* (c_n |n\rangle) = |c_n|^2 |n\rangle\)。
因此有:
\[ 1 \cdot |n\rangle = |c_n|^2 |n\rangle \]
这意味着 \(|c_n|^2 = 1\)。对于非简并能级,可以选择本征态的相位,使得 \(c_n = 1\)。所以:
\[ \Theta |n\rangle = |n\rangle \]

\item \textbf{轨道角动量算符在时间反演下的变换}

轨道角动量算符 \(\mathbf{L} = \mathbf{x} \times \mathbf{p}\)。其在时间反演下的变换为:
\[ \Theta \mathbf{L} \Theta^{-1} = (\Theta \mathbf{x} \Theta^{-1}) \times (\Theta \mathbf{p} \Theta^{-1}) \]
\[ = \mathbf{x} \times (-\mathbf{p}) = -(\mathbf{x} \times \mathbf{p}) = -\mathbf{L} \]

\item \textbf{证明 \(\langle \mathbf{L} \rangle = 0\)}

考虑在能量本征态 \(|n\rangle\) 下轨道角动量的期望值 \(\langle \mathbf{L} \rangle_n = \langle n | \mathbf{L} | n \rangle\)。
利用 \(\Theta |n\rangle = |n\rangle\),可以写出:
\[ \langle \mathbf{L} \rangle_n = \langle n | \mathbf{L} | n \rangle = \langle \Theta n | \mathbf{L} | \Theta n \rangle \]
由于时间反演算符 \(\Theta\) 是反幺正的,对于任意算符 \(A\) 和任意态 \(|\psi\rangle, |\phi\rangle\),有如下关系:
\[ \langle \Theta \psi | A | \Theta \phi \rangle = \langle \psi | \Theta^{-1} A \Theta | \phi \rangle^* \]
将此应用于 \(\langle \Theta n | \mathbf{L} | \Theta n \rangle\),得到:
\[ \langle \Theta n | \mathbf{L} | \Theta n \rangle = \langle n | \Theta^{-1} \mathbf{L} \Theta | n \rangle^* \]
将 \(\Theta^{-1} \mathbf{L} \Theta = -\mathbf{L}\) 代入上式 (因为 \(\Theta \mathbf{L} \Theta^{-1} = -\mathbf{L} \implies \mathbf{L} = \Theta^{-1} (-\mathbf{L}) \Theta \implies \Theta^{-1} \mathbf{L} \Theta = -\mathbf{L}\)):
\[ \langle \Theta n | \mathbf{L} | \Theta n \rangle = \langle n | (-\mathbf{L}) | n \rangle^* = - \langle n | \mathbf{L} | n \rangle^* \]
因此,得到:
\[ \langle n | \mathbf{L} | n \rangle = - \langle n | \mathbf{L} | n \rangle^* \]
即:
\[ \langle \mathbf{L} \rangle_n = - \langle \mathbf{L} \rangle_n^* \]
由于轨道角动量算符 \(\mathbf{L}\) 是厄米算符,其在任意态下的期望值是一个实矢量。这意味着 \(\langle \mathbf{L} \rangle_n^* = \langle \mathbf{L} \rangle_n\)。
将此代入上面的等式:
\[ \langle \mathbf{L} \rangle_n = - \langle \mathbf{L} \rangle_n \]
\[ 2 \langle \mathbf{L} \rangle_n = 0 \]
\[ \langle \mathbf{L} \rangle_n = 0 \]
这证明了对于一个无自旋粒子,如果其束缚在固定点的势能 \(V(\mathbf{x})\) 不是中心势,且能级都是非简并的,那么其轨道角动量的期望值为零。由于这是对任意非简并本征态 \(|n\rangle\) 成立的,所以一般结论 \(\langle \mathbf{L} \rangle = 0\) 成立。
\end{enumerate}
\newpage
\section*{问题 3 (40分)}
\noindent 三个 \(S = 1/2\) 粒子构成的系统,哈密顿量为
\[ H = J(\mathbf{S}_1 \cdot \mathbf{S}_2 + \mathbf{S}_2 \cdot \mathbf{S}_3 + \mathbf{S}_3 \cdot \mathbf{S}_1). \]
\noindent (1) 请问系统是否具有时间反演不变性? (2) 能级是否简并?

\rule[1ex]{\textwidth}{0.4pt}

\subsection*{1. 系统的时间反演不变性}
系统具有时间反演不变性。

\textbf{证明:}
时间反演算符 \(\Theta\) 是一个反幺正算符。对于自旋算符 \(\mathbf{S}_i\),其变换性质为 \(\Theta \mathbf{S}_i \Theta^{-1} = -\mathbf{S}_i\)。
考虑哈密顿量中的任意一项 \(\mathbf{S}_i \cdot \mathbf{S}_j = S_{ix}S_{jx} + S_{iy}S_{jy} + S_{iz}S_{jz}\)。在时间反演操作下:
\[ \Theta (\mathbf{S}_i \cdot \mathbf{S}_j) \Theta^{-1} = \sum_{k=x,y,z} \Theta (S_{ik} S_{jk}) \Theta^{-1} \]
由于 \(\Theta\) 是反幺正的,且 \(\Theta S_{ik} \Theta^{-1} = -S_{ik}\) 和 \(\Theta S_{jk} \Theta^{-1} = -S_{jk}\),我们有:
\[ \Theta (S_{ik} S_{jk}) \Theta^{-1} = (\Theta S_{ik} \Theta^{-1}) (\Theta S_{jk} \Theta^{-1}) = (-S_{ik})(-S_{jk}) = S_{ik} S_{jk} \]
因此,\(\Theta (\mathbf{S}_i \cdot \mathbf{S}_j) \Theta^{-1} = \mathbf{S}_i \cdot \mathbf{S}_j\)。
假设耦合常数 \(J\) 是实数 (通常物理哈密顿量中的参数是实数),则 \(\Theta J \Theta^{-1} = J^* = J\)。
哈密顿量在时间反演下的变换为:
\[ \Theta H \Theta^{-1} = \Theta [J(\mathbf{S}_1 \cdot \mathbf{S}_2 + \mathbf{S}_2 \cdot \mathbf{S}_3 + \mathbf{S}_3 \cdot \mathbf{S}_1)] \Theta^{-1} \]
\[ = J(\Theta (\mathbf{S}_1 \cdot \mathbf{S}_2) \Theta^{-1} + \Theta (\mathbf{S}_2 \cdot \mathbf{S}_3) \Theta^{-1} + \Theta (\mathbf{S}_3 \cdot \mathbf{S}_1) \Theta^{-1}) \]
\[ = J(\mathbf{S}_1 \cdot \mathbf{S}_2 + \mathbf{S}_2 \cdot \mathbf{S}_3 + \mathbf{S}_3 \cdot \mathbf{S}_1) = H \]
由于 \(\Theta H \Theta^{-1} = H\),即 \([H, \Theta] = 0\),系统具有时间反演不变性。


\subsection*{2. 能级简并}

\textbf{证明:}

引入总自旋算符 \(\mathbf{S}_{tot} = \mathbf{S}_1 + \mathbf{S}_2 + \mathbf{S}_3\)。
其平方为 \(\mathbf{S}_{tot}^2 = (\mathbf{S}_1 + \mathbf{S}_2 + \mathbf{S}_3)^2 = \mathbf{S}_1^2 + \mathbf{S}_2^2 + \mathbf{S}_3^2 + 2(\mathbf{S}_1 \cdot \mathbf{S}_2 + \mathbf{S}_1 \cdot \mathbf{S}_3 + \mathbf{S}_2 \cdot \mathbf{S}_3)\)。
对于 \(S=1/2\) 的粒子,\(\mathbf{S}_i^2 = s_i(s_i+1)\hbar^2 = \frac{1}{2}(\frac{1}{2}+1)\hbar^2 = \frac{3}{4}\hbar^2\)。
所以,\(\mathbf{S}_1^2 + \mathbf{S}_2^2 + \mathbf{S}_3^2 = 3 \cdot \frac{3}{4}\hbar^2 = \frac{9}{4}\hbar^2\)。
则哈密顿量中的点乘项可以表示为:
\[ \mathbf{S}_1 \cdot \mathbf{S}_2 + \mathbf{S}_2 \cdot \mathbf{S}_3 + \mathbf{S}_3 \cdot \mathbf{S}_1 = \frac{1}{2} (\mathbf{S}_{tot}^2 - (\mathbf{S}_1^2 + \mathbf{S}_2^2 + \mathbf{S}_3^2)) = \frac{1}{2} \left(\mathbf{S}_{tot}^2 - \frac{9}{4}\hbar^2\right) \]
哈密顿量 \(H\) 可以改写为:
\[ H = J \left( \frac{1}{2} \mathbf{S}_{tot}^2 - \frac{9}{8}\hbar^2 \right) = \frac{J}{2} \mathbf{S}_{tot}^2 - \frac{9J}{8}\hbar^2 \]
系统的能量本征值 \(E\) 取决于总自旋量子数 \(S\) (其中 \(\mathbf{S}_{tot}^2\) 的本征值为 \(S(S+1)\hbar^2\)):
\[ E_S = \frac{J}{2} S(S+1)\hbar^2 - \frac{9J}{8}\hbar^2 \]
对于三个 \(S=1/2\) 的粒子耦合,总自旋 \(S\) 的可能取值:
首先耦合 \(\mathbf{S}_1\) 和 \(\mathbf{S}_2\) 得到 \(\mathbf{S}_{12} = \mathbf{S}_1 + \mathbf{S}_2\)。\(S_{12}\) 可以取 \(1/2+1/2=1\) 或 \(1/2-1/2=0\)。
然后耦合 \(\mathbf{S}_{12}\) 和 \(\mathbf{S}_3\):
\begin{itemize}
    \item 如果 \(S_{12}=1\),与 \(S_3=1/2\) 耦合,得到 \(S_{tot} = 1+1/2 = 3/2\) 或 \(1-1/2 = 1/2\)。
    \item 如果 \(S_{12}=0\),与 \(S_3=1/2\) 耦合,得到 \(S_{tot} = 0+1/2 = 1/2\)。
\end{itemize}
所以,总自旋 \(S_{tot}\) 的可能取值为 \(3/2\) (出现一次) 和 \(1/2\) (出现两次)。

对应的能量本征值为:
\begin{itemize}
    \item 对于 \(S = 3/2\):
    \[ E_{3/2} = \frac{J}{2} \frac{3}{2}\left(\frac{3}{2}+1\right)\hbar^2 - \frac{9J}{8}\hbar^2 = \frac{J}{2} \frac{3}{2} \frac{5}{2}\hbar^2 - \frac{9J}{8}\hbar^2 = \frac{15J}{8}\hbar^2 - \frac{9J}{8}\hbar^2 = \frac{6J}{8}\hbar^2 = \frac{3J}{4}\hbar^2 \]
    此能级的简并度为 \(2S+1 = 2(3/2)+1 = 4\)。
    \item 对于 \(S = 1/2\):
    \[ E_{1/2} = \frac{J}{2} \frac{1}{2}\left(\frac{1}{2}+1\right)\hbar^2 - \frac{9J}{8}\hbar^2 = \frac{J}{2} \frac{1}{2} \frac{3}{2}\hbar^2 - \frac{9J}{8}\hbar^2 = \frac{3J}{8}\hbar^2 - \frac{9J}{8}\hbar^2 = -\frac{6J}{8}\hbar^2 = -\frac{3J}{4}\hbar^2 \]
    由于总自旋 \(S=1/2\) 的态出现了两次 (即有两个独立的态系列可以耦合得到总自旋 \(S=1/2\)),每个 \(S=1/2\) 的态对应的简并度是 \(2S+1 = 2(1/2)+1 = 2\)。因此,对应能量 \(E_{1/2}\) 的总简并度为 \(2 \times 2 = 4\)。
\end{itemize}
系统有两个不同的能量本征值,\(E_{3/2} = \frac{3J}{4}\hbar^2\) 和 \(E_{1/2} = -\frac{3J}{4}\hbar^2\)。
每个能量本征值都对应一个 4 重简并的能级。因此,能级是简并的。

此外,由于系统由奇数个 (3个) 半整数自旋粒子组成,且哈密顿量具有时间反演不变性,根据 Kramers' 定理,所有能级至少是两重简并的。我们的计算结果 (4重简并) 与 Kramers' 定理是一致的。

\end{document}
