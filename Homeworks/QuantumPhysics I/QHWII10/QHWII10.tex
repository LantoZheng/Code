\documentclass{article}
\usepackage{amsmath}
\usepackage{amsfonts}
\usepackage{amssymb}
\usepackage{ctex}
\usepackage{geometry}
\begin{document}

\title{QHWII 10}
\author{郑晓旸 \\ 202111030007}
\date{\today}
\maketitle

该论证的关键在于,在磁单极子场中运动的带电粒子的波函数必须是单值的。

\section{背景设置:带电粒子与磁单极子}

我们考虑一个电荷为 \(e\) 的带电粒子,在一个位于原点的、磁荷为 \(g\) 的静态磁单极子的存在下运动。
该磁单极子产生的磁场为:
\[ \mathbf{B} = g \frac{\mathbf{r}}{r^3} \]
关键在于,这样的磁场不能用一个单一的、全局明确定义的矢势 \(\mathbf{A}\)(其中 \(\mathbf{B} = \nabla \times \mathbf{A}\))来描述。因为如果可以,那么 \(\nabla \cdot \mathbf{B} = \nabla \cdot (\nabla \times \mathbf{A}) = 0\) 将处处成立。然而,对于磁单极子,\(\nabla \cdot \mathbf{B} = 4\pi g \delta^3(\mathbf{r})\)(这里我们使用高斯单位制;\(\delta^3(\mathbf{r})\) 是狄拉克δ函数)。

为了克服这个问题,我们可以用至少两个区域(例如,“北半球” \(R_N\) 和“南半球” \(R_S\))来覆盖围绕磁单极子的球面,并在每个区域定义不同的矢势 \(\mathbf{A}_N\) 和 \(\mathbf{A}_S\)。这些矢势必须在它们各自的区域内产生相同的磁场 \(\mathbf{B}\)。

例如,\(\mathbf{A}_N\) 的一种选择(在除负z轴以外的任何地方都有效)是:
\[ \mathbf{A}_N = g \frac{1 - \cos\theta}{r\sin\theta} \hat{\mathbf{\phi}} \]
而 \(\mathbf{A}_S\) 的一种选择(在除正z轴以外的任何地方都有效)是:
\[ \mathbf{A}_S = -g \frac{1 + \cos\theta}{r\sin\theta} \hat{\mathbf{\phi}} \]
其中 \((\theta, \phi)\) 是通常的球坐标,\(\hat{\mathbf{\phi}}\) 是 \(\phi\) 方向的单位矢量。

\section{规范变换与波函数的单值性}

在重叠区域(例如赤道),\(\mathbf{A}_N\) 和 \(\mathbf{A}_S\) 必须通过一个规范变换相关联:
\[ \mathbf{A}_N - \mathbf{A}_S = \nabla \chi \]
计算这个差值:
\[ \mathbf{A}_N - \mathbf{A}_S = g \left( \frac{1 - \cos\theta}{r\sin\theta} + \frac{1 + \cos\theta}{r\sin\theta} \right) \hat{\mathbf{\phi}} = \frac{2g}{r\sin\theta} \hat{\mathbf{\phi}} \]
我们认识到 \(\nabla\phi = \frac{1}{r\sin\theta} \hat{\mathbf{\phi}}\)(这是坐标 \(\phi\) 的梯度,而不是某个标量函数 \(\phi(r,\theta,\phi)\))。所以,我们可以写出:
\[ \nabla \chi = \frac{2g}{r\sin\theta} \hat{\mathbf{\phi}} \]
这意味着规范函数 \(\chi\) 可以取为:
\[ \chi = 2g\phi \]
现在,考虑带电荷 \(e\) 的粒子的波函数 \(\psi\)。在区域 \(R_N\) 中,其哈密顿量包含 \(\mathbf{A}_N\),得到波函数 \(\psi_N\)。在区域 \(R_S\) 中,其哈密顿量包含 \(\mathbf{A}_S\),得到波函数 \(\psi_S\)。在重叠区域,这些波函数必须通过与规范变换相应的相位因子相关联:
\[ \psi_N = e^{i \frac{e}{\hbar c} \chi} \psi_S = e^{i \frac{e}{\hbar c} (2g\phi)} \psi_S \]
(我们在分母中使用 \(\hbar c\) 是因为采用高斯单位制;如果使用国际单位制,则为 \(\hbar\))。

为了使总波函数是明确定义且单值的,如果我们沿着重叠区域中围绕z轴的闭合路径(例如,沿着赤道)行进一圈,\(\phi\) 将改变 \(2\pi\)。波函数必须回到其初始值。
所以,当 \(\phi \rightarrow \phi + 2\pi\) 时,相位因子 \(e^{i \frac{e}{\hbar c} (2g\phi)}\) 的指数部分必须改变 \(2\pi i\) 的整数倍:
\[ \frac{e}{\hbar c} (2g \cdot 2\pi) = 2\pi n \]
其中 \(n\) 是一个整数。简化后得到:
\[ \frac{2eg}{\hbar c} = n \]
这意味着乘积 \(eg\) 的量子化:
\[ eg = n \frac{\hbar c}{2} \]
如果我们令 \(g_m\) 为磁单极子的磁荷,则 \(g_m = g\)。基本电荷是 \(e_0\)(通常用 \(e\) 表示,但为了区分我们这里明确用 \(e_0\))。因此,如果粒子的电荷是 \(e_0\),那么磁荷 \(g_m\) 必须是:
\[ g_m = n \frac{\hbar c}{2e_0} \]
这就是狄拉克量子化条件:磁荷必须是基本单位 \(\frac{\hbar c}{2e_0}\) 的整数倍。

\section{与贝里相位的联系}

带电粒子在具有矢势 \(\mathbf{A}\) 的区域中沿闭合路径 \(C\) 运动时获得的阿哈罗诺夫-玻姆相位是:
\[ \gamma_{AB} = \frac{e}{\hbar c} \oint_C \mathbf{A} \cdot d\mathbf{l} \]
这是一个几何相位,也是贝里相位的一个典型例子。

在我们的情境中,波函数单值性的要求意味着,当遍历任何不可收缩的回路(例如环绕狄拉克弦的回路,或两个规范区域重叠处的回路)时,累积的贝里相位必须是 \(2\pi\) 的整数倍。
通过包围磁单极子的闭合曲面 \(S\) 的贝里曲率(其形式为 \((e/\hbar c)\mathbf{B}\))的总“通量”是:
\[ \Phi_{\text{Berry}} = \frac{e}{\hbar c} \oint_S \mathbf{B} \cdot d\mathbf{S} \]
使用磁单极子的高斯定律(\(\oint_S \mathbf{B} \cdot d\mathbf{S} = 4\pi g\),在高斯单位制中):
\[ \Phi_{\text{Berry}} = \frac{e}{\hbar c} (4\pi g) \]
这个与磁单极子周围场构型的拓扑相关的总相位,必须以 \(2\pi\) 为单位进行量子化。
令 \(\Phi_{\text{Berry}} = 2\pi n\),其中 \(n\) 为某个整数:
\[ \frac{4\pi eg}{\hbar c} = 2\pi n \]
\[ \frac{2eg}{\hbar c} = n \]
这再次导出:
\[ g = n \frac{\hbar c}{2e} \]

因此,贝里相位形式论提供了一种现代且几何直观的方式来理解狄拉克的量子化条件。在磁单极子存在的情况下,量子力学的一致性(波函数的单值性)要求磁荷必须量子化,其基本磁荷单位为 \(g_0 = \frac{\hbar c}{2e_0}\)(其中 \(e_0\) 是基本电荷)。

总而言之,该论证表明,如果存在一个磁荷为 \(g\) 的磁单极子,并且同时也存在电荷 \(e\),那么量子力学要求它们的乘积 \(eg\) 是量子化的。这意味着磁荷本身必须以与 \(\frac{\hbar c}{2e}\) 成比例的离散单位存在。

\end{document}
