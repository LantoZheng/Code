\documentclass{article}

\usepackage{fancyhdr}
\usepackage{extramarks}
\usepackage{amsmath}
\usepackage{amsthm}
\usepackage{amsfonts}
\usepackage{tikz}
\usepackage[plain]{algorithm}
\usepackage{algpseudocode}
\usepackage{ctex}

\usetikzlibrary{automata,positioning}

%
% Basic Document Settings
%

\topmargin=-0.45in
\evensidemargin=0in
\oddsidemargin=0in
\textwidth=6.5in
\textheight=9.0in
\headsep=0.25in

\linespread{1.1}

\pagestyle{fancy}
\lhead{\hmwkAuthorName}
\chead{\hmwkClass\ (\hmwkClassInstructor\ \hmwkClassTime): \hmwkTitle}
\rhead{\firstxmark}
\lfoot{\lastxmark}
\cfoot{\thepage}

\renewcommand\headrulewidth{0.4pt}
\renewcommand\footrulewidth{0.4pt}

\setlength\parindent{0pt}

%
% Create Problem Sections
%

\newcommand{\enterProblemHeader}[1]{
    \nobreak\extramarks{}{Problem \arabic{#1} continued on next page\ldots}\nobreak{}
    \nobreak\extramarks{Problem \arabic{#1} (continued)}{Problem \arabic{#1} continued on next page\ldots}\nobreak{}
}

\newcommand{\exitProblemHeader}[1]{
    \nobreak\extramarks{Problem \arabic{#1} (continued)}{Problem \arabic{#1} continued on next page\ldots}\nobreak{}
    \stepcounter{#1}
    \nobreak\extramarks{Problem \arabic{#1}}{}\nobreak{}
}

\setcounter{secnumdepth}{0}
\newcounter{partCounter}
\newcounter{homeworkProblemCounter}
\setcounter{homeworkProblemCounter}{1}
\nobreak\extramarks{Problem \arabic{homeworkProblemCounter}}{}\nobreak{}

%
% Homework Problem Environment
%
% This environment takes an optional argument. When given, it will adjust the
% problem counter. This is useful for when the problems given for your
% assignment aren't sequential. See the last 3 problems of this template for an
% example.
%
\newenvironment{homeworkProblem}[1][-1]{
    \ifnum#1>0
        \setcounter{homeworkProblemCounter}{#1}
    \fi
    \section{Problem \arabic{homeworkProblemCounter}}
    \setcounter{partCounter}{1}
    \enterProblemHeader{homeworkProblemCounter}
}{
    \exitProblemHeader{homeworkProblemCounter}
}

%
% Homework Details
%   - Title
%   - Due date
%   - Class
%   - Section/Time
%   - Instructor
%   - Author
%

\newcommand{\hmwkTitle}{Homework\ \#6}
\newcommand{\hmwkDueDate}{\today}
\newcommand{\hmwkClass}{Thermo Dynamics}
\newcommand{\hmwkClassTime}{}
\newcommand{\hmwkClassInstructor}{Professor Jingdong Bao}
\newcommand{\hmwkAuthorName}{\textbf{郑晓旸}}

%
% Title Page
%

\title{
    \vspace{2in}
    \textmd{\textbf{\hmwkClass:\ \hmwkTitle}}\\
    \vspace{0.1in}\large{\textit{\hmwkClassInstructor\ \hmwkClassTime}}
    \vspace{3in}
}

\author{\hmwkAuthorName}
\date{}

\renewcommand{\part}[1]{\textbf{\large Part \Alph{partCounter}}\stepcounter{partCounter}\\}

%
% Various Helper Commands
%

% Useful for algorithms
\newcommand{\alg}[1]{\textsc{\bfseries \footnotesize #1}}

% For derivatives
\newcommand{\deriv}[1]{\frac{\mathrm{d}}{\mathrm{d}x} (#1)}

% For partial derivatives
\newcommand{\pderiv}[2]{\frac{\partial}{\partial #1} (#2)}

% Integral dx
\newcommand{\dx}{\mathrm{d}x}

% Alias for the Solution section header
\newcommand{\solution}{\textbf{\large Solution}}

% Probability commands: Expectation, Variance, Covariance, Bias
\newcommand{\E}{\mathrm{E}}
\newcommand{\Var}{\mathrm{Var}}
\newcommand{\Cov}{\mathrm{Cov}}
\newcommand{\Bias}{\mathrm{Bias}}

\begin{document}

\maketitle

\pagebreak

\begin{homeworkProblem}
   在体积 \(V\) 中有 \(N\) 个可区分的粒子,系统的能量为 \(E = \sum_{i=1}^{N} cp_i\),其中 \(c\) 为光速,\(p_i\) 为第 \(i\) 粒子的动量。若气体的温度为 \(T\),试求:
\begin{enumerate}
    \item[(a)] 物态方程
    \item[(b)] 内能
\end{enumerate}

    \textbf{Solution}
由于粒子是可区分的,并且系统的 \(N, V, T\) 给定,我们使用正则系综进行处理。
\subsection*{1. 单粒子配分函数 \(z\)}
单个粒子的能量为 \(\epsilon = cp\),其中 \(p = |\vec{p}|\) 是动量的大小。
单粒子配分函数 \(z\) 定义为:
\[z = \frac{1}{h^3} \int e^{-\beta \epsilon} d^3r d^3p\]
其中 \(\beta = \frac{1}{k_B T}\),\(k_B\) 是玻尔兹曼常数,\(h\) 是普朗克常数。
由于能量 \(\epsilon\) 与位置 \(\vec{r}\) 无关,对空间的积分 \(\int d^3r = V\)。
\[z = \frac{V}{h^3} \int e^{-\beta cp} d^3p\]
我们将动量空间积分转换到球坐标系,\(d^3p = 4\pi p^2 dp\):
\[z = \frac{4\pi V}{h^3} \int_0^{\infty} p^2 e^{-\beta cp} dp\]
为了计算这个积分,我们可以使用伽马函数 \(\Gamma(n) = \int_0^{\infty} x^{n-1}e^{-x}dx\)。对于 \(n=3\),\(\Gamma(3) = (3-1)! = 2! = 2\)。
令 \(x = \beta c p\),则 \(p = \frac{x}{\beta c}\),\(dp = \frac{dx}{\beta c}\)。
积分变为:
\[\int_0^{\infty} \left(\frac{x}{\beta c}\right)^2 e^{-x} \frac{dx}{\beta c} = \frac{1}{(\beta c)^3} \int_0^{\infty} x^2 e^{-x} dx = \frac{1}{(\beta c)^3} \Gamma(3) = \frac{2}{(\beta c)^3}\]
所以,单粒子配分函数为:
\[z = \frac{4\pi V}{h^3} \frac{2}{(\beta c)^3} = \frac{8\pi V}{(h\beta c)^3}\]
代入 \(\beta = \frac{1}{k_B T}\):
\[z = 8\pi V \left(\frac{k_B T}{hc}\right)^3\]
\subsection*{2. N粒子系统的配分函数 \(Z_N\)}
由于粒子是可区分的,\(N\) 粒子系统的总配分函数 \(Z_N\) 为:
\[Z_N = z^N = \left[ \frac{8\pi V}{(h\beta c)^3} \right]^N\]
\subsection*{3. 亥姆霍兹自由能 \(F\)}
亥姆霍兹自由能 \(F\) 与配分函数 \(Z_N\) 的关系是:
\[F = -k_B T \ln Z_N\]
\[F = -k_B T \ln \left[ \left( \frac{8\pi V}{(h\beta c)^3} \right)^N \right]\]
\[F = -N k_B T \ln \left( \frac{8\pi V}{(h\beta c)^3} \right)\]
\[F = -N k_B T \left[ \ln(8\pi V) - 3\ln(h\beta c) \right]\]
将 \(\beta = 1/(k_B T)\) 代入,可以写成:
\[F = -N k_B T \left[ \ln(8\pi V) - 3\ln\left(\frac{hc}{k_B T}\right) \right]\]
\[F = -N k_B T \left[ \ln(8\pi V) - 3\ln(hc) + 3\ln(k_B T) \right]\]
\subsection*{(a) 物态方程}
物态方程可以通过亥姆霍兹自由能对体积的偏导数得到压强 \(P\):
\[P = -\left(\frac{\partial F}{\partial V}\right)_{T,N}\]
\[P = -\frac{\partial}{\partial V} \left( -N k_B T \left[ \ln(8\pi V) - 3\ln(h\beta c) \right] \right)_{T,N}\]
\[P = N k_B T \frac{\partial}{\partial V} \left[ \ln(8\pi V) - 3\ln(h\beta c) \right]_{T,N}\]
由于 \(T\) 和 \(N\) 是常数,\(\ln(h\beta c)\) 项对 \(V\) 的导数为零。
\[P = N k_B T \frac{\partial}{\partial V} \ln(8\pi V) = N k_B T \frac{1}{8\pi V} (8\pi)\]
\[P = \frac{N k_B T}{V}\]
所以,物态方程为:
\[PV = N k_B T\]
\subsection*{(b) 内能 \(U\)}
内能 \(U\) (题目中用 \(E\) 表示总能量,这里我们用统计物理中常用的 \(U\) 表示内能) 可以通过以下关系得到:
\[U = -\left(\frac{\partial \ln Z_N}{\partial \beta}\right)_{V,N}\]
首先计算 \(\ln Z_N\):
\[\ln Z_N = N \ln z = N \ln \left( \frac{8\pi V}{(h\beta c)^3} \right) = N \left[ \ln(8\pi V) - 3\ln(h c) - 3\ln\beta \right]\]
然后对 \(\beta\) 求偏导:
\[\frac{\partial \ln Z_N}{\partial \beta} = N \frac{\partial}{\partial \beta} \left[ \ln(8\pi V) - 3\ln(h c) - 3\ln\beta \right]\]
\[\frac{\partial \ln Z_N}{\partial \beta} = N \left( 0 - 0 - \frac{3}{\beta} \right) = -\frac{3N}{\beta}\]
所以,内能 \(U\) 为:
\[U = -\left(-\frac{3N}{\beta}\right) = \frac{3N}{\beta}\]
代入 \(\beta = \frac{1}{k_B T}\):
\[U = 3N k_B T\]
\\
对于所描述的 \(N\) 个可区分的、能量为 \(\epsilon = cp_i\) 的粒子组成的系统:
\begin{enumerate}
    \item[(a)] 物态方程为:
    \[PV = N k_B T\]
    \item[(b)] 内能为:
    \[U = 3N k_B T\]
\end{enumerate}

\end{homeworkProblem}

\pagebreak

\begin{homeworkProblem}
   体积 \(V\) 内盛有两种组元的单原子分子混合理想气体,其物质的量分别为 \(\nu_1\) 和 \(\nu_2\),温度为 \(T\)。试用正则分布导出混合理想气体的内能、熵以及物态方程
    \textbf{Solution}
    我们将使用正则系综来处理这个混合理想气体系统。假设两种组元的单原子分子质量分别为 \(m_1\) 和 \(m_2\)。对应的粒子数分别为 \(N_1 = \nu_1 N_A\) 和 \(N_2 = \nu_2 N_A\),其中 \(N_A\) 是阿伏伽德罗常数。
\subsection*{1. 单种单原子理想气体的配分函数}
对于体积 \(V\) 中由 \(N\) 个同种单原子理想气体分子组成的系统,其能量仅为平动动能 \(\epsilon = \frac{p^2}{2m}\)。
单个粒子的配分函数 \(z\) 为:
\[z = \frac{1}{h^3} \int e^{-\beta \epsilon} d^3r d^3p = \frac{V}{h^3} \int_{-\infty}^{\infty} e^{-\beta p_x^2/(2m)} dp_x \int_{-\infty}^{\infty} e^{-\beta p_y^2/(2m)} dp_y \int_{-\infty}^{\infty} e^{-\beta p_z^2/(2m)} dp_z\]
利用高斯积分 \(\int_{-\infty}^{\infty} e^{-ax^2} dx = \sqrt{\frac{\pi}{a}}\),我们得到:
\[\int_{-\infty}^{\infty} e^{-\beta p_x^2/(2m)} dp_x = \sqrt{\frac{2\pi m}{\beta}}\]
因此,单粒子配分函数为:
\[z = \frac{V}{h^3} \left(\frac{2\pi m}{\beta}\right)^{3/2} = V \left(\frac{2\pi m k_B T}{h^2}\right)^{3/2}\]
其中 \(\beta = 1/(k_B T)\),\(k_B\) 是玻尔兹曼常数,\(h\) 是普朗克常数。
对于 \(N\) 个不可区分的同种粒子,其正则配分函数 \(Z_N\) 为:
\[Z_N = \frac{z^N}{N!} = \frac{1}{N!} \left[ V \left(\frac{2\pi m k_B T}{h^2}\right)^{3/2} \right]^N\]
\subsection*{2. 两种组元混合理想气体的配分函数 \(Z_{mix}\)}
对于两种组元的混合理想气体,由于不同组元的粒子是可区分的,而同组元的粒子是不可区分的,总配分函数是各组分配分函数的乘积:
\[Z_{mix} = Z_{N_1} Z_{N_2} = \frac{z_1^{N_1}}{N_1!} \frac{z_2^{N_2}}{N_2!}\]
其中:
\[z_1 = V \left(\frac{2\pi m_1 k_B T}{h^2}\right)^{3/2}\]
\[z_2 = V \left(\frac{2\pi m_2 k_B T}{h^2}\right)^{3/2}\]
所以,
\[Z_{mix} = \frac{1}{N_1! N_2!} \left[ V \left(\frac{2\pi m_1 k_B T}{h^2}\right)^{3/2} \right]^{N_1} \left[ V \left(\frac{2\pi m_2 k_B T}{h^2}\right)^{3/2} \right]^{N_2}\]
\subsection*{3. 亥姆霍兹自由能 \(F\)}
亥姆霍兹自由能 \(F = -k_B T \ln Z_{mix}\)。
\[\ln Z_{mix} = N_1 \ln z_1 + N_2 \ln z_2 - \ln N_1! - \ln N_2!\]
使用斯特林近似 \(\ln N! \approx N \ln N - N\):
\[F \approx -k_B T \left[ N_1 \left( \ln \left( \frac{V}{N_1} \left(\frac{2\pi m_1 k_B T}{h^2}\right)^{3/2} \right) + 1 \right) + N_2 \left( \ln \left( \frac{V}{N_2} \left(\frac{2\pi m_2 k_B T}{h^2}\right)^{3/2} \right) + 1 \right) \right]\]
将 \(N_1 = \nu_1 N_A\) 和 \(N_2 = \nu_2 N_A\) 代入,并使用 \(R = N_A k_B\):
\[F \approx -\sum_{i=1}^{2} \nu_i R T \left[ \ln \left( \frac{V}{\nu_i N_A} \left(\frac{2\pi m_i k_B T}{h^2}\right)^{3/2} \right) + 1 \right]\]
\subsection*{4. 内能 \(U\)}
内能 \(U = -\left(\frac{\partial \ln Z_{mix}}{\partial \beta}\right)_{V, N_1, N_2}\)。
\[\frac{\partial \ln Z_{mix}}{\partial \beta} = -\frac{3N_1}{2\beta} - \frac{3N_2}{2\beta} = -\frac{3(N_1+N_2)}{2\beta}\]
所以,内能为:
\[U = - \left( -\frac{3(N_1+N_2)}{2\beta} \right) = \frac{3}{2}(N_1+N_2)k_B T\]
用物质的量 \(\nu_1, \nu_2\) 和气体常数 \(R = N_A k_B\) 表示:
\[U = \frac{3}{2}(\nu_1 + \nu_2)RT\]
\subsection*{5. 物态方程}
压强 \(P = -\left(\frac{\partial F}{\partial V}\right)_{T, N_1, N_2} = k_B T \left(\frac{\partial \ln Z_{mix}}{\partial V}\right)_{T, N_1, N_2}\)。
\[\left(\frac{\partial \ln Z_{mix}}{\partial V}\right)_{T, N_1, N_2} = \frac{N_1}{V} + \frac{N_2}{V} = \frac{N_1+N_2}{V}\]
因此,压强为:
\[P = k_B T \frac{N_1+N_2}{V}\]
物态方程为:
\[PV = (N_1+N_2)k_B T\]
用物质的量表示:
\[PV = (\nu_1+\nu_2)RT\]
\subsection*{6. 熵 \(S\)}
熵 \(S = \frac{U-F}{T}\)。
\[S = k_B \left[ N_1 \left( \ln \left( \frac{V}{N_1} \left(\frac{2\pi m_1 k_B T}{h^2}\right)^{3/2} \right) + \frac{5}{2} \right) + N_2 \left( \ln \left( \frac{V}{N_2} \left(\frac{2\pi m_2 k_B T}{h^2}\right)^{3/2} \right) + \frac{5}{2} \right) \right]\]
用物质的量表示:
\[S = \sum_{i=1}^{2} \nu_i R \left[ \ln \left( \frac{V}{\nu_i N_A} \left(\frac{2\pi m_i k_B T}{h^2}\right)^{3/2} \right) + \frac{5}{2} \right]\]
\section*{总结}
对于体积 \(V\) 内,温度为 \(T\),由物质的量分别为 \(\nu_1\) 和 \(\nu_2\) 的两种单原子理想气体组成的混合物:
\begin{enumerate}
    \item[(a)] \textbf{内能 \(U\)}:
    \[U = \frac{3}{2}(\nu_1 + \nu_2)RT\]
    \item[(b)] \textbf{物态方程}:
    \[PV = (\nu_1 + \nu_2)RT\]
    \item[(c)] \textbf{熵 \(S\)}:
    \[S = \sum_{i=1}^{2} \nu_i R \left[ \ln \left( \frac{V}{\nu_i N_A} \left(\frac{2\pi m_i k_B T}{h^2}\right)^{3/2} \right) + \frac{5}{2} \right]\]
\end{enumerate}
其中 \(m_i\) 是第 \(i\) 种分子的质量,\(N_A\) 是阿伏伽德罗常数,\(R\) 是理想气体常数,\(k_B\) 是玻尔兹曼常数,\(h\) 是普朗克常数。
\end{homeworkProblem}
\newpage
\begin{homeworkProblem}
    求证:
$$
 \overline{\left(\frac{\mathrm{d} \ln D(E)}{\mathrm{d}E}\right)}_{\text{正则}} = \frac{1}{k_B T} 
$$
其中 $D(E)$ 是系统的能态密度,$k_B$ 是玻尔兹曼常数,$T$ 是正则系综的温度。上划线和下标“正则”表示在正则系综中对物理量取平均值。
    \\
    
    \textbf{Solution}
我们知道 \(P(E) = D(E)e^{-\beta E}/Z\),所以 \(\ln D(E) = \ln P(E) + \beta E - \ln(1/Z) = \ln P(E) + \beta E + \ln Z\)。
对能量 \(E\)求导:
\[ \frac{\mathrm{d} \ln D(E)}{\mathrm{d}E} = \frac{\mathrm{d} \ln P(E)}{\mathrm{d}E} + \beta \]
(\(\beta\) 和 \(Z\) 在此对 \(E\) 的导数中视为常数,因为这里的 \(E\) 是微观状态的能量,而不是系统的平均能量 \(U\))。
对上式两边取正则系综平均:
\[ \overline{\left(\frac{\mathrm{d} \ln D(E)}{\mathrm{d}E}\right)} = \overline{\left(\frac{\mathrm{d} \ln P(E)}{\mathrm{d}E} + \beta\right)} = \overline{\left(\frac{\mathrm{d} \ln P(E)}{\mathrm{d}E}\right)} + \overline{\beta} \]
由于 \(\beta\) 是由热库决定的常数,\(\overline{\beta} = \beta\)。
计算 \(\overline{\left(\frac{\mathrm{d} \ln P(E)}{\mathrm{d}E}\right)}\):
\[ \overline{\left(\frac{\mathrm{d} \ln P(E)}{\mathrm{d}E}\right)} = \int_{E_0}^{\infty} \left(\frac{1}{P(E)}\frac{\mathrm{d}P(E)}{\mathrm{d}E}\right) P(E) \mathrm{d}E = \int_{E_0}^{\infty} \frac{\mathrm{d}P(E)}{\mathrm{d}E} \mathrm{d}E \]
\[ = [P(E)]_{E_0}^{\infty} = P(\infty) - P(E_0) \]
由于 \(P(E)\) 是概率密度,它在 \(E \to \infty\) 时必须为0(保证归一化积分收敛),所以 \(P(\infty)=0\)。
因此,\(\overline{\left(\frac{\mathrm{d} \ln P(E)}{\mathrm{d}E}\right)} = -P(E_0)\)。
代回原式:
\[ \overline{\left(\frac{\mathrm{d} \ln D(E)}{\mathrm{d}E}\right)} = \beta - P(E_0) = \beta - \frac{D(E_0)e^{-\beta E_0}}{Z} \]
同样,在 \(P(E_0)=0\) (例如 \(D(E_0)=0\)) 的条件下:
\[ \overline{\left(\frac{\mathrm{d} \ln D(E)}{\mathrm{d}E}\right)}_{\text{正则}} = \beta = \frac{1}{k_B T} \]
证毕。
\end{homeworkProblem}
\newpage
\begin{homeworkProblem}
    证明正则系综的配分函数 \(Z(N,V,T)\) 满足:
\[ N \left(\frac{\partial \ln Z}{\partial N}\right)_{V,T} + V \left(\frac{\partial \ln Z}{\partial V}\right)_{N,T} = \ln Z \]
\\

\textbf{Solution}\\
证明正则系综的配分函数 \(Z(N,V,T)\) 满足:
\[ N \left(\frac{\partial \ln Z}{\partial N}\right)_{V,T} + V \left(\frac{\partial \ln Z}{\partial V}\right)_{N,T} = \ln Z \]
\section*{证明}
我们从正则系综中的热力学关系出发。
亥姆霍兹自由能 \(F\) 与正则配分函数 \(Z\) 的关系是:
\[ F(N,V,T) = -k_B T \ln Z(N,V,T) \]
其中 \(k_B\) 是玻尔兹曼常数。
由此可得:
\[ \ln Z = -\frac{F}{k_B T} \]
我们需要计算 \(\ln Z\) 对粒子数 \(N\) 和体积 \(V\) 的偏导数。
\subsection*{1. 对粒子数 \(N\) 的偏导数}
\[ \left(\frac{\partial \ln Z}{\partial N}\right)_{V,T} = \frac{\partial}{\partial N} \left(-\frac{F}{k_B T}\right)_{V,T} = -\frac{1}{k_B T} \left(\frac{\partial F}{\partial N}\right)_{V,T} \]
根据热力学定义,化学势 \(\mu\) 为:
\[ \mu = \left(\frac{\partial F}{\partial N}\right)_{V,T} \]
所以,
\begin{equation} \label{eq:dlnZ_dN}
\left(\frac{\partial \ln Z}{\partial N}\right)_{V,T} = -\frac{\mu}{k_B T}
\end{equation}
\subsection*{2. 对体积 \(V\) 的偏导数}
\[ \left(\frac{\partial \ln Z}{\partial V}\right)_{N,T} = \frac{\partial}{\partial V} \left(-\frac{F}{k_B T}\right)_{N,T} = -\frac{1}{k_B T} \left(\frac{\partial F}{\partial V}\right)_{N,T} \]
根据热力学定义,压强 \(P\) 为:
\[ P = -\left(\frac{\partial F}{\partial V}\right)_{N,T} \]
所以,
\begin{equation} \label{eq:dlnZ_dV}
\left(\frac{\partial \ln Z}{\partial V}\right)_{N,T} = -\frac{1}{k_B T} (-P) = \frac{P}{k_B T}
\end{equation}
现在,我们将式 \eqref{eq:dlnZ_dN} 和 \eqref{eq:dlnZ_dV} 代入待证明等式的左边 (LHS):
\[ \text{LHS} = N \left(-\frac{\mu}{k_B T}\right) + V \left(\frac{P}{k_B T}\right) \]
\[ \text{LHS} = \frac{PV - N\mu}{k_B T} \]
我们知道吉布斯自由能 \(G\) 的定义是 \(G = F + PV\)。
对于单组分系统,吉布斯自由能也可以表示为 \(G = N\mu\)。
因此,我们有:
\[ N\mu = F + PV \]
这意味着 \(PV - N\mu = -F\)。
将此关系代入 LHS 的表达式:
\[ \text{LHS} = \frac{-F}{k_B T} \]
因为 \(F = -k_B T \ln Z\),所以 \(-F = k_B T \ln Z\)。
代入上式:
\[ \text{LHS} = \frac{k_B T \ln Z}{k_B T} = \ln Z \]
这正是待证明等式的右边 (RHS)。
因此,我们证明了:
\[ N \left(\frac{\partial \ln Z}{\partial N}\right)_{V,T} + V \left(\frac{\partial \ln Z}{\partial V}\right)_{N,T} = \ln Z \]
证毕。
\end{homeworkProblem}
\newpage
\begin{homeworkProblem}
    (a) 求证巨正则系综的粒子数的方均涨落为
\[ \overline{(\Delta N)^2} = k_B T \frac{\partial \overline{N}}{\partial \mu} \]
(b) 据此求单原子分子和双原子分子理想气体的粒子数相对涨落。

\textbf{Solution}

\subsection*{(a) 求证巨正则系综的粒子数的方均涨落公式}
在巨正则系综中,系统的状态由化学势 \(\mu\)、体积 \(V\) 和温度 \(T\) 决定。巨配分函数 \(\Xi(\mu, V, T)\) 定义为:
\[ \Xi(\mu, V, T) = \sum_{N=0}^{\infty} \sum_{i} e^{-\beta(E_i(N,V) - \mu N)} = \sum_{N=0}^{\infty} e^{\beta \mu N} Z(N,V,T) \]
其中 \(\beta = 1/(k_B T)\),\(Z(N,V,T)\) 是包含 \(N\) 个粒子时的正则配分函数,\(E_i(N,V)\) 是系统粒子数为 \(N\) 时第 \(i\) 个能级的能量。
平均粒子数 \(\overline{N}\) 由下式给出:
\[ \overline{N} = \frac{1}{\Xi} \sum_{N=0}^{\infty} N e^{\beta \mu N} Z(N,V,T) \]
我们可以注意到,\(\sum_{N=0}^{\infty} N e^{\beta \mu N} Z(N,V,T) = \frac{1}{\beta} \left(\frac{\partial \Xi}{\partial \mu}\right)_{V,T}\)。
所以,
\begin{equation}
\overline{N} = \frac{1}{\beta \Xi} \left(\frac{\partial \Xi}{\partial \mu}\right)_{V,T} = k_B T \frac{1}{\Xi} \left(\frac{\partial \Xi}{\partial \mu}\right)_{V,T} = k_B T \left(\frac{\partial \ln \Xi}{\partial \mu}\right)_{V,T} \label{eq:N_avg}
\end{equation}
粒子数的方均值 \(\overline{N^2}\) 由下式给出:
\[ \overline{N^2} = \frac{1}{\Xi} \sum_{N=0}^{\infty} N^2 e^{\beta \mu N} Z(N,V,T) \]
我们可以注意到,\(\sum_{N=0}^{\infty} N^2 e^{\beta \mu N} Z(N,V,T) = \frac{1}{\beta^2} \left(\frac{\partial^2 \Xi}{\partial \mu^2}\right)_{V,T}\)。
所以,
\begin{equation}
\overline{N^2} = \frac{1}{\beta^2 \Xi} \left(\frac{\partial^2 \Xi}{\partial \mu^2}\right)_{V,T} = (k_B T)^2 \frac{1}{\Xi} \left(\frac{\partial^2 \Xi}{\partial \mu^2}\right)_{V,T} \label{eq:N2_avg}
\end{equation}
粒子数的方均涨落 \(\overline{(\Delta N)^2}\) 定义为:
\[ \overline{(\Delta N)^2} = \overline{(N - \overline{N})^2} = \overline{N^2 - 2N\overline{N} + (\overline{N})^2} = \overline{N^2} - 2\overline{N}\overline{N} + (\overline{N})^2 = \overline{N^2} - (\overline{N})^2 \]
代入 \(\overline{N}\) 和 \(\overline{N^2}\) 的表达式:
\[ \overline{(\Delta N)^2} = (k_B T)^2 \frac{1}{\Xi} \left(\frac{\partial^2 \Xi}{\partial \mu^2}\right)_{V,T} - \left(k_B T \frac{1}{\Xi} \left(\frac{\partial \Xi}{\partial \mu}\right)_{V,T}\right)^2 \]
\[ \overline{(\Delta N)^2} = (k_B T)^2 \left[ \frac{1}{\Xi} \left(\frac{\partial^2 \Xi}{\partial \mu^2}\right)_{V,T} - \left(\frac{1}{\Xi} \left(\frac{\partial \Xi}{\partial \mu}\right)_{V,T}\right)^2 \right] \]
括号中的项正是 \(\ln \Xi\) 对 \(\mu\) 的二阶偏导数:
\[ \left(\frac{\partial^2 \ln \Xi}{\partial \mu^2}\right)_{V,T} = \frac{\partial}{\partial \mu} \left(\frac{1}{\Xi} \left(\frac{\partial \Xi}{\partial \mu}\right)_{V,T}\right)_{V,T} = \frac{1}{\Xi} \left(\frac{\partial^2 \Xi}{\partial \mu^2}\right)_{V,T} - \frac{1}{\Xi^2} \left(\left(\frac{\partial \Xi}{\partial \mu}\right)_{V,T}\right)^2 \]
因此,
\begin{equation}
\overline{(\Delta N)^2} = (k_B T)^2 \left(\frac{\partial^2 \ln \Xi}{\partial \mu^2}\right)_{V,T} \label{eq:DeltaN2_lnXi}
\end{equation}
从式 \eqref{eq:N_avg},我们有 \(\overline{N} = k_B T \left(\frac{\partial \ln \Xi}{\partial \mu}\right)_{V,T}\)。对这个表达式两边再关于 \(\mu\) 求偏导(保持 \(V,T\) 不变):
\[ \left(\frac{\partial \overline{N}}{\partial \mu}\right)_{V,T} = \frac{\partial}{\partial \mu} \left(k_B T \left(\frac{\partial \ln \Xi}{\partial \mu}\right)_{V,T}\right)_{V,T} = k_B T \left(\frac{\partial^2 \ln \Xi}{\partial \mu^2}\right)_{V,T} \]
比较上式与式 \eqref{eq:DeltaN2_lnXi}:
\[ \overline{(\Delta N)^2} = k_B T \left[ k_B T \left(\frac{\partial^2 \ln \Xi}{\partial \mu^2}\right)_{V,T} \right] = k_B T \left(\frac{\partial \overline{N}}{\partial \mu}\right)_{V,T} \]
通常简写为:
\[ \overline{(\Delta N)^2} = k_B T \frac{\partial \overline{N}}{\partial \mu} \]
证毕。
\subsection*{(b) 据此求单原子分子和双原子分子理想气体的粒子数相对涨落}
对于经典理想气体(无论是单原子分子还是双原子分子),其巨配分函数 \(\Xi\) 可以表示为:
\[ \Xi = \sum_{N=0}^{\infty} \frac{(Z_1 e^{\beta \mu})^N}{N!} = \exp(Z_1 e^{\beta \mu}) \]
其中 \(Z_1(V,T)\) 是单个粒子的配分函数, \(Z_1 = Z_{1,\text{tr}} \cdot Z_{1,\text{int}}\)。
从巨配分函数,我们可以得到平均粒子数 \(\overline{N}\):
\[ \ln \Xi = Z_1 e^{\beta \mu} \]
\[ \overline{N} = k_B T \left(\frac{\partial \ln \Xi}{\partial \mu}\right)_{V,T} = k_B T \frac{\partial}{\partial \mu} (Z_1 e^{\beta \mu}) = k_B T (Z_1 \beta e^{\beta \mu}) = Z_1 e^{\beta \mu} \]
所以,对于理想气体,\(\overline{N} = \ln \Xi\)。
现在我们计算 \(\left(\frac{\partial \overline{N}}{\partial \mu}\right)_{V,T}\):
\[ \left(\frac{\partial \overline{N}}{\partial \mu}\right)_{V,T} = \frac{\partial}{\partial \mu} (Z_1 e^{\beta \mu}) = Z_1 \beta e^{\beta \mu} \]
因为 \(Z_1 e^{\beta \mu} = \overline{N}\) 且 \(\beta = 1/(k_B T)\),所以:
\[ \left(\frac{\partial \overline{N}}{\partial \mu}\right)_{V,T} = \beta \overline{N} = \frac{\overline{N}}{k_B T} \]
将此结果代入 (a) 中证明的公式:
\[ \overline{(\Delta N)^2} = k_B T \left(\frac{\partial \overline{N}}{\partial \mu}\right)_{V,T} = k_B T \left(\frac{\overline{N}}{k_B T}\right) = \overline{N} \]
粒子数的相对涨落定义为 \(\frac{\sqrt{\overline{(\Delta N)^2}}}{\overline{N}}\)。
因此,对于单原子分子和双原子分子理想气体:
\[ \frac{\sqrt{\overline{(\Delta N)^2}}}{\overline{N}} = \frac{\sqrt{\overline{N}}}{\overline{N}} = \frac{1}{\sqrt{\overline{N}}} \]
这个结果表明,对于经典理想气体,其粒子数的相对涨落与气体是单原子还是双原子无关,仅取决于平均粒子数的平方根的倒数。

\end{homeworkProblem}
\newpage
\begin{homeworkProblem}
    巨正则系统中,能量的方均涨落可以表示为:
$$
(\Delta E)^2_{\text{巨}} = -\frac{\partial \overline{E}}{\partial \beta} = k_B T^2 \left(\frac{\partial \overline{E}}{\partial T}\right)_{\mu,V}
$$
    其中 \(\overline{E}\) 是巨正则系综的平均能量,\(\beta = \frac{1}{k_B T}\) 是倒温度。证明这个公式,并解释其物理意义。

    \textbf{Solution}
从巨正则系综的基本定义出发。巨正则系综的配分函数为:
\[\Xi(\mu,V,T) = \sum_{N=0}^{\infty}\sum_{i}e^{-\beta(E_i-\mu N)}\]
平均能量可以通过巨配分函数 $\Xi$ 表示为:
\[\overline{E} = -\frac{\partial \ln \Xi}{\partial \beta}\bigg|_{\mu,V}\]
能量的平方平均值可以表示为:
\[\overline{E^2} = \frac{\partial^2 \ln \Xi}{\partial \beta^2}\bigg|_{\mu,V}\]
因此,能量的方均涨落为:
\[(\Delta E)^2_{\text{巨}} = \overline{E^2} - \overline{E}^2 = \frac{\partial^2 \ln \Xi}{\partial \beta^2} - \left(\frac{\partial \ln \Xi}{\partial \beta}\right)^2\]
通过数学运算,可以证明:
\[(\Delta E)^2_{\text{巨}} = -\frac{\partial \overline{E}}{\partial \beta}\bigg|_{\mu,V}\]
再利用求导链式法则,考虑到 $\beta = \frac{1}{k_B T}$,所以 $\frac{\partial \beta}{\partial T} = -\frac{1}{k_B T^2}$,因此:
\[\frac{\partial \overline{E}}{\partial \beta}\bigg|_{\mu,V} = \frac{\partial \overline{E}}{\partial T}\bigg|_{\mu,V} \cdot \frac{\partial T}{\partial \beta}\bigg|_{\mu,V} = -k_B T^2 \frac{\partial \overline{E}}{\partial T}\bigg|_{\mu,V}\]
带入前面的方均涨落公式,最终得到:
\[(\Delta E)^2_{\text{巨}} = -\frac{\partial \overline{E}}{\partial \beta}\bigg|_{\mu,V} = k_B T^2 \left(\frac{\partial \overline{E}}{\partial T}\right)_{\mu,V}\]
\subsection*{物理意义与应用}
这个公式揭示了巨正则系统中能量涨落与系统热容量之间的关系。注意到定容热容量可以表示为:
\[C_V = \left(\frac{\partial \overline{E}}{\partial T}\right)_V\]
因此,对于巨正则系统,能量涨落可以重写为:
\[(\Delta E)^2_{\text{巨}} = k_B T^2 C_{\mu,V}\]

\end{homeworkProblem}
\end{document}
