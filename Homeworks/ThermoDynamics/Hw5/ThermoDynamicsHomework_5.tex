\documentclass{article}
\usepackage{ctex} % For Chinese characters
\usepackage{amsmath} % For mathematical environments and symbols like \mathbf
\usepackage{amssymb} % For additional math symbols
\usepackage{physics} % Optional, provides useful physics macros like \exp

\begin{document}

\section*{5.1}
(30分) (a) 振子能级 $\epsilon_n = (n+1/2) h\nu$.

处于每一能级上的概率为 $P_n = \exp(-\beta \epsilon_n)$.

振子处于第一激发态与基态的概率之比:
\[ \frac{P_1}{P_0} = \frac{\exp(-\beta \epsilon_1)}{\exp(-\beta \epsilon_0)} = \exp(-\beta(\epsilon_1 - \epsilon_0)) = \exp(-\beta h\nu). \]
(b) 平均能量
\[ \bar{\epsilon} = \frac{\epsilon_0 P_0 + \epsilon_1 P_1}{P_0 + P_1}. \]

\section*{5.4}
配分函数 $Z_N = (Z_1)^N$.
\[ Z_1 = \frac{1}{h^3} \iint e^{-\frac{1}{2k_B T}(P_x^2+P_y^2+P_z^2)} dx dy dz dP_x dP_y dP_z. \]
假设体积为 $V$,
\[ Z_1 = \frac{V}{h^3} \int_0^\infty e^{-\frac{p^2}{2m k_B T}} 4\pi p^2 dp = \frac{V}{h^3} (2\pi m k_B T)^{3/2}. \]
\[ Z_N = \left(\frac{V}{h^3} (2\pi m k_B T)^{3/2}\right)^N. \]
内能 $U = -\pdv{}{\beta} \ln Z_N = -\pdv{}{\beta} (N \ln Z_1) = N \frac{3}{2} k_B T$.
定容热容量 $C_V = \pdv{U}{T}_V = \frac{3}{2} N k_B$.

\section*{5.2}
解: $Z = \frac{1}{h^3} \iint e^{-\frac{1}{2m k_B T}(\mathbf{p}^2) - \beta mgz} d^3p d^3r$.
假设 $z$ 积分从 $0$ 到 $L$:
\[ Z_1 = \left(\frac{1}{h^3} \int e^{-\frac{\mathbf{p}^2}{2m k_B T}} d^3p\right) \left(\int_0^L e^{-\beta mgz} dz\right) \]
\[ Z_1 = \frac{(2\pi m k_B T)^{3/2}}{h^3} \left[-\frac{1}{\beta mg} e^{-\beta mgz}\right]_0^L = \frac{(2\pi m k_B T)^{3/2}}{h^3} \frac{1}{\beta mg}(1 - e^{-\beta mgL}). \]
对于 $N$ 个粒子, $Z = (Z_1)^N$.
内能 $U = -N \pdv{}{\beta} \ln Z_1$.
\[ \ln Z_1 = \frac{3}{2} \ln(2\pi m k_B) + \frac{3}{2} \ln T - 3 \ln h + \ln V - \ln(\beta mg) + \ln(1 - e^{-\beta mgL}). \]
Using $\beta = 1/k_B T$,
\[ \ln Z_1 = C + \frac{3}{2} \ln T + \ln T - \ln(mg/k_B) + \ln(1 - e^{-mgL/k_B T}). \]
\[ U = N k_B T^2 \pdv{}{T} \ln Z_1 = N k_B T^2 \left( \frac{3}{2T} + \frac{1}{T} + \frac{1}{1 - e^{-mgL/k_B T}} e^{-mgL/k_B T} \frac{mgL}{k_B T^2} \right). \]
\[ U = N \left( \frac{5}{2} k_B T + \frac{mgL e^{-mgL/k_B T}}{1 - e^{-mgL/k_B T}} \right) = N \left( \frac{5}{2} k_B T + \frac{mgL}{e^{mgL/k_B T} - 1} \right). \]
图像中的公式为: $U = U_0 + NK_BT - \frac{NmgH}{e^{\beta mgH}-1}$. Assuming $H=L$ and $U_0$ includes $\frac{5}{2}NK_BT$, or there is a mistake in the image. I will transcribe the formula as it appears:
\[ U = U_0 + NK_BT - \frac{NmgH}{e^{\beta mgH}-1}. \]
气体热容量 $C_V = \pdv{U}{T}_V$.
\[ C_V = C_V^0 + NK_B - K_BT \frac{N(mgH)^2 e^{\beta mgH}}{(e^{\beta mgH}-1)^2}. \]

\section*{5.3}
分子能量 $\epsilon = \frac{1}{2m}(p_x^2 + p_y^2 + p_z^2) + mgz$.
由能均分定理 $\bar{\epsilon}_{动} = \frac{3}{2} k_B T$.
$\bar{\epsilon}_{势} = \overline{mgz}$.
\[ \overline{mgz} = \frac{\int mgz e^{-\beta \epsilon} d\tau}{\int e^{-\beta \epsilon} d\tau}. \]
积分代表全空间积分. Assuming integration over $z$ from 0 to $L$.
\[ \overline{mgz} = \frac{\int_0^L mgz e^{-\beta mgz} dz}{\int_0^L e^{-\beta mgz} dz}. \]
图像中的结果是:
\[ \overline{mgz} = k_B T \frac{mgL}{e^{\beta mgL}-1}. \]

\section*{5.6}
配分函数 $Z = \sum_{n=0}^\infty e^{-\beta \epsilon_n}$.
$\epsilon_n = h\nu (n + 1/2)$.
\[ Z = \sum_{n=0}^\infty e^{-\beta h\nu (n+1/2)} = e^{-\beta h\nu/2} \sum_{n=0}^\infty (e^{-\beta h\nu})^n = \frac{e^{-\beta h\nu/2}}{1 - e^{-\beta h\nu}}. \]
收至平均能量 $U = -\pdv{}{\beta} \ln Z$.
\[ \ln Z = -\frac{\beta h\nu}{2} - \ln(1 - e^{-\beta h\nu}). \]
\[ U = \frac{h\nu}{2} + \frac{h\nu e^{-\beta h\nu}}{1 - e^{-\beta h\nu}} = \frac{h\nu}{2} + \frac{h\nu}{e^{\beta h\nu} - 1}. \]
\[ U = h\nu\left(\frac{1}{2}\right) + \frac{h\nu}{e^{\beta h\nu}-1}. \]

\section*{5.7}
用: 哈密顿量 $H = \frac{1}{2I} P_\theta^2 + \frac{1}{2I \sin^2\theta} P_\phi^2$.
能谱面为: $P_\theta^2/(2IE) + P_\phi^2/(2IE\sin^2\theta) = 1$.
相体积为: $\Sigma(E) = \int_0^\pi d\theta \int_0^{2\pi} d\phi \int \int_{H \le E} dP_\theta dP_\phi$.
\[ \Sigma(E) = \int_0^\pi d\theta \int_0^{2\pi} d\phi (2\pi \sqrt{2IE} \sqrt{2IE\sin^2\theta}) = \int_0^\pi d\theta \int_0^{2\pi} d\phi (4\pi IE |\sin\theta|). \]
\[ \Sigma(E) = (2\pi)(4\pi IE) \int_0^\pi \sin\theta d\theta = 8\pi^2 IE [-\cos\theta]_0^\pi = 8\pi^2 IE (1 - (-1)) = 16\pi^2 IE. \]
Wait, the integral limit in the image is $H=E$, not $H \le E$. This suggests calculation of the area of the energy shell, not the volume of phase space up to E.
Let's re-calculate based on the formula given in the image:
\[ \Sigma(E) = \int_0^\pi d\theta \int_0^{2\pi} d\phi (2\pi IE |\sin\theta|). \]
\[ \Sigma(E) = (2\pi) \int_0^\pi (2\pi IE \sin\theta) d\theta = 4\pi^2 IE \int_0^\pi \sin\theta d\theta = 4\pi^2 IE (2) = 8\pi^2 IE. \]
This matches the image. This $\Sigma(E)$ is related to the density of states. $\rho(E) = \frac{1}{h^2} \frac{d\Sigma}{dE} = \frac{8\pi^2 I}{h^2}$.
配分函数 $Z = \int_0^\infty e^{-\beta E} \rho(E) dE = \int_0^\infty e^{-\beta E} \frac{8\pi^2 I}{h^2} dE = \frac{8\pi^2 I}{h^2 \beta}$.
Alternatively, using the momentum integrals directly:
\[ Z = \frac{1}{h^2} \int_0^\pi d\theta \int_0^{2\pi} d\phi \int_{-\infty}^\infty dP_\theta \int_{-\infty}^\infty dP_\phi e^{-\beta (\frac{P_\theta^2}{2I} + \frac{P_\phi^2}{2I \sin^2\theta})}. \]
\[ Z = \frac{1}{h^2} \int_0^\pi d\theta \int_0^{2\pi} d\phi \sqrt{\frac{2\pi I}{\beta}} \sqrt{\frac{2\pi I \sin^2\theta}{\beta}} = \frac{1}{h^2} \int_0^\pi d\theta \int_0^{2\pi} d\phi \frac{2\pi I |\sin\theta|}{\beta}. \]
\[ Z = \frac{1}{h^2} (2\pi) \frac{2\pi I}{\beta} \int_0^\pi \sin\theta d\theta = \frac{4\pi^2 I}{h^2 \beta} (2) = \frac{8\pi^2 I}{h^2 \beta}. \]
This matches the image.
故 $U = NK_BT$.
$C_V = NK_B$.

\section*{5.8}
(a) $P_{平行} = \frac{1}{Z} e^{\mu H / k_B T}$, $P_{反平行} = \frac{1}{Z} e^{-\mu H / k_B T}$.
配分函数 $Z = e^{\mu H / k_B T} + e^{-\mu H / k_B T}$.
(b) 平均磁矩 $\bar{\mu} = P_{平行} \cdot \mu - P_{反平行} \cdot \mu$.
\[ \bar{\mu} = \frac{e^{\beta \mu H}}{e^{\beta \mu H} + e^{-\beta \mu H}} \mu - \frac{e^{-\beta \mu H}}{e^{\beta \mu H} + e^{-\beta \mu H}} \mu = \mu \frac{e^{\beta \mu H} - e^{-\beta \mu H}}{e^{\beta \mu H} + e^{-\beta \mu H}} = \mu \tanh\left(\frac{\mu H}{k_B T}\right). \]
(c) 磁化强度 $M = N_0 \bar{\mu} = N_0 \mu \tanh\left(\frac{\mu H}{k_B T}\right)$.
高温时, $\mu H \ll k_B T$.
$\tanh(x) \approx x$.
$M \approx N_0 \mu \left(\frac{\mu H}{k_B T}\right) = N_0 \frac{\mu^2 H}{k_B T}$.
图像中的结果是 $M \approx N_0 \frac{2\mu H}{k_B T}$. I will transcribe the image result.
\[ M \approx N_0 \frac{2\mu H}{k_B T}. \]
低温时 $\tanh\left(\frac{\mu H}{k_B T}\right) \approx 1$.
$M \approx N_0 \mu$.

\end{document}
