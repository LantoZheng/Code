\documentclass{article}

\usepackage{ctex}
\usepackage{fancyhdr}
\usepackage{extramarks}
\usepackage{amsmath}
\usepackage{amsthm}
\usepackage{amsfonts}
\usepackage{tikz}
\usepackage[plain]{algorithm}
\usepackage{algpseudocode}

\usetikzlibrary{automata,positioning}

%
% Basic Document Settings
%

\topmargin=-0.45in
\evensidemargin=0in
\oddsidemargin=0in
\textwidth=6.5in
\textheight=9.0in
\headsep=0.25in

\linespread{1.1}

\pagestyle{fancy}
\lhead{\hmwkAuthorName}
\chead{\hmwkClass\ (\hmwkClassInstructor\ \hmwkClassTime): \hmwkTitle}
\rhead{\firstxmark}
\lfoot{\lastxmark}
\cfoot{\thepage}

\renewcommand\headrulewidth{0.4pt}
\renewcommand\footrulewidth{0.4pt}

\setlength\parindent{0pt}

%
% Create Problem Sections
%

\newcommand{\enterProblemHeader}[1]{
    \nobreak\extramarks{}{Problem \arabic{#1} continued on next page\ldots}\nobreak{}
    \nobreak\extramarks{Problem \arabic{#1} (continued)}{Problem \arabic{#1} continued on next page\ldots}\nobreak{}
}

\newcommand{\exitProblemHeader}[1]{
    \nobreak\extramarks{Problem \arabic{#1} (continued)}{Problem \arabic{#1} continued on next page\ldots}\nobreak{}
    \stepcounter{#1}
    \nobreak\extramarks{Problem \arabic{#1}}{}\nobreak{}
}

\setcounter{secnumdepth}{0}
\newcounter{partCounter}
\newcounter{homeworkProblemCounter}
\setcounter{homeworkProblemCounter}{1}
\nobreak\extramarks{Problem \arabic{homeworkProblemCounter}}{}\nobreak{}

%
% Homework Problem Environment
%
% This environment takes an optional argument. When given, it will adjust the
% problem counter. This is useful for when the problems given for your
% assignment aren't sequential. See the last 3 problems of this template for an
% example.
%
\newenvironment{homeworkProblem}[1][-1]{
    \ifnum#1>0
        \setcounter{homeworkProblemCounter}{#1}
    \fi
    \section{Problem \arabic{homeworkProblemCounter}}
    \setcounter{partCounter}{1}
    \enterProblemHeader{homeworkProblemCounter}
}{
    \exitProblemHeader{homeworkProblemCounter}
}

%
% Homework Details
%   - Title
%   - Due date
%   - Class
%   - Section/Time
%   - Instructor
%   - Author
%

\newcommand{\hmwkTitle}{第1次作业}
\newcommand{\hmwkDueDate}{2024.3.4}
\newcommand{\hmwkClass}{热学}
\newcommand{\hmwkClassTime}{Section A}
\newcommand{\hmwkClassInstructor}{周欣}
\newcommand{\hmwkAuthorName}{\textbf{郑晓旸} \and \textbf{202111030007}}

%
% Title Page
%

\title{
    \vspace{2in}
    \textmd{\textbf{\hmwkClass:\ \hmwkTitle}}\\
    \normalsize\vspace{0.1in}\small{Due\ on\ \hmwkDueDate\ }\\
    \vspace{0.1in}\large{\textit{\hmwkClassInstructor\ \hmwkClassTime}}
    \vspace{3in}
}

\author{\hmwkAuthorName}
\date{}

\renewcommand{\part}[1]{\textbf{\large Part \Alph{partCounter}}\stepcounter{partCounter}\\}

%
% Various Helper Commands
%

% Useful for algorithms
\newcommand{\alg}[1]{\textsc{\bfseries \footnotesize #1}}

% For derivatives
\newcommand{\deriv}[1]{\frac{\mathrm{d}}{\mathrm{d}x} (#1)}

% For partial derivatives
\newcommand{\pderiv}[2]{\frac{\partial}{\partial #1} (#2)}

% Integral dx
\newcommand{\dx}{\mathrm{d}x}

% Alias for the Solution section header
\newcommand{\solution}{\textbf{\large Solution}}

% Probability commands: Expectation, Variance, Covariance, Bias
\newcommand{\E}{\mathrm{E}}
\newcommand{\Var}{\mathrm{Var}}
\newcommand{\Cov}{\mathrm{Cov}}
\newcommand{\Bias}{\mathrm{Bias}}

\begin{document}

\maketitle

\pagebreak

\begin{homeworkProblem}
    道尔顿提出一种温标:规定理想气体体积的相对增量正比于温度的增量,在标准大气压下,规定水的冰点温度为零度,沸水温度为100度。试用摄氏度\(t\)来表示道尔顿温标的温度\(\tau\)
    \\\\
    \solution
    \\
    设大气压强为\(P_{atm}\),\(T_0=273.15K\)为摄氏0度,\(T_{100}=373.15K\)为摄氏100度,\(t\)为摄氏度,\(\tau\)为道尔顿温标的温度。\(V_0\)为理想气体在摄氏0度下的体积。
    \\
    由题意可得:
    \begin{align*}
        P_{atm} V_0 &= \nu R T_0
        \\
        P_{atm} V_{100} &= \nu R T_{100}
    \end{align*}        
    由定义:
    \[
        \tau=\frac{V-V_0}{V_{100}-V_0} \times 100
    \]
    并带入气体体积和摄氏度的关系:
    \[
        V=\frac{\nu R(t+273.15)}{P_{atm}}\]
    得到道尔顿温度与摄氏温度的转化关系:
    \begin{align*}
        \tau&=\frac{\frac{\nu R(t+273.15)}{P_{atm}}-V_0}{V_{100}-V_0}\times 100
        \\
        &=\frac{\frac{\nu R(t+273.15)}{P_{atm}}-\nu R T_0}{\nu R T_{100}-\nu R T_0}\times 100
        \\
        &=\frac{t}{100}\times 100=t
    \end{align*}




\end{homeworkProblem}

\begin{homeworkProblem}
    国际实用温标(1990年)规定:用于13.803 (平衡氢三相点)到
961.78°C(银在0.101MPa下的凝固点)的标准测量仪器是铂电阻温
度计。设铂电阻在0°C及°C时电阻的值分别为\(R_0\)及\(R(t)\),定义
\(W(t)=R(t)/R_0\),且在不同测温区内\(W(t)\)对\(t\)的函数关系是不同的,
在上述测温范围内大致有\(W(t)=1+At+Bt^2\)
若在0.101MPa下,对应于冰的熔点、水的沸点、硫的沸点(温度为444.67°C)电阻的阻值分别为11.000Ω、15.247Ω、28.887Ω,
试确定上式中的常数A和B。(正确标注常数A和B的单位)
\\\\
\solution
\\
由题意可得:
\begin{align*}
    W(0)&=1
    \\
    W(100)&=1+100^\circ C \cdot A+10000^\circ C^2\cdot B
    \\
    W(444.67)&=1+444.67^\circ C\cdot A+(444.67^\circ C)^2\cdot B
\end{align*}
同时代入电阻的阻值:
\begin{align*}
    11/11&=R_0/R_0=1
    \\
    15.247/11&=R_{100}/R_0=1+100^\circ C \cdot A+10000^\circ C^2 \cdot B
    \\
    28.887/11&=R_0=1+444.67^\circ C \cdot A+(444.67^\circ C)^2\cdot B
\end{align*}
得到A、B、C的解:

    \[\begin{cases}
    A=3.9201 ^\circ C^{-1}\\
    B=-5.9205\times 10^{-7^\circ} C^{-2}
\end{cases}\]
\end{homeworkProblem}
\\\\\\

\begin{homeworkProblem}
    求氧气压强为0.1MPa、温度为27°C时的密度
    \\\\
    \solution
    \\
    由理想气体状态方程:
    \[
        PV=\nu RT
    \]
    导出:
    \[
        \rho = \frac{M_{mol}P}{RT}
    \]
    以及氧气的摩尔质量为32g/mol,和标准状态(0℃,101kPA)下每摩尔理想气体体积\\
    \[V^{\Theta}=22.4L \ \ M_{mol}=32g\]    
    得到氧气在标准状态下密度:\(\rho^{\Theta}=M_{mol}/V^{\Theta}\)
    代入氧气压强为0.1MPa、温度为27°C时的密度:
    \begin{align*}
    \rho = \frac{M_{mol}P}{RT}&=\frac{M_{mol}P^{\Theta}}{RT^{\Theta}}\frac{T^{\Theta}P}{TP^{\Theta}}=\rho^{\Theta}\frac{T^{\Theta}P}{TP^{\Theta}}\\
    \rho &= 0.707g/L    
    \end{align*}
\end{homeworkProblem}



\begin{homeworkProblem}
    容积为10L的瓶内贮有氢气,因开关损坏而漏气,在温度为7.0°C时,压
    强计的读数为50atm 。过了些时候,温度上升为17°C,压强计的读数未变,
    问漏去了多少质量的氢?
    \\\\
    \solution
    \\  
    由理想气体状态方程:
    \begin{align*}
        \nu &= \frac{PV}{RT}
        \\
        m&=\nu M_{mol}=\frac{PV}{RT}M_{mol}
        \\
        \Delta m &= \frac{M_{mol}PV}{R}\Delta (\frac{1}{T})\\
        &= 727.21g
    \end{align*}    

\end{homeworkProblem}



\begin{homeworkProblem}
    现有一气球,体积为\(8.7m^3\),冲入温度为15°C的氢气。当温度升高到37°C时,维持其气压\(p\)及体积\(V\)不变,气球中多余的氢气跑掉了,而
使其质量减少了 ,试求氢气在0°C、压强\(p\)下的密度
\\\\
\solution
\\
由理想气体状态方程:
\begin{align*}
    PV&=\nu RT
    \\
    \nu &= \frac{VP}{RT}
    \\
    \Delta \nu &= \frac{Vp}{R}\Delta \frac{1}{T}
    \\
    \frac{Vp}{R}&=\frac{\Delta \nu}{\Delta \frac{1}{T}}
    \\
    \rho &= \frac{M_{mol}p}{RT}=\frac{pV}{R}\frac{1}{VT}M_{mol}\\
    &=\frac{\Delta m}{\Delta \frac{1}{T}}\frac{1}{VT}\\
    &=0.00889 kg/m^3
\end{align*}
\end{homeworkProblem}
\\\\\\

\begin{homeworkProblem}
    一个不漏气的氢气球可以膨胀,保持气球内外压强相等,随着气球不断升高,大气压强不断减少,氢气球就不断膨胀。如果忽略掉大气温度和摩尔质量随高度的变化,假设氢气和大气都是理想气体,试问:
    \\
    (A)气球在上升过程中所受的净浮力(浮力-氢气自身重量)是否变化?\\
    (B)在标准状态下给氢气球充气后,体积\(V=556m^3\)球壳的体积可以忽略不计,球壳的质量\(m=12.5kg\),在0℃的等温大气中,这个气球还可悬挂多重物体而不坠下?已知\(M_{h_2}^{mol}=2g/mol,M_{air}^{mol}=28.9g/mol\)
    \\\\
    \solution 
    \\\\
    \part
    \\
    由理想气体状态方程:
    \begin{align*}
        PV &= \nu RT \\
        PM_{mol} &= \rho RT
    \end{align*}
    计算浮力:
    \begin{align*}
        F_b &= \rho_{air}V_{air}g-\rho_{h_2}V_{h_2}g\\
        &=\frac{PM_{air}^{mol}V}{RT}g-\frac{PM_{h_2}^{mol}V}{RT}g\\
        &=\nu_{h2}(M_{air}^{mol}-M_{h_2}^{mol})g\\
    \end{align*}
    由上式可知,气球在上升过程中所受的净浮力不变。
    \\\\
    \part
    \\
    受力平衡方程:
    \begin{align*}
        F_{\text{浮}}&=Mg+mg+\rho_{h2}gV\\
        (M_{air}^{mol}-M_{h_2}^{mol})gV&=(M+m)g\\
        M&=\frac{PV}{RT}(M_{air}^{mol}-M_{h_2}^{mol})V-m\\
        &=664.6kg
    \end{align*}
\end{homeworkProblem}
\\\\\\
\begin{homeworkProblem}
    某混合气体由\(H_2,CO_2,CH_4,C_2H_4\)组成,其摩尔质量分别为\(2g\cdot mol^{-1},44g\cdot mol^{-1},16g\cdot mol^{-1},28g\cdot mol^{-1}\)。
    在20℃时,上述四种气体对应的分压强为\(200mmHg,150mmHg,320mmHg,105mmHg\)\\
    (A)求混合气体的总压强\\
    (B)求氢气的质量分数\\
    \\\\
    \solution
    \\
    \part
    \\
    \[
        P=\Sigma p_i = 775 mmHg
    \]
    \part
    \\
    \[
        \eta_{H_2}^m=\frac{M_{H_2}^{mol}p_{H_2}}{\sum_i M_i^{mol}p_i}=\frac{20}{753} 
    \]
\end{homeworkProblem}
\begin{homeworkProblem}
    所谓某混合理想气体中各组分的体积百分比,是指各组分单独处在与混合理想气体相同压强和温度状态下,其体积在混合理想气体中所占的百分数。
    空气可当作理想气体,空气中几种主要组分的体积百分比是:氮78\%,氧21\%,氩1\% \\
    求标准情况下空气的密度\\
    已知:\(M_{N_2}^{mol}=28.0g/mol,M_{O_2}^{mol}=32.0g/mol,M_{Ar}^{mol}=39.9g/mol\)
    \\\\
    \solution
    \\
    \begin{align*}
        \rho &= \frac{M_{N_2}^{mol}p_{N_2}+M_{O_2}^{mol}p_{O_2}+M_{Ar}^{mol}p_{Ar}}{RT}\\
        &= \frac{P(M_{N_2}^{mol}\eta_{N_2}+M_{O_2}^{mol}p\eta_{O_2}+M_{Ar}^{mol}\eta_{Ar})}{RT}\\
        &= \frac{M_{N_2}^{mol}\eta_{N_2}+M_{O_2}^{mol}p\eta_{O_2}+M_{Ar}^{mol}\eta_{Ar}}{V_{mol}^{\Theta}}\\
        &=1.29\ kg/m^3
    \end{align*}
\end{homeworkProblem}
\begin{homeworkProblem}
    一段开口、横截面积处处相等的长管中冲有压强为p的空气。先对管子加热,使其形成从开口温度1000K均匀变为200K的温度分布,然后把管子开口段密封,再使整体温度降为100K。
    求管中最终压强。
    \\\\
    \solution
    \\
    由理想气体状态方程:
    \begin{align*}
        \nu &= \int d\nu \\
        &= \int \frac{P}{RT}dV \\
        &= \int_0^l \frac{P}{RT}Adx \\
        &= \int_0^l \frac{PA}{R(T_{200}+x/l(T_{1000}-T_{200}))}dx\\
        &=\frac{PA}{R}\frac{l}{T_{1000}-T_{200}}\ln\frac{T_{1000}}{T_{200}}
    \end{align*}
    将上式代入理想气体状态方程:
    \[
        P'=\frac{\nu RT'}{Al}=P\frac{T'}{T_{1000}-T_{200}}\ln{\frac{T_{1000}}{T_{200}}}=\frac{\ln{5}}{8}P
    \]
\end{homeworkProblem}
\end{document}
