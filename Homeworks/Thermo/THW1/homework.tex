\documentclass{article}

\usepackage{ctex}
\usepackage{fancyhdr}
\usepackage{extramarks}
\usepackage{amsmath}
\usepackage{amsthm}
\usepackage{amsfonts}
\usepackage{tikz}
\usepackage[plain]{algorithm}
\usepackage{algpseudocode}

\usetikzlibrary{automata,positioning}

%
% Basic Document Settings
%

\topmargin=-0.45in
\evensidemargin=0in
\oddsidemargin=0in
\textwidth=6.5in
\textheight=9.0in
\headsep=0.25in

\linespread{1.1}

\pagestyle{fancy}
\lhead{\hmwkAuthorName}
\chead{\hmwkClass\ (\hmwkClassInstructor\ \hmwkClassTime): \hmwkTitle}
\rhead{\firstxmark}
\lfoot{\lastxmark}
\cfoot{\thepage}

\renewcommand\headrulewidth{0.4pt}
\renewcommand\footrulewidth{0.4pt}

\setlength\parindent{0pt}

%
% Create Problem Sections
%

\newcommand{\enterProblemHeader}[1]{
    \nobreak\extramarks{}{Problem \arabic{#1} continued on next page\ldots}\nobreak{}
    \nobreak\extramarks{Problem \arabic{#1} (continued)}{Problem \arabic{#1} continued on next page\ldots}\nobreak{}
}

\newcommand{\exitProblemHeader}[1]{
    \nobreak\extramarks{Problem \arabic{#1} (continued)}{Problem \arabic{#1} continued on next page\ldots}\nobreak{}
    \stepcounter{#1}
    \nobreak\extramarks{Problem \arabic{#1}}{}\nobreak{}
}

\setcounter{secnumdepth}{0}
\newcounter{partCounter}
\newcounter{homeworkProblemCounter}
\setcounter{homeworkProblemCounter}{1}
\nobreak\extramarks{Problem \arabic{homeworkProblemCounter}}{}\nobreak{}

%
% Homework Problem Environment
%
% This environment takes an optional argument. When given, it will adjust the
% problem counter. This is useful for when the problems given for your
% assignment aren't sequential. See the last 3 problems of this template for an
% example.
%
\newenvironment{homeworkProblem}[1][-1]{
    \ifnum#1>0
        \setcounter{homeworkProblemCounter}{#1}
    \fi
    \section{Problem \arabic{homeworkProblemCounter}}
    \setcounter{partCounter}{1}
    \enterProblemHeader{homeworkProblemCounter}
}{
    \exitProblemHeader{homeworkProblemCounter}
}

%
% Homework Details
%   - Title
%   - Due date
%   - Class
%   - Section/Time
%   - Instructor
%   - Author
%

\newcommand{\hmwkTitle}{第1次作业}
\newcommand{\hmwkDueDate}{2024.3.4}
\newcommand{\hmwkClass}{热学}
\newcommand{\hmwkClassTime}{Section A}
\newcommand{\hmwkClassInstructor}{周欣}
\newcommand{\hmwkAuthorName}{\textbf{郑晓旸} \and \textbf{202111030007}}

%
% Title Page
%

\title{
    \vspace{2in}
    \textmd{\textbf{\hmwkClass:\ \hmwkTitle}}\\
    \normalsize\vspace{0.1in}\small{Due\ on\ \hmwkDueDate\ }\\
    \vspace{0.1in}\large{\textit{\hmwkClassInstructor\ \hmwkClassTime}}
    \vspace{3in}
}

\author{\hmwkAuthorName}
\date{}

\renewcommand{\part}[1]{\textbf{\large Part \Alph{partCounter}}\stepcounter{partCounter}\\}

%
% Various Helper Commands
%

% Useful for algorithms
\newcommand{\alg}[1]{\textsc{\bfseries \footnotesize #1}}

% For derivatives
\newcommand{\deriv}[1]{\frac{\mathrm{d}}{\mathrm{d}x} (#1)}

% For partial derivatives
\newcommand{\pderiv}[2]{\frac{\partial}{\partial #1} (#2)}

% Integral dx
\newcommand{\dx}{\mathrm{d}x}

% Alias for the Solution section header
\newcommand{\solution}{\textbf{\large Solution}}

% Probability commands: Expectation, Variance, Covariance, Bias
\newcommand{\E}{\mathrm{E}}
\newcommand{\Var}{\mathrm{Var}}
\newcommand{\Cov}{\mathrm{Cov}}
\newcommand{\Bias}{\mathrm{Bias}}

\begin{document}

\maketitle

\pagebreak

\begin{homeworkProblem}
    道尔顿提出一种温标:规定理想气体体积的相对增量正比于温度的增量,在标准大气压下,规定水的冰点温度为零度,沸水温度为100度。试用摄氏度\(t\)来表示道尔顿温标的温度\(\tau\)
\\
    \textbf{Solution}
    \\
    设大气压强为\(P_{atm}\),\(T_0=273.15K\)为摄氏0度,\(T_{100}=373.15K\)为摄氏100度,\(t\)为摄氏度,\(\tau\)为道尔顿温标的温度。\(V_0\)为理想气体在摄氏0度下的体积。
    \\
    由题意可得:
    \begin{align*}
        P_{atm} V_0 &= \nu R T_0
        \\
        P_{atm} V_{100} &= \nu R T_{100}
    \end{align*}        
    由定义:
    \[
        \tau=\frac{V-V_0}{V_{100}-V_0} \times 100
    \]
    并带入气体体积和摄氏度的关系:
    \[
        V=\frac{\nu R(t+273.15)}{P_{atm}}\]
    得到道尔顿温度与摄氏温度的转化关系:
    \begin{align*}
        \tau&=\frac{\frac{\nu R(t+273.15)}{P_{atm}}-V_0}{V_{100}-V_0}\times 100
        \\
        &=\frac{\frac{\nu R(t+273.15)}{P_{atm}}-\nu R T_0}{\nu R T_{100}-\nu R T_0}\times 100
        \\
        &=\frac{t}{100}\times 100=t
    \end{align*}




\end{homeworkProblem}
\pagebreak
\begin{homeworkProblem}
    国际实用温标(1990年)规定:用于13.803 (平衡氢三相点)到
961.78°C(银在0.101MPa下的凝固点)的标准测量仪器是铂电阻温
度计。设铂电阻在0°C及°C时电阻的值分别为\(R_0\)及\(R(t)\),定义
\(W(t)=R(t)/R_0\),且在不同测温区内\(W(t)\)对\(t\)的函数关系是不同的,
在上述测温范围内大致有\(W(t)=1+At+Bt^2\)
若在0.101MPa下,对应于冰的熔点、水的沸点、硫的沸点(温度为444.67°C)电阻的阻值分别为11.000Ω、15.247Ω、28.887Ω,
试确定上式中的常数A和B。(正确标注常数A和B的单位)
\\
\textbf{Solution}
\\
由题意可得:
\begin{align*}
    W(0)&=1
    \\
    W(100)&=1+100^\circ C \cdot A+10000^\circ C^2\cdot B
    \\
    W(444.67)&=1+444.67^\circ C\cdot A+(444.67^\circ C)^2\cdot B
\end{align*}
同时代入电阻的阻值:
\begin{align*}
    11/11&=R_0/R_0=1
    \\
    15.247/11&=R_{100}/R_0=1+100^\circ C \cdot A+10000^\circ C^2 \cdot B
    \\
    28.887/11&=R_{444.67}/R_0=1+444.67^\circ C \cdot A+(444.67^\circ C)^2\cdot B
\end{align*}
得到A、B、C的解:

\[\begin{cases}
    A=3.9201 ^\circ C^{-1}\\
    B=-5.9205\times 10^{-7^\circ} C^{-2}
\end{cases}\]
\end{homeworkProblem}
\pagebreak

\begin{homeworkProblem}
    Write part of \alg{Quick-Sort($list, start, end$)}

    \begin{algorithm}[]
        \begin{algorithmic}[1]
            \Function{Quick-Sort}{$list, start, end$}
                \If{$start \geq end$}
                    \State{} \Return{}
                \EndIf{}
                \State{} $mid \gets \Call{Partition}{list, start, end}$
                \State{} \Call{Quick-Sort}{$list, start, mid - 1$}
                \State{} \Call{Quick-Sort}{$list, mid + 1, end$}
            \EndFunction{}
        \end{algorithmic}
        \caption{Start of QuickSort}
    \end{algorithm}
\end{homeworkProblem}

\pagebreak

\begin{homeworkProblem}
    Suppose we would like to fit a straight line through the origin, i.e.,
    \(Y_i = \beta_1 x_i + e_i\) with \(i = 1, \ldots, n\), \(\E [e_i] = 0\),
    and \(\Var [e_i] = \sigma^2_e\) and \(\Cov[e_i, e_j] = 0, \forall i \neq
    j\).
    \\

    \part

    Find the least squares esimator for \(\hat{\beta_1}\) for the slope
    \(\beta_1\).
    \\

    \solution

    To find the least squares estimator, we should minimize our Residual Sum
    of Squares, RSS:

    \[
        \begin{split}
            RSS &= \sum_{i = 1}^{n} {(Y_i - \hat{Y_i})}^2
            \\
            &= \sum_{i = 1}^{n} {(Y_i - \hat{\beta_1} x_i)}^2
        \end{split}
    \]

    By taking the partial derivative in respect to \(\hat{\beta_1}\), we get:

    \[
        \pderiv{
            \hat{\beta_1}
        }{RSS}
        = -2 \sum_{i = 1}^{n} {x_i (Y_i - \hat{\beta_1} x_i)}
        = 0
    \]

    This gives us:

    \[
        \begin{split}
            \sum_{i = 1}^{n} {x_i (Y_i - \hat{\beta_1} x_i)}
            &= \sum_{i = 1}^{n} {x_i Y_i} - \sum_{i = 1}^{n} \hat{\beta_1} x_i^2
            \\
            &= \sum_{i = 1}^{n} {x_i Y_i} - \hat{\beta_1}\sum_{i = 1}^{n} x_i^2
        \end{split}
    \]

    Solving for \(\hat{\beta_1}\) gives the final estimator for \(\beta_1\):

    \[
        \begin{split}
            \hat{\beta_1}
            &= \frac{
                \sum {x_i Y_i}
            }{
                \sum x_i^2
            }
        \end{split}
    \]

    \pagebreak

    \part

    Calculate the bias and the variance for the estimated slope
    \(\hat{\beta_1}\).
    \\

    \solution

    For the bias, we need to calculate the expected value
    \(\E[\hat{\beta_1}]\):

    \[
        \begin{split}
            \E[\hat{\beta_1}]
            &= \E \left[ \frac{
                \sum {x_i Y_i}
            }{
                \sum x_i^2
            }\right]
            \\
            &= \frac{
                \sum {x_i \E[Y_i]}
            }{
                \sum x_i^2
            }
            \\
            &= \frac{
                \sum {x_i (\beta_1 x_i)}
            }{
                \sum x_i^2
            }
            \\
            &= \frac{
                \sum {x_i^2 \beta_1}
            }{
                \sum x_i^2
            }
            \\
            &= \beta_1 \frac{
                \sum {x_i^2 \beta_1}
            }{
                \sum x_i^2
            }
            \\
            &= \beta_1
        \end{split}
    \]

    Thus since our estimator's expected value is \(\beta_1\), we can conclude
    that the bias of our estimator is 0.
    \\

    For the variance:

    \[
        \begin{split}
            \Var[\hat{\beta_1}]
            &= \Var \left[ \frac{
                \sum {x_i Y_i}
            }{
                \sum x_i^2
            }\right]
            \\
            &=
            \frac{
                \sum {x_i^2}
            }{
                \sum x_i^2 \sum x_i^2
            } \Var[Y_i]
            \\
            &=
            \frac{
                \sum {x_i^2}
            }{
                \sum x_i^2 \sum x_i^2
            } \Var[Y_i]
            \\
            &=
            \frac{
                1
            }{
                \sum x_i^2
            } \Var[Y_i]
            \\
            &=
            \frac{
                1
            }{
                \sum x_i^2
            } \sigma^2
            \\
            &=
            \frac{
                \sigma^2
            }{
                \sum x_i^2
            }
        \end{split}
    \]

\end{homeworkProblem}

\pagebreak

\begin{homeworkProblem}
    Prove a polynomial of degree \(k\), \(a_kn^k + a_{k - 1}n^{k - 1} + \hdots
    + a_1n^1 + a_0n^0\) is a member of \(\Theta(n^k)\) where \(a_k \hdots a_0\)
    are nonnegative constants.

    \begin{proof}
        To prove that \(a_kn^k + a_{k - 1}n^{k - 1} + \hdots + a_1n^1 +
        a_0n^0\), we must show the following:

        \[
            \exists c_1 \exists c_2 \forall n \geq n_0,\ {c_1 \cdot g(n) \leq
            f(n) \leq c_2 \cdot g(n)}
        \]

        For the first inequality, it is easy to see that it holds because no
        matter what the constants are, \(n^k \leq a_kn^k + a_{k - 1}n^{k - 1} +
        \hdots + a_1n^1 + a_0n^0\) even if \(c_1 = 1\) and \(n_0 = 1\).  This
        is because \(n^k \leq c_1 \cdot a_kn^k\) for any nonnegative constant,
        \(c_1\) and \(a_k\).
        \\

        Taking the second inequality, we prove it in the following way.
        By summation, \(\sum\limits_{i=0}^k a_i\) will give us a new constant,
        \(A\). By taking this value of \(A\), we can then do the following:

        \[
            \begin{split}
                a_kn^k + a_{k - 1}n^{k - 1} + \hdots + a_1n^1 + a_0n^0 &=
                \\
                &\leq (a_k + a_{k - 1} \hdots a_1 + a_0) \cdot n^k
                \\
                &= A \cdot n^k
                \\
                &\leq c_2 \cdot n^k
            \end{split}
        \]

        where \(n_0 = 1\) and \(c_2 = A\). \(c_2\) is just a constant. Thus the
        proof is complete.
    \end{proof}
\end{homeworkProblem}

\pagebreak

%
% Non sequential homework problems
%

% Jump to problem 18
\begin{homeworkProblem}[18]
    Evaluate \(\sum_{k=1}^{5} k^2\) and \(\sum_{k=1}^{5} (k - 1)^2\).
\end{homeworkProblem}

% Continue counting to 19
\begin{homeworkProblem}
    Find the derivative of \(f(x) = x^4 + 3x^2 - 2\)
\end{homeworkProblem}

% Go back to where we left off
\begin{homeworkProblem}[6]
    Evaluate the integrals
    \(\int_0^1 (1 - x^2) \dx\)
    and
    \(\int_1^{\infty} \frac{1}{x^2} \dx\).
\end{homeworkProblem}

\end{document}
