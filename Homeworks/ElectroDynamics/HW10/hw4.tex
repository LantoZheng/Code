\documentclass{assignment}
\ProjectInfos{电动力学A}{1410100401}{2024-2025学年第一学期}{第10次作业}{截止时间:2024. 11. 12(周二)}{郑晓旸}[https://github.com/LantoZheng]{202111030007}
\begin{document}
    \begin{prob}
    证明:$A \times (\nabla \times A) = \frac{1}{2} \nabla A^2 -(A \dot \nabla)A$
    \end{prob}
    \begin{sol}
    \begin{equation}
        A \times (\nabla \times A) = \epsilon_{ijk}A_j(\nabla \times A)_k = \epsilon_{ijk}A_j\epsilon_{klm}\partial_l A_m
    \end{equation}
    交换求和顺序得到:
    \begin{equation}
        A \times (\nabla \times A) = \epsilon_{jik}\epsilon_{klm}A_j\partial_l A_m = (\delta_{il}\delta_{jm} - \delta_{im}\delta_{jl})A_j\partial_l A_m
    \end{equation}
    展开得到:
    \begin{equation}
        A \times (\nabla \times A) = A_i\partial_j A_j - A_j\partial_j A_i
    \end{equation}
    由于$A_i\partial_j A_j = \nabla \cdot (A A)$,因此有:
    \begin{equation}
        A \times (\nabla \times A) = \nabla \cdot (A A) - (A \cdot \nabla)A
    \end{equation}
    由于$\nabla \cdot (A A) = \nabla \cdot \left(\frac{1}{2}A^2\right)$,因此有:
    \begin{equation}
        A \times (\nabla \times A) = \frac{1}{2}\nabla A^2 - (A \cdot \nabla)A
    \end{equation}

    \end{sol}
    \begin{prob}
        设$r = \sqrt{(x-x')^2+(y-y')^2+(z-z')^2}$, $\mathbf{r}$ 为源点到场点的矢量,证明:
        \[
            \begin{aligned}
            \nabla r = -\nabla ' r &= \frac{\mathbf{r}}{r}\\
            \nabla \frac{1}{r} = -\nabla ' \frac{1}{r} &= -\frac{\mathbf{r}}{r^3}\\
            \nabla \times \frac{\mathbf{r}}{r^3} &= 0 \\
            \nabla \cdot \frac{\mathbf{r}}{r^3} = \nabla ' \cdot \frac{\mathbf{r}}{r^3} &= 0
            \end{aligned}
        \]
        求 $\nabla \mathbf{r}$, $\nabla \times \mathbf{r}$, $\mathbf{a} \cdot \nabla \mathbf{r}$, $\nabla \mathbf{a} \cdot \mathbf{r}$, $\nabla \cdot [ \mathbf{E}_0 \sin{(\mathbf{k_0} \cdot \mathbf{r})}]$,$\nabla \times [ \mathbf{E}_0 \sin{(\mathbf{k_0} \cdot \mathbf{r})}]$
    \end{prob}
    \begin{sol}
        由于$r = \sqrt{(x-x')^2+(y-y')^2+(z-z')^2}$,因此有:
        \begin{equation}
            \begin{aligned}
            \nabla r &= \left(
                \begin{array}{c}
                    \frac{\partial r}{\partial x}\\
                    \frac{\partial r}{\partial y}\\
                    \frac{\partial r}{\partial z}
                \end{array}
            \right) = \left(
                \begin{array}{c}
                    \frac{x-x'}{r}\\
                    \frac{y-y'}{r}\\
                    \frac{z-z'}{r}
                \end{array}
            \right) = -\nabla ' r = \frac{\mathbf{r}}{r}\\
            \nabla \frac{1}{r} &= -\nabla ' \frac{1}{r} = -\frac{\mathbf{r}}{r^3}\\
            \nabla \times \frac{\mathbf{r}}{r^3} &= 0 \\
            \nabla \cdot \frac{\mathbf{r}}{r^3} = \nabla ' \cdot \frac{\mathbf{r}}{r^3} &= 0
            \end{aligned}
        \end{equation}
        由于
    \end{sol}
    \begin{prob}
        在地面系,静止的物体$A$在$x$方向受到恒力$\overrightarrow{F}$ ,求地面系中物体的运动轨迹;\\
        设物体$B$与物体$A$同时开始运动,$B$沿着$y$方向匀速直线运动,以$B$为参考系,求$B$参考系中$A$的速度和运动轨迹。
    \end{prob}
    \begin{sol}
        在地面系中,物体$A$受到的恒力为$\overrightarrow{F} = F\hat{i}$,因此有:
        \begin{equation}
            \frac{dp^x}{dt} = \frac{d \gamma m u^x}{dt} = F
        \end{equation}
    \end{sol}
    其中,$\gamma = \frac{1}{\sqrt{1-u^2}}$,这里取$c = 1$.
    我们得到参数方程:
    \begin{equation}
        \gamma m u^x = Ft
    \end{equation}
    展开得到:
    \begin{equation}
        \frac{dx}{dt} = u^x = \frac{Ft}{\sqrt{m^2 + F^2t^2}} = \frac{t}{\sqrt{t^2 + \frac{m^2}{F^2}}}
    \end{equation}
    积分得到:
    \begin{equation}
        x = \sqrt{t^2 + \frac{m^2}{F^2}} - \frac{m}{F}
    \end{equation}
    我们可以写出$A$的四位置:
    \begin{equation}
        x^\mu = \left(
            \begin{array}{c}
                t\\
                \sqrt{t^2 + \frac{m^2}{F^2}} - \frac{m}{F}\\
                0\\
                0
            \end{array}
        \right)
    \end{equation}
    在$B$参考系中,使用从地面系到$B$系的参数形式洛伦兹变换,其中$\gamma = \frac{1}{\sqrt{1-v^2}}$:
    \begin{equation}
        \Lambda^\nu_{\ \mu} = \left(
            \begin{array}{cccc}
                \gamma & 0 & -\gamma v & 0\\
                0 & 1 & 0 & 0\\
                -\gamma v & 0 & \gamma & 0\\
                0 & 0 & 0 & 0
            \end{array}
        \right)
    \end{equation}
    得到$A$在$B$参考系中的四位置:
    \begin{equation}
        x'^\mu = \Lambda^\mu_{\ \nu}x^\nu = \left(
            \begin{array}{c}
                \gamma t\\
                \sqrt{t^2 + \frac{m^2}{F^2}} - \frac{m}{F}\\
                -\gamma vt\\
                0
            \end{array}
        \right)
    \end{equation}
    因此$A$在$B$参考系中的速度为:
    \begin{equation}
        \begin{aligned}
        v'^x &= \frac{dx'}{dt'} = \frac{t'}{\sqrt{\gamma^2 t'^2 + \gamma^4 \frac{m^2}{F^2}}}\\
        v'^y &= -v
        \end{aligned}
    \end{equation}
\end{document}
