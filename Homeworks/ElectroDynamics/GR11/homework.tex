\documentclass{article}
\usepackage{ctex}
\usepackage{braket}
\usepackage{fancyhdr}
\usepackage{extramarks}
\usepackage{amsmath}
\usepackage{amsthm}
\usepackage{amsfonts}
\usepackage{tikz}
\usepackage[plain]{algorithm}
\usepackage{algpseudocode}

\usetikzlibrary{automata,positioning}

%
% Basic Document Settings
%

\topmargin=-0.45in
\evensidemargin=0in
\oddsidemargin=0in
\textwidth=6.5in
\textheight=9.0in
\headsep=0.25in

\linespread{1.1}

\pagestyle{fancy}
\lhead{\hmwkAuthorName}
\chead{\hmwkClass\ (\hmwkClassInstructor\ \hmwkClassTime): \hmwkTitle}
\rhead{\firstxmark}
\lfoot{\lastxmark}
\cfoot{\thepage}

\renewcommand\headrulewidth{0.4pt}
\renewcommand\footrulewidth{0.4pt}

\setlength\parindent{0pt}

%
% Create Problem Sections
%

\newcommand{\enterProblemHeader}[1]{
    \nobreak\extramarks{}{Problem \arabic{#1} continued on next page\ldots}\nobreak{}
    \nobreak\extramarks{Problem \arabic{#1} (continued)}{Problem \arabic{#1} continued on next page\ldots}\nobreak{}
}

\newcommand{\exitProblemHeader}[1]{
    \nobreak\extramarks{Problem \arabic{#1} (continued)}{Problem \arabic{#1} continued on next page\ldots}\nobreak{}
    \stepcounter{#1}
    \nobreak\extramarks{Problem \arabic{#1}}{}\nobreak{}
}

\setcounter{secnumdepth}{0}
\newcounter{partCounter}
\newcounter{homeworkProblemCounter}
\setcounter{homeworkProblemCounter}{1}
\nobreak\extramarks{Problem \arabic{homeworkProblemCounter}}{}\nobreak{}

%
% Homework Problem Environment
%
% This environment takes an optional argument. When given, it will adjust the
% problem counter. This is useful for when the problems given for your
% assignment aren't sequential. See the last 3 problems of this template for an
% example.
%
\newenvironment{homeworkProblem}[1][-1]{
    \ifnum#1>0
        \setcounter{homeworkProblemCounter}{#1}
    \fi
    \section{Problem \arabic{homeworkProblemCounter}}
    \setcounter{partCounter}{1}
    \enterProblemHeader{homeworkProblemCounter}
}{
    \exitProblemHeader{homeworkProblemCounter}
}

%
% Homework Details
%   - Title
%   - Due date
%   - Class
%   - Section/Time
%   - Instructor
%   - Author
%

\newcommand{\hmwkTitle}{Homework\ 11}
\newcommand{\hmwkDueDate}{November 26,2024}
\newcommand{\hmwkClass}{General Relativity}
\newcommand{\hmwkClassTime}{}
\newcommand{\hmwkClassInstructor}{Professor Mingyong Guo}
\newcommand{\hmwkAuthorName}{\textbf{郑晓旸}}

%
% Title Page
%

\title{
    \vspace{2in}
    \textmd{\textbf{\hmwkClass:\ \hmwkTitle}}\\
    \normalsize\vspace{0.1in}\small{Due\ on\ \hmwkDueDate\ at 3:10pm}\\
    \vspace{0.1in}\large{\textit{\hmwkClassInstructor\ \hmwkClassTime}}
    \vspace{3in}
}

\author{\hmwkAuthorName}
\date{}

\renewcommand{\part}[1]{\textbf{\large Part \Alph{partCounter}}\stepcounter{partCounter}\\}



\begin{document}


\begin{homeworkProblem}
    使用拉普拉斯变换或者傅里叶变换求解常微分方程:
    \begin{equation}
        \frac{d^2v}{d\varphi ^2}+u=\cos{\varphi}
    \end{equation}
\end{homeworkProblem}

\subsubsection{拉普拉斯变换}
\textbf{Solution}

对任意函数$f(t)$,其拉普拉斯变换定义为:
\begin{equation}
    F(s)=\int_0^{\infty}f(t)e^{-st}dt
\end{equation}
其中$s$是复变量。对于函数$f(t)$的导数$f'(t)$,有:
\begin{equation}
    \mathcal{L}\{f'(t)\}=sF(s)-f(0)
\end{equation}
对于函数$f(t)$的二阶导数$f''(t)$,有:
\begin{equation}
    \mathcal{L}\{f''(t)\}=s^2F(s)-sf(0)-f'(0)
\end{equation}
几个有用的拉普拉斯变换如下:
\begin{equation}
    \mathcal{L}\{\cos{\varphi}\}=\frac{p}{p^2+1}
\end{equation}
\begin{equation}
    \mathcal{L}\{\sin{\varphi}\}=\frac{1}{p^2+1}
\end{equation}
\begin{equation}
    \mathcal{L}\{\varphi\exp{i\varphi}\}=i\frac{d}{dp}\left(\frac{1}{p^2+1}\right)
\end{equation}
对上述常微分方程两端做拉普拉斯变换,得:
\begin{equation}
    p^2V(p)-pv(0)-v'(0)+V(p)=\frac{p}{p^2+1}
\end{equation}
其中$V(p)$是$v(\varphi)$的拉普拉斯变换。对上式整理,得:
\begin{equation}
    V(p)=\frac{p}{(p^2+1)^2}+\frac{pv(0)+v'(0)}{p^2+1}
\end{equation}
对上式右端两项分别求逆拉普拉斯变换,得:

\begin{equation}
    v(\varphi)=\frac{1}{2}\varphi\sin{\varphi}+v(0)\cos{\varphi}+v'(0)\sin{\varphi}
\end{equation}


\end{document}