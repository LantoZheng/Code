\documentclass{article}
\usepackage{amsmath, amssymb, amsthm}
\usepackage{geometry}
\geometry{a4paper, margin=1in}

\newtheorem*{problem}{Problem}
\newcommand{\ket}[1]{|#1\rangle}
\newcommand{\bra}[1]{\langle#1|}
\newcommand{\braket}[2]{\langle#1|#2\rangle}

\begin{document}

\section*{Problem Set 5: Gate Errors}

\subsection*{5.1: Bounding the Error of an Imperfect Rotation}

\begin{proof}
We are asked to find an upper bound for the inner product $\bra{\psi} U_{\theta}^\dagger U_{\theta+\delta} \ket{\psi}$. This quantity measures the fidelity between the state we wanted to produce, $U_\theta \ket{\psi}$, and the state we actually produced, $U_{\theta+\delta} \ket{\psi}$. A more informative measure of error is how much this inner product deviates from 1 (the ideal case with no error, $\delta=0$). We will therefore find an upper bound for $|1 - \bra{\psi} U_{\theta}^\dagger U_{\theta+\delta} \ket{\psi}|$.

First, let's simplify the operator product. The unitary operator is $U_\alpha = e^{i\alpha Z}$. Its adjoint is $U_\alpha^\dagger = (e^{i\alpha Z})^\dagger = e^{-i\alpha Z}$.
\[ U_{\theta}^\dagger U_{\theta+\delta} = e^{-i\theta Z} e^{i(\theta+\delta)Z} \]
Since the operators are functions of the same matrix $Z$, they commute, and we can add the exponents:
\[ U_{\theta}^\dagger U_{\theta+\delta} = e^{(-i\theta + i\theta + i\delta)Z} = e^{i\delta Z} = U_\delta \]
So the expression we need to analyze is $|1 - \bra{\psi} U_\delta \ket{\psi}|$.

Let an arbitrary single-qubit state be $\ket{\psi} = \alpha\ket{0} + \beta\ket{1}$, where $|\alpha|^2 + |\beta|^2 = 1$. The matrix for $U_\delta$ is:
\[ U_\delta = \begin{bmatrix} e^{i\delta} & 0 \\ 0 & e^{-i\delta} \end{bmatrix} \]
Applying this to $\ket{\psi}$:
\[ U_\delta\ket{\psi} = \alpha e^{i\delta}\ket{0} + \beta e^{-i\delta}\ket{1} \]
Now, we can compute the inner product $\bra{\psi} U_\delta \ket{\psi}$:
\[ \bra{\psi} U_\delta \ket{\psi} = (\alpha^*\bra{0} + \beta^*\bra{1}) (\alpha e^{i\delta}\ket{0} + \beta e^{-i\delta}\ket{1}) = |\alpha|^2 e^{i\delta} + |\beta|^2 e^{-i\delta} \]
Now we analyze the error term. Using the identity $1 = |\alpha|^2 + |\beta|^2$:
\begin{align*}
    1 - \bra{\psi} U_\delta \ket{\psi} &= (|\alpha|^2 + |\beta|^2) - (|\alpha|^2 e^{i\delta} + |\beta|^2 e^{-i\delta}) \\
    &= |\alpha|^2(1 - e^{i\delta}) + |\beta|^2(1 - e^{-i\delta})
\end{align*}
To find the bound, we take the magnitude and apply the triangle inequality:
\begin{align*}
    |1 - \bra{\psi} U_\delta \ket{\psi}| &= \left| |\alpha|^2(1 - e^{i\delta}) + |\beta|^2(1 - e^{-i\delta}) \right| \\
    &\leq |\alpha|^2|1 - e^{i\delta}| + |\beta|^2|1 - e^{-i\delta}|
\end{align*}
We note that $|1 - e^{-i\delta}| = |(e^{i\delta} - 1)(-e^{-i\delta})| = |e^{i\delta}-1| \cdot |-e^{-i\delta}| = |e^{i\delta}-1|$. The two magnitude terms are equal.
\begin{align*}
    |1 - \bra{\psi} U_\delta \ket{\psi}| &\leq |\alpha|^2|e^{i\delta} - 1| + |\beta|^2|e^{i\delta} - 1| \\
    &= (|\alpha|^2 + |\beta|^2) |e^{i\delta} - 1| \\
    &= |e^{i\delta} - 1|
\end{align*}
Using the hint provided, $|e^{i\delta} - 1| \leq |\delta|$, we arrive at the final bound for the deviation from unity:
\[ |1 - \bra{\psi} U_{\theta}^\dagger U_{\theta+\delta} \ket{\psi}| \leq |\delta| \]
This shows that the error in the final state scales linearly with the error in the rotation angle.
\end{proof}

\subsection*{5.2: Compounded Errors and the Nature of Quantum Computation}

\paragraph{Required Precision for Sequential Rotations}
When we compose a sequence of $s$ imperfect gates, each with a small error $\delta$, these errors accumulate. While the total error is a complex sum, a reasonable first-order approximation suggests that the errors add up. The result from 5.1 shows that the error introduced by one gate is bounded by $|\delta|$. For a sequence of $s$ such gates, the total accumulated error, $E_{total}$, will be approximately bounded by the sum of individual errors:
\[ E_{total} \lesssim s|\delta| \]
For the total error to be "not significant," it must be less than some small constant, say $\epsilon \ll 1$.
\[ s|\delta| < \epsilon \implies |\delta| < \frac{\epsilon}{s} \]
This result is highly significant: it means that for an algorithm's length (or depth) $s$ to double, the required precision of each individual gate must also double (i.e., the error magnitude $\delta$ must be halved). The required hardware precision scales inversely with the length of the desired computation.

\paragraph{Implications for the Nature of Quantum Computing}
This finding directly addresses whether quantum computing is fundamentally analog or digital.

\begin{itemize}
    \item \textbf{Analog Computation:} An analog computer encodes information in continuous physical quantities (e.g., voltage, rotation angle). A defining characteristic of analog systems is that small, continuous errors from physical components accumulate over time and are not corrected. The linear accumulation of error ($E \propto s\delta$) is the hallmark of an analog computational model. The "weak noise regime" described for near-term processors, where $\delta$ must shrink as $s$ grows, indicates that these devices are operating as \textbf{analog computers}.

    \item \textbf{Digital Computation:} A digital computer encodes information in discrete states (e.g., 0 and 1). Gates are designed to be restorative, meaning they map a range of imperfect inputs (e.g., a voltage close to 0V) to a discrete, ideal output (exactly 0V), thus preventing the accumulation of small errors. This fault tolerance is what allows for extremely deep and complex computations.
\end{itemize}

\textbf{Conclusion:}
The discussion suggests that quantum computing is fundamentally \textbf{analog at the physical level}. Qubits are represented by continuous physical states, and gates are continuous physical processes, both of which are susceptible to the linear accumulation of errors.

The eventual goal of creating "error-corrected, fault-tolerant quantum computers" that can tolerate a constant error $\delta$ represents a paradigm shift from analog to digital. Quantum Error Correction (QEC) is the process that "digitizes" the computation. By encoding a single logical qubit into many physical qubits and constantly checking for and correcting errors, QEC creates a restorative digital abstraction over the noisy analog hardware. This transition is essential for building scalable quantum computers capable of solving large-scale problems.

\end{document}
