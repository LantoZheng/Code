\documentclass[10pt, a4paper]{article}

% --- 宏包设置 ---

% 编码与字体
\usepackage[utf8]{inputenc}
\usepackage[T1]{fontenc}
\usepackage{lmodern} % 使用更平滑的 Latin Modern 字体
\usepackage{amsmath} % 用于数学公式 (如GPA)

% 页面布局
\usepackage[left=0.8in, right=0.8in, top=0.75in, bottom=0.75in]{geometry}

% 超链接 (用于Email)
\usepackage[colorlinks=true, urlcolor=blue, linkcolor=blue]{hyperref}

% 列表自定义
\usepackage{enumitem}

% 章节标题自定义
\usepackage{titlesec}

% --- 格式定义 ---

% 移除页码
\pagestyle{empty}

% 定义简历中的主要分区 (Education, Skills, etc.)
% 格式为: 大号、小型大写字母、左对齐、下方带一条横线
\titleformat{\section}
  {\Large\scshape\raggedright} % 格式
  {}                            % 标签 (无)
  {0em}                         % 标签与标题的距离
  {}                            % 标题前的代码
  [\titlerule]                  % 标题后的代码 (一条横线)
\titlespacing*{\section}{0pt}{1.0\baselineskip}{0.4\baselineskip} % 间距 (左, 上, 下)

% 为所有 itemize 列表设置统一的间距
\setlist[itemize]{itemsep=0.15em, topsep=0.15em, parsep=0pt, partopsep=0pt}

% --- 文档开始 ---
\begin{document}

% --- 1. 页眉:个人信息 ---
\begin{center}
    % 姓名
    \textbf{\LARGE Xiaoyang Zheng}\\[0.3em]
    % 联系信息
    \small
    \href{mailto:xiaoyangzheng@mail.bnu.edu.cn}{xiaoyangzheng@mail.bnu.edu.cn} / \href{mailto:Xiaoyang_zheng@berkeley.edu}{Xiaoyang\_zheng@berkeley.edu} $\cdot$
    +86 13955190184 $\cdot$
    2217 Channing Way Apt. C, CA Berkeley, 94704
\end{center}
\vspace{-0.2em}


% --- 2. 教育背景 ---
\section{Education}

% 使用 itemize 来组织条目,不显示标签 (label={})
\begin{itemize}[leftmargin=*, label={}]

    % BNU
    \item
    \textbf{Beijing Normal University, Beijing, China} \hfill \textit{2021.09-Now}
    \begin{itemize}
        \item Undergraduate of Science in Physics, Liyun Elite Program \hfill (expected in 2026.07)
        \item Undergraduate of Business in Economics \hfill (expected in 2026.07)
    \end{itemize}

    % UC Berkeley
    \item
    \textbf{University of California, Berkely, Berkeley, US} \hfill \textit{(2025.8-2025.12)}
    \begin{itemize}
        \item Berkely Physics International Education (BPIE) Program
    \end{itemize}

    % 详细信息 (GPA, 课程, 奖学金等)
    \item
    \begin{itemize}[leftmargin=1.5em]
        \item \textbf{Ranking:} 2/23 $\quad$ \textbf{GPA:} $3.7/4.0$
        \item \textbf{Core courses:} Optics (93), Quantum Mechanics I\&II (89), Computational Physics (95), Seminar on Optics (95), Mechanics (95), Electromagnetism (97), Electrodynamics (91), Solid-state Physics (82)
        \item \textbf{Language:} IELTS 7.5 (R:8.0, L:8.5, S:6.5, W:6.0); TOEFL 101 (R:30, L:28, S:20, W:23)
        \item \textbf{Scholarships:} Outstanding Freshman (2021), First-class Scholarship (2024, 2022), First-class Incentive Scholarship (2024, 2023, 2022)
    \end{itemize}
    
\end{itemize}


% --- 3. 技能 ---
\section{Skills}

\begin{itemize}[leftmargin=*]
    \item \textbf{Computer:} Python (PyTorch, SciPy), MatLab, C++
    \item \textbf{Lab Skills:} SEM and TEM for materials characterization and analysis
\end{itemize}


% --- 4. 研究经历 ---
\section{Research Experiences}

\begin{itemize}[leftmargin=*, label={}]

    % 项目 1: D2NN
    \item
    \textbf{Single-layer Diffractive Neural Network (D2NN)} \hfill \textit{2024.12-Now}
    \begin{itemize}[leftmargin=1.5em]
        \item Designed a single-layer D2NN to transfer neural network inference to the optical domain for \textbf{high-speed parallel computation}. Addressed output layer intensity loss in traditional multi-layer D2NNs by achieving similar performance using a single phase mask within a 4f optical system.
        \item Designed optical weights using a \textbf{Spatial Light Modulator (SLM)} in the Fourier plane; formulated the optical neural network transfer function based on diffraction principles; trained the D2NN with \textbf{PyTorch} on the MNIST dataset.
        \item Achieved \textbf{97\%+ accuracy} on MNIST through optical simulation, demonstrating the potential of single-layer D2NNs for optical computing.
    \end{itemize}
    \vspace{0.2em}

    % 项目 2: Beam Shaping
    \item
    \textbf{Self-calibrating Beam Shaping Based on Reflective SLM | Course Project}
    \begin{itemize}[leftmargin=1.5em]
        \item Developed a \textbf{self-calibrating optical system} for real-time beam shaping and wavefront correction to compensate distortions and higher-order modes using a reflective SLM in a feedback control loop.
        \item Designed and constructed the optical system integrating a reflective SLM for phase modulation; incorporated a CCD camera with microlens array as wavefront sensor; developed \textbf{Python program} implementing optimization algorithms (SGD, simulated annealing) to minimize wavefront error.
        \item Successfully demonstrated \textbf{real-time wavefront correction} and generation of Gaussian and Laguerre-Gaussian beam profiles.
    \end{itemize}
    \vspace{0.2em}

    % 项目 3: Micro and Nano Optics
    \item
    \textbf{Micro and Nano Optics | Undergraduate Research Assistant} \hfill \textit{2022.4-2024.6}
    \begin{itemize}[leftmargin=1.5em]
        \item \textit{Advisor: Jinwei Shi (Professor, Beijing Normal University)} $\quad$ \textit{Funding: Beijing Undergraduate Research Training Program}
        \item Utilized \textbf{SEM} for nanoparticle characterization and \textbf{optical spectroscopy} for plasmon resonance analysis.
        \item Conducted \textbf{FDTD simulations} to study electromagnetic modes of gold nanorods excited by visible light and investigated effects of nanorod dimensions on absorption peaks in 2D materials.
    \end{itemize}
    \vspace{0.2em}

    % 项目 4: Atomic Co-Magnetometry
    \item
    \textbf{Atomic Co-Magnetometry | Undergraduate Research Assistant} \hfill \textit{2023.06-2023.08}
    \begin{itemize}[leftmargin=1.5em]
        \item \textit{Advisor: Dong Sheng (Professor, University of Science and Technology of China)}
        \item Developed knowledge of \textbf{atomic magnetometers and co-magnetometers} for precision measurements.
        \item Conducted \textbf{COMSOL thermal simulations} to optimize component thermal stability; participated in optical path construction and electronic circuit implementation for signal acquisition and noise reduction.
    \end{itemize}

\end{itemize}


% --- 5. 工作经历 ---
\section{Work Experiences}

\begin{itemize}[leftmargin=*, label={}]

    % 经历 1: BNUPA
    \item
    \textbf{BNUPA (Beijing Normal University Photographer Association) | Chairman} \hfill \textit{2023.09-2024.09}
    \begin{itemize}[leftmargin=1.5em]
        \item Organized \textbf{3 lectures and 2 interviews} with external experts, engaging approximately \textbf{200 students}.
        \item Delivered lecture series including \textit{"Optical Concepts in Photography"} and \textit{"Imaging System Quality from Photographic Equipment Perspective"}.
    \end{itemize}
    \vspace{0.2em}
    
    % 经历 2: Homoludens Archive
    \item
    \textbf{Homoludens Archive | Undergraduate Researcher \& Archives Administrator} \hfill \textit{2022.09-2023.07}

\end{itemize}

% --- 文档结束 ---
\end{document}