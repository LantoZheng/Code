% Compact XeLaTeX document for the Research Plan
\documentclass[11pt,a4paper]{article}
% Use fontspec for XeLaTeX
\usepackage{fontspec}
% Use a common system font; change if you prefer another installed font
\setmainfont{Times New Roman}
\usepackage[margin=0.75in]{geometry}
\usepackage{setspace}
\singlespacing
\usepackage{microtype}
\setlength{\parskip}{2pt}
\setlength{\parindent}{0pt}
\usepackage{titlesec}
\titlespacing*{\section}{0pt}{8pt}{4pt}
% Use biblatex for bibliography management with biber backend
\usepackage[backend=biber,style=numeric,sorting=none]{biblatex}
\usepackage{hyperref}
\usepackage{graphicx}
\usepackage{amsmath}
\usepackage{csquotes}
\addbibresource{references_tiny.bib}

% Bibliography: use biblatex + biber. Run sequence: xelatex -> biber -> xelatex -> xelatex
% Citations in the document use \cite{key} and will be rendered by biblatex.
\title{Research Plan}
\author{Xiaoyang Zheng}
\date{\today}


\begin{document}
\begin{center}
    \Large \textbf{AI-Enabled Multi-Objective Inverse Design of Chiral Metasurfaces\\for Biosensing via Moth-Eye Template Nanofabrication}\\
    \vspace{4pt}
    \normalsize Xiaoyang Zheng
\end{center}
\vspace{4pt}

\section*{Research Motivation and Background}
Detecting biomolecular chirality is essential for drug discovery and diagnostics, yet traditional circular dichroism (CD) spectroscopy requires expensive instrumentation and large sample volumes~\cite{nanophotonic_biosensors_acs}. Professor Mana Toma's group has pioneered plasmonic metasurface biosensors leveraging collective plasmon modes for label-free colorimetric detection using silver nanodome arrays~\cite{toma_researches,toma_colorimetric_biosensor,toma_plasmonic_metasurface}, enabling practical spectrometer-free biosensing through industrially scalable moth-eye nanoimprint lithography~\cite{nil_metasurface_review}.

I aim to extend this platform to chiral sensing, where superchiral electromagnetic fields achieve ultrasensitive detection of molecular handedness~\cite{chiral_biosensing_review}. Current AI-driven metasurface design methods cannot efficiently generate manufacturable chiral structures optimized for both signal enhancement and biosensing performance~\cite{dl_nanophotonics_rg}, creating urgent need for an AI framework generating moth-eye-compatible designs with multi-objective optimization.

\section*{Research Objectives}
I propose developing a conditional Generative Adversarial Network (cGAN) framework tailored to Professor Toma's moth-eye fabrication platform for rapid generation of optimized chiral plasmonic metasurface biosensor designs~\cite{conditional_gan_nanophotonics,gan_nanophotonic_inverse}:

\begin{enumerate}
    \setlength{\itemsep}{0.5pt}
    \setlength{\parskip}{0pt}
  \item \textbf{Define parametric design space:} Mathematical model describing achievable 3D chiral nanostructures within moth-eye constraints, inspired by Professor Toma's silver nanodome architectures.
  \item \textbf{Generate training dataset:} High-quality dataset ($>$10,000 designs) via automated FDTD simulations evaluating CD enhancement and LSPR wavelength shifts.
  \item \textbf{Develop cGAN architecture:} PyTorch-based cGAN with dual performance targets (chiroptical response + colorimetric sensitivity) as inputs and manufacturable parameters as outputs.
  \item \textbf{Experimental validation:} Fabricate top designs using moth-eye templates and characterize via Professor Toma's protocols (CD spectroscopy, colorimetric immunoassays).
\end{enumerate}

\section*{Methodology}
\textbf{Phase 1 (Months 1--8):} Develop automated Python-FDTD pipeline sampling moth-eye parameter space (silver thickness, template depth, asymmetry), generating paired geometric and performance data (CD enhancement, refractive index sensitivity).

\textbf{Phase 2 (Months 9--16):} Train forward-predicting surrogate network for rapid evaluation, then train cGAN with multi-objective loss (adversarial learning, dual-target reconstruction, physics constraints), validated via FDTD and comparison with Professor Toma's published designs.

\textbf{Phase 3 (Months 17--24):} Fabricate selected designs using Professor Toma's moth-eye process with silver deposition. Characterize via SEM, CD spectroscopy, and label-free binding assays with model biomolecules (chiral amino acids, IgG)~\cite{toma_plasmonic_metasurface,toma_enhanced_fluorescence}.

\section*{Expected Outcomes and Impact}
Deliverables: (1) validated cGAN tool for rapid chiral biosensor customization without re-training, (2) design library demonstrating chiroptical-colorimetric trade-offs, and (3) proof-of-concept devices with $<$10\% prediction-experiment deviation. This directly advances Professor Toma's mission of democratizing biosensing~\cite{toma_achievements} by combining her scalable moth-eye platform with AI-driven multi-objective design for pharmaceutical quality control and point-of-care diagnostics.

\newpage
\printbibliography[title={References}]

\end{document}
