\documentclass[11pt,a4paper]{article}

% --- 宏包设置 ---
\usepackage{fontspec}
\setmainfont{Times New Roman}
\usepackage{geometry}
\geometry{margin=0.7in}
\usepackage{setspace}
\setstretch{0.95}
\usepackage{microtype}
\usepackage{hyperref}
\hypersetup{
    colorlinks=true,
    linkcolor=blue,
    urlcolor=blue,
    citecolor=blue
}
\usepackage{enumitem}
\setlist{nosep, leftmargin=*, itemsep=1pt, topsep=2pt, parsep=0pt}
\usepackage{titlesec}

% 极度紧凑的章节标题格式
\titleformat{\section}
  {\normalsize\bfseries}
  {}
  {0em}
  {}
\titlespacing*{\section}{0pt}{6pt}{3pt}

\titleformat{\subsection}
  {\small\bfseries}
  {}
  {0em}
  {}
\titlespacing*{\subsection}{0pt}{4pt}{2pt}

% 移除页码
\pagenumbering{gobble}

% --- 文档开始 ---
\begin{document}

% --- 标题 ---
\begin{center}
    {\large \textbf{Statement of Purpose}}\\[0.2em]
    {\normalsize Master's Program, Graduate School of Science, University of Tokyo}\\[0.5em]
    \textbf{Xiaoyang Zheng} $\cdot$ Beijing Normal University $\cdot$ \href{mailto:xiaoyangzheng@mail.bnu.edu.cn}{xiaoyangzheng@mail.bnu.edu.cn}
\end{center}

\vspace{-0.4em}

% --- 正文 ---
\section{Academic Background and Motivation}

I am a final-year undergraduate Physics major at Beijing Normal University (ranked 2/23, GPA 3.7/4.0) with research experience in optical physics and nanophotonics. My projects include diffractive neural networks (D2NN achieving 97\% MNIST accuracy), spatial light modulator (SLM)-based beam shaping with real-time optimization, and nanophotonics characterization using SEM/TEM and FDTD simulations. Currently at UC Berkeley through the Berkeley Physics International Education Program (Aug--Dec 2025), I have been exposed to cutting-edge condensed matter physics and ultrafast spectroscopy, crystallizing my goal to transition from applied optics to fundamental quantum materials research.

My technical skills include precision optical alignment, wavefront control, signal optimization, and computational methods (Python/PyTorch, MATLAB [95/100 in Computational Physics]). These capabilities, combined with strong experimental foundations, position me to contribute meaningfully to advanced spectroscopic research on quantum materials at the University of Tokyo.

\section{Why School of Science, University of Tokyo, and Tentative Research Plan}

\subsection{Research Alignment with Professor Takao Kondo's Laboratory (ISSP)}

I am strongly motivated to join Prof.~Kondo's laboratory due to exceptional research synergy. The lab's world-leading work in angle-resolved photoemission spectroscopy (ARPES) on topological materials and strongly correlated systems—including \textbf{recent development of time-, spin-, and angle-resolved ARPES using pump-probe femtosecond lasers}—represents precisely the intersection of ultrafast optics and quantum materials physics where I aspire to build my career.

\textbf{Key alignment between my background and laboratory techniques:}
\begin{itemize}
    \item My D2NN project in 4f optical systems trained me in Fourier-plane optics and phase control, directly applicable to laser-ARPES systems and OPA tuning for pump-probe experiments.
    \item My SLM feedback control work (using SGD and simulated annealing algorithms) parallels signal optimization challenges in time-resolved spectroscopy.
    \item My computational physics background enables contribution to high-dimensional ARPES data analysis and machine learning implementation for complex spectroscopic datasets.
\end{itemize}

The laboratory's groundbreaking discoveries—direct observation of topological surface states (\textit{Nature} 2019, 2021), resolution of cuprate superconductor Fermi surface mysteries (\textit{Science} 2020), and microscopic observation of complex magnetic phases (\textit{Nature Communications} 2020)—demonstrate the power of advanced ARPES to address fundamental condensed matter physics questions. The University of Tokyo's ISSP offers unparalleled resources: multiple ultrahigh-resolution laser-ARPES systems, global synchrotron facility access, and collaborative networks with theory groups and materials synthesis teams.

\subsection{Tentative Research Plan: Time-Resolved ARPES on Quantum Materials}

\textbf{Primary research direction:} I propose to contribute to the laboratory's development of time-resolved ARPES by combining pump-probe spectroscopy with momentum-resolved electronic structure measurements. Specific scientific objectives include:

\begin{enumerate}
    \item \textbf{Band-selective photoexcitation dynamics}: Using tunable pump wavelengths to selectively excite electrons at specific k-points (e.g., van Hove singularities, Dirac/Weyl points) in kagome metals (CsV$_3$Sb$_5$) or topological materials, then probe transient band structure evolution to reveal electron-electron, electron-phonon, and electron-spin interaction timescales.
    
    \item \textbf{Non-equilibrium electronic structure}: Observe Fermi surface and band dispersion evolution on femtosecond-to-picosecond timescales following optical excitation, providing direct insights into quantum many-body dynamics.
    
    \item \textbf{Photo-induced phenomena}: If time permits, explore light-induced metastable phases or hidden quantum states inaccessible in equilibrium conditions.
\end{enumerate}

I am fully flexible to adapt based on laboratory priorities and sample availability. \textbf{Alternative research directions} include: time-resolved studies of cuprate superconductors (pseudogap/Cooper pair dynamics), spin dynamics in topological materials using pump-probe spin-resolved ARPES, or two-dimensional electronic states in thin film quantum materials—all leveraging the laboratory's cutting-edge capabilities.

\textbf{Two-year timeline:}
\begin{itemize}
    \item \textbf{Year 1 (Months 1--12):} Laboratory integration, coursework, intensive Japanese language study (targeting JLPT N3), mastery of laser-ARPES operation, preliminary time-resolved measurements on reference materials.
    \item \textbf{Year 2 (Months 13--24):} Deep spectroscopic characterization of quantum materials, data analysis with theorists, manuscript preparation (targeting \textit{Physical Review B} or \textit{JPSJ}), conference presentations, thesis completion.
\end{itemize}

\textbf{Expected outcomes:} At least one first-author publication, comprehensive Master's thesis, and acquisition of advanced experimental/analytical skills in ultrafast quantum materials spectroscopy.

\subsection{World-Class Environment and Career Development}

The GSGC program's interdisciplinary training framework and international collaboration emphasis align perfectly with my goal to become an experimental physicist addressing frontier problems at the interface of physics, materials science, and advanced instrumentation. Studying in Japan provides unique opportunities to immerse myself in a leading academic culture for condensed matter physics, develop Japanese language proficiency, and establish international research connections valuable for my long-term career in research (academia or national laboratory).

\section{Career Aspirations and Commitment}

\textbf{Immediate goals:} Excel in the Master's program and contribute high-quality research to Prof.~Kondo's laboratory. I am strongly considering continuing to a doctoral program if my Master's work proves successful, as frontier questions in quantum materials physics—how topology, correlation, and symmetry breaking produce exotic phases—represent challenges I am passionate about addressing.

\textbf{Long-term vision:} Research career where I can (1) develop next-generation spectroscopic techniques with unprecedented resolution, (2) bridge experiment and theory through collaborative research, and (3) mentor future physicists. The training at the University of Tokyo in precision laser spectroscopy, cryogenic techniques, data analysis, and international collaboration will be foundational for this career path.

\textbf{My commitment to success:}
\begin{itemize}
    \item \textbf{Full dedication:} Long laboratory hours, active seminar participation, deep engagement with scientific literature.
    \item \textbf{Japanese language mastery:} Intensive study targeting JLPT N3 by Year 1 completion for effective communication and cultural integration.
    \item \textbf{Effective collaboration:} Learning from senior researchers, contributing to group discussions, maintaining high standards for experimental rigor.
    \item \textbf{Scientific communication:} Publishing in peer-reviewed journals and presenting at domestic/international conferences.
\end{itemize}

I seek a transformative educational experience that will shape me into a rigorous, creative, and collaborative experimental physicist capable of addressing fundamental questions in quantum materials.

\section{Conclusion}

The School of Science at the University of Tokyo, particularly Prof.~Takao Kondo's laboratory at ISSP, represents my ideal environment for pursuing quantum materials research through advanced optical spectroscopy. The laboratory's world-leading ARPES expertise, development of time-resolved and spin-resolved techniques, and focus on topological and strongly correlated materials align perfectly with my scientific interests and technical background.

My experience in optical system design (D2NN, SLM beam shaping), computational analysis (machine learning, optimization algorithms), and hands-on experimentation has prepared me to contribute quickly while learning advanced spectroscopic methods. I am confident that I possess both the technical foundation and intellectual passion to succeed in this program and make meaningful contributions to frontier research in quantum materials physics.

Thank you for considering my application. I am excited about the prospect of joining the University of Tokyo community and collaborating with leading researchers in condensed matter physics.

\vspace{0.5em}

\noindent
\textbf{Xiaoyang Zheng} $\cdot$ Beijing Normal University (Class Rank 2/23, GPA 3.7/4.0) $\cdot$ UC Berkeley Exchange (Aug--Dec 2025)\\
Email: \href{mailto:xiaoyangzheng@mail.bnu.edu.cn}{xiaoyangzheng@mail.bnu.edu.cn} $\cdot$ Phone: +86-13955190184 $\cdot$ Language: IELTS 7.5, TOEFL 101\\
\textbf{Intended Supervisor:} Professor Takeshi Kondo, Institute for Solid State Physics (ISSP)

\end{document}
