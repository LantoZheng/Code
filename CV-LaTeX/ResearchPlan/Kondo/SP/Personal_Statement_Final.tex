\documentclass[11pt,a4paper]{article}

% --- 宏包设置 ---
\usepackage{fontspec}
\setmainfont{Times New Roman}
\usepackage{geometry}
\geometry{margin=0.75in}
\usepackage{setspace}
\setstretch{1.0}
\usepackage{microtype}
\usepackage{hyperref}
\hypersetup{
    colorlinks=true,
    linkcolor=blue,
    urlcolor=blue,
    citecolor=blue
}
\usepackage{enumitem}
\setlist{nosep, leftmargin=*, itemsep=2pt, topsep=2pt, parsep=2pt}
\usepackage{titlesec}

% 极度紧凑的章节标题格式
\titleformat{\section}
  {\normalsize\bfseries}
  {}
  {0em}
  {}
\titlespacing*{\section}{1pt}{10pt}{4pt}

\titleformat{\subsection}
  {\small\bfseries}
  {}
  {0em}
  {}
\titlespacing*{\subsection}{1pt}{8pt}{4pt}


% --- 文档开始 ---
\begin{document}

% --- 标题 ---
\begin{center}
    {\large \textbf{Personal Statement}}\\[0.2em]
    {\normalsize Doctoral Program, Graduate School of Science, University of Tokyo}\\[0.4em]
    \textbf{Xiaoyang Zheng} $\cdot$ Beijing Normal University $\cdot$ \href{mailto:xiaoyangzheng@mail.bnu.edu.cn}{xiaoyangzheng@mail.bnu.edu.cn}
\end{center}

\vspace{-0.4em}

% --- 正文 ---
\section{Introduction: From Optical Engineering to Quantum Physics Questions}

As a senior undergraduate majored in Physics at Beijing Normal University (ranked 2/23, GPA 3.7/4.0), my journey has evolved from \textbf{applied optics engineering} to asking \textbf{fundamental physics questions}. Currently at UC Berkeley (Aug--Dec 2025) engaging with cutting-edge condensed matter physics and ultrafast spectroscopy, I have solidified my scientific passion: \textbf{How can we disentangle electron-phonon coupling, electron-electron correlations, and topological band structure that govern exotic quantum phases?} In this statement I'll highlight my accomplishments demonstrating technical skills, independence of mind, and readiness for a five-year doctoral program addressing frontier questions in quantum materials physics.

\section{Research Accomplishments: Initiative and Innovation}

\subsection{Pioneering Single-Layer Diffractive Neural Networks (2024-Present)}

My most significant independent achievement demonstrates the transition from \textbf{applied optics to physics-driven innovation}. Designing a \textbf{single-layer diffractive neural network (D2NN)} addressed critical limitations in existing multi-layer architectures:

\textbf{Innovation:} Traditional multi-layer D2NNs suffer from cascaded intensity loss. I proposed using a \textbf{single phase mask in a 4f optical system}, deriving the optical transfer function from first-principles diffraction theory, designing SLM phase patterns for the Fourier plane, and implementing a custom PyTorch training pipeline.

\textbf{Physics Insight:} Achieved \textbf{97\%+ accuracy} on MNIST classification. More importantly, this taught me to think about \textbf{physical constraints as design principles}—how diffraction limits information transfer, how Fourier optics enables parallel processing. This mindset shift from "making it work" to "understanding why it works" prepared me for fundamental research in quantum materials.

\subsection{Self-Calibrating Beam Shaping System (2024)}

For an advanced optics course, I \textbf{designed and constructed a complete optical system from scratch}. I developed a \textbf{self-calibrating feedback control system} integrating reflective SLM for dynamic phase modulation, CCD camera with microlens array as wavefront sensor, and custom Python software implementing optimization algorithms (SGD, simulated annealing) for real-time wavefront error minimization.

\textbf{Transferable skills:} Successfully demonstrated \textbf{real-time wavefront correction} and generation of Gaussian and Laguerre-Gaussian beam profiles. This built the \textbf{hands-on experimental ability, systems integration skills (optics + electronics + software), and optimization expertise} essential for laser-ARPES and pump-probe spectroscopy—techniques I aim to master for studying quantum materials.

\subsection{Nanophotonics Research (2022-2024)}

As an \textbf{undergraduate research assistant} under Prof.~Jinwei Shi, I contributed to nanophotonics research bridging experiment and simulation: characterized gold nanostructures using \textbf{SEM/TEM}, performed optical spectroscopy analyzing plasmon resonances, and conducted \textbf{FDTD electromagnetic simulations} studying gold nanorod interactions with 2D materials.

\textbf{Scientific growth:} This two-year experience taught me to transition from following protocols to \textbf{asking independent research questions}—Why do specific geometries enhance light-matter coupling? How do plasmons mediate energy transfer? The iterative experiment-simulation-refinement process instilled patience and rigor, while the focus on \textbf{light-matter interactions} foreshadowed my current interest in how photons reveal electronic structure in quantum materials.

\subsection{Atomic Magnetometry (Summer 2023)}

During summer research at USTC under Prof.~Dong Sheng, I contributed to developing \textbf{atomic co-magnetometers} for precision measurements. I quickly learned atomic physics fundamentals and contributed meaningfully: conducted \textbf{COMSOL thermal simulations} optimizing temperature stability and participated in optical system construction. This demonstrates my rapid learning ability and adaptability to new research fields.

\section{Leadership and Scientific Communication}

As \textbf{Chairman of BNU Photographer Association (2023-2024)}, I transformed the organization into a platform for scientific education, organizing \textbf{3 expert lectures and 2 interviews engaging ~200 students}. I delivered two lecture series bridging physics and photography: \textit{"Optical Concepts in Photography"} (diffraction limits, lens aberrations, Fourier optics) and \textit{"Imaging System Quality"} (MTF, resolution, sensor technology). This required effective public speaking, ability to communicate complex physics to non-specialists, and team management—skills invaluable in research requiring collaboration across diverse backgrounds.

\section{Academic Resilience and International Adaptation}

Pursuing a \textbf{double major in Physics and Economics} while maintaining competitive standing (2/23, GPA 3.7) required exceptional time management and sustained high performance in technical courses: Optics (93), Computational Physics (95), Electromagnetism (97), Quantum Mechanics (89). This dual-degree experience sharpened my ability to work efficiently under pressure and integrate knowledge across disciplines.

My participation in the \textbf{Berkeley Physics International Education (BPIE) Program} represents a pivotal turning point. At Berkeley, I am \textbf{directly engaging with cutting-edge condensed matter physics and ultrafast spectroscopy}. This exposure has:

\begin{itemize}
    \item \textbf{Crystallized my research vision:} Understanding how electron-phonon coupling, electron-electron correlations, and topological structure produce exotic quantum phases through ultrafast spectroscopy
    \item \textbf{Validated technical preparation:} My D2NN optical systems, SLM control algorithms, and computational physics background apply directly to laser-ARPES, OPA tuning, and high-dimensional data analysis
    \item \textbf{Built international capacity:} Adapting to English-language academic culture (IELTS 7.5, TOEFL 101), thriving in world-leading research environment
\end{itemize}

This experience reinforces my confidence in succeeding in a five-year doctoral program at the University of Tokyo, where I can bridge my optical engineering expertise with Prof.~Takeshi Kondo's world-leading ARPES work on quantum materials.

\section{Research Philosophy and Conclusion}

My experiences have crystallized a clear evolution: \textbf{the best physics happens when innovative experimental tools meet fundamental questions about nature}. I consistently:

\begin{enumerate}
    \item \textbf{Question-driven, not technique-driven}: View experimental methods as means to answer physics questions
    \item \textbf{Bridge theory and practice}: Combine mathematical rigor with hands-on validation
    \item \textbf{Embrace complexity}: Tackle challenges requiring sustained effort (2-year nanophotonics, iterative D2NN development)
    \item \textbf{Communicate across boundaries}: Teaching, collaboration, effective scientific writing
\end{enumerate}

My accomplishments—from pioneering single-layer D2NNs (97\%+ accuracy) to leading scientific outreach (200+ students)—demonstrate the \textbf{technical skills, physics intuition, independence of mind, and communication abilities} necessary for success in a rigorous five-year doctoral program. I have proven my ability to build complex experimental systems, bridge computation and experiment, ask deep physics questions, and persevere through long-term challenges.

I am excited to bring this combination of \textbf{hands-on ultrafast optics expertise, computational problem-solving, and physics question-driven mindset} to the University of Tokyo. Working with Prof.~Takeshi Kondo's laboratory, I aim to master band-selective ultrafast spectroscopy to disentangle electron-phonon coupling, electron-electron correlations, and topological structure in quantum materials—contributing transformative advances to experimental many-body physics while growing as an independent scientist capable of asking and answering frontier questions about the quantum world.

Thank you for considering my application.

\vspace{3em}

\noindent
\textbf{Xiaoyang Zheng} $\cdot$ Beijing Normal University $\cdot$ UC Berkeley (BPIE, 2025)\\
Email: \href{mailto:xiaoyangzheng@mail.bnu.edu.cn}{xiaoyangzheng@mail.bnu.edu.cn} $\cdot$ Phone: +86-13955190184 $\cdot$ Language: IELTS 7.5, TOEFL 101

\end{document}
