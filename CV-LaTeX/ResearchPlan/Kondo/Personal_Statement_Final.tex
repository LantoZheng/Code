\documentclass[11pt,a4paper]{article}

% --- 宏包设置 ---
\usepackage{fontspec}
\setmainfont{Times New Roman}
\usepackage{geometry}
\geometry{margin=0.65in}
\usepackage{setspace}
\setstretch{0.92}
\usepackage{microtype}
\usepackage{hyperref}
\hypersetup{
    colorlinks=true,
    linkcolor=blue,
    urlcolor=blue,
    citecolor=blue
}
\usepackage{enumitem}
\setlist{nosep, leftmargin=*, itemsep=1pt, topsep=2pt, parsep=0pt}
\usepackage{titlesec}

% 极度紧凑的章节标题格式
\titleformat{\section}
  {\normalsize\bfseries}
  {}
  {0em}
  {}
\titlespacing*{\section}{0pt}{6pt}{2pt}

\titleformat{\subsection}
  {\small\bfseries}
  {}
  {0em}
  {}
\titlespacing*{\subsection}{0pt}{4pt}{1pt}

% 移除页码
\pagenumbering{gobble}

% --- 文档开始 ---
\begin{document}

% --- 标题 ---
\begin{center}
    {\large \textbf{Personal Statement}}\\[0.2em]
    {\normalsize Master's Program, Graduate School of Science, University of Tokyo}\\[0.4em]
    \textbf{Xiaoyang Zheng} $\cdot$ Beijing Normal University $\cdot$ \href{mailto:xiaoyangzheng@mail.bnu.edu.cn}{xiaoyangzheng@mail.bnu.edu.cn}
\end{center}

\vspace{-0.4em}

% --- 正文 ---
\section{Introduction: Building Bridges Between Applied and Fundamental Physics}

As a final-year undergraduate Physics major at Beijing Normal University (ranked 2/23, GPA 3.7/4.0), I have developed a research profile combining \textbf{experimental innovation}, \textbf{computational problem-solving}, and \textbf{scientific leadership}. My journey from designing diffractive neural networks to characterizing nanophotonic materials has taught me the independence of mind, perseverance, and collaborative spirit essential for frontier research. This statement highlights key accomplishments demonstrating my ability and readiness for advanced graduate studies.

\section{Research Accomplishments: Initiative and Innovation}

\subsection{Pioneering Single-Layer Diffractive Neural Networks (2024-Present)}

My most significant independent achievement has been designing a \textbf{single-layer diffractive neural network (D2NN)} for optical computing, addressing critical limitations in existing multi-layer architectures:

\textbf{Problem \& Innovation:} Traditional multi-layer D2NNs suffer from significant intensity loss due to cascaded diffraction. I proposed using a \textbf{single phase mask in a 4f optical system}, requiring me to derive the optical transfer function from first-principles diffraction theory, design spatial light modulator (SLM) phase patterns for the Fourier plane, and implement a custom PyTorch training pipeline bridging optical physics and machine learning.

\textbf{Results:} Achieved \textbf{97\%+ accuracy} on MNIST classification through optical simulation, demonstrating single-layer architectures can rival multi-layer designs while eliminating cascaded losses. This showcases my ability to identify problems in established approaches, develop innovative solutions independently, and integrate optics, physics, and AI.

\subsection{Self-Calibrating Beam Shaping System (2024)}

For an advanced optics course, I went beyond standard requirements by \textbf{designing and constructing a complete optical system from scratch}. Recognizing that real-world laser beams suffer from wavefront distortions, I developed a \textbf{self-calibrating feedback control system} integrating: (1) a reflective SLM for dynamic phase modulation, (2) a CCD camera with microlens array as wavefront sensor, and (3) custom Python software implementing optimization algorithms (SGD, simulated annealing) for real-time wavefront error minimization.

\textbf{Overcoming challenges:} Required systematic troubleshooting of optical alignment precision, SLM calibration, camera-SLM synchronization, and algorithm convergence. Successfully demonstrated \textbf{real-time wavefront correction} and generation of Gaussian and Laguerre-Gaussian beam profiles. This project exemplifies my hands-on experimental ability, systems integration skills (optics + electronics + software), and capacity to deliver functional prototypes from conceptual designs.

\subsection{Nanophotonics Research (2022-2024)}

As an \textbf{undergraduate research assistant} under Prof.~Jinwei Shi (Beijing Undergraduate Research Training Program), I contributed to nanophotonics research on plasmon resonances in gold nanostructures: characterized nanoparticles using \textbf{SEM/TEM}, performed optical spectroscopy to analyze plasmon resonances, conducted \textbf{FDTD electromagnetic simulations} studying gold nanorod interactions with 2D materials, and investigated how nanorod dimensions affect absorption spectra.

\textbf{Scientific growth:} This two-year experience taught me to transition from following protocols to \textbf{asking independent research questions}, critically analyze experimental data, propose hypotheses, and design simulations to test predictions. The iterative experiment-simulation-refinement process instilled the patience and rigor required for experimental physics research.

\subsection{Atomic Magnetometry: Rapid Cross-Disciplinary Learning (Summer 2023)}

During summer research at USTC under Prof.~Dong Sheng, I contributed to developing \textbf{atomic co-magnetometers} for precision measurements—a field outside my optics background. I quickly learned atomic physics fundamentals and contributed meaningfully: conducted \textbf{COMSOL thermal simulations} optimizing temperature stability, participated in optical system construction for atomic excitation/detection, and assisted in developing signal acquisition electronics. This demonstrates my rapid learning ability, adaptability to new research fields, and capacity to contribute effectively in fast-paced, interdisciplinary environments.

\section{Leadership and Scientific Communication}

\subsection{Chairman, BNU Photographer Association (2023-2024)}

As BNUPA chairman, I transformed the organization into a platform for scientific education, organizing \textbf{3 expert lectures and 2 interviews engaging ~200 students}. I delivered two lecture series bridging physics and photography: \textit{"Optical Concepts in Photography"} (explaining diffraction limits, lens aberrations, Fourier optics in camera design) and \textit{"Imaging System Quality"} (discussing MTF, resolution, sensor technology). This required effective public speaking, ability to communicate complex physics to non-specialists, team management, and initiative in creating educational opportunities.

\textbf{Communication philosophy:} Effective scientific communication requires not only deep understanding but also empathy—anticipating audience knowledge and crafting resonant explanations. This skill is invaluable in graduate research requiring collaboration across diverse backgrounds.

\subsection{Homoludens Archive (2022-2023)}

As archives administrator, I contributed to \textbf{preserving institutional academic history}, requiring meticulous documentation and systematic organization—qualities equally important in experimental research where proper data management and reproducibility are paramount.

\section{Academic Resilience and Continuous Growth}

\subsection{Balancing Dual Degrees}

Pursuing a \textbf{double major in Physics and Economics} while maintaining competitive standing (2/23, GPA 3.7) required exceptional time management, strategic prioritization, and sustained high performance in technical courses: Optics (93), Computational Physics (95), Electromagnetism (97), Quantum Mechanics (89). This dual-degree experience sharpened my ability to work efficiently under pressure, integrate knowledge across disciplines, and maintain focus on long-term goals despite obstacles.

\subsection{International Adaptation: UC Berkeley Exchange (2025)}

My participation in the \textbf{Berkeley Physics International Education (BPIE) Program} represents proactive initiative-taking: adapting to an entirely English-language environment (IELTS 7.5, TOEFL 101), engaging with cutting-edge condensed matter physics at a world-leading institution, building cross-cultural communication skills, and establishing international research networks. This experience has reinforced my confidence in thriving in diverse academic cultures—crucial preparation for graduate study in Japan.

\subsection{Proactive Skill Development}

My strong performance in \textbf{Computational Physics (95/100)} reflects deliberate skill acquisition. I taught myself Python, PyTorch, and numerical methods—skills not traditionally emphasized in Chinese physics curricula—recognizing their importance for contemporary research. This proactive approach demonstrates my independence of mind and commitment to becoming a well-rounded researcher.

\section{Research Philosophy: Intersection of Disciplines}

My experiences have crystallized a clear philosophy: \textbf{the best science happens at the intersection of disciplines, where innovative techniques meet fundamental questions}. I consistently:

\begin{enumerate}
    \item \textbf{Take initiative}: Identify problems independently and propose novel solutions
    \item \textbf{Bridge theory and practice}: Combine mathematical rigor with hands-on experimental validation
    \item \textbf{Embrace challenges}: View setbacks as learning opportunities
    \item \textbf{Communicate effectively}: Share knowledge through teaching, presentations, collaboration
    \item \textbf{Think long-term}: Commit to sustained effort on complex projects requiring persistence
\end{enumerate}

\section{Conclusion}

My accomplishments—from pioneering single-layer D2NNs to leading scientific outreach—demonstrate the \textbf{technical skills, independence of mind, communication abilities, and resilience} necessary for success in rigorous graduate programs. I have proven my ability to work independently on novel problems, collaborate effectively, communicate complex ideas to diverse audiences, and persevere through challenges.

I am excited to bring this combination of hands-on experimental expertise, computational problem-solving, and collaborative spirit to the University of Tokyo, where I can contribute to frontier research in ultrafast spectroscopy of quantum materials while continuing to grow as an independent scientist and effective communicator.

Thank you for considering my application.

\vspace{0.3em}

\noindent
\textbf{Xiaoyang Zheng} $\cdot$ Beijing Normal University (Physics, Rank 2/23, GPA 3.7/4.0) $\cdot$ UC Berkeley (BPIE, 2025)\\
Email: \href{mailto:xiaoyangzheng@mail.bnu.edu.cn}{xiaoyangzheng@mail.bnu.edu.cn} $\cdot$ Phone: +86-13955190184

\end{document}
