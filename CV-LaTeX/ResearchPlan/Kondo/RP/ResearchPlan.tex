\documentclass[12pt,a4paper]{article}
% Use fontspec for XeLaTeX
\usepackage{fontspec}
% Use a common system font; change if you prefer another installed font
\setmainfont{Times New Roman}
\usepackage{geometry}
\geometry{margin=1in}
\usepackage{setspace}
\onehalfspacing
\usepackage{microtype}
% Use biblatex for bibliography management with biber backend
\usepackage[backend=biber,style=numeric,sorting=none]{biblatex}
\usepackage{hyperref}
\usepackage{graphicx}
\usepackage{amsmath}
\usepackage{csquotes}
\usepackage{enumitem}
\usepackage{xcolor}
\usepackage{soul}
\addbibresource{ResearchPlan.bib}

% Define custom commands for advisor review notes
\newcommand{\advisornote}[1]{{\color{red}\textbf{[ADVISOR REVIEW: #1]}}}
\newcommand{\verify}[1]{{\color{blue}\textbf{[VERIFY: #1]}}}
\newcommand{\todo}[1]{{\color{orange}\textbf{[TODO: #1]}}}

% Bibliography: use biblatex + biber. Run sequence: xelatex -> biber -> xelatex -> xelatex
% Citations in the document use \cite{key} and will be rendered by biblatex.
\title{Research Plan}
\author{Xiaoyang Zheng}
\date{\today}

% --- Document ---
\begin{document}

\begin{center}
    \large \textbf{Band-Selective Ultrafast Dynamics and Non-Thermal Control in the Topological Kagome Metal CsV$_3$Sb$_5$}
\end{center}
\sloppy

% Context Box for PhD Application
\noindent\fbox{\parbox{\dimexpr\textwidth-2\fboxsep-2\fboxrule}{%
\textbf{APPLICATION CONTEXT}

\textbf{Program:} Five-Year Doctoral Program at the University of Tokyo, GSGC\\
\textbf{Potential Supervisor:} Professor Takao Kondo\\
\textbf{Background:} Final-year undergraduate with experience in optical systems (D2NN, SLM-based beam shaping) and nanophotonics research

\textbf{Research Plan Objectives:}
\begin{itemize}[noitemsep,topsep=0pt]
    \item Demonstrate deep understanding of frontier physics questions in ultrafast quantum materials
    \item Define key measurable observables and quantitative success criteria
    \item Outline a comprehensive 5-year doctoral research trajectory addressing fundamental questions
    \item Show alignment with Kondo Laboratory's core expertise and research directions
\end{itemize}
}}

\vspace{0.3cm}

\begin{abstract}
The AV$_3$Sb$_5$ kagome metals represent a unique platform where geometric frustration, electronic topology, and strong correlations converge. Despite foundational ultrafast studies revealing coherent CDW oscillations, a fundamental question remains unresolved: \textbf{Can we disentangle the contributions of electron-phonon (e-ph) versus electron-electron (e-e) interactions in driving the CDW formation, and how does the interplay between electronic correlations and topological band structure manifest in non-equilibrium states?} This five-year doctoral research program proposes to answer these questions through \textbf{band-selective ultrafast spectroscopy} combined with momentum-resolved techniques. By systematically tuning pump photon energy across key inter-band transitions—targeting van Hove singularities at M-points versus Dirac-like states at K-points—and employing dual-modality measurements (pump-probe reflectivity for time resolution + time-resolved ARPES for momentum resolution), I will extract \textbf{quantitative observables}: transient energy shifts $\Delta E(t)$ (meV precision), carrier relaxation timescales $\tau_{e-e} \sim 50$--100 fs and $\tau_{e-ph} \sim 0.5$--2 ps, electron-phonon coupling strength $\lambda \sim 0.3$--0.8, and quasiparticle renormalization factor $Z(k,t)$ at topologically non-trivial points. The research is structured in three phases: (1) \textbf{Years 1--2 (Foundation)}: Establish band-selective methodology in CsV$_3$Sb$_5$, disentangle e-ph/e-e channels, achieve 1--2 publications (PRB); (2) \textbf{Years 3--4 (Platform Extension)}: Expand to cuprate superconductors (leveraging Kondo Lab's 5/5 expertise) to map pseudogap dynamics and test universality of correlation-topology interplay, target high-impact publication (PRL/Nature Communications); (3) \textbf{Year 5 (Frontier)}: Explore photo-induced metastable phases or develop machine-learning-accelerated spectroscopy. This program addresses Kondo Laboratory's strategic direction in ultrafast quantum materials while establishing band-selective spectroscopy as a quantitative tool for many-body physics. My optical systems background (D2NN, SLM wavefront control) provides direct technical preparation for OPA tuning and signal optimization in pump-probe experiments.
\end{abstract}

\section{Research Background}
The recent discovery of the AV$_3$Sb$_5$ (A = K, Rb, Cs) family of kagome metals has galvanized the condensed matter physics community \cite{Ortiz2019, Wilson2021}. These materials exhibit a cascade of correlated phases, including a unique charge density wave (CDW) with a Star-of-David (SoD) distortion below $T_{CDW} \approx 94$~K, followed by superconductivity at low temperatures \cite{Neupert2022}. A central, unresolved question is the microscopic origin of this CDW. While Fermi surface nesting involving the prominent van Hove singularities (vHS) at the M-points of the Brillouin zone is a leading theory, strongly supported by ARPES data \cite{Kang2022}, other mechanisms involving local correlations are also debated. Further complicating this picture is the mounting evidence for a time-reversal symmetry-breaking, \textit{chiral} charge order, hinting at an even more exotic ground state with intertwined orders \cite{Jiang2021, Shrestha2023}.

Time-resolved optical spectroscopy is a powerful, all-optical technique capable of directly probing the fundamental interactions, collective modes, and energy relaxation pathways that define such correlated ground states \cite{Demsar2007, Giannetti2016}. Seminal pump-probe studies on CsV$_3$Sb$_5$ and the related compound KV$_3$Sb$_5$ have successfully identified the coherent oscillations of several A$_{1g}$ symmetry phonons, including the amplitude mode of the CDW order parameter (at $\sim$2.9 THz) \cite{Uykur2022}.

However, these pioneering works have a crucial limitation: they were performed using a fixed, non-resonant pump photon energy (typically 1.55 eV or 800 nm). This simultaneously excites electrons across a wide range of states and momentum space, yielding a \textit{band-integrated} response. The core innovation of this proposal is to transition from this band-integrated to a \textbf{band-selective} viewpoint. This methodology has been successfully employed to disentangle competing energy gaps and coupling channels in other strongly correlated systems, such as the high-temperature cuprate superconductors \cite{Coslovich2015, Giannetti2016}. By tuning the pump photon energy, we can directly test the central hypothesis that the CDW is predominantly coupled to the vHS at the M-points \cite{Miao2021, Liang2021}.

\section{Proposed Research and Objectives}

\subsection{Core Physics Questions}

This doctoral research addresses two fundamental questions in quantum many-body physics:

\textbf{Question 1: How can we disentangle electron-phonon versus electron-electron interactions in correlated materials?} In the AV$_3$Sb$_5$ kagome metals, the CDW formation mechanism remains debated—is it driven by momentum-specific Fermi surface nesting (e-ph dominated) or by local Coulomb correlations (e-e dominated)? Conventional fixed-energy pump-probe studies cannot distinguish these channels because they excite all electronic states simultaneously.

\textbf{Question 2: How does the interplay between electronic correlations and topological band structure manifest in non-equilibrium states?} The van Hove singularities and Dirac-like crossings in these materials provide momentum-space markers. Do correlation effects (many-body renormalization) and topological features (Berry curvature, orbital textures) exhibit different responses to ultrafast perturbations?

\subsection{Objective 1: Disentangle Electron-Phonon and Electron-Electron Interactions via Band-Selective Dynamics}

\textbf{Measurable Observables:}
\begin{itemize}
    \item \textbf{Carrier relaxation times}: $\tau_{e-e} \sim 50$--100 fs (electron-electron thermalization) vs.~$\tau_{e-ph} \sim 0.5$--2 ps (electron-phonon coupling), extracted from multi-exponential fits of $\Delta R/R(t)$
    \item \textbf{CDW gap dynamics}: Transient gap magnitude $\Delta_{\text{CDW}}(t)$ with meV-scale precision, revealing order parameter melting timescale $\tau_{\text{CDW}} \sim 5$--50 ps
    \item \textbf{Electron-phonon coupling constant}: Dimensionless $\lambda \sim 0.3$--0.8, extracted from Allen's formula using measured $\tau_{e-ph}$ and phonon density of states
    \item \textbf{Momentum-resolved coupling strength}: Excitation spectrum $A_i(E_{\text{pump}})$ mapping CDW amplitude mode coupling across Brillouin zone
\end{itemize}

\textbf{Approach:} Perform pump-energy-dependent measurements of transient reflectivity ($\Delta R/R$) by tuning across 0.8--1.6 eV. By fitting the oscillatory component $A(t) = \sum_{i} A_i e^{-t/\tau_i} \cos(\omega_i t + \phi_i)$, we extract the CDW mode amplitude ($A_i$). Plotting $A_i$ versus $E_{\text{pump}}$ yields the "excitation spectrum," revealing momentum-selective coupling.

\textit{Theoretical Expectation:} If CDW is driven by vHS at M-points (as ARPES suggests \cite{Kang2022}), $A_i$ should peak when $E_{\text{pump}}$ matches M-point inter-band transitions ($\sim$1.0--1.5 eV). Pumping at K-point Dirac states ($\sim$0.8--1.0 eV) should yield weaker coupling. The distinct timescales ($\tau_{e-e}$ vs.~$\tau_{e-ph}$) and their pump-energy dependence directly distinguish interaction channels.

\textbf{Success Criteria:} (1) Map $A_i(E_{\text{pump}})$ across 10+ pump energies with statistical significance $>3\sigma$; (2) Extract $\tau_{e-e}$ and $\tau_{e-ph}$ with $<20\%$ error bars; (3) Determine dominant CDW driving mechanism with quantitative coupling constants.

\subsection{Objective 2: Probe Correlation-Topology Interplay via Momentum-Resolved Dynamics}

\textbf{Measurable Observables:}
\begin{itemize}
    \item \textbf{Quasiparticle renormalization}: $Z(k,t)$ at van Hove (M-point) and Dirac (K-point) regions, tracking many-body mass enhancement dynamics
    \item \textbf{Transient bandwidth}: $W(t)$ evolution near topological features, revealing correlation-induced renormalization timescales
    \item \textbf{Energy shift at van Hove singularity}: $\Delta E_{\text{vHS}}(t)$ with meV precision, directly measuring electronic temperature and chemical potential shifts
    \item \textbf{Momentum-resolved carrier population}: $n(k,t)$ from time-resolved ARPES (dual-modality approach), mapping non-equilibrium distribution functions
\end{itemize}

\textbf{Approach:} Combine pump-probe reflectivity (time resolution) with time-resolved ARPES (momentum resolution) in a dual-modality strategy. At key pump energies identified in Objective 1, perform trARPES to directly observe band dispersion $E(k,t)$ and spectral function $A(k,\omega,t)$ evolution on femtosecond timescales.

\textit{Physical Insight:} Topological features (Dirac cones, van Hove points) act as "momentum-space probes." If correlations are momentum-dependent, we expect: (1) Faster thermalization at flat-band regions (high density of states); (2) Different $Z$ renormalization near topological vs.~trivial bands; (3) Asymmetric dynamics above/below Fermi level revealing particle-hole asymmetry.

\textbf{Success Criteria:} (1) Obtain trARPES spectra with $\Delta k < 0.05$ Å$^{-1}$ resolution and $<100$ fs time steps; (2) Quantify $Z(k)$ evolution with $<30\%$ uncertainty; (3) Establish correlation between band topology markers and dynamics timescales.

\subsection{Objective 3: Ultrafast Probe of Chiral CDW and Time-Reversal Symmetry Breaking} Recent scanning tunneling microscopy (STM) and X-ray scattering studies have provided tantalizing evidence for a time-reversal symmetry-breaking chiral charge order in AV$_3$Sb$_5$ \cite{Jiang2021, Shrestha2023}. However, direct dynamical evidence remains elusive. We propose to perform the first circularly-polarized pump-probe experiment on this system.

\textit{Physical Mechanism:} A chiral CDW, characterized by a complex order parameter $\Delta e^{i\phi(r)}$ with non-trivial Berry phase, couples differently to left- and right-handed circularly polarized light ($\sigma^+$ and $\sigma^-$) through orbital angular momentum selection rules \cite{Wang2020}. This circular dichroism should manifest as a differential transient reflectivity signal: $\Delta R_{\text{CD}} = (\Delta R/R)_{\sigma^+} - (\Delta R/R)_{\sigma^-}$.

\textit{Expected Signal:} Based on theoretical estimates of the chiral order parameter magnitude ($\sim$1--5\% of the total CDW amplitude) and analogous studies in magnetic Weyl semimetals, we anticipate $\Delta R_{\text{CD}} / \Delta R \sim 10^{-3}$--$10^{-2}$. This sensitivity is achievable with modern balanced detection and lock-in amplification techniques \cite{Giannetti2016}, which routinely reach $\Delta R/R \sim 10^{-5}$--$10^{-6}$ in ultrafast spectroscopy. This small but measurable signal requires careful lock-in detection and averaging over multiple scans. We will perform these measurements as a function of temperature across $T_{\text{CDW}}$ and pump fluence to map the phase diagram of the chiral component. A null result would also be significant, constraining theories of the CDW ground state.

\section{Methodology}

\subsection{Experimental Setup}
The experiments will be conducted using the Kondo Laboratory's Ti:Sapphire regenerative amplifier laser system (800 nm, 35 fs pulse duration, 1 kHz repetition rate) coupled with the \textbf{Optical Parametric Amplifier (OPA)} system to provide tunable pump wavelengths covering 0.8--1.6 eV (corresponding to 775--1550 nm). My experience with SLM-based optical systems (diffractive neural networks and beam shaping projects) has prepared me to quickly master the OPA tuning procedures and wavelength optimization. This energy range spans the key features in the electronic structure: the K-point Dirac states ($\sim$0.8--1.0 eV) and the M-point van Hove singularities ($\sim$1.0--1.5 eV). The probe pulse will be the fundamental 800 nm beam (1.55 eV), with $\sim$50 fs temporal resolution set by the pump-probe cross-correlation.

Sample reflectivity changes will be measured using a balanced photodetector with lock-in amplification, providing sensitivity of $\Delta R/R \sim 10^{-5}$. The pump fluence will be varied from 10 to 500 $\mu$J/cm$^2$ to study both the linear response regime and non-thermal melting dynamics. For circularly-polarized measurements (Objective 3), we will employ achromatic quarter-wave plates optimized for the visible to near-infrared range, with polarization modulation at 500 Hz to enable differential detection.

\subsection{Sample Preparation and Characterization}
High-quality single crystals of CsV$_3$Sb$_5$ will be obtained through the Kondo Laboratory's established collaborations with materials synthesis groups. Sample quality will be verified by: (i) X-ray diffraction to confirm the P6/mmm crystal structure and absence of secondary phases; (ii) transport measurements to confirm $T_{\text{CDW}} \approx 94$ K; and (iii) visual inspection for typical dimensions of $\sim$1--3 mm lateral size and 50--200 $\mu$m thickness. Samples will be cleaved in vacuum or inert atmosphere immediately before optical measurements to ensure a pristine surface. All measurements will be performed in the continuous-flow helium cryostat with temperature control from 10 to 300 K (stability $\pm$0.5 K).

\subsection{Data Analysis}
My computational background (Python, MATLAB, and experience with signal processing from my undergraduate research) will be applied to: (i) multi-exponential fitting of transient reflectivity traces using non-linear least-squares algorithms; (ii) Fourier analysis with zero-padding and Welch windowing to extract phonon frequencies and Fano lineshapes; (iii) principal component analysis to identify temperature- and fluence-dependent trends; and (iv) Monte Carlo simulations to estimate error bars and confidence intervals for extracted parameters.

\section{Research Timeline and Skill Development Plan}

The five-year doctoral program is structured in three progressive phases aligned with skill development, research depth, and publication goals:

\subsection{Phase 1: Foundation (Years 1--2)}

\textbf{Year 1 Goals:} Laboratory integration, technique mastery, establish band-selective methodology in CsV$_3$Sb$_5$

\textbf{Months 1--6:}
\begin{itemize}
    \item \textbf{Coursework:} Advanced Quantum Mechanics, Many-Body Physics, Experimental Methods in Condensed Matter
    \item \textbf{Language:} Intensive Japanese (target JLPT N4 by Month 6, daily lab communication fluency)
    \item \textbf{Lab training:} Laser safety, Ti:Sapphire/OPA systems, cryogenics, lock-in detection
    \item \textbf{Literature mastery:} Systematic review of ultrafast spectroscopy in correlated materials (50+ papers)
    \item \textbf{Preliminary experiments:} Room-temperature pump-probe on reference materials (Au, graphite)
\end{itemize}

\textbf{Months 7--12:}
\begin{itemize}
    \item \textbf{OPA mastery:} Wavelength tuning, power optimization, spectral characterization across 0.8--1.6 eV
    \item \textbf{CsV$_3$Sb$_5$ characterization:} Baseline reflectivity, verify $T_{\text{CDW}}$, optimize sample mounting
    \item \textbf{Pump-energy mapping:} Systematic scan of $A_i(E_{\text{pump}})$ at T=10 K (10+ energies, establish excitation spectrum)
    \item \textbf{Data pipeline:} Develop Python analysis suite for multi-exponential fitting, Fourier analysis, uncertainty quantification
    \item \textbf{Milestone:} First conference presentation (JPS Annual Meeting), complete Objective 1 preliminary data
\end{itemize}

\textbf{Year 2 Goals:} Complete e-ph/e-e disentanglement, begin trARPES collaboration, first major publication

\textbf{Months 13--18:}
\begin{itemize}
    \item \textbf{Temperature/fluence mapping:} Detailed $T$-dependent (10--120 K) and fluence-dependent (10--500 $\mu$J/cm$^2$) studies at key pump energies
    \item \textbf{Timescale extraction:} Quantify $\tau_{e-e}$, $\tau_{e-ph}$, $\tau_{\text{CDW}}$ with $<20\%$ error, extract $\lambda$ coupling constant
    \item \textbf{trARPES initiation:} Begin collaboration with Kondo Lab's trARPES system development, learn angle-resolved techniques
    \item \textbf{Theory collaboration:} Work with theorists on band structure interpretation, DFT+DMFT comparison
\end{itemize}

\textbf{Months 19--24:}
\begin{itemize}
    \item \textbf{Circular dichroism:} Attempt chiral CDW detection via polarization-resolved pump-probe
    \item \textbf{Manuscript preparation:} Draft first major paper targeting \textit{Physical Review B}: "Band-Selective Disentanglement of Electron-Phonon and Electron-Electron Interactions in CsV$_3$Sb$_5$"
    \item \textbf{International conference:} APS March Meeting presentation (oral or poster)
    \item \textbf{Milestone:} Submit/publish first-author PRB paper, pass doctoral candidacy examination
\end{itemize}

\subsection{Phase 2: Platform Extension and Research Leadership (Years 3--4)}

\textbf{Year 3 Goals:} Expand technique to multiple material platforms, achieve momentum-resolved dynamics via trARPES, high-impact publication

\textbf{Months 25--30:}
\begin{itemize}
    \item \textbf{trARPES mastery:} Hands-on training with time-resolved ARPES system, achieve $<100$ fs temporal and $<0.05$ Å$^{-1}$ momentum resolution
    \item \textbf{Dual-modality experiments:} Combine pump-probe reflectivity + trARPES on CsV$_3$Sb$_5$ to directly observe $E(k,t)$, $Z(k,t)$ evolution
    \item \textbf{Correlation-topology mapping:} Quantify dynamics differences at van Hove (M-point) vs.~Dirac (K-point) features (Objective 2)
    \item \textbf{Platform comparison:} Begin measurements on sister compound KV$_3$Sb$_5$ or related kagome metal to test universality
\end{itemize}

\textbf{Months 31--36:}
\begin{itemize}
    \item \textbf{Advanced techniques:} Implement pump-pump-probe or coherent 2D spectroscopy (if equipment available)
    \item \textbf{Manuscript 2:} Draft high-impact paper targeting \textit{Physical Review Letters} or \textit{Nature Communications}: "Momentum-Resolved Correlation-Topology Interplay in Non-Equilibrium Kagome Metals"
    \item \textbf{Fellowship application:} Apply for JSPS DC2 fellowship (if eligible) or GSGC excellence scholarship
    \item \textbf{Mentorship:} Begin supervising 1--2 undergraduate/Master's students on related projects
    \item \textbf{Milestone:} Submit high-impact paper, achieve 3+ total publications (1 submitted/published PRL/Nat.Comm., 1--2 PRB)
\end{itemize}

\textbf{Year 4 Goals:} Strategic expansion to cuprate superconductors, leverage Kondo Lab core expertise, establish research independence

\textbf{Rationale for Cuprate Focus:} Kondo Laboratory has world-leading expertise (5/5 rating) in cuprate ARPES and pseudogap physics. Applying band-selective ultrafast techniques to cuprates addresses fundamental questions (e-ph vs.~e-e in pseudogap formation, Cooper pair dynamics) while maximizing lab synergy and demonstrating technique transferability. This strategic choice balances novelty (new technique application) with feasibility (lab infrastructure + sample access).

\textbf{Months 37--42:}
\begin{itemize}
    \item \textbf{Cuprate sample preparation:} Collaborate with synthesis groups for high-quality BSCCO, LSCO, or YBaCuO single crystals
    \item \textbf{Pseudogap dynamics:} Apply band-selective pump-probe to map $\Delta_{\text{PG}}(T,k,t)$ evolution, distinguish pairing vs.~competing order scenarios
    \item \textbf{Antinodal/nodal comparison:} Leverage momentum selectivity to compare dynamics at antinodal (strong correlation) vs.~nodal (weak correlation) regions
    \item \textbf{Theory integration:} Collaborate with cuprate theorists on interpreting results within t-J model, Hubbard model frameworks
\end{itemize}

\textbf{Months 43--48:}
\begin{itemize}
    \item \textbf{Photo-doping experiments:} Explore transient superconductivity enhancement or photo-induced pairing via controlled carrier injection
    \item \textbf{Manuscript 3:} Target \textit{Physical Review X} or \textit{npj Quantum Materials}: "Band-Selective Ultrafast Spectroscopy Reveals Pseudogap Dynamics in Cuprate Superconductors"
    \item \textbf{International collaboration:} Research visit to collaborating group (1--2 months) for technique exchange or complementary measurements
    \item \textbf{Milestone:} Establish band-selective spectroscopy as premier tool for cuprate dynamics, 4--5 total publications
\end{itemize}

\subsection{Phase 3: Frontier Exploration and Dissertation (Year 5)}

\textbf{Year 5 Goals:} Explore high-risk/high-reward directions, complete dissertation, transition to postdoctoral career

\textbf{Months 49--54:}
\begin{itemize}
    \item \textbf{Frontier direction (choose one based on prior results):}
    \begin{itemize}
        \item \textit{Option A - Photo-induced phases:} Systematically map light-induced metastable CDW/SC states, explore Floquet engineering
        \item \textit{Option B - ML acceleration:} Develop machine learning models for real-time spectral analysis, automated phase classification
        \item \textit{Option C - New materials:} Extend to twisted bilayer graphene, moiré superlattices, or other emerging platforms
    \end{itemize}
    \item \textbf{Final experiments:} Address committee feedback, fill remaining gaps in story
    \item \textbf{Student mentorship:} Transfer knowledge to 2--3 junior lab members, ensure project continuity
\end{itemize}

\textbf{Months 55--60:}
\begin{itemize}
    \item \textbf{Dissertation writing:} Comprehensive 200--300 page thesis covering all three phases, emphasizing physics insights and methodological innovations
    \item \textbf{Final manuscripts:} Complete any pending publications (target 5--6 total first-author papers)
    \item \textbf{Defense preparation:} Practice presentation, prepare for committee questions
    \item \textbf{Career transition:} Apply for postdoctoral positions at world-leading institutions (Stanford, MIT, Max Planck, etc.)
    \item \textbf{Job talks:} Present research at target institutions
    \item \textbf{Milestone:} Successfully defend PhD dissertation, accept postdoctoral offer
\end{itemize}

\subsection{Expected Publications and Impact}

\textbf{Target Publications (5--6 first-author papers):}
\begin{enumerate}
    \item \textit{Physical Review B} (Year 2): Band-selective e-ph/e-e disentanglement in kagome metals
    \item \textit{Physical Review Letters} or \textit{Nature Communications} (Year 3): Correlation-topology interplay from momentum-resolved dynamics
    \item \textit{Physical Review B} (Year 3): Platform extension to sister compounds or related materials
    \item \textit{Physical Review X} or \textit{npj Quantum Materials} (Year 4): Cuprate pseudogap dynamics with band-selective spectroscopy
    \item \textit{Review of Scientific Instruments} or \textit{Physical Review Applied} (Year 4--5): Methodological paper on dual-modality techniques
    \item \textit{Nature Physics/Materials} or high-impact specialized journal (Year 5): Frontier exploration (photo-induced phases or ML application)
\end{enumerate}

\textbf{Broader Impact:}
\begin{itemize}
    \item Establish band-selective ultrafast spectroscopy as quantitative tool for many-body physics (cited methodology)
    \item Train 2--3 junior researchers in advanced techniques (mentorship legacy)
    \item Build international collaboration network through conferences, visits, joint publications
    \item Contribute to Kondo Lab's strategic direction in ultrafast quantum materials
\end{itemize}

\subsection{Integration and Cultural Adaptation Plan}

\textbf{Language Development (5-year progression):}
\begin{itemize}
    \item \textbf{Year 1:} Intensive Japanese courses, daily practice (target JLPT N4 by Month 6, N3 by Month 12)
    \item \textbf{Year 2:} Conversational fluency in lab meetings, read Japanese physics papers (target N2 by Year 2 end)
    \item \textbf{Years 3--4:} Technical presentation skills, seminar talks in Japanese (target N1 by Year 4)
    \item \textbf{Year 5:} Full professional fluency for thesis defense, job interviews, teaching (if TA)
\end{itemize}

\textbf{Professional Development:}
\begin{itemize}
    \item \textbf{Leadership transition:} From student (Y1--2) → junior researcher (Y3--4) → research leader (Y5)
    \item \textbf{Mentorship:} Supervise 2--3 undergraduates/Master's students starting Year 3
    \item \textbf{Fellowship applications:} JSPS DC2, GSGC excellence scholarship, travel grants
    \item \textbf{Teaching experience:} Serve as TA for undergraduate physics labs (Years 2--4)
    \item \textbf{Grant writing:} Contribute to Prof.~Kondo's grant proposals (experience for future PI career)
\end{itemize}

\textbf{International Network Building:}
\begin{itemize}
    \item \textbf{Conferences:} Annual presentations at APS, MRS, ICSCE, JPS (domestic + international)
    \item \textbf{Research visit:} 1--2 month visit to collaborating group (Year 4) for technique exchange
    \item \textbf{Collaboration:} Co-author papers with theory groups, synthesis collaborators
    \item \textbf{Postdoc networking:} Attend career workshops, connect with potential postdoc advisors at conferences
\end{itemize}

\section{Risk Assessment and Mitigation Strategies}

\textbf{Risk 1: Weak or absent pump-energy dependence.} If the excitation spectrum $A_i(E_{\text{pump}})$ shows no clear resonance features, it could indicate local correlation dominance rather than momentum-specific nesting. \textit{Mitigation:} (1) Extend energy range down to 0.5 eV with second OPA stage; (2) Implement complementary fluence-dependent studies; (3) Pursue alternative Objective (correlation-topology via trARPES). A null result remains scientifically valuable, ruling out simple nesting and motivating advanced theoretical models—suitable for PRB publication and thesis chapter.

\textbf{Risk 2: Sample availability challenges.} High-quality CsV$_3$Sb$_5$ or cuprate crystals may face synthesis delays. \textit{Mitigation:} (1) Leverage Kondo Lab's established collaborations and institutional networks; (2) Use sister compounds (KV$_3$Sb$_5$, RbV$_3$Sb$_5$) as backup platforms; (3) For cuprates, access multiple material families (BSCCO, LSCO, YBaCuO) through diverse collaborators; (4) 5-year timeline provides buffer to wait for high-quality samples rather than compromising on material quality.

\textbf{Risk 3: trARPES technical challenges.} Time-resolved ARPES requires simultaneous optimization of temporal ($<100$ fs), energy ($<50$ meV), and momentum ($<0.05$ Å$^{-1}$) resolution—technically demanding. \textit{Mitigation:} (1) Years 1--2 focus on pump-probe reflectivity (mature technique) to build foundation; (2) Collaborate closely with Kondo Lab's trARPES development team; (3) If technical hurdles persist, dual-modality strategy remains viable with separate static ARPES + time-resolved reflectivity correlation; (4) Consider alternative momentum-resolved probes (angle-resolved second-harmonic generation).

\textbf{Risk 4: Cuprate complexity exceeds timeline.} Cuprates involve multiple competing orders (CDW, pseudogap, superconductivity), complicating data interpretation. \textit{Mitigation:} (1) Focus on specific phase diagram region (e.g., underdoped pseudogap regime) rather than complete mapping; (2) Leverage Kondo Lab's extensive cuprate expertise for rapid interpretation; (3) Collaborate with established cuprate theorists; (4) Year 4 timing allows incorporating lessons from Years 1--3 kagome work; (5) Even partial cuprate results demonstrate technique transferability.

\textbf{Risk 5: Publication competition.} Ultrafast spectroscopy is highly active field with potential for scooping. \textit{Mitigation:} (1) Band-selective methodology provides unique angle vs.~conventional fixed-energy studies; (2) Dual-modality (reflectivity + trARPES) offers differentiation; (3) Rapid publication strategy (Years 2, 3, 4, 5) maintains priority; (4) Diversified platform strategy (kagome + cuprates + frontier) reduces single-topic risk; (5) Close monitoring of arXiv and conference proceedings for competitive awareness.

\textbf{Risk 6: Career transition uncertainties.} Postdoc job market competitiveness. \textit{Mitigation:} (1) Target 5--6 first-author publications with $\geq$1 high-impact (PRL/Nature family) for strong CV; (2) Build international network through conferences and research visit; (3) Develop versatile skillset (ultrafast + ARPES + ML/data science); (4) Pursue teaching experience as TA; (5) Apply to diverse postdoc opportunities (academia, national labs, industry research); (6) Year 5 timing allows job application cycles during dissertation writing.

\section{Preparation and Fit with Kondo Laboratory}

\subsection{How My Background Prepares Me for This Project}

My undergraduate research experience provides strong technical foundation for this doctoral program:

\textbf{Optical Systems Expertise:}
\begin{itemize}
    \item \textbf{D2NN Project:} Designed single-layer diffractive neural networks using SLM in 4f optical systems—directly transferable to OPA tuning, pump-probe beam alignment, Fourier-plane optics
    \item \textbf{Self-Calibrating Beam Shaping:} Developed feedback-controlled wavefront correction with reflective SLMs—applicable to real-time signal optimization, noise reduction, automated data acquisition
\end{itemize}

\textbf{Computational and Data Analysis Skills:}
\begin{itemize}
    \item Proficient in Python (PyTorch, SciPy), MATLAB—ready for multi-exponential fitting, Fourier analysis, machine learning applications
    \item Experience with optimization algorithms (SGD, simulated annealing)—transferable to parameter extraction, automated phase classification
    \item Strong computational physics background (95/100)—foundation for numerical modeling, Monte Carlo simulations
\end{itemize}

\textbf{Research Mindset:}
\begin{itemize}
    \item Top academic performance (2/23 ranking)—demonstrated ability to master complex material quickly
    \item International experience (UC Berkeley program)—prepared for research environment, English proficiency
    \item Interdisciplinary background (Physics + Economics double major)—broad perspective for problem-solving
\end{itemize}

\subsection{Why Kondo Laboratory and Five-Year Doctoral Program}

The Kondo Laboratory represents the ideal environment for several reasons:

\textbf{Technical Infrastructure:}
\begin{itemize}
    \item World-class ARPES systems with momentum resolution $<0.01$ Å$^{-1}$
    \item Developing time-resolved capabilities (laser-ARPES, pump-probe)—perfect timing to contribute to frontier technique development
    \item Access to synchrotron facilities through institutional partnerships
    \item Established cryogenic, sample preparation, and characterization infrastructure
\end{itemize}

\textbf{Expertise Alignment:}
\begin{itemize}
    \item Cuprate superconductors (5/5 expertise)—enables Year 4 strategic expansion
    \item Topological materials (emerging strength)—aligns with kagome focus
    \item Strong theory collaborations—essential for interpreting complex many-body dynamics
    \item Materials synthesis networks—ensures sample access
\end{itemize}

\textbf{Mentorship and Career Development:}
\begin{itemize}
    \item Prof.~Kondo's track record in high-impact publications (\textit{Nature}, \textit{Science})
    \item Lab culture emphasizing both technical excellence and physics depth
    \item International collaboration network for postdoc opportunities
    \item Located at University of Tokyo ISSP—hub for condensed matter physics in Asia
\end{itemize}

\textbf{Five-Year Commitment Rationale:}

A five-year doctoral program is essential to achieve the research depth and breadth outlined in this proposal:
\begin{itemize}
    \item Years 1--2: Master foundational techniques (cannot be rushed)
    \item Years 3--4: Expand to multiple platforms and high-impact publications (requires maturity)
    \item Year 5: Explore frontier directions and complete comprehensive dissertation (distinguishes PhD from extended Master's)
\end{itemize}

This timeline allows transitioning from student to research leader, developing mentorship skills, and establishing myself as expert in band-selective ultrafast spectroscopy—positioning me for tenure-track or national lab PI career.

\subsection{Expected Outcomes for Doctoral Degree}

\textbf{Concrete Deliverables:}
\begin{itemize}
    \item \textbf{Publications:} 5--6 first-author papers including $\geq$1 PRL/Nature family high-impact publication
    \item \textbf{Doctoral dissertation:} Comprehensive 200--300 page thesis establishing band-selective ultrafast spectroscopy as quantitative tool for many-body physics
    \item \textbf{Conference presentations:} 10+ talks/posters at major international conferences (APS, MRS, ICSCE, JPS)
    \item \textbf{Dataset legacy:} Well-documented experimental databases for future lab members and collaborators
    \item \textbf{Methodological innovations:} Dual-modality protocols, analysis pipelines, potentially patentable techniques
\end{itemize}

\textbf{Skills Acquired:}
\begin{itemize}
    \item \textbf{Experimental mastery:} Femtosecond laser systems, OPA, trARPES, cryogenics, sample preparation
    \item \textbf{Analytical expertise:} Advanced data analysis, statistical modeling, machine learning for spectroscopy
    \item \textbf{Theoretical understanding:} Many-body physics, non-equilibrium dynamics, correlation effects
    \item \textbf{Professional skills:} Scientific writing, presentation, grant writing, mentorship, collaboration
    \item \textbf{Language proficiency:} Japanese (target JLPT N1), enabling full integration into academic culture
\end{itemize}

\textbf{Career Positioning:}

This doctoral program will position me as an internationally recognized expert in ultrafast quantum materials spectroscopy, qualified for:
\begin{itemize}
    \item \textbf{Postdoctoral research:} Top-tier institutions (Stanford, MIT, Max Planck, Berkeley, etc.)
    \item \textbf{Tenure-track faculty:} Emphasis on experimental condensed matter physics programs
    \item \textbf{National laboratory PI:} Staff scientist positions at facilities (Brookhaven, SLAC, NIST, etc.)
    \item \textbf{Industry R\&D:} Quantum technology companies, advanced materials development
\end{itemize}

The five-year investment in this doctoral program is not merely degree completion—it is essential training to become a creative, rigorous, and collaborative research leader capable of addressing frontier questions in quantum materials physics.

\newpage
\printbibliography

\end{document}
