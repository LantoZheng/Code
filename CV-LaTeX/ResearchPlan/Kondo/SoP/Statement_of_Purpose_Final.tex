\documentclass[11pt,a4paper]{article}

% --- 宏包设置 ---
\usepackage{fontspec}
\setmainfont{Times New Roman}
\usepackage{geometry}
\geometry{margin=0.7in}
\usepackage{setspace}
\setstretch{0.95}
\usepackage{amsmath}
\usepackage{microtype}
\usepackage{hyperref}
\hypersetup{
    colorlinks=true,
    linkcolor=blue,
    urlcolor=blue,
    citecolor=blue
}
\usepackage{enumitem}
\setlist{nosep, leftmargin=*, itemsep=1pt, topsep=2pt, parsep=0pt}
\usepackage{titlesec}

% 极度紧凑的章节标题格式
\titleformat{\section}
  {\large\bfseries}
  {}
  {0em}
  {}
\titlespacing*{\section}{0pt}{10pt}{5pt}

\titleformat{\subsection}
  {\normalsize\bfseries}
  {}
  {0em}
  {}
\titlespacing*{\subsection}{0pt}{6pt}{4pt}


% --- 文档开始 ---
\begin{document}

% --- 标题 ---
\begin{center}
    {\large \textbf{Statement of Purpose}}\\[0.2em]
    {\normalsize Doctoral Program, Graduate School of Science, University of Tokyo}\\[0.5em]
    \textbf{Xiaoyang Zheng} $\cdot$ Beijing Normal University $\cdot$ \href{mailto:xiaoyangzheng@mail.bnu.edu.cn}{xiaoyangzheng@mail.bnu.edu.cn}
\end{center}

\vspace{-0.4em}

% --- 正文 ---
\section{Academic Background and Motivation}

I am a senior undergraduate majored in Physics at Beijing Normal University (ranked 2/23, GPA 3.7/4.0) with research in optical physics (diffractive neural networks, SLM beam shaping) and nanophotonics. Currently at UC Berkeley (Aug--Dec 2025), I have been exposed to cutting-edge condensed matter physics and ultrafast spectroscopy. This solidified my scientific passion: \textbf{How can we disentangle electron-phonon coupling, electron-electron correlations, and topological band structure that govern exotic quantum phases?} This fundamental question motivates my commitment to a five-year doctoral program in experimental quantum materials physics.

My technical skills—precision optics, signal optimization, computational methods (Python/PyTorch, MATLAB)—are tools to solve deep physics questions through advanced ultrafast spectroscopy, where experiment meets theory to probe non-equilibrium many-body phenomena.

\section{Why School of Science, University of Tokyo, and Tentative Research Plan}

\subsection{Research Alignment with Professor Takeshi Kondo's Laboratory (ISSP)}

I am strongly motivated to join Prof.~Kondo's laboratory due to exceptional research synergy. The lab's world-leading ARPES work on topological materials and strongly correlated systems—\textbf{combined with strategic development of time-resolved and spin-resolved ARPES} —represents the intersection of ultrafast optics and quantum materials physics where I aspire to build my career.

My D2NN optical systems experience, SLM feedback control algorithms, and computational physics background directly apply to laser-ARPES, OPA tuning, and high-dimensional data analysis. The lab's discoveries (\textit{Nature} 2019, 2021 topological states; \textit{Science} 2020 cuprate mysteries) and ISSP resources (ultrahigh-resolution laser-ARPES, global synchrotron access, theory/synthesis networks) provide ideal environment.

\textbf{Alignment with the laboratory's trARPES development:} The lab's time-resolved ARPES development complements my dual-modality approach. My pump-probe expertise enables rapid parameter screening (\textit{when/what conditions}), while trARPES collaboration (Years 3--5) reveals momentum-space dynamics (\textit{how}). This synergy contributes to technical development while advancing ultrafast many-body physics.

\subsection{Tentative Research Plan: Band-Selective Ultrafast Spectroscopy for Many-Body Physics}

\textbf{Core Physics Questions:}
\begin{enumerate}
    \item \textbf{Can we disentangle electron-phonon versus electron-electron interactions?} Band-selective spectroscopy in kagome metals (CsV$_3$Sb$_5$) and cuprates extracts quantitative timescales ($\tau_{e-e} \sim 50$--100 fs vs.~$\tau_{e-ph} \sim 0.5$--2 ps) and coupling constants ($\lambda$, $Z$).
    \item \textbf{How does correlation-topology interplay manifest in non-equilibrium?} Dual-modality measurements (pump-probe + trARPES) map Van Hove singularities and Dirac cone dynamics.
    \item \textbf{Can we access photo-induced metastable phases?} Light-driven state engineering (Floquet, hidden order) via systematic fluence/wavelength studies.
\end{enumerate}

\textbf{Technical Approach and Feasibility:}
My methodology builds directly on Kondo Lab's established platforms:
\begin{itemize}
    \item \textbf{Band-selective pump-probe (Y1--2):} Leverage lab's ultrahigh-resolution laser-ARPES ($\Delta E < 1$ meV) for momentum-resolved equilibrium mapping. Add tunable pump (OPA 1.2--6 eV, $<$100 fs) + broadband probe (white-light generation 1.5--3.5 eV) to track $\Delta R/R(k, \omega, t)$. Initial work on CsV$_3$Sb$_5$ exploits lab's topological material expertise.
    \item \textbf{Dual-modality synergy (Y3--5):} Pump-probe identifies \textit{when/where} dynamics occur (fluence/wavelength parameter space, $\sim$100 conditions/week); collaborate with the lab's trARPES development to resolve \textit{how} ($k$-space evolution, quasiparticle lifetime). Cuprate focus (BSCCO, LSCO) leverages Kondo Lab's world-leading sample growth (\textit{Science} 2020) and antinodal/nodal ARPES mastery.
    \item \textbf{Risk mitigation:} Strong electron-phonon screening (kagome)? $\to$ Switch to Bi$_2$Se$_3$/topological insulators (lab expertise). Pump-induced damage? $\to$ Cryogenic environments ($<$20 K), fluence optimization ($<$1 mJ/cm$^2$). Data complexity? $\to$ ML-based analysis (my PyTorch/computational background).
\end{itemize}

\textbf{Five-year trajectory with milestones:}
\begin{itemize}
    \item \textbf{Years 1--2 (Foundation):} CsV$_3$Sb$_5$ band-selective methodology development. Extract $\Delta E(t)$ (meV precision), $\tau_{e-e/e-ph}$ (fs/ps regime), $\lambda$ (0.1--0.5 range), $Z(k)$ at Van Hove singularities. \textit{Milestones:} First pump-probe setup (Month 6), initial kagome data (Month 12), 1--2 \textit{PRB} submissions (Month 18--24).
    \item \textbf{Years 3--4 (Cuprate Focus):} Probe pseudogap dynamics in BSCCO/LSCO (antinodal vs.~nodal regions) using lab's cuprate samples (\textit{Science} 2020; 5/5 mastery rating). Map $\Delta_\text{PG}(k, t)$ collapse/recovery timescales, test competing order scenarios. \textit{Milestones:} Cuprate measurement campaign (Month 30--36), high-impact paper submission \textit{PRL}/\textit{Nature Communications} (Month 42--48).
    \item \textbf{Year 5 (Frontier \& Dissertation):} Explore photo-induced phases (Floquet engineering, hidden order), implement ML-accelerated analysis pipelines, or extend to new platforms (nickelates, twisted bilayers). 200--300 page dissertation integrating all findings. \textit{Milestone:} Defense + postdoc applications (Month 60).
\end{itemize}

\textbf{Expected outcomes:} 5--6 first-author papers ($\geq$1 high-impact \textit{PRL}/Nature-family), dual-modality technical innovations (potential patent/technique papers), mentorship of 2--3 junior students (Years 3--5), international collaborations (synchrotron beamtimes, theory groups), strong positioning for postdoc at world-leading institutions.

\subsection{Career Goals}

The GSGC program's interdisciplinary training and international collaboration align with my goal to become an experimental physicist at the physics-materials-instrumentation interface. Studying in Japan offers immersion in leading condensed matter physics culture, Japanese language proficiency (target JLPT N1 by Year 4), and international research connections.

\section{Career Aspirations and Commitment}

\textbf{Doctoral goals:} Become expert in band-selective ultrafast spectroscopy; publish 5--6 papers ($\geq$1 \textit{PRL}/Nature-family); develop novel dual-modality techniques; build international collaborations; mentor 2--3 students (Years 3--5); achieve JLPT N1 for full integration.

\textbf{Long-term vision:} Postdoc at world-leading institution (Stanford, MIT, Max Planck, Berkeley) developing advanced spectroscopies, then establish research group as tenure-track faculty or national lab PI, bridging experiment-theory through data-driven modeling and ML, mentoring next generation, addressing quantum many-body far-from-equilibrium physics.

\textbf{My commitment:}
\begin{itemize}
    \item \textbf{Full dedication:} Prepared for intensive research, thriving in challenges (double major Physics+Economics, rank 2/23), viewing obstacles as growth opportunities.
    \item \textbf{Cultural integration:} Intensive Japanese study (JLPT N1 by Year 4), active university engagement beyond research.
    \item \textbf{Leadership transition:} Student (Y1--2) $\to$ junior researcher (Y3--4) $\to$ research leader (Y5), taking ownership, mentoring, initiating collaborations.
    \item \textbf{Scientific excellence:} Top-tier publications; annual international conferences (APS, MRS); competitive fellowships (JSPS DC2, GSGC); teaching/grant experience.
\end{itemize}

\section{Conclusion}

The University of Tokyo's School of Science, particularly Prof.~Kondo's ISSP laboratory, represents my ideal environment for a five-year doctoral program addressing fundamental quantum materials physics through ultrafast spectroscopy. The lab's ARPES expertise (\textit{Nature} 2019, 2021; \textit{Science} 2020), time-resolved technique development, and focus on correlated/topological materials align perfectly with my vision of disentangling competing interactions via band-selective methods.

My optical systems experience (D2NN, SLM), computational skills (ML, optimization), and hands-on experimentation provide technical foundation. More importantly, my intellectual passion for understanding how electron-phonon coupling, electron-electron correlations, and topological structure produce exotic quantum phases drives my five-year commitment. I am prepared to transition from student to research leader, contributing transformative advances to ultrafast quantum materials physics.

Thank you for considering my application. I am excited to join the University of Tokyo and collaborate with Professor Kondo's group to advance quantum many-body phenomena understanding.

\vspace{5em}

\noindent
\textbf{Xiaoyang Zheng} $\cdot$ Beijing Normal University $\cdot$ UC Berkeley Exchange (Aug--Dec 2025)\\
Email: \href{mailto:xiaoyangzheng@mail.bnu.edu.cn}{xiaoyangzheng@mail.bnu.edu.cn} $\cdot$ Phone: +86-13955190184 $\cdot$ Language: IELTS 7.5, TOEFL 101\\
\textbf{Intended Supervisor:} Professor Takeshi Kondo, Institute for Solid State Physics (ISSP)

\end{document}
