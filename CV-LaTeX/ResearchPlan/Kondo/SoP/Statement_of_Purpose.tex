\documentclass[11pt,a4paper]{article}

% --- 宏包设置 ---
\usepackage{fontspec}
\setmainfont{Times New Roman}
\usepackage{geometry}
\geometry{margin=0.8in}  % 略微缩小边距以适应2页限制
\usepackage{setspace}
\setstretch{1.15}  % 减小行距以压缩长度
\usepackage{microtype}
\usepackage{hyperref}
\hypersetup{
    colorlinks=true,
    linkcolor=blue,
    urlcolor=blue,
    citecolor=blue
}
\usepackage{enumitem}
\setlist{nosep, leftmargin=*}  % 减小列表间距
\usepackage{titlesec}

% 章节标题格式 - 更紧凑
\titleformat{\section}
  {\large\bfseries}
  {}
  {0em}
  {}
\titlespacing*{\section}{0pt}{0.8\baselineskip}{0.3\baselineskip}

\titleformat{\subsection}
  {\normalsize\bfseries}
  {}
  {0em}
  {}
\titlespacing*{\subsection}{0pt}{0.5\baselineskip}{0.2\baselineskip}

% 移除页码
\pagenumbering{gobble}

% --- 文档开始 ---
\begin{document}

% --- 标题 ---
\begin{center}
    {\LARGE \textbf{Statement of Purpose}}\\[0.5em]
    {\large Master's Program Application}\\[0.3em]
    {\large Graduate School of Science, University of Tokyo}\\[1em]
    \textbf{Xiaoyang Zheng}\\[0.3em]
    Beijing Normal University\\
    \href{mailto:xiaoyangzheng@mail.bnu.edu.cn}{xiaoyangzheng@mail.bnu.edu.cn} $\cdot$ 
    \href{mailto:Xiaoyang_zheng@berkeley.edu}{Xiaoyang\_zheng@berkeley.edu}
\end{center}

\vspace{0.5em}

% --- 正文 ---
\section{Academic Background and Motivation}

I am a final-year undergraduate student majoring in Physics at Beijing Normal University, currently ranked 2nd out of 23 students in my cohort with a GPA of 3.7/4.0. My undergraduate research has focused on optical physics and nanophotonics, where I have developed strong expertise in advanced optical systems through projects on diffractive neural networks (D2NN) and spatial light modulator (SLM)-based beam shaping. Through these experiences, I have cultivated not only technical proficiency in precision optical design and control, but also a deep fascination with how light can be used as a probe to unveil fundamental physics in quantum materials.

During my exchange semester at UC Berkeley through the Berkeley Physics International Education Program (2025.8--2025.12), I have been exposed to cutting-edge research in condensed matter physics and ultrafast spectroscopy. This experience has crystallized my desire to transition from applied optics to fundamental physics research, specifically to investigate the microscopic electronic structures and dynamics that govern exotic quantum phenomena. The intersection of ultrafast laser technology and quantum materials represents a frontier where my optical engineering skills can directly contribute to addressing profound questions in modern physics.

\section{Why the Kondo Laboratory and University of Tokyo}

My decision to apply to Professor Kondo's laboratory at the Institute for Solid State Physics (ISSP), University of Tokyo, is driven by three key factors that make this an ideal match for my academic goals:

\subsection{Alignment with Research Philosophy and Techniques}

The Kondo Laboratory's pioneering work in angle-resolved photoemission spectroscopy (ARPES) on strongly correlated systems and topological materials represents exactly the type of rigorous, technique-driven approach to fundamental physics that I aspire to master. What particularly excites me is the laboratory's recent development of \textbf{time-, spin-, and angle-resolved photoemission spectroscopy}, where pump-probe techniques are being integrated with spin-resolved ARPES using femtosecond lasers. This represents a natural synergy with my background:

\begin{itemize}[leftmargin=1.5em, itemsep=0.3em]
    \item My experience with \textbf{D2NN in 4f optical systems} has trained me in Fourier-plane optics, phase modulation, and precision optical alignment---skills directly transferable to laser-based ARPES systems and the OPA tuning required for pump-probe experiments.
    
    \item My work on \textbf{SLM-based beam shaping with feedback control} involved real-time optimization algorithms (SGD, simulated annealing) and wavefront sensing, which parallels the experimental control and signal optimization challenges in time-resolved spectroscopy.
    
    \item My computational physics background (Python, PyTorch, MATLAB; 95/100 in coursework) positions me to contribute to the data analysis pipelines for high-dimensional ARPES data and to implement machine learning approaches for extracting physical insights from complex spectroscopic datasets.
\end{itemize}

The laboratory's emphasis on developing state-of-the-art experimental techniques---from ultrahigh-resolution laser-ARPES to cryogenic $^3$He systems---aligns perfectly with my hands-on experimental aptitude and my desire to become an expert in advanced spectroscopic methods.

\subsection{Scientific Interest in Kagome Metals and Topological Materials}

While my initial research interests centered on kagome metals (specifically CsV$_3$Sb$_5$ and its charge density wave dynamics), exploring the Kondo Laboratory's research has broadened my perspective to encompass the wider landscape of topological quantum materials and strongly correlated electron systems. The laboratory's groundbreaking work on:

\begin{itemize}[leftmargin=1.5em, itemsep=0.3em]
    \item \textbf{Topological insulators and Weyl semimetals} (\textit{Nature} 2019, 2021; \textit{Nature Materials} 2017, 2021): The direct observation of weak topological insulator surface states and higher-order topological edge states demonstrates the power of ARPES in revealing exotic topological phenomena.
    
    \item \textbf{Cuprate high-temperature superconductors} (\textit{Science} 2020): The resolution of the ``small vs.\ large Fermi surface'' problem through combined laser-ARPES and quantum oscillation measurements showcases how advanced spectroscopy can address decades-old mysteries.
    
    \item \textbf{Strongly correlated systems like CeSb} (\textit{Nature Communications} 2020; \textit{Nature Materials} 2021): The microscopic observation of the ``devil's staircase'' magnetic phase transitions illustrates how ARPES can unravel complex many-body physics.
\end{itemize}

These research directions resonate deeply with my interest in understanding how electronic structure---momentum-resolved and time-resolved---governs the macroscopic quantum phases in materials. Kagome metals like CsV$_3$Sb$_5$, which exhibit the interplay of topology, charge density waves, and superconductivity, would be a natural extension of the laboratory's expertise. However, I am equally excited about the possibility of working on other quantum materials within the laboratory's portfolio, including topological superconductors or magnetic Weyl semimetals, where time-resolved and spin-resolved ARPES could reveal non-equilibrium dynamics.

\subsection{World-Class Research Environment and Career Development}

The University of Tokyo's ISSP is globally recognized as a premier institution for solid-state physics research. The Kondo Laboratory's access to:

\begin{itemize}[leftmargin=1.5em, itemsep=0.3em]
    \item Multiple in-house laser-ARPES systems with ultrahigh resolution, spin resolution, and sub-Kelvin temperature capabilities
    \item Synchrotron facilities worldwide (Diamond, BESSY, Photon Factory, UVSOR)
    \item Collaborative networks with theory groups and materials synthesis teams
    \item The 10.7~eV short-pulse laser developed by Prof.\ Kobayashi's group for pump-probe ARPES
\end{itemize}

\noindent provides an unparalleled environment for me to develop as an experimental physicist. The GSGC program's emphasis on interdisciplinary training and international collaboration will further enhance my ability to work at the intersection of physics, materials science, and advanced instrumentation.

Moreover, studying in Japan offers a unique opportunity to immerse myself in a different academic culture, learn Japanese (I have already planned intensive language study: JLPT N4 by Month 6, N3 by Month 12), and establish international research connections that will be invaluable throughout my career---whether I pursue academia, national laboratories, or industrial R\&D in quantum technologies.

\section{Research Plan for Master's Program}

For my Master's research, I propose to contribute to the laboratory's ongoing development of \textbf{time-resolved ARPES} by combining pump-probe spectroscopy with momentum-resolved electronic structure measurements on quantum materials. Specifically, I am interested in:

\subsection{Primary Research Direction: Ultrafast Dynamics in Kagome Metals or Topological Materials}

Building on the laboratory's expertise in equilibrium ARPES and leveraging the femtosecond laser capabilities being developed for pump-probe measurements, I aim to investigate:

\begin{enumerate}[leftmargin=1.5em, itemsep=0.3em]
    \item \textbf{Band-selective photoexcitation dynamics}: Using tunable pump wavelengths (via OPA or the high-order harmonic laser), selectively excite electrons at specific k-points (e.g., van Hove singularities vs.\ Dirac points in kagome metals, or Weyl points in magnetic Weyl semimetals) and probe the subsequent relaxation dynamics and transient band structure changes.
    
    \item \textbf{Non-equilibrium electronic structure}: Observe how the Fermi surface and band dispersion evolve on femtosecond to picosecond timescales following optical excitation, revealing the timescales of electron-electron, electron-phonon, and electron-spin interactions.
    
    \item \textbf{Photo-induced phase transitions}: If time and sample availability permit, explore light-induced metastable phases or hidden states that are inaccessible in equilibrium conditions---a frontier area in ultrafast quantum materials science.
\end{enumerate}

\subsection{Alternative Research Directions (Flexible)}

I am fully prepared to adapt my research focus based on the laboratory's priorities and available samples. Potential alternatives include:

\begin{itemize}[leftmargin=1.5em, itemsep=0.3em]
    \item \textbf{Time-resolved studies of cuprate superconductors}: Investigating the ultrafast dynamics of pseudogap formation or Cooper pair breaking/reformation.
    
    \item \textbf{Spin dynamics in topological materials}: Using the pump-probe spin-resolved ARPES under development to probe ultrafast spin polarization dynamics in topological surface states or Weyl semimetals.
    
    \item \textbf{Thin film quantum materials}: Applying time-resolved techniques to two-dimensional electronic states in engineered thin films or heterostructures.
\end{itemize}

The two-year Master's timeline I have outlined in my research plan is structured to ensure both technical skill development and scientific output:

\begin{itemize}[leftmargin=1.5em, itemsep=0.3em]
    \item \textbf{Year 1 (Months 1--12)}: Laboratory integration, coursework, Japanese language study, mastery of laser-ARPES systems, and preliminary time-resolved measurements on reference materials.
    
    \item \textbf{Year 2 (Months 13--24)}: Deep characterization of selected quantum materials, data analysis in collaboration with theorists, manuscript preparation (targeting \textit{Physical Review B} or \textit{Journal of the Physical Society of Japan}), and thesis writing.
\end{itemize}

I am committed to producing high-quality research output---at least one first-author publication and a comprehensive Master's thesis---while acquiring the experimental and analytical skills necessary for a successful career in quantum materials physics.

\section{Connection Between My Background and Your Research}

At first glance, my undergraduate work on diffractive neural networks and beam shaping may seem disconnected from ARPES and condensed matter physics. However, I view these projects as providing a strong foundation for the transition I wish to make:

\textbf{Optical Systems Expertise}:
\begin{itemize}[leftmargin=1.5em, itemsep=0.2em]
    \item My D2NN project required precise control of optical phase and amplitude in the Fourier plane using SLMs---conceptually similar to manipulating laser parameters (wavelength, polarization, pulse duration) in pump-probe ARPES.
    \item The self-calibrating beam shaping system I developed involved feedback loops and optimization algorithms to minimize wavefront errors, directly applicable to optimizing signal-to-noise ratios in time-resolved spectroscopy.
\end{itemize}

\textbf{Computational Skills}:
\begin{itemize}[leftmargin=1.5em, itemsep=0.2em]
    \item Training neural networks on MNIST using PyTorch taught me to work with high-dimensional datasets, extract meaningful patterns, and implement machine learning algorithms---skills increasingly relevant for analyzing large ARPES datasets and identifying subtle spectroscopic signatures.
    \item My computational physics coursework included numerical methods, Fourier analysis, and signal processing, all of which are essential for time-resolved spectroscopy data analysis.
\end{itemize}

\textbf{Hands-On Experimental Ability}:
\begin{itemize}[leftmargin=1.5em, itemsep=0.2em]
    \item My experience with SEM, TEM, and optical spectroscopy in nanophotonics research has taught me the patience, attention to detail, and troubleshooting mindset required for complex experimental physics.
    \item Constructing optical systems from scratch and integrating hardware with software control has prepared me for the interdisciplinary nature of modern experimental research.
\end{itemize}

\textbf{Rapid Learning Ability}:
\begin{itemize}[leftmargin=1.5em, itemsep=0.2em]
    \item Achieving a class rank of 2/23 while simultaneously pursuing a double major in Physics and Economics demonstrates my capacity to master new material quickly and work efficiently.
    \item My successful adaptation to the English-language Berkeley program (IELTS 7.5, TOEFL 101) shows my ability to thrive in unfamiliar academic environments---a skill I will apply to learning Japanese and integrating into the Kondo Laboratory.
\end{itemize}

What I bring to the Kondo Laboratory is not just technical skills, but also a mindset oriented toward continuous learning, interdisciplinary thinking, and collaborative problem-solving. I am eager to transition from applied optics to fundamental physics, and I believe the rigorous training in ARPES and ultrafast spectroscopy at the Kondo Laboratory will provide the ideal pathway.

\section{Career Aspirations and Long-Term Vision}

My immediate goal is to excel in the Master's program, master advanced spectroscopic techniques, and contribute high-quality research to the Kondo Laboratory. However, I am strongly considering continuing to a doctoral program if my Master's work proves successful. The frontier questions in quantum materials---how topology, correlation, and symmetry breaking conspire to produce exotic phases---represent problems that will occupy physicists for decades to come, and I want to be part of this endeavor.

In the long term, I aspire to a research career where I can:

\begin{itemize}[leftmargin=1.5em, itemsep=0.3em]
    \item \textbf{Push the boundaries of experimental techniques}: Developing next-generation spectroscopic methods that can probe quantum materials with unprecedented resolution in energy, momentum, time, and spin.
    
    \item \textbf{Bridge experiment and theory}: Working collaboratively with theorists to design experiments that test fundamental predictions and using experimental data to guide new theoretical frameworks.
    
    \item \textbf{Mentor the next generation}: Whether in academia or a national laboratory, I want to train students in advanced experimental techniques and foster a collaborative, intellectually rigorous research environment.
\end{itemize}

The skills I will acquire at the Kondo Laboratory---precision laser spectroscopy, cryogenic techniques, data analysis, collaboration with international teams---will be essential for any of these paths. Moreover, the experience of working in Japan's unique academic culture and developing fluency in Japanese will enable me to maintain research connections in one of the world's leading regions for condensed matter physics.

\section{Commitment to Success}

I understand that Master's study at the University of Tokyo, particularly in an advanced experimental laboratory, will be demanding. I am prepared to:

\begin{itemize}[leftmargin=1.5em, itemsep=0.3em]
    \item \textbf{Dedicate myself fully to research}: I will treat this as a full-time commitment, spending long hours in the laboratory, attending seminars, and engaging deeply with the scientific literature.
    
    \item \textbf{Learn Japanese diligently}: I have planned intensive Japanese language study (targeting JLPT N3 by the end of Year 1) to facilitate daily communication, attend lab meetings, and integrate into Japanese academic culture.
    
    \item \textbf{Collaborate effectively}: I will actively participate in group meetings, learn from senior students and postdocs, and contribute to the laboratory's collaborative atmosphere.
    
    \item \textbf{Adapt to feedback}: I will be receptive to Professor Kondo's guidance, adjust my research approach based on experimental results and theoretical insights, and maintain high standards for data quality and scientific rigor.
    
    \item \textbf{Publish and present}: I will work diligently to produce publication-quality research and present my work at domestic and international conferences to develop as a scientific communicator.
\end{itemize}

I am not seeking merely a degree, but a transformative educational experience that will shape me into a rigorous, creative, and collaborative experimental physicist.

\section{Conclusion}

The Kondo Laboratory at the University of Tokyo's ISSP represents the ideal environment for me to pursue my passion for understanding quantum materials through advanced optical spectroscopy. The laboratory's world-leading expertise in ARPES, its development of time-resolved and spin-resolved techniques, and its focus on topological materials and strongly correlated systems align perfectly with my interests and background.

My experience in optical system design, computational analysis, and hands-on experimentation has prepared me to quickly contribute to the laboratory's research program, while the rigorous training I will receive will equip me with the skills to pursue a long-term career in experimental quantum materials physics.

I am excited about the prospect of joining the Kondo Laboratory, immersing myself in Japanese academic culture, and contributing to the frontier of ultrafast spectroscopy of quantum materials. I am confident that I have both the technical preparation and the intellectual passion to succeed in this program and to make meaningful contributions to the laboratory's research mission.

Thank you for considering my application. I look forward to the opportunity to discuss my research interests and how I can contribute to the Kondo Laboratory.

\vspace{1em}

\noindent
\textbf{Xiaoyang Zheng}\\
Beijing Normal University\\
Email: \href{mailto:xiaoyangzheng@mail.bnu.edu.cn}{xiaoyangzheng@mail.bnu.edu.cn}\\
Phone: +86 13955190184

\vspace{0.5em}

\noindent
\textit{Application for: Master's Program, Graduate School of Science}\\
\textit{University of Tokyo, Graduate School for Global and Comprehensive Studies (GSGC)}\\
\textit{Potential Supervisor: Professor Takao Kondo, Institute for Solid State Physics}

\end{document}
