\documentclass[11pt,a4paper]{article}

% --- 宏包设置 ---
\usepackage{fontspec}
\setmainfont{Times New Roman}
\usepackage{geometry}
\geometry{margin=0.75in}
\usepackage{setspace}
\setstretch{1.0}
\usepackage{microtype}
\usepackage{hyperref}
\hypersetup{
    colorlinks=true,
    linkcolor=blue,
    urlcolor=blue,
    citecolor=blue
}
\usepackage{enumitem}
\setlist{nosep, leftmargin=*, itemsep=2pt, topsep=3pt}
\usepackage{titlesec}

% 紧凑的章节标题格式
\titleformat{\section}
  {\large\bfseries}
  {}
  {0em}
  {}
\titlespacing*{\section}{0pt}{8pt}{4pt}

\titleformat{\subsection}
  {\normalsize\bfseries}
  {}
  {0em}
  {}
\titlespacing*{\subsection}{0pt}{6pt}{3pt}

% 移除页码
\pagenumbering{gobble}

% --- 文档开始 ---
\begin{document}

% --- 标题 ---
\begin{center}
    {\Large \textbf{Statement of Purpose}}\\[0.4em]
    {\large Master's Program, Graduate School of Science, University of Tokyo}\\[0.8em]
    \textbf{Xiaoyang Zheng} $\cdot$ Beijing Normal University\\
    \href{mailto:xiaoyangzheng@mail.bnu.edu.cn}{xiaoyangzheng@mail.bnu.edu.cn}
\end{center}

\vspace{-0.3em}

% --- 正文 ---
\section{Academic Background and Motivation}

I am a final-year undergraduate Physics major at Beijing Normal University, ranked 2/23 with GPA 3.7/4.0. My research experience spans optical physics and nanophotonics, where I developed expertise in diffractive neural networks (D2NN), spatial light modulator (SLM)-based beam shaping, and computational optimization. Currently at UC Berkeley through the Berkeley Physics International Education Program (Aug--Dec 2025), I have been exposed to cutting-edge condensed matter physics and ultrafast spectroscopy, crystallizing my goal to transition from applied optics to fundamental quantum materials research.

My academic journey has equipped me with strong foundations in both experimental techniques (precision optical alignment, wavefront control, signal optimization) and computational methods (Python, PyTorch, MATLAB; 95/100 in Computational Physics). These skills, combined with hands-on experience in SEM/TEM characterization and FDTD simulations in nanophotonics, position me to contribute meaningfully to advanced spectroscopic research on quantum materials.

\section{Why School of Science, University of Tokyo}

\subsection{Research Alignment with Professor Takao Kondo's Laboratory (ISSP)}

I am strongly motivated to join Prof.~Kondo's laboratory due to the exceptional match between my background and the laboratory's research frontier. The lab's world-leading work in angle-resolved photoemission spectroscopy (ARPES) on topological materials and strongly correlated systems—combined with recent development of \textbf{time-, spin-, and angle-resolved ARPES using pump-probe techniques}—represents precisely the intersection of ultrafast optics and quantum materials physics where I aspire to build my career.

\textbf{Key research synergies:}
\begin{itemize}
    \item My D2NN project in 4f optical systems trained me in Fourier-plane optics and phase control, directly applicable to laser-ARPES systems and OPA tuning for pump-probe experiments.
    \item My SLM feedback control work involved real-time optimization algorithms (SGD, simulated annealing), paralleling signal optimization challenges in time-resolved spectroscopy.
    \item My computational physics background enables me to contribute to high-dimensional ARPES data analysis and implement machine learning approaches for complex spectroscopic datasets.
\end{itemize}

The laboratory's groundbreaking discoveries—including direct observation of topological surface states (\textit{Nature} 2019, 2021), resolution of cuprate superconductor Fermi surface mysteries (\textit{Science} 2020), and microscopic observation of complex magnetic phases in CeSb (\textit{Nature Communications} 2020)—demonstrate the power of advanced ARPES techniques to address fundamental questions in condensed matter physics. I am excited to learn these techniques and contribute to the laboratory's ongoing development of ultrafast spectroscopic methods.

\subsection{World-Class Research Environment}

The University of Tokyo's Institute for Solid State Physics (ISSP) offers unparalleled resources: multiple ultrahigh-resolution laser-ARPES systems, global synchrotron facility access, and collaborative networks with theory groups and materials synthesis teams. The GSGC program's interdisciplinary training framework and international collaboration emphasis align perfectly with my goal to become an experimental physicist capable of addressing frontier problems at the interface of physics, materials science, and advanced instrumentation.

Additionally, studying in Japan provides a unique opportunity to immerse myself in a leading academic culture for condensed matter physics, develop Japanese language proficiency (planned: JLPT N4 by Month 6, N3 by Month 12), and establish international research connections valuable for my long-term career.

\section{Tentative Research Plan}

\subsection{Primary Research Direction: Time-Resolved ARPES on Quantum Materials}

I propose to contribute to the laboratory's development of time-resolved ARPES by combining pump-probe spectroscopy with momentum-resolved electronic structure measurements. Specifically:

\textbf{Scientific objectives:}
\begin{enumerate}
    \item \textbf{Band-selective photoexcitation dynamics in kagome metals or topological materials}: Using tunable pump wavelengths to selectively excite electrons at specific k-points (e.g., van Hove singularities, Dirac/Weyl points) and probe transient band structure evolution, revealing electron-electron, electron-phonon, and electron-spin interaction timescales.
    
    \item \textbf{Non-equilibrium electronic structure}: Observe Fermi surface and band dispersion evolution on femtosecond-to-picosecond timescales following optical excitation.
    
    \item \textbf{Photo-induced phenomena}: If time permits, explore light-induced metastable phases or hidden quantum states inaccessible in equilibrium.
\end{enumerate}

Materials of interest include CsV$_3$Sb$_5$ (kagome metal exhibiting topology, charge density waves, and superconductivity), magnetic Weyl semimetals, or other quantum materials within the laboratory's research portfolio. I am fully flexible to adapt based on laboratory priorities, sample availability, and equipment development progress.

\textbf{Alternative research directions} include: time-resolved studies of cuprate superconductors (pseudogap dynamics, Cooper pair dynamics), spin dynamics in topological materials using pump-probe spin-resolved ARPES, or two-dimensional electronic states in thin film quantum materials.

\subsection{Two-Year Timeline}

\textbf{Year 1 (Months 1--12):} Laboratory integration, coursework completion, intensive Japanese language study, mastery of laser-ARPES operation and maintenance, preliminary time-resolved measurements on reference materials, and collaboration with senior students/postdocs.

\textbf{Year 2 (Months 13--24):} Deep spectroscopic characterization of selected quantum materials, data analysis in collaboration with theorists, manuscript preparation (targeting \textit{Physical Review B} or \textit{Journal of the Physical Society of Japan}), conference presentations, and Master's thesis completion.

\textbf{Expected outcomes:} At least one first-author publication, comprehensive Master's thesis, and acquisition of advanced experimental/analytical skills in ultrafast spectroscopy of quantum materials.

\section{Career Aspirations}

My immediate goal is to excel in the Master's program and contribute high-quality research to Prof.~Kondo's laboratory. I am strongly considering continuing to a doctoral program if my Master's work proves successful, as the frontier questions in quantum materials physics—how topology, correlation, and symmetry breaking conspire to produce exotic phases—represent challenges I am passionate about addressing.

Long-term, I aspire to a research career (academia or national laboratory) where I can: (1) develop next-generation spectroscopic techniques with unprecedented resolution, (2) bridge experiment and theory through collaborative research, and (3) mentor future physicists in advanced experimental methods. The training I will receive at the University of Tokyo—in precision laser spectroscopy, cryogenic techniques, data analysis, and international collaboration—will be foundational for this career path.

\section{Commitment to Success}

I am fully committed to:
\begin{itemize}
    \item \textbf{Full dedication to research}: Long laboratory hours, active seminar participation, and deep engagement with scientific literature.
    \item \textbf{Japanese language mastery}: Intensive study targeting JLPT N3 by Year 1 completion for effective communication and cultural integration.
    \item \textbf{Effective collaboration}: Learning from senior researchers, contributing to group discussions, and maintaining high standards for experimental rigor.
    \item \textbf{Scientific communication}: Publishing in peer-reviewed journals and presenting at domestic/international conferences.
\end{itemize}

I seek not merely a degree, but a transformative educational experience that will shape me into a rigorous, creative, and collaborative experimental physicist capable of addressing fundamental questions in quantum materials.

\section{Conclusion}

The School of Science at the University of Tokyo, particularly Prof.~Takao Kondo's laboratory at ISSP, represents my ideal environment for pursuing quantum materials research through advanced optical spectroscopy. The laboratory's world-leading ARPES expertise, development of time-resolved and spin-resolved techniques, and focus on topological and strongly correlated materials align perfectly with my scientific interests and technical background.

My experience in optical system design (D2NN, SLM beam shaping), computational analysis (machine learning, optimization algorithms), and hands-on experimentation (nanophotonics characterization) has prepared me to contribute quickly while learning advanced spectroscopic methods. I am confident that I possess both the technical foundation and intellectual passion to succeed in this program and make meaningful contributions to frontier research in quantum materials physics.

I am excited about the prospect of joining the University of Tokyo community, collaborating with leading researchers in condensed matter physics, and contributing to the advancement of ultrafast spectroscopy techniques. Thank you for considering my application.

\vspace{0.8em}

\noindent
\textbf{Applicant:} Xiaoyang Zheng $\cdot$ \textbf{Email:} \href{mailto:xiaoyangzheng@mail.bnu.edu.cn}{xiaoyangzheng@mail.bnu.edu.cn} $\cdot$ \textbf{Phone:} +86-13955190184\\
\textbf{Current Institution:} Beijing Normal University (Physics Major, Class Rank 2/23, GPA 3.7/4.0)\\
\textbf{Exchange Program:} UC Berkeley BPIE (Aug--Dec 2025) $\cdot$ \textbf{Language:} IELTS 7.5, TOEFL 101\\
\textbf{Intended Supervisor:} Professor Takao Kondo, Institute for Solid State Physics (ISSP)

\end{document}
