%%%%%%%%%%%%%%%%%%%%%%%%%%%%%%%%%%%%%%%%%%%%%%%%%
% FINAL PROJECT PROPOSAL - COMPRESSED VERSION
%%%%%%%%%%%%%%%%%%%%%%%%%%%%%%%%%%%%%%%%%%%%%%%%%

\documentclass[a4paper,11pt,twoside]{article}
\hyphenpenalty=8000
\textwidth=130mm
\textheight=220mm
\usepackage[top=2cm, bottom=2cm, inner=2.5cm, outer=2.5cm, includehead]{geometry}
\usepackage{fancyhdr}
\usepackage{physics}
\usepackage{amsmath,amssymb}
\pagestyle{fancy}
\fancyhf{}
\fancyhead[LE,RO]{\thepage}
\fancyhead[RE]{Final Project Proposal}
\fancyhead[LO]{Quantum Money Inflation Control}
\setlength{\headheight}{15pt}
\setlength{\parskip}{4pt}
\setlength{\parindent}{0pt}

\begin{document}

\begin{center}
{\Large \textbf{Quantum Money and Inflation Control}} \\[4pt]
\textbf{Final Project Proposal for PHYS C191A}\\[2pt]
\textit{Juncheng Ding, Tian Ariyaratrangsee, Xiaoyang Zheng}\\[2pt]
University of California, Berkeley – Fall 2025
\end{center}

\vspace{-8pt}
\section*{1. Problem Statement}
\vspace{-4pt}

The quantum no-cloning theorem prevents copying states but not unlimited generation of new valid quantum money states. As quantum power $Q(t)$ grows, generation rate $R \propto Q/D$ yields unbounded supply $M(t) \to \infty$—\textbf{quantum inflation} (Zhandry 2017). Recent theoretical work (Coladangelo \& Sattath 2020) coupled quantum states to classical systems for supply control, but left open \textit{physical implementation}.

\textbf{Our Approach:} We investigate whether \textbf{quantum circuit complexity} naturally limits state generation through: (1) inflation dynamics modeling via circuit parameters; (2) Resource Token (RT) mechanism using physical constraints (gates, coherence, ancillas); (3) NISQ implementation demonstrating bounded $M(t) \to M_{\max}$.

\vspace{-4pt}
\section*{2. Technical Approach}
\vspace{-6pt}

\subsection*{2.1 Quantum Lightning Framework \& Inflation Dynamics}
\vspace{-4pt}

Zhandry's Quantum Lightning: each currency unit $\ket{\psi_i}$ is a superposition over polynomial-degree hash pre-images. Verification: public hash $H$ with criterion $H(\ket{\psi_i}) < D$.

\textbf{Unbounded Case:} Fixed difficulty $D$, growing capability $Q(t) = Q_0 e^{\lambda t}$ (quantum Moore's law), probability $P \sim Q/2^D$ gives:
\vspace{-6pt}
\[
\frac{dM}{dt} = \frac{Q_0 e^{\lambda t}}{2^D} \quad \Rightarrow \quad M(t) \sim e^{\lambda t}.
\]
\vspace{-8pt}

\vspace{-4pt}
\subsection*{2.2 Resource Token (RT) Mechanism — Quantum Complexity Constraints}
\vspace{-4pt}

\textbf{Principle:} Couple generation to \textit{physical quantum resources}. For bolt $\ket{\psi_y}$ with circuit depth $L$, $G$ gates on $m$ qubits:
\vspace{-6pt}
\[
\text{RT}_{\text{cost}} = \alpha G + \beta L + \gamma m, \quad M_{\max} = \frac{R_{\text{total}}}{\langle \text{RT}_{\text{cost}} \rangle}.
\]
\vspace{-8pt}

\textbf{Three Physical Implementations:} (A) \textit{Gate-Count}: $\text{RT} = \alpha G + \beta L$; tracks computational complexity via Qiskit transpilation. (B) \textit{Decoherence-Limited}: $\text{RT} = \gamma \int_0^T \Gamma(t) dt$; models hardware errors from $T_1, T_2$ times. (C) \textit{Ancilla Budget}: Finite $N_{\text{ancilla}}$ pool; each generation consumes $a$ ancillas (NISQ-realistic).

\textbf{Protocol:} (1) Initialize seed $\ket{\phi_0}$, allocate RT; (2) Apply $U_{\text{mint}}$ circuit; (3) Measure serial $s$, verify $|\braket{\psi|\psi_s}|^2 > 1-\epsilon$; (4) Deduct RT based on $(G,L,m)$; (5) Adjust $D(t)$ if $R_{\text{obs}} > R_{\text{target}}$.

\textbf{Security:} Under $(2k+2)$-NAMCR for degree-2 polynomials, RT preserves quantum lightning uniqueness. Adversaries cannot: clone states (no-cloning); generate non-affine multi-collisions (NAMCR); bypass RT (circuit measurement enforced).

\vspace{-4pt}
\subsection*{2.3 NISQ Implementation Strategy}
\vspace{-4pt}

\textbf{Parameters:} Toy $(n=3, k=2, m=12)$ requires $\sim$36 qubits (IBM/IonQ accessible). Degree-2 polynomial hash over $\mathbb{F}_2$ with Zhandry verification via Hadamard transforms.

\textbf{Circuits:} Generation: seed $\to$ polynomial evaluation $\to$ Grover-like amplification. Verification: measure $y$, Hadamard for derivatives, solve linear system. RT: Qiskit transpiler tracks $(G,L)$; noise model estimates $\Gamma(t)$.

\textbf{Study:} Compare unbounded (fixed $D$) vs. RT-bounded (adaptive $R_{\text{avail}}$). Run 1000 attempts; measure supply $M(t)$, RT depletion rates, verification fidelity.

\vspace{-4pt}
\subsection*{2.4 Validation Framework}
\vspace{-4pt}

\textbf{Mathematical Model:} Derive coupled differential equations:
\vspace{-6pt}
\[
\frac{dM}{dt} = R(Q, D, R_{\text{avail}}), \quad \frac{dR_{\text{avail}}}{dt} = -\langle \text{RT}_{\text{cost}} \rangle \cdot R,
\]
\vspace{-8pt}
where $R = \min\{Q/2^D, R_{\text{avail}}/\text{RT}_{\text{cost}}\}$. Solve numerically; show equilibrium $M_{\infty} = R_{\text{total}}/\langle\text{RT}\rangle$ is stable.

\textbf{Qiskit Simulation:} Compare quantum growth scenarios $\lambda \in \{0.1, 0.5, 1.0\}$ (annual doubling to monthly) across RT budgets $R_{\text{total}} \in \{10^3, 10^5\}$. Implement all three RT variants (gate, decoherence, ancilla) with realistic noise models from IBM hardware.

\textbf{Physics Metrics:}
\vspace{-4pt}
\begin{itemize}
    \item \textbf{Inflation reduction:} $I_{\text{RT}} / I_{\text{unbounded}} < 0.01$ (target: 99\% suppression)
    \item \textbf{Circuit complexity:} Verify $O(n^3)$ gate scaling via log-log regression
    \item \textbf{Fidelity dependence:} Measure success rate vs. hardware error rates $(10^{-2}, 10^{-3}, 10^{-4})$
    \item \textbf{Resource efficiency:} Compare RT costs across three implementations; identify optimal for NISQ
\end{itemize}

\vspace{-4pt}
\subsection*{2.5 Expected Deliverables}
\vspace{-4pt}

(1) Analytical solutions for $M(t)$ in unbounded and RT-bounded regimes with stability proofs; (2) Qiskit implementation of miniaturized quantum lightning with all three RT variants; (3) Comparative plots: supply curves, RT depletion rates, fidelity sensitivity; (4) Circuit complexity analysis confirming polynomial scaling; (5) Discussion of physical feasibility: qubit requirements, coherence time constraints, error mitigation needs.

\vspace{-4pt}
\section*{3. Expected Outcomes}
\vspace{-4pt}

\begin{itemize}
    \item Demonstration of unbounded inflation $M(t) \sim e^{\lambda t}$ in baseline quantum lightning model
    \item Proof-of-concept showing RT-based quantum resource constraints achieve bounded $M(t) \to M_{\max}$
    \item NISQ-compatible Qiskit circuits implementing polynomial hash verification on $<$50 qubits
    \item Quantitative comparison: gate-count vs. decoherence vs. ancilla-based RT mechanisms
    \item Analysis of quantum hardware requirements: optimal circuit depth, error rates, qubit connectivity
    \item Connection to broader quantum complexity theory: linking no-cloning to computational resource bounds
\end{itemize}

\vspace{-4pt}
\section*{4. Timeline}
\vspace{-4pt}
\begin{center}
\begin{tabular}{|l|p{9.5cm}|}
\hline
\textbf{Date} & \textbf{Milestone} \\
\hline
Oct 30 & Finalize proposal; set up Qiskit environment \\
Nov 10 & Implement baseline model; validate exponential inflation \\
Nov 17 & Study Zhandry's verification protocol; design RT circuits \\
Nov 24 & Implement \& test three RT mechanisms (gate, decoherence, ancilla) \\
Nov 30 & Run comparative simulations; analyze circuit complexity \\
Dec 5 & Complete report with physics analysis; prepare poster \\
Dec 9 & Poster presentation \& defense \\
\hline
\end{tabular}
\end{center}
\vspace{-4pt}
\section*{5. Division of Labor}
\vspace{-4pt}

\textbf{Xiaoyang Zheng:} Theoretical development (no-cloning under RT, Lindblad dynamics, Lyapunov stability), quantum algorithm simulation in Qiskit (Grover, HHL, SWAP test, noise modeling), project integration, LaTeX report writing. \\
\textbf{Tian Ariyaratrangsee:} Poster design and presentation, quantum circuit complexity calculations (gate counts, depth analysis, scaling verification), circuit implementation (oracle construction, gate decomposition, transpilation), Fisher information analysis. \\
\textbf{Juncheng Ding:} Inflation dynamics modeling (ODE derivation, scipy numerical integration), mathematical analysis (equilibrium computation, convergence rates, Jacobian eigenvalues), comparative simulation (9-scenario parameter sweep), RT mechanism feasibility studies.

\vspace{-4pt}
\section*{6. Evaluation Criteria \& Risk Mitigation}
\vspace{-4pt}

\textbf{Success Criteria:} (1) Clear demonstration of $>$99\% inflation reduction under RT constraints; (2) Stable equilibrium $M_{\max}$ achieved; (3) Circuit complexity confirms polynomial scaling $O(n^3)$; (4) All simulations complete within Qiskit's computational limits.

\textbf{Risks:} Circuit too large for simulation (mitigation: use $n=2$ toy parameters); RT mechanism breaks quantum security (mitigation: formal no-cloning proof); Time constraints (priority: baseline + one RT variant as minimum viable project).

\vspace{-4pt}
\section*{7. References}
\vspace{-4pt}

\begin{itemize}
    \item Wiesner, S. "Conjugate Coding." \textit{ACM SIGACT News}, 15(1), 78–88 (1983).
    \item Zhandry, M. "Quantum Lightning Never Strikes the Same State Twice." \textit{EUROCRYPT 2019}, arXiv:1711.02276v3.
    \item Coladangelo, A. \& Sattath, O. "A Quantum Money Solution to the Blockchain Scalability Problem." \textit{Quantum}, 4, 297 (2020).
    \item Aaronson, S. \& Christiano, P. "Quantum Money from Hidden Subspaces." \textit{STOC 2012}.
    \item Lutomirski, A. et al. "Breaking and Making Quantum Money." \textit{ICS 2010}, arXiv:0912.3825.
\end{itemize}

\end{document}
