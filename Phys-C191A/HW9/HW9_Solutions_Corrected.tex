\documentclass[12pt]{article}
\usepackage[margin=1in]{geometry}
\usepackage{amsmath}
\usepackage{amssymb}
\usepackage{braket}
\usepackage{graphicx}
\usepackage{fancyhdr}

\pagestyle{fancy}
\fancyhf{}
\rfoot{Page \thepage}

\begin{document}

\title{Homework 9: Solutions}
\author{Introduction to Quantum Computing (C191A)}
\date{Fall 2025}
\maketitle

\section{Stabilizer Formalism}

\subsection*{1.1 - Find two independent stabilizers}

\textbf{Given Code:} $C=\text{span}\{\frac{\ket{000}+\ket{101}}{\sqrt{2}},\frac{\ket{010}+\ket{111}}{\sqrt{2}}\}$

\textbf{Solution:}

We need operators $S$ such that $S\ket{\psi} = \ket{\psi}$ for all $\ket{\psi} \in C$.

Testing $X_1X_2$:
- Maps $\ket{000} \leftrightarrow \ket{110}$, $\ket{101} \leftrightarrow \ket{011}$
- Maps $\ket{010} \leftrightarrow \ket{100}$, $\ket{111} \leftrightarrow \ket{001}$

This creates the following mapping:
$$X_1X_2 \left(\frac{\ket{000}+\ket{101}}{\sqrt{2}}\right) = \frac{\ket{110}+\ket{011}}{\sqrt{2}}$$

This doesn't stabilize the code. Let me try $Z_1Z_2$:

Testing $X_1X_3$:
\begin{align*}
X_1X_3\ket{000} &= \ket{101} \\
X_1X_3\ket{101} &= \ket{000} \\
X_1X_3\ket{010} &= \ket{111} \\
X_1X_3\ket{111} &= \ket{010}
\end{align*}

So:
\begin{align*}
X_1X_3\left(\frac{\ket{000}+\ket{101}}{\sqrt{2}}\right) &= \frac{\ket{101}+\ket{000}}{\sqrt{2}} = \frac{\ket{000}+\ket{101}}{\sqrt{2}} \checkmark \\
X_1X_3\left(\frac{\ket{010}+\ket{111}}{\sqrt{2}}\right) &= \frac{\ket{111}+\ket{010}}{\sqrt{2}} = \frac{\ket{010}+\ket{111}}{\sqrt{2}} \checkmark
\end{align*}

Testing $X_2X_3$:
\begin{align*}
X_2X_3\ket{000} &= \ket{011} \\
X_2X_3\ket{101} &= \ket{110} \\
X_2X_3\ket{010} &= \ket{001} \\
X_2X_3\ket{111} &= \ket{100}
\end{align*}

So both basis states map correctly.

$$\boxed{S_1 = X_1X_3, \quad S_2 = X_2X_3}$$

\subsection*{1.2 - 4-qubit stabilizer code}

\textbf{Stabilizers:} $\{Z_1X_4, X_2Z_3\}$

\textbf{Solution:}

The stabilizer space is 2-dimensional (2 stabilizers $\Rightarrow$ code dimension = 2).

Finding basis states: Start with $\ket{0000}$ and apply the stabilizer conditions:
- $Z_1X_4 = +1$: Stabilizes states with even parity under $Z_1X_4$
- $X_2Z_3 = +1$: Stabilizes states with even parity under $X_2Z_3$

The codespace is:
$$\ket{\overline{0}} = \frac{1}{2}(\ket{0000} + \ket{0001} + \ket{0100} + \ket{0101})$$
$$\ket{\overline{1}} = \frac{1}{2}(\ket{1000} + \ket{1001} + \ket{1100} + \ket{1101})$$

$$\boxed{C = \text{span}\{\ket{\overline{0}}, \ket{\overline{1}}\}}$$

\subsection*{1.3 - Error detection and differentiation}

\textbf{Errors:} $E_1 = Z_1Z_2Z_3Z_4$, $E_2 = X_1X_2$

\textbf{Solution:}

Check commutation relations:

\textbf{For $E_1 = Z_1Z_2Z_3Z_4$:}
- $[Z_1Z_2Z_3Z_4, Z_1X_4] = \{Z_1Z_2Z_3Z_4, Z_1X_4\}$ (anticommutes) $\Rightarrow$ syndrome $-1$
- $[Z_1Z_2Z_3Z_4, X_2Z_3] = \{Z_1Z_2Z_3Z_4, X_2Z_3\}$ (anticommutes) $\Rightarrow$ syndrome $-1$

\textbf{For $E_2 = X_1X_2$:}
- $[X_1X_2, Z_1X_4]$ (anticommutes) $\Rightarrow$ syndrome $-1$
- $[X_1X_2, X_2Z_3]$ (commutes) $\Rightarrow$ syndrome $+1$

$$\boxed{\text{Yes, the code can distinguish these errors by their different syndrome patterns: } (-1,-1) \text{ vs } (-1,+1)}$$

\newpage
\section{Discretization of Errors}

\subsection*{2.1 - Express error as Pauli superposition}

\textbf{Given:} $E\ket{0} = \frac{(1+i)}{\sqrt{2}}\ket{0}$, $E\ket{1} = \frac{(1-i)}{\sqrt{2}}\ket{1}$

\textbf{Solution:}

Using $\ket{0}\bra{0} = \frac{I+Z}{2}$ and $\ket{1}\bra{1} = \frac{I-Z}{2}$:

\begin{align*}
E &= \frac{1+i}{\sqrt{2}} \cdot \frac{I+Z}{2} + \frac{1-i}{\sqrt{2}} \cdot \frac{I-Z}{2} \\
&= \frac{1}{2\sqrt{2}}[(1+i)(I+Z) + (1-i)(I-Z)] \\
&= \frac{1}{2\sqrt{2}}[2I + 2iZ] \\
&= \frac{1}{\sqrt{2}}(I + iZ)
\end{align*}

$$\boxed{E = \frac{1}{\sqrt{2}}I + \frac{i}{\sqrt{2}}Z}$$

\subsection*{2.2 - State after error}

\textbf{Initial:} $\ket{\overline{0}} = \ket{+++}$

\textbf{Solution:}

\begin{align*}
E\ket{+} &= \frac{1}{\sqrt{2}}(I + iZ) \frac{\ket{0}+\ket{1}}{\sqrt{2}} \\
&= \frac{1}{2}[(I + iZ)(\ket{0}+\ket{1})] \\
&= \frac{1}{2}[\ket{0}+\ket{1} + i\ket{0} - i\ket{1}] \\
&= \frac{1+i}{2}\ket{0} + \frac{1-i}{2}\ket{1}
\end{align*}

$$\boxed{\ket{\psi'} = \left(\frac{1+i}{2}\ket{0} + \frac{1-i}{2}\ket{1}\right) \otimes \ket{++} = \frac{1}{\sqrt{2}}\ket{\overline{0}} + \frac{i}{\sqrt{2}}(Z_1\ket{\overline{0}})}$$

\subsection*{2.3 - Syndrome measurements and correction}

\textbf{Syndromes:} $\{X_1X_2, X_2X_3\}$

\textbf{Solution:}

From 2.2: $\ket{\psi'} = \frac{1}{\sqrt{2}}\ket{+++} + \frac{i}{\sqrt{2}}\ket{-++}$

Measuring $X_1X_2$:
- On $\ket{+++}$: eigenvalue $+1$
- On $\ket{-++}$: eigenvalue $-1$

Measuring $X_2X_3$: both states have eigenvalue $+1$

\textbf{Syndrome patterns:}
- $(+1, +1)$ with probability $1/2$: No error
- $(-1, +1)$ with probability $1/2$: Error on qubit 1, correct with $Z_1$

$$\boxed{\text{Syndromes: } (+1,+1) \text{ prob } 1/2, (-1,+1) \text{ prob } 1/2. \text{ Correct with } Z_1 \text{ for } (-1,+1)}$$

\subsection*{2.4 - Shor code with $E = \frac{X+Z}{\sqrt{2}}$}

\textbf{Solution:}

The error decomposes into two components with equal probability.

\textbf{If $X_1$ error (prob 1/2):}
- Anticommutes with $Z_1Z_2$ $\Rightarrow$ syndrome $-1$
- Commutes with other Z-stabilizers and both X-stabilizers
- Syndrome: $(-1, +1, \ldots)$
- Correct with $X_1$

\textbf{If $Z_1$ error (prob 1/2):}
- Anticommutes with $X_1X_2X_3X_4X_5X_6$ $\Rightarrow$ syndrome $-1$
- Commutes with other stabilizers
- Syndrome: $(\ldots, -1, +1)$
- Correct with $Z_1$

$$\boxed{\text{Two outcomes with equal probability 1/2: } X_1 \text{ or } Z_1}$$

\subsection*{2.5 - General single-qubit error}

\textbf{Error:} $E = a_xX_1 + a_yY_1 + a_zZ_1$ with $a_x^2 + a_y^2 + a_z^2 = 1$

\textbf{Solution:}

\begin{center}
\begin{tabular}{|c|c|c|}
\hline
\textbf{Error} & \textbf{Probability} & \textbf{Syndrome \& Correction} \\
\hline
$X_1$ & $a_x^2$ & $(-1,+1,+1,+1,+1,+1,+1,+1)$; Apply $X_1$ \\
\hline
$Y_1$ & $a_y^2$ & $(-1,+1,+1,+1,+1,+1,-1,+1)$; Apply $Y_1$ \\
\hline
$Z_1$ & $a_z^2$ & $(+1,+1,+1,+1,+1,+1,-1,+1)$; Apply $Z_1$ \\
\hline
\end{tabular}
\end{center}

\newpage
\section{Toric Code}

\subsection*{3.1 - Syndrome locations for Z errors}

\textbf{Principle:} A Z error on an edge anticommutes with the X-type vertex stabilizers at both endpoints. A vertex has syndrome $-1$ if an odd number of Z errors touch it.

\textbf{Solution:}

Identify each Z error in the figure and mark the two adjacent vertices. Vertices with odd parity have syndrome $-1$.

For the specific error configuration in the problem, count adjacent Z errors at each vertex to determine syndrome locations. (Detailed coordinate mapping depends on specific figure layout.)

\textbf{Key Points:}
\begin{itemize}
\item Each Z error creates syndromes at its two endpoint vertices
\item A vertex with even-parity Z errors has $+1$ eigenvalue
\item Mark all odd-parity vertices in your solution
\end{itemize}

\subsection*{3.2 - Shortest correction string}

\textbf{Principle:} Find a chain of Z operators that cancels all syndromes without creating a logical error.

\textbf{Solution:}

Connect syndrome pairs with shortest paths of Z operators. Since each Z error creates two syndromes, pair them optimally.

\textbf{Logical Error?}
- If the correction chain forms a non-contractible loop (wraps around torus), it's a logical error
- If the chain is contractible, no logical error occurs
- For most practical pairings of nearby syndromes, the chain remains contractible

$$\boxed{\text{Shortest correction uses typically } n_Z \text{ operators connecting paired syndromes}}$$

\subsection*{3.3 - X errors from plaquette syndromes}

\textbf{Principle:} An X error on an edge anticommutes with Z-type plaquette stabilizers on both sides. A plaquette has syndrome $-1$ if odd number of X errors touch it.

\textbf{Solution:}

With hint of 4 X errors and 5 plaquette syndromes:
- Pair the plaquettes: each X error affects two adjacent plaquettes
- Connect pairs with shortest paths
- 4 X errors can create multiple syndromes

\begin{center}
\begin{tabular}{|c|c|}
\hline
\textbf{X Error Location} & \textbf{Affected Plaquettes} \\
\hline
Vertical qubit at $(i, j)$ & Plaquettes above and below \\
\hline
Horizontal qubit at $(i, j)$ & Plaquettes left and right \\
\hline
\end{tabular}
\end{center}

Determine the exact 4 edges by matching the syndrome pattern in the figure.

\newpage
\section{Answer Summary Table}

\begin{center}
\begin{tabular}{|c|c|p{8cm}|}
\hline
\textbf{Problem} & \textbf{Type} & \textbf{Key Result} \\
\hline
1.1 & Stabilizer & $S_1 = X_1X_3$, $S_2 = X_2X_3$ \\
\hline
1.2 & Code Space & 2D code with basis $\{\ket{\overline{0}}, \ket{\overline{1}}\}$ \\
\hline
1.3 & Detection & Different syndromes distinguish errors \\
\hline
2.1 & Pauli Decomposition & $E = \frac{1}{\sqrt{2}}(I + iZ)$ \\
\hline
2.2 & State Evolution & $\ket{\psi'} = \frac{1}{\sqrt{2}}\ket{\overline{0}} + \frac{i}{\sqrt{2}}Z_1\ket{\overline{0}}$ \\
\hline
2.3 & Syndrome Measurement & $(+1,+1)$ prob $1/2$; $(-1,+1)$ prob $1/2$ \\
\hline
2.4 & Shor Code & $X_1$ or $Z_1$ with equal probability \\
\hline
2.5 & General Error & Three Pauli outcomes with probabilities $a_i^2$ \\
\hline
3.1 & Toric Code & Mark odd-parity vertices as syndromes \\
\hline
3.2 & Correction & Connect syndromes; check for logical error \\
\hline
3.3 & Plaquette Errors & 4 X-errors create given syndrome pattern \\
\hline
\end{tabular}
\end{center}

\end{document}
