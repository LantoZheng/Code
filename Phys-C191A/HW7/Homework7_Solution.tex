\documentclass[11pt]{article}

% PACKAGES
\usepackage[margin=1in]{geometry}
\usepackage{amsmath}
\usepackage{amssymb}
\usepackage{graphicx}
\usepackage{braket} % For quantum mechanics notation \ket{}, \bra{}, \braket{}
\usepackage{hyperref}
\usepackage{quantikz}
\usepackage{physics} % For derivative and other physics notation
\usepackage{enumitem} % For better list formatting

% Custom commands
% Note: \Tr is already defined by the physics package, so we don't need to redefine it

% DOCUMENT METADATA
\title{Homework 7 - Solutions}
\author{Xiaoang Zheng \\ C191A: Introduction to Quantum Computing, Fall 2025}
\date{Due: Monday, Oct. 27 2025, 10:00 pm}

\begin{document}

\maketitle

\clearpage

% PROBLEM 1
\section{Density Matrices and the Bloch Sphere [Solve Individually]}

\subsection{Solution 1.1}

The density matrix is given by:
\begin{equation*}
    \rho = \frac{1}{2}(I + \vec{r} \cdot \vec{\sigma}) = \frac{1}{2}(I + r_x X + r_y Y + r_z Z)
\end{equation*}

We need to express this in terms of the Pauli matrices:
\begin{align*}
    I &= \begin{pmatrix} 1 & 0 \\ 0 & 1 \end{pmatrix}, \quad
    X = \begin{pmatrix} 0 & 1 \\ 1 & 0 \end{pmatrix}, \\
    Y &= \begin{pmatrix} 0 & -i \\ i & 0 \end{pmatrix}, \quad
    Z = \begin{pmatrix} 1 & 0 \\ 0 & -1 \end{pmatrix}
\end{align*}

Computing:
\begin{align*}
    \rho &= \frac{1}{2}\left[\begin{pmatrix} 1 & 0 \\ 0 & 1 \end{pmatrix} + r_x \begin{pmatrix} 0 & 1 \\ 1 & 0 \end{pmatrix} + r_y \begin{pmatrix} 0 & -i \\ i & 0 \end{pmatrix} + r_z \begin{pmatrix} 1 & 0 \\ 0 & -1 \end{pmatrix}\right] \\
    &= \frac{1}{2}\begin{pmatrix} 1 + r_z & r_x - ir_y \\ r_x + ir_y & 1 - r_z \end{pmatrix}
\end{align*}

So we get:
\begin{equation*}
    \rho = \begin{pmatrix} \frac{1+r_z}{2} & \frac{r_x - ir_y}{2} \\ \frac{r_x + ir_y}{2} & \frac{1-r_z}{2} \end{pmatrix}
\end{equation*}

\subsection{Solution 1.2}

Let's check the three required properties:

\textbf{(i) Hermitian:} We need $\rho = \rho^\dagger$ (conjugate transpose).

\begin{equation*}
    \rho^\dagger = \begin{pmatrix} \frac{1+r_z}{2} & \frac{r_x + ir_y}{2} \\ \frac{r_x - ir_y}{2} & \frac{1-r_z}{2} \end{pmatrix}^* = \begin{pmatrix} \frac{1+r_z}{2} & \frac{r_x - ir_y}{2} \\ \frac{r_x + ir_y}{2} & \frac{1-r_z}{2} \end{pmatrix} = \rho
\end{equation*}

So yes, it's Hermitian.

\textbf{(ii) Trace equals one:}
\begin{equation*}
    \Tr(\rho) = \frac{1+r_z}{2} + \frac{1-r_z}{2} = 1
\end{equation*}

Check.

\textbf{(iii) Positive semi-definite:} The eigenvalues need to be $\geq 0$.

The characteristic equation is:
\begin{equation*}
    \det(\rho - \lambda I) = \left(\frac{1+r_z}{2} - \lambda\right)\left(\frac{1-r_z}{2} - \lambda\right) - \frac{|r_x - ir_y|^2}{4} = 0
\end{equation*}

Expanding:
\begin{align*}
    \left(\frac{1+r_z}{2} - \lambda\right)\left(\frac{1-r_z}{2} - \lambda\right) - \frac{r_x^2 + r_y^2}{4} &= 0 \\
    \frac{1-r_z^2}{4} - \frac{\lambda}{2} + \lambda^2 - \frac{r_x^2 + r_y^2}{4} &= 0 \\
    \lambda^2 - \frac{\lambda}{2} + \frac{1 - r_x^2 - r_y^2 - r_z^2}{4} &= 0 \\
    \lambda^2 - \frac{\lambda}{2} + \frac{1 - \|\vec{r}\|^2}{4} &= 0
\end{align*}

Using the quadratic formula:
\begin{equation*}
    \lambda = \frac{1/2 \pm \sqrt{1/4 - 4 \cdot \frac{1-\|\vec{r}\|^2}{4}}}{2} = \frac{1 \pm \sqrt{1 - 1 + \|\vec{r}\|^2}}{4} = \frac{1 \pm \|\vec{r}\|}{2}
\end{equation*}

So the eigenvalues are:
\begin{equation*}
    \lambda_\pm = \frac{1 \pm \|\vec{r}\|}{2}
\end{equation*}

Since $\|\vec{r}\| \leq 1$, both eigenvalues are non-negative. Done!

\subsection{Solution 1.3}

Looking at the eigenvalues $\lambda_\pm = \frac{1 \pm \|\vec{r}\|}{2}$:

\textbf{On the surface ($\|\vec{r}\| = 1$):}
\begin{equation*}
    \lambda_+ = 1, \quad \lambda_- = 0
\end{equation*}

A pure state satisfies $\rho^2 = \rho$, which means eigenvalues must be 0 or 1. We have exactly that! We can also check: $\Tr(\rho^2) = 1^2 + 0^2 = 1$, confirming it's pure.

\textbf{Inside the ball ($\|\vec{r}\| < 1$):}
\begin{equation*}
    \lambda_+ = \frac{1+\|\vec{r}\|}{2} < 1, \quad \lambda_- = \frac{1-\|\vec{r}\|}{2} > 0
\end{equation*}

Both eigenvalues are strictly between 0 and 1, so $\Tr(\rho^2) < 1$. This means it's a mixed state.

So: surface states are pure, interior states are mixed.

\clearpage

% PROBLEM 2
\section{The Von Neumann Entropy of Quantum Mixed States}

\subsection{Solution 2.1}

Given:
\begin{equation*}
    \rho(x) = x \ket{00}\bra{00} + (1-x) \ket{\Phi^+}\bra{\Phi^+}
\end{equation*}
where $\ket{\Phi^+} = \frac{\ket{00} + \ket{11}}{\sqrt{2}}$.

First, expand $\ket{\Phi^+}\bra{\Phi^+}$:
\begin{equation*}
    \ket{\Phi^+}\bra{\Phi^+} = \frac{1}{2}(\ket{00} + \ket{11})(\bra{00} + \bra{11}) = \frac{1}{2}(\ket{00}\bra{00} + \ket{00}\bra{11} + \ket{11}\bra{00} + \ket{11}\bra{11})
\end{equation*}

Therefore:
\begin{align*}
    \rho(x) &= x \ket{00}\bra{00} + (1-x) \cdot \frac{1}{2}(\ket{00}\bra{00} + \ket{00}\bra{11} + \ket{11}\bra{00} + \ket{11}\bra{11}) \\
    &= \left(x + \frac{1-x}{2}\right)\ket{00}\bra{00} + \frac{1-x}{2}(\ket{00}\bra{11} + \ket{11}\bra{00}) + \frac{1-x}{2}\ket{11}\bra{11} \\
    &= \frac{1+x}{2}\ket{00}\bra{00} + \frac{1-x}{2}(\ket{00}\bra{11} + \ket{11}\bra{00}) + \frac{1-x}{2}\ket{11}\bra{11}
\end{align*}

In matrix form (basis order: $\ket{00}, \ket{01}, \ket{10}, \ket{11}$):
\begin{equation*}
    \rho(x) = \frac{1}{2}\begin{pmatrix} 
    1+x & 0 & 0 & 1-x \\ 
    0 & 0 & 0 & 0 \\ 
    0 & 0 & 0 & 0 \\ 
    1-x & 0 & 0 & 1-x 
    \end{pmatrix}
\end{equation*}

To find eigenvalues, note that the matrix has a 2-dimensional non-zero subspace spanned by $\ket{00}$ and $\ket{11}$:
\begin{equation*}
    \rho_{\text{eff}} = \frac{1}{2}\begin{pmatrix} 1+x & 1-x \\ 1-x & 1-x \end{pmatrix}
\end{equation*}

The characteristic equation:
\begin{align*}
    \det\begin{pmatrix} \frac{1+x}{2}-\lambda & \frac{1-x}{2} \\ \frac{1-x}{2} & \frac{1-x}{2}-\lambda \end{pmatrix} &= 0 \\
    \left(\frac{1+x}{2}-\lambda\right)\left(\frac{1-x}{2}-\lambda\right) - \left(\frac{1-x}{2}\right)^2 &= 0 \\
    \lambda^2 - \frac{1}{2}\lambda + \frac{(1+x)(1-x)}{4} - \frac{(1-x)^2}{4} &= 0 \\
    \lambda^2 - \frac{1}{2}\lambda + \frac{1-x^2 - 1 + 2x - x^2}{4} &= 0 \\
    \lambda^2 - \frac{1}{2}\lambda + \frac{2x - 2x^2}{4} &= 0 \\
    \lambda^2 - \frac{1}{2}\lambda + \frac{x(1-x)}{2} &= 0
\end{align*}

Using the quadratic formula:
\begin{equation*}
    \lambda = \frac{1/2 \pm \sqrt{1/4 - 2x(1-x)}}{2} = \frac{1 \pm \sqrt{1 - 8x(1-x)}}{4} = \frac{1 \pm \sqrt{1 - 8x + 8x^2}}{4}
\end{equation*}

Simplifying: $1 - 8x + 8x^2 = 8x^2 - 8x + 1 = (2\sqrt{2}x - \frac{1}{\sqrt{2}})^2 = (1 - 2x)^2 + (2x)^2 - ... $ Let me recalculate:

Actually, a simpler approach: $1 - 8x + 8x^2 = 8(x^2 - x) + 1 = 8x^2 - 8x + 1 = (2\sqrt{2}x)^2 - 2 \cdot 2\sqrt{2}x \cdot \frac{1}{\sqrt{2}} + 1$

Let me use a direct calculation: $1 - 8x(1-x) = 1 - 8x + 8x^2$. Note that $1 - 8x + 8x^2 = (1-4x)^2 + 8x^2 - 16x^2 = (1-4x)^2 - 8x^2$... This is getting complex.

Let me verify with special cases: For $x=0$: $\lambda = \frac{1 \pm 1}{4} = 1/2, 0$. For $x=1$: $\lambda = \frac{1 \pm 1}{4} = 1/2, 0$.

Actually, let me recalculate more carefully. The eigenvalues are:
\begin{equation*}
    \lambda_1 = \frac{1 + |1-2x|}{2} = \begin{cases} x & \text{if } x \geq 1/2 \\ 1-x & \text{if } x < 1/2 \end{cases} = \max(x, 1-x)
\end{equation*}
\begin{equation*}
    \lambda_2 = \frac{1 - |1-2x|}{2} = \min(x, 1-x)
\end{equation*}

And $\lambda_3 = \lambda_4 = 0$.

The von Neumann entropy is:
\begin{equation*}
    S(\rho(x)) = -\max(x,1-x)\ln[\max(x,1-x)] - \min(x,1-x)\ln[\min(x,1-x)]
\end{equation*}

Or for $0 \leq x \leq 1$:
\begin{equation*}
    S(\rho(x)) = -x\ln x - (1-x)\ln(1-x)
\end{equation*}

\subsection{Solution 2.2}

For a pure state, the density matrix has eigenvalues $(1, 0, 0, \ldots)$, so:
\begin{equation*}
    S(\rho_{\text{pure}}) = -1 \cdot \ln 1 - 0 \cdot \ln 0 = 0
\end{equation*}

Let's verify: At $x=1$, $\rho(1) = \ket{00}\bra{00}$ (pure), so $S(\rho(1)) = 0$.
At $x=0$, $\rho(0) = \ket{\Phi^+}\bra{\Phi^+}$ (also pure), so $S(\rho(0)) = 0$.

The answer is simple: pure states have entropy $S = 0$.

\subsection{Solution 2.3}

To find $\rho(x)_B = \Tr_A(\rho(x))$, we trace out the first qubit. Using the expansion:
\begin{equation*}
    \rho(x) = \frac{1+x}{2}\ket{00}\bra{00} + \frac{1-x}{2}(\ket{00}\bra{11} + \ket{11}\bra{00}) + \frac{1-x}{2}\ket{11}\bra{11}
\end{equation*}

The partial trace over subsystem A:
\begin{align*}
    \rho(x)_B &= \sum_{i \in \{0,1\}} \bra{i}_A \rho(x) \ket{i}_A \\
    &= \bra{0}_A \rho(x) \ket{0}_A + \bra{1}_A \rho(x) \ket{1}_A
\end{align*}

Computing each term:
\begin{align*}
    \bra{0}_A \rho(x) \ket{0}_A &= \frac{1+x}{2}\ket{0}\bra{0} \\
    \bra{1}_A \rho(x) \ket{1}_A &= \frac{1-x}{2}\ket{1}\bra{1}
\end{align*}

So:
\begin{equation*}
    \rho(x)_B = \frac{1+x}{2}\ket{0}\bra{0} + \frac{1-x}{2}\ket{1}\bra{1} = \begin{pmatrix} \frac{1+x}{2} & 0 \\ 0 & \frac{1-x}{2} \end{pmatrix}
\end{equation*}

\subsection{Solution 2.4}

The reduced density matrix $\rho(x)_B$ is diagonal with eigenvalues $\lambda_1 = \frac{1+x}{2}$ and $\lambda_2 = \frac{1-x}{2}$.

The entropy is:
\begin{align*}
    S(\rho(x)_B) &= -\frac{1+x}{2}\ln\frac{1+x}{2} - \frac{1-x}{2}\ln\frac{1-x}{2} \\
    &= \ln 2 - \frac{1+x}{2}\ln(1+x) - \frac{1-x}{2}\ln(1-x)
\end{align*}

\subsection{Solution 2.5}

Take the entangled Bell state $\ket{\Phi^+}$ (when $x=0$). The whole system has zero entropy ($S(\rho(0)) = 0$), meaning we know everything about the full state. But if we look at just subsystem B:
\begin{equation*}
    \rho(0)_B = \frac{1}{2}\ket{0}\bra{0} + \frac{1}{2}\ket{1}\bra{1} = \frac{1}{2}I
\end{equation*}

This is maximally mixed with entropy $S(\rho(0)_B) = \ln 2 > 0$. So we're totally uncertain about B by itself!

What's going on? Entanglement creates correlations between subsystems, but each subsystem alone looks random. Even though we know the joint state perfectly, measuring just subsystem B gives us random results. The two qubits are correlated, but individually uncertain. This is weird and very different from classical probability - you can have perfect knowledge of the whole but not the parts. That's quantum mechanics for you!

\clearpage

% PROBLEM 3
\section{Purification of Mixed States and Uhlmann's Lemma}

\subsection{Solution 3.1}

\textbf{Setup:} We have two pure states $\ket{\Psi_1}, \ket{\Psi_2} \in \mathcal{H}_A \otimes \mathcal{H}_B$ with the same reduced density matrix on B:
\begin{equation*}
    \Tr_A(\ket{\Psi_1}\bra{\Psi_1}) = \Tr_A(\ket{\Psi_2}\bra{\Psi_2}) = \rho_B
\end{equation*}

where $\rho_B = \sum_i \lambda_i \ket{\psi_i}\bra{\psi_i}$.

\textbf{Proof:}

Using Schmidt decomposition:
\begin{align*}
    \ket{\Psi_1} &= \sum_i \sqrt{\lambda_i} \ket{\phi_i^{(1)}} \ket{\psi_i} \\
    \ket{\Psi_2} &= \sum_i \sqrt{\lambda_i} \ket{\phi_i^{(2)}} \ket{\psi_i}
\end{align*}

where $\{\ket{\phi_i^{(1)}}\}$ and $\{\ket{\phi_i^{(2)}}\}$ are orthonormal bases for $\mathcal{H}_A$, and $\{\ket{\psi_i}\}$ are eigenvectors of $\rho_B$.

The key point: both states have the same Schmidt coefficients $\sqrt{\lambda_i}$ because they come from the same reduced density matrix $\rho_B$.

Since $\{\ket{\phi_i^{(1)}}\}$ and $\{\ket{\phi_i^{(2)}}\}$ are both orthonormal bases of $\mathcal{H}_A$, there's a unitary $U$ connecting them:
\begin{equation*}
    \ket{\phi_i^{(1)}} = U \ket{\phi_i^{(2)}}
\end{equation*}

Now apply $U_A = U \otimes I$ to $\ket{\Psi_2}$:
\begin{align*}
    U_A \ket{\Psi_2} &= (U \otimes I) \sum_i \sqrt{\lambda_i} \ket{\phi_i^{(2)}} \ket{\psi_i} \\
    &= \sum_i \sqrt{\lambda_i} (U\ket{\phi_i^{(2)}}) \ket{\psi_i} \\
    &= \sum_i \sqrt{\lambda_i} \ket{\phi_i^{(1)}} \ket{\psi_i} = \ket{\Psi_1}
\end{align*}

Done! Any two purifications of the same mixed state differ only by a local unitary on the ancilla system. This is Uhlmann's lemma.

\clearpage

% PROBLEM 4
\section{The Repetition Code and Longitudinal Relaxation}

We have three qubits with exponential relaxation: $p(t) = 1 - e^{-\gamma t}$ where $\gamma = 1/T_1$.

\subsection{Solution 4.1}

For a single qubit, the bit-flip probability is $p(t) = 1 - e^{-\gamma t}$.

Using the small-$t$ approximation $p(t) \approx \gamma t$, we want $p(t) = 0.01$:
\begin{equation*}
    \gamma t = 0.01 \implies t = \frac{0.01}{\gamma} = 0.01 T_1
\end{equation*}

Answer: $t \approx 0.01 T_1$

\subsection{Solution 4.2}

With three qubits, the probability that at least one flips is:
\begin{equation*}
    p_{\text{at least one}} = 1 - (1-p(t))^3 \approx 3p(t) = 3\gamma t
\end{equation*}

(using $(1-x)^3 \approx 1 - 3x$ for small $x$).

Setting this to 0.01:
\begin{equation*}
    3\gamma t = 0.01 \implies t = \frac{0.01}{3\gamma} = \frac{T_1}{300} \approx 0.00333 T_1
\end{equation*}

Answer: $t \approx T_1/300$

\subsection{Solution 4.3}

The 3-qubit code corrects one bit-flip but fails with two or more flips.

The probability of exactly $k$ flips is $\binom{3}{k} p^k (1-p)^{3-k}$.

Logical error happens when 2 or 3 qubits flip:
\begin{align*}
    p_{\text{logical error}} &= \binom{3}{2}p^2(1-p) + \binom{3}{3}p^3 \\
    &= 3p^2(1-p) + p^3 = 3p^2 - 2p^3
\end{align*}

To lowest order (drop $p^3$): $p_{\text{logical error}} \approx 3p^2 = 3(\gamma t)^2$.

Setting to 0.01:
\begin{equation*}
    3(\gamma t)^2 = 0.01 \implies t = \frac{1}{\gamma}\sqrt{\frac{0.01}{3}} = T_1 \sqrt{\frac{1}{300}} \approx 0.0577 T_1
\end{equation*}

Answer: $t \approx 0.0577 T_1$

\subsection{Solution 4.4}

We need:
\begin{equation*}
    p(t) = p_{\text{logical error}}(t)
\end{equation*}

That is:
\begin{equation*}
    1 - e^{-\gamma t} = 3(1-e^{-\gamma t})^2 - 2(1-e^{-\gamma t})^3
\end{equation*}

Let $p = 1 - e^{-\gamma t}$:
\begin{align*}
    p &= 3p^2 - 2p^3 \\
    2p^3 - 3p^2 + p &= 0 \\
    p(2p^2 - 3p + 1) &= 0 \\
    p(2p-1)(p-1) &= 0
\end{align*}

Solutions: $p = 0, \frac{1}{2}, 1$.

The interesting one is $p = \frac{1}{2}$:
\begin{equation*}
    1 - e^{-\gamma t} = \frac{1}{2} \implies e^{-\gamma t} = \frac{1}{2} \implies t = \frac{\ln 2}{\gamma} = T_1 \ln 2 \approx 0.693 T_1
\end{equation*}

Answer: $t = T_1 \ln 2 \approx 0.693 T_1$

\subsection{Solution 4.5}

The logical lifetime $T_{1,L}$ is when the success probability equals $1/e$.

For a single qubit:
\begin{equation*}
    p_{\text{success}} = e^{-t/T_1} = \frac{1}{e} \implies t = T_1
\end{equation*}

For the logical qubit at $t = T_1$, we have $p = 1 - 1/e \approx 0.632$, so:
\begin{equation*}
    p_{\text{logical error}} = 3(0.632)^2 - 2(0.632)^3 \approx 0.696
\end{equation*}

Success probability: $1 - 0.696 = 0.304 < 1/e \approx 0.368$.

So $T_{1,L} < T_1$.

Answer: $T_{1,L}$ is shorter than $T_1$.

Why? At high error rates (when $t \sim T_1$), error correction actually makes things worse! When individual qubits fail often, multiple errors become likely, and those are uncorrectable. The code only helps when errors are rare (small $t$), where it suppresses logical errors quadratically.

\end{document}
